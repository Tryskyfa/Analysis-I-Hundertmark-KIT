\documentclass[11pt, twoside, a4paper]{article}

% Setup
\usepackage[margin=2.4cm, top=3.5cm]{geometry}
\usepackage[utf8]{inputenc}
\usepackage[ngerman]{babel}

% Package imports
\usepackage{amsfonts}
\usepackage{amsmath}
\usepackage{amssymb}
\usepackage{amsthm}
\usepackage{mathtools}
\usepackage{setspace}
\usepackage{float}
\usepackage{enumitem}
\usepackage{hyperref}
\usepackage[pagestyles]{titlesec}
\usepackage{fancyhdr}
\usepackage{colonequals}
\usepackage{caption}
\usepackage{tikz}
\usepackage{marginnote}
\usepackage{etoolbox}
\usepackage{mdframed}
\usepackage{aligned-overset}

% Font-Encoding
\usepackage[T1]{fontenc}
\usepackage{lmodern}

% Theorems
\newtheorem{blockelement}{Blockelement}[subsection]
\newtheoremstyle{plain}{}{}{}{}{\bfseries}{.}{ }{}
\theoremstyle{plain}
\newtheorem{bemerkung}[blockelement]{Bemerkung}
\newtheorem{definition}[blockelement]{Definition}
\newtheorem{lemma}[blockelement]{Lemma}
\newtheorem{satz}[blockelement]{Satz}
\newtheorem{notation}[blockelement]{Notation}
\newtheorem{korollar}[blockelement]{Korollar}
\newtheorem{uebung}[blockelement]{Übung}
\newtheorem{beispiel}[blockelement]{Beispiel}
\newtheorem{folgerung}[blockelement]{Folgerung}
\newtheorem{axiom}[blockelement]{Axiom}
\newtheorem{beobachtung}[blockelement]{Beobachtung}
\newtheorem{konzept}[blockelement]{Konzept}
\newtheorem{visualisierung}[blockelement]{Visualisierung}
\newtheorem{anwendung}[blockelement]{Anwendung}
\newtheorem{skizze}[blockelement]{Skizze}

% Marginnotes left
\makeatletter
\patchcmd{\@mn@@@marginnote}{\begingroup}{\begingroup\@twosidefalse}{}{\fail}
\reversemarginpar
\makeatother

% Long equations
\allowdisplaybreaks

% \left \right
\newcommand{\set}[1]{\left\{#1\right\}}
\newcommand{\pair}[1]{\left(#1\right)}
\newcommand{\of}[1]{\mathopen{}\mathclose{}\bgroup\left(#1\aftergroup\egroup\right)}
\newcommand{\abs}[1]{\left\lvert#1\right\rvert}
\newcommand{\norm}[1]{\left\lVert#1\right\rVert}
\newcommand{\linterv}[1]{\left[#1\right)}
\newcommand{\rinterv}[1]{\left(#1\right]}
\newcommand{\interv}[1]{\left[#1\right]}
\newcommand{\sprod}[1]{\left<#1\right>}

% Shorten commands
\newcommand{\equivalent}[0]{\Leftrightarrow{}}
\newcommand{\impl}[0]{\Rightarrow{}}
\newcommand{\fromto}{\rightarrow{}}
\newcommand{\definedas}[0]{\coloneqq}
\newcommand{\definedasbackwards}[0]{\eqqcolon}
\newcommand{\definedasequiv}[0]{\ratio\Leftrightarrow{}}
\newcommand{\exclude}[0]{\setminus}
\renewcommand{\emptyset}{\varnothing}
\newcommand{\sbset}{\subseteq}

\newcommand{\ntoinf}[0]{n\fromto\infty}
\newcommand{\toinf}{\fromto\infty}
\newcommand{\fa}{\;\forall\,}
\newcommand{\ex}{\;\exists\,}
\newcommand{\conj}[1]{\overline{#1}}

\newcommand{\annot}[3][]{\overset{\text{#3}}#1{#2}}
\newcommand{\biglim}[1]{{\displaystyle \lim_{#1}}}
\newcommand{\nn}[0]{\\[2\baselineskip]}
\newcommand{\anf}[1]{\glqq{}#1\grqq}
\newcommand{\OBDA}{o.B.d.A. }
\newcommand{\theoremescape}{\leavevmode}
\newcommand{\aligntoright}[2]{\hfill#1\hspace{#2\textwidth}~}
\newcommand{\horizontalline}[0]{\par\noindent\rule{0.05\textwidth}{0.1pt}\\}
\newcommand{\rgbcolor}[3]{rgb,255:red,#1;green,#2;blue,#3}
\newcommand{\fixedspace}[2]{\makebox[#1][l]{#2}}

\let\Re\relax
\let\Im\relax

% MathOperators
\DeclareMathOperator{\grad}{Grad}
\DeclareMathOperator{\bild}{Bild}
\DeclareMathOperator{\Re}{Re}
\DeclareMathOperator{\Im}{Im}

% Mengenbezeichner
\newcommand{\R}{\mathbb{R}}
\newcommand{\N}{\mathbb{N}}
\newcommand{\C}{\mathbb{C}}
\newcommand{\Z}{\mathbb{Z}}
\newcommand{\Q}{\mathbb{Q}}
\newcommand{\K}{\mathbb{K}}

\newcommand\imaginarysubsection[1]{
    \refstepcounter{subsection}
    \subsectionmark{#1}
}

% Unfassbar hässlich, aber effektiv für temporäre schnelle Lösungen
\def\:={\coloneqq}
\def\->{\fromto}
\def\=>{\impl}
\def\<={\leq}
\def\>={\geq}
\def\!={\neq}

% Envs
\newenvironment{induktionsanfang}{
    \rule{0pt}{3ex}\noindent
    \begin{minipage}[t]{0.11\textwidth}
    {I-Anfang}
    \end{minipage}
    \hfill
    \begin{minipage}[t]{0.89\textwidth}
    }
    {
    \end{minipage}
}
\newenvironment{induktionsvoraussetzung}{
    \rule{0pt}{3ex}\noindent
    \begin{minipage}[t]{0.11\textwidth}
    {I-Vor.}
    \end{minipage}
    \hfill
    \begin{minipage}[t]{0.89\textwidth}
    }
    {
    \end{minipage}
}
\newenvironment{induktionsschritt}{
    \rule{0pt}{3ex}\noindent
    \begin{minipage}[t]{0.11\textwidth}
    {I-Schritt}
    \end{minipage}
    \hfill
    \begin{minipage}[t]{0.89\textwidth}
    }
    {
    \end{minipage}
}

% Section style
\titleformat*{\section}{\LARGE\bfseries}
\titleformat*{\subsection}{\large\bfseries}

% Page styles
\newpagestyle{pagenumberonly}{
    \sethead{}{}{}
    \setfoot[][][\thepage]{\thepage}{}{}
}
\newpagestyle{headfootdefault}{
    \sethead[][][\thesubsection~\textit{\subsectiontitle}]{\thesection~\textit{\sectiontitle}}{}{}
    \setfoot[][][\thepage]{\thepage}{}{}
}
\pagestyle{headfootdefault}

\begin{document}
    \title{\vspace{3cm} Skript zur Vorlesung\\Analysis I\\bei Prof. Dr. Dirk Hundertmark}
    \author{Karlsruher Institut für Technologie}
    \date{Wintersemester 2023/24}
    \maketitle
    \begin{center}
        Dieses Skript ist inoffiziell. Es besteht kein\\ Anspruch auf Vollständigkeit oder Korrektheit.
    \end{center}
    \thispagestyle{empty}
    \newpage

    \tableofcontents
    ~\\
    Alle mit [*] markierten Kapitel sind noch nicht korrektur gelesen und bedürfen eventuell noch Änderungen.
    \newpage


    \section{Aussagenlogik}
    %%%%%%%%%%%%%%%%%%%%%%%%
% 26. Oktober 2023
%%%%%%%%%%%%%%%%%%%%%%%%

\thispagestyle{pagenumberonly}
\setcounter{subsection}{1}

\begin{definition}[Aussage]
    Eine Aussage ist eine Behauptung sprachlich oder mittels Formeln, welche entweder wahr oder falsch ist.
\end{definition}

\begin{beispiel}[Zulässige Aussagen]
    \theoremescape
    \begin{enumerate}[label=(\roman*)]
        \item Bielefeld existiert (w)
        \item $2+2=5$ (f)
        \item Es gibt unendlich viele Primzahlen (w)
    \end{enumerate}
\end{beispiel}

\begin{definition}[Aussageform]
    Eine Aussage, die von mindestens einer Variablen abhängt, nennt sich Aussageform.
    Wir schreiben zum Beispiel $H(x)$ für eine Aussage für die Variable $x$.
\end{definition}

\begin{beispiel}[Mögliche Aussageformen]
    \theoremescape
    \begin{enumerate}[label=(\roman*)]
        \item $H(x) \definedasequiv \pair{x^2-3x+2=0}$
        \item $G(x) \definedasequiv \pair{x=1\lor x=2}$
    \end{enumerate}
\end{beispiel}

\begin{konzept}[Beweisstruktur]
    \begin{equation*}
        \begin{split}
            p\\
            \text{Vorraussetzung}\\
            \text{hinreichend für }q
        \end{split}
        \begin{split}
            \qquad\impl\qquad
        \end{split}
        \begin{split}
            q\\
            \text{Behauptung}\\
            \text{notwendig für }p
        \end{split}
    \end{equation*}
    \noindent Beweis: $p\impl r_1 \impl r_2 \impl r_3 \impl \dots\impl r_n \impl q$. ($r_1,\dots, r_n$ sind bereits bekannte wahre Aussagen oder Axiome)
\end{konzept}

\begin{satz}[Regeln der Aussagenlogik]
    Seien $p,q,r$ Aussagen.
    Dann sind folgende Aussagen wahr:
    \begin{enumerate}[label=(\roman*)]
        \item $p \lor \neg p$ \aligntoright{(Tertium non datur)}{0.1}\\
        $p \impl p$\\
        $\neg (p \land \neg p)$
        \item $p\land q \equivalent q \land p$ \aligntoright{(Kommutativität)}{0.1}\\
        $p\lor q \equivalent q \lor p$
        \item $(p\land q) \land r \equivalent p \land (q\land r)$ \aligntoright{(Assoziativität)}{0.1}\\
        $(p\lor q) \lor r \equivalent p \lor (q\land r)$
        \item $\neg(p\land q) \equivalent \neg p \lor \neg q$ \aligntoright{(De Morgan)}{0.1}\\
        $\neg(p\lor q) \equivalent \neg p \land \neg q$
        \item $p\impl q \equivalent \neg p \lor q$ \aligntoright{(Definition der Implikation)}{0.1}
        \item $(p\equivalent q) \equivalent (p\impl q) \land (q\impl p)$ \aligntoright{(Definition der Äquivalenz)}{0.1}\\
        $(p\equivalent q) \equivalent (p\land q) \lor (\neg p \land \neg q)$
        \item $(p\equivalent q) \land (q\equivalent r) \impl (p \equivalent r)$ \aligntoright{(Transitivität)}{0.1}
    \end{enumerate}
\end{satz}

\newpage


    \section{Mengen}
    \thispagestyle{pagenumberonly}

\subsection{Eigenschaften von Mengen}

\begin{notation}[Konkrete Beschreibung von Mengen]
    Eine Menge ist informell formuliert eine Ansammlung von Objekten.
    Um diese Objekte konkret anzugeben, schreiben wir $M = \set{1, 2, 3}$.\\
    Eine andere Möglichkeit, eine Menge anzugeben ist über die Eigenschaften der Elemente.
    Wir schreiben: $M=\set{x|~H(x)}$. Das bedeutet, dass ein Element $x$ genau dann ein Element der Menge ist, wenn $H(x)$ gilt.
    Wir schreiben $x\in M \equivalent H(x)$.
\end{notation}

\begin{beispiel}[Mengendeklaration über Aussagenform]
    \begin{align*}
        H(x) &\definedas (x^2-3x+2 = 0)\\
        \impl \set{x|~H(x)} &= \set{1,2}
    \end{align*}
\end{beispiel}

\begin{definition}[Eigenschaften von Mengen]
    Allgemein gilt für Mengen:
    \theoremescape
    \begin{enumerate}
        \item 2 Mengen sind gleich, wenn sie die selben Elemente enthalten.
        \item Die leere Menge ($\emptyset$) ist die einzige Menge, die keine Elemente enthält.
        \item Wenn für jedes $x\in A$ auch $x\in B$ folgt, dann ist $A$ eine Teilmenge von $B$. ($A\subseteq B$)
        \item Ist $A\subseteq B$ und $A\neq B$, dann nennen wir $A$ eine echte Teilmenge von $B$. ($A\subsetneq B$)
        \item $A$ und $B$ sind disjunkt, falls aus $x\in A$ folgt, dass $x\not\in B$.
    \end{enumerate}
\end{definition}

\begin{bemerkung}
    Allgemein gilt für zwei Mengen $A$ und $B$, dass $A\subseteq B \land B\subseteq A \equivalent A = B$.\footnote{Diese Äquivalenz wird insbesondere in Beweisen häufig eingesetzt, indem durch das Zeigen, dass zwei Mengen gegenseitige Teilmenge sind, deren Gleichheit gezeigt wird.}
\end{bemerkung}

\subsection{Operationen mit Mengen}

\begin{definition}[Mengenoperationen]
    Seien $A, B$ Mengen. Dann gilt:
    \begin{align*}
        A\cap B &\definedas \set{x|~x\in A \land x\in B}\tag{Schnitt}\\
        A\cup B &\definedas \set{x|~x\in A \lor x\in B}\tag{Vereinigung}\\
        A\exclude B &\definedas \set{x|~x\in A \land x\notin B}\tag{Differenz}\\[10pt]
        \text{Wenn } A\subseteq M&\colon A^{C} = A^{C}_{M} \definedas M\exclude A\tag{Komplement}
    \end{align*}
    Allgemein gilt: $A$ und $B$ disjunkt $\equivalent A\cap B = \emptyset$
\end{definition}

\begin{visualisierung}[Darstellung von Mengen als Venn-Diagramm]
    Die Operationen auf zwei Mengen $A$ und $B$ lassen sich mittels eines Venn-Diagramms veranschaulichen:
    \begin{figure}[H]
        \centering
        \begin{tikzpicture}
            \node [draw,
            circle,
            minimum size =3cm,
            label={135:$A$}] (A) at (0,0){};

            \node [draw,
            circle,
            minimum size =3cm,
            label={45:$B$}] (B) at (1.8,0){};
            \node at (0.9,0) {$A\cap B$};
        \end{tikzpicture}
        \caption{Schnittmenge zweier Mengen als Venn-Diagramm}
    \end{figure}
\end{visualisierung}

%%%%%%%%%%%%%%%%%%%%%%%%
% 31. Oktober 2023
%%%%%%%%%%%%%%%%%%%%%%%%

\begin{lemma}[Kommutativität des Schnitts]
    \marginnote{[31. Okt]}
    $A\cap B = B \cap A$
    \begin{proof}
        \begin{align*}
            A\cap B &= \set{x|~x\in A \land x \in B}\\
            &= \set{x|~x\in B \land x \in A}\\
            &= B\cap A\qedhere
        \end{align*}
    \end{proof}
\end{lemma}

\begin{lemma}[Distributivität]
    $A\cup (B \cap C) = (A\cup B) \cap (A\cup C)$
    \begin{proof}
        \begin{align*}
            x\in A \cup \pair{B \cap C} &\equivalent x\in A \lor x\in B \cap C\\
            &\equivalent x\in A \lor \pair{x\in B \land x \in B}\\
            &\equivalent \pair{x\in A \lor x\in B}\land\pair{x\in A \lor x\in C}\\
            &\equivalent x\in A \cup B\land x \in A \cup C\\
            &\equivalent x\in \pair{A \cup B}\cap\pair{A\cup C}\qedhere
        \end{align*}
    \end{proof}
\end{lemma}

\begin{definition}[Familie von Mengen]
    Sei $J$ eine Indexmenge mit $J\neq \emptyset$. Die Mengenfamilie ist gegeben durch Mengen $A_{j}$ für jedes $j\in J$. Wir schreiben $\set{A_j}_{j\in J}$
\end{definition}

\begin{definition}[Schnitt und Vereinigung über mehrere Mengen]
    Für eine Mengenfamilie $\set{A_j}j\in J$ gilt:
    \begin{align*}
        \bigcap_{j\in J} A_j &\definedas \set{x|~\forall j\in J: x\in A_j}\\[10pt]
        \bigcup_{j\in J} A_j &\definedas \set{x|~\exists j\in J: x\in A_j}
    \end{align*}
\end{definition}

\subsection{Quantoren}

\begin{definition}[Quantoren]
    Wir definieren drei unterschiedliche Quantoren:
    \theoremescape
    \begin{enumerate}[label=(\roman*)]
        \item $\forall$: \anf{Für alle}
        \item $\exists$: \anf{Es existiert ein}
        \item $\exists!$: \anf{Es existiert genau ein}
    \end{enumerate}
    \vspace{0.2cm}
    Es seien $A,B$ Mengen und $H(x,y)$ eine Aussageform mit $x\in A$ und $y \in B$. Dann gilt:
    \begin{align*}
        \forall x \in A~\exists y \in B\colon H(x,y) \equivalent \text{Für alle $x\in A$ existiert ein $y \in B$, so dass $H(x,y)$ wahr ist}
    \end{align*}
\end{definition}

\begin{folgerung}[Negation von Quantoren]
    Es seien $A,B$ Mengen und $H(x,y)$ Aussageform mit $x\in A$ und $y \in B$. Dann gilt:
    \begin{align*}
        \neg\pair{\forall x \in A\colon H(x)} &\equivalent \exists x\in A\colon \neg H(x)\\
        \neg\pair{\exists x \in A\colon H(x)} &\equivalent \forall x\in A\colon \neg H(x)\\
        \neg \pair{\forall x \in A~\exists y\in B\colon H(x,y)}&\equivalent\exists x \in A\colon \neg\pair{\exists y\in B\colon H(x,y)}\\
        &\equivalent \exists x\in A~\forall y \in B\colon \neg H(x,y)\\
    \end{align*}
\end{folgerung}

\begin{definition}[Potenzmenge und Mengensystem]
    Sei $A$ eine Menge, so heißt $\mathcal{P}(A) \definedas \set{N|~N \subseteq A}$ Potenzmenge von $A$. Eine Teilmenge $A\subseteq \mathcal{P}(A)$ heißt Mengensystem über $A$.
\end{definition}

\begin{bemerkung}[Russels Paradoxon]
    Russel definiert: $R\definedas\set{M|~M \text{ ist Menge und $M \notin M$}}$\\
    Falls $R$ eine Menge, dann kann man fragen, ob $R\in R$ oder nicht.\\
    1. Fall: $R\notin R\impl R\in R$\qquad (Widerspruch)\\
    2. Fall: $R\in R\impl R \notin R$\qquad (Widerspruch)\\
    Lösung: $R$ ist keine Menge, sondern eine Klasse.
\end{bemerkung}

\vfill

\subsection{Kartesisches Produkt und Relationen}

\begin{definition}[Tupel]
    Seien $A$ und $B$ Mengen. Für $x\in A$ und $y\in B$ ist $(a,b) \definedas \set{a, \set{a,b}}$ das geordnete Paar oder Tupel bestehend aus $a$ und $b$.\\[10pt]
    Zwei Tupel $(a_1, b_1)$ und $(a_2, b_2)$ mit $a_1, a_2\in A, b_1, b_2\in B$ sind genau dann gleich, wenn ihr jeweils erstes und zweites Element gleich ist:
    \begin{align*}
    (a_1, b_1)
        = (a_2, b_2) \equivalent a_1 = a_2 \land b_1 = b_2
    \end{align*}
\end{definition}

\begin{definition}[Kartesisches Produkt]
    Somit ist $A \times B \definedas \set{\pair{a,b}|~a\in A \land b \in B}$ wieder eine Menge, genannt das Kartesische Produkt von $A$ und $B$.
\end{definition}

\begin{beispiel}
    \begin{align*}
        M &\definedas \set{1,2,3}\\
        N &\definedas \set{3,4}\\
        M\times N &= \set{(1,3), (1,4), (2,3), (2,4), (3,3), (3,4)}
    \end{align*}
\end{beispiel}


\begin{definition}[Relation]
    Seien $A, B$ Mengen. Eine Relation $R=\pair{A, B, G}$ besteht aus einer Menge $A$, einer Menge $B$ und einer Menge $G\subseteq A\times B$.\\
    $G$ ist der Graph von $R$, auch geschrieben als $G_R$.
    Ist $\pair{a,b}\in G$ so sagt man \anf{$a$ ist $R$-verwandt zu $b$}. Wir schreiben $aRb$ (Infix-Schreibweise).\\
    $A$ ist die Definitionsmenge von $R$ und $B$ ist die Zielmenge von $R$.\\[10pt]
    Seien $R_1 = \pair{A_1, B_1, G_1}$ und $R_2 = \pair{A_2, B_2, G_2}$ Relationen. Dann gilt:
    \begin{align*}
        R_1 = R_2 \equivalent A_1 = A_2 \land B_1 = B_2 \land G_1 = G_2
    \end{align*}
    Wir können eine Umkehrrelation $R^{-1}$ wie folgt definieren:
    \begin{align*}
        R^{-1} &\definedas \pair{B,A,G_{R^{-1}}}\\
        G_{R^{-1}} &\definedas \set{\pair{b,a} |~\pair{a,b} \in G_R}
    \end{align*}
\end{definition}

\begin{beispiel}[Kleiner-Relation]
    Es sei $A = \set{1,2,3,4}$. Wir definieren die Kleiner-Relation $R=\set{A, A, G_{<}}.$\\
    Dann gilt: $a_1 < a_2 \equivalent \pair{a_1, a_2} \in G_{<}$ und somit $G_{<} = \set{(1,2), (1,3), (1,4), (2,3), (2,4), (3,4)}$
\end{beispiel}

\vfill
\newpage

\begin{definition}[Äquivalenzrelation]
    Sei $R=\pair{A, A, G}$ eine Relation. Dann definieren wir unterschiedliche Eigenschaften, die die Relation haben kann:
    \begin{enumerate}[label=(\roman*)]
        \item $R$ ist reflexiv:\quad $\forall a\in A\colon aRa\quad\pair{\forall a \in A\colon (a,a) \in G}$
        \item $R$ ist symmetrisch:\quad $\forall a_1, a_2\in A\colon a_{1}Ra_2\equivalent a_{2}Ra_1$
        \item $R$ ist transitiv:\quad $\forall a_1, a_2, a_3 \in A\colon a_{1}Ra_{2} \land a_{2}Ra_{3} \impl a_{1}Ra_{3}$
    \end{enumerate}
    Eine Äquivalenzrelation ist eine reflexive, symmetrische und transitive Relation auf $A$.\\
    Ist $R$ eine Äquivalenzrelation und $a_{1}Ra_{2}$ so nennt man $a_{1}$ äquivalent zu $a_{2}$ bezüglich $R$.
\end{definition}

\begin{notation}[Äquivalenzklassen]
    Sei $R$ Äquivalenzrelation auf $A$. Dann gilt:
    \begin{align*}
    [a]
        _{R} \definedas \set{b\in A|~aRb}
    \end{align*} ist die Äquivalenzklasse von $a$. Wir schreiben auch $a\sim_R b$ für $aRb$ oder $a = b$ modulo $R$.
\end{notation}

\begin{beobachtung}
    Allgemein gilt für Äquivalenzklassen damit:
    \theoremescape
    \begin{enumerate}[label=(\roman*)]
        \item $\forall a\in A\colon [a]_R \neq \emptyset$
        \item $aRa \impl a\in [a]_R$
        \item $a_1, a_2\in [a]_R \impl a_1 \sim_R a, a_2 \sim_R a \annot{\impl}{sym.} a_1 \sim_R a, a \sim_R a_2 \annot{\impl}{trans.} a_1\sim_R a_2$
    \end{enumerate}
\end{beobachtung}

\begin{lemma}
    Sei $R$ Äquivalenzrelation auf $A$. Für $a_1,a_2\in A$ ist entweder $[a_1]_R = [a_2]_R$ oder $[a_1]_R \cap [a_2]_R = \emptyset$.
    \begin{proof}
        Da $[a_1]_R$, $[a_2]_R \neq \emptyset$ reicht es zu zeigen, dass: $[a_1]_R \cap [a_2]_R \neq \emptyset \impl [a_1]_R = [a_2]_R$.
        \begin{align*}
            \text{Sei }b\in [a_1]_R\cap [a_2]_R\text{ und }c\in [a_1]_R\\
            \impl b\sim_R a_1 \land c\sim_R a_1 \annot{\impl}{trans.} c\sim_R b\\
            \intertext{Da  $b\sim_R a_2$ muss nach der Transitivität gelten:}
            c\sim_R a_2 \impl [a_1]_R \subseteq [a_2]_R
        \end{align*}
        Symmetrisch lässt sich argumentieren, dass $[a_2]_R \subseteq [a_1]_R$.
    \end{proof}
\end{lemma}

\begin{korollar}
    Es sei $R$ eine Äquivalenzrelation auf $A\neq \emptyset$. Dann sind $a_1,a_2\in A$ entweder äquivalent oder sie gehören zu disjunkten Äquivalenzklassen.
\end{korollar}

\begin{definition}[Zerlegung einer Menge]
    Sei $A\neq \emptyset$ eine Menge. Dann ist eine Zerlegung $F = \set{A_j}_{j\in J}, A_j\subseteq A$ mit folgenden Eigenschaften definiert:
    \begin{enumerate}
        \item $\forall j\in J\colon A_j\neq \emptyset$
        \item Für $j_1, j_2\in J, j_1\neq j_2\colon A_{j1} \cap A_{j2} = \emptyset$
        \item $\bigcup_{j\in J}A_j = A$
    \end{enumerate}
\end{definition}

\begin{notation}[Quotient]
    Es sei $R$ Äquivalenzrelation auf $A$.
    \begin{align*}
        F\definedas\set{[a]_R|~a\in A}
    \end{align*}
    ist eine Zerlegung von $A$ (bezüglich der Äquivalenzrelation $R$). Wir schreiben $F = A/R$.
    $A/R$ ist der \anf{Quotient} von $A$ bezüglich $R$
\end{notation}

\begin{beispiel}[Restklassendefinition über Äquivalenzrelationen]
    Es sei $A = \naturalnumbers_{0} = \set{0,1,2,3,4,\dots}$ und $p\in \mathbb{N}$. $m,n\in\mathbb{N}_0$ seien genau dann äquivalent, wenn $m=n+k\cdot p$ für ein $k\in\mathbb{Z}$:
    \begin{align*}
        R_{p} = \set{\pair{m,n}\in \mathbb{N}_0 \times \mathbb{N}_0 |~\exists k\in \mathbb{Z}\text{ mit } m=n+k\cdot p}
    \end{align*}
    So definieren wir die Restklassen von $\mathbb{N}_0$ bezüglich Division mit $p$.
    \begin{align*}
        m\in [j]_{R} \equivalent m=n+k\cdot p\text{ für ein }k\in\mathbb{Z}
    \end{align*}
\end{beispiel}

%%%%%%%%%%%%%%%%%%%%%%%%
% 2. November 2023
%%%%%%%%%%%%%%%%%%%%%%%%

\subsection{Funktionen}

\begin{bemerkung}[Moralische Definition einer Funktion]
    \marginnote{[2. Nov]}
    Gegeben Mengen $A, B$, eine Relation $f$ von $A$ nach $B$. $f$ ist eine Funktion, wenn es jedem Element in $A$ genau ein Element in $B$ zuordnet.
\end{bemerkung}

\begin{notation}[Pfeilnotation]
    Wir schreiben $f: A \fromto B,~a\mapsto f\pair{a}$.
\end{notation}
\begin{folgerung}
    Zu $a\in A$ gibt es $f\pair{a}\in B \leadsto$ Tupel($a, f\pair{a}$) $\in A\times B \impl \set{\pair{a, f(a)}: a\in A} \subseteq A\times B$.
\end{folgerung}

\begin{definition}[Funktion]
    Eine Relation $R = (A, B, G_R)$ heißt Funktion (oder Abbildung), wenn
    \begin{align*}
        \forall a\in A~\exists! b\in B\colon (a,b)\in G_f.
    \end{align*}
    Wir setzen dann $f(a)\definedas b$.
\end{definition}

\begin{beispiel}[Mögliche Funktionen]
    \theoremescape
    \begin{alignat*}{2}
        f\colon &\mathbb{Z} \rightarrow \realnumbers,\quad && n\mapsto 3n^2+7\\
        g\colon &\mathbb{Z} \rightarrow \mathbb{Z},\quad && n\mapsto 3n^2+7\\
        h\colon &\linterv{0, \infinity} \rightarrow \realnumbers,\quad && x\mapsto x^2+3x+4\\
        j\colon &\realnumbers \rightarrow \realnumbers,\quad && x\mapsto x^2+3x+4\\
    \end{alignat*}
\end{beispiel}
\begin{bemerkung}
    $f$ und $g$ haben zwar die gleiche Funktionsvorschrift, sind aber dennoch unterschiedliche Funktionen, da diese wie Relationen auch über Definitionsmenge und Zielmenge definiert sind.
\end{bemerkung}

\begin{notation}[Bild und Urbild]
    Sei $f: A\rightarrow B$ eine Funktion.
    Dann gilt
    \begin{alignat*}{2}
        & M \subseteq A:~ & f\pair{M} &\definedas \set{b\in B|~\exists x\in M\colon b=f\pair{x}}\tag{Bild von $M$ unter $f$}\\
        & N \subseteq B:~ & f^{-1}\pair{N} &\definedas \set{a\in A|~f(a)\in N}\tag{Urbild von $N$ unter $f$}
    \end{alignat*}
    Außerdem ist das gesamte Bild von $f$ definiert als
    \begin{align*}
        Bild(f) \definedas f(A)
    \end{align*}
\end{notation}
\begin{definition}[Einschränkungen von Funktionen]
    Sei $f\colon~A \rightarrow B$ eine Funktion und $M \subseteq A$.
    Dann ist die Einschränkung (oder Restriktion) von $f$ auf $M$ definiert als
    \begin{align*}
        f|_{M}\colon&~M \rightarrow B,\quad x \mapsto f\pair{x}
    \end{align*}
\end{definition}
\begin{definition}[Besondere Eigenschaften von Funktionen]
    Es sei $f\colon A\fromto B$ eine Funktion.
    \theoremescape
    \begin{enumerate}[label=(\roman*)]
        \item $f$ ist injektiv, falls $\forall a_1, a_2 \in A\colon f\pair{a_1} = f\pair{a_2} \impl a_1 = a_2$.
        \item $f$ ist surjektiv, falls $Bild\pair{f} = B$.
        \item $f$ ist bijektiv, falls es injektiv und surjektiv ist. In diesem Fall existiert eine Inverse\\ $f^{-1}: B\fromto A, b\mapsto a$ mit $f(a)=b$.
    \end{enumerate}
\end{definition}

\subsection{Geordnete Mengen}

\begin{definition}[Ordnungsrelation und teilweise geordnete Menge]
    Es sei $A$ eine Menge und $R$ eine Relation auf $A$. $R$ heißt Ordnungsrelation (geschrieben \anf{$\prec$}), falls
    \begin{enumerate}[label=(\roman*)]
        \item $\forall a\in A\colon a \prec a$ (Reflexivität)
        \item $\forall a_1, a_2, a_3\in A\colon a_1\prec a_2 \land a_2 \prec a_3 \impl a_1 \prec a_3$ (Transitivität)
        \item $\forall a_1, a_2\in A\colon a_1\prec a_2 \land a_2 \prec a_1 \impl a_1 = a_2$ (Antisymmetrie)
    \end{enumerate}
    $\pair{A, \prec}$ heißt \textbf{teilweise} geordnete Menge. Nicht alle Paare $a_1$, $a_2$ müssen vergleichbar sein.
\end{definition}

\begin{notation}
    Wir schreiben $a_1\prec a_2$ als $a_{1}Ra_{2}$ oder $\pair{a_1, a_2}\in G_R$.
\end{notation}
\begin{definition}[Kette]
    $T\subseteq A$ heißt Kette (oder geordnete Menge), falls
    \begin{align*}
        a_1, a_2\in T \impl a_1\prec a_2 \lor a_2 \prec a_1
    \end{align*}
\end{definition}

\begin{beispiel}[Ordnungsrelation]
    Ordnungsrelation auf $\mathcal{P}(A)$:
    \begin{align*}
        M, N \subseteq A \quad M \prec N, \text{ falls } M\subseteq N
    \end{align*}
    \noindent Ordnungsrelation auf $\naturalnumbers$:
    \begin{align*}
    (\naturalnumbers, \prec)
        : n\prec m, \text{ falls $n$ teilt $m$}
    \end{align*}
    $2$ und $3$ sind dabei nicht vergleichbar, aber $2$ und $4$ sind vergleichbar.
\end{beispiel}

\newpage


    \section{Die Axiome der reellen Zahlen}
    \thispagestyle{pagenumberonly}

Es gibt eine Menge $\realnumbers$, genannt Reelle Zahlen, die 3 Gruppen von Axiomen erfüllt:
\begin{enumerate}
    \item Algebraische Axiome
    \item Anordnungsaxiome
    \item Das Vollständigkeitsaxiom
\end{enumerate}

\subsection{Algebraische Axiome}
In $\realnumbers$ gibt es 2 Operationen:
\begin{enumerate}
    \item Addition \anf{+}
    \item Multiplikation \anf{$\cdot$}
\end{enumerate}

\begin{folgerung}
    $a,b\in\realnumbers\impl a+b\in \realnumbers$ und $a \cdot b\in\realnumbers$
\end{folgerung}

\begin{definition}[Eigenschaften eines Körpers]
    \theoremescape
    \begin{enumerate}[label=(I.\arabic*)]
        \item $(a+b) + c = a + (b+c)$\quad \textit{Assoziativität der Addition}
        \item $a + b = b + a$\quad \textit{Kommutativität der Addition}
        \item Es gibt genau eine Zahl genannt Null, geschrieben $0$, mit $\forall a \in\realnumbers: a+0 = a$ \quad \textit{Existenz eines neutralen Elements der Addition}
        \item $\forall a\in\realnumbers~\exists! b\in \realnumbers: a + b = 0$, geschrieben $b=-a$\quad \textit{Existenz eines inversen Elements der Addition}
        \item $\pair{a\cdot b}\cdot c = a\cdot\pair{b\cdot c}$\quad \textit{Assoziativität der Multiplikation}
        \item $a\cdot b = b \cdot a$\quad \textit{Kommutativität der Multiplikation}
        \item $\forall a \in\realnumbers, a \neq 0$ gibt es ein eindeutiges $b\neq 0$ mit $a\cdot b = 1$. Wir schreiben $b = a^{-1} = \frac{1}{a}$\quad \textit{Existenz eines inversen Elements der Multiplikation}
        \item Es gibt genau eine Zahl Eins, geschrieben 1, die von 0 verschieden ist, mit $\forall a\in\realnumbers: a\cdot 1 = a$\quad \textit{Existenz eines neutralen Elements der Multiplikation}
        \item $a\cdot\pair{b+c} = a \cdot b + a \cdot c$\quad \textit{Distributivität}
    \end{enumerate}

    \noindent Jede Menge $\mathbb{K}$, welche (I.1) bis (I.9) erfüllt, heißt \textbf{Körper}.
\end{definition}

\begin{bemerkung}
    Dass die Eindeutigkeit von 0 und 1 durch die Axiome gefordert wird, ist nicht unbedingt erforderlich.\footnote{Seien $0, 0'$ neutrale Elemente bezüglich der Addition. $\impl 0 = 0+0' = 0'+0=0'$}
\end{bemerkung}

\begin{bemerkung}
    Das inverse Elemente bezüglich Addition und Multiplikation ist eindeutig.
    \begin{proof}
        Annahme: $a+b=0$ und $a+b'=0$
        \begin{align*}
            &\impl b + 0 = b + \pair{a+b'} = b' + \pair{a+b} = b' + 0\\
            &\impl b = b'\qedhere
        \end{align*}
    \end{proof}
\end{bemerkung}

\begin{notation}
    \begin{align*}
        a-b &\definedas a+(-b)\tag{Differenz}\\[10pt]
        \frac{a}{b} &\definedas a \cdot b^{-1}\tag{Quotient}
    \end{align*}
\end{notation}

\begin{satz}[Abgeleitete Regeln]
    Es gilt\\
    (I.10)
    \begin{align}
        -\pair{-a} &= a\\[8pt]
        \pair{-a} + \pair{-b} &= -\pair{a+b}\\[8pt]
        \pair{a^{-1}}^{-1} &= a\\[8pt]
        a^{-1}\cdot b^{-1} &= \pair{a\cdot b}^{-1}\\[8pt]
        a\cdot 0 &= 0\\[8pt]
        a\cdot\pair{-b} &= -\pair{a\cdot b}\\[8pt]
        (-a)\cdot(-b) &= a\cdot b\\[8pt]
        a\cdot\pair{b-c} &= a\cdot b - a\cdot c
    \end{align}
    \noindent (I.11) Ist $a\cdot b = 0$ so ist mindestens eine der Zahlen $a$ oder $b$ gleich Null.

    \begin{proof}[Beweis zu (I.10.5)]
        Zu zeigen: $a \cdot 0 = 0$
        \begin{align*}
            a\cdot 0 + a \cdot 0 &= a \cdot\pair{0+0}\\
            &= a \cdot 0\\
            \impl \pair{a\cdot 0 + a \cdot 0} + \pair{-a \cdot 0} &= a \cdot 0 + \pair{-a\cdot 0}\\
            \impl a \cdot 0 + \pair{a\cdot 0 + \pair{-a\cdot 0}} &= 0\\
            \impl a\cdot 0 +0 &= 0\\
            \impl a\cdot 0 &= 0\qedhere
        \end{align*}
    \end{proof}
    \begin{proof}[Beweis zu (I.11)]
        Sei $a\cdot b = 0$.\\
        Ist $a\neq 0 \impl b = 1\cdot b = a^{-1}\cdot a \cdot b = a^{-1}\cdot (a\cdot b) = a^{-1} = 0 = 0$\\
        Ist $b\neq 0,$ so gilt analog, dass $a = 0$.
    \end{proof}
    \begin{uebung}
        Beweisen Sie die verbleibenden Regeln aus (I.10).
    \end{uebung}
\end{satz}

\begin{satz}[Regeln des Bruchrechnens]
    \label{satz:bruchrechnen}
    ~\\(I.12) Es gilt:
    \setcounter{equation}{0}
    \begin{align}
        \frac{a}{b} + \frac{c}{d} &= \frac{ad+cb}{bd} &b,d&\neq 0\\[10pt]
        \frac{a}{b} \cdot \frac{c}{d} &= \frac{ac}{bd} &b,d&\neq 0\\[10pt]
        \frac{\frac{a}{b}}{\frac{c}{d}} &= \frac{ad}{bc} &b,c,d&\neq 0
    \end{align}
    \begin{uebung}
        Beweisen Sie Satz~\ref{satz:bruchrechnen}.
    \end{uebung}
\end{satz}

\subsection{Die Anordnungsaxiome}

Allgemein gilt: $a,b\in\realnumbers \impl a = b \lor a \neq b$.\\
Ist $a\neq b$, besteht eine Anordnung \anf{$<$}, die verlangt, dass genau eine der Relationen $a<b$ oder $b<a$ gilt.
Das heißt $\forall a,b\in\realnumbers$ gilt genau eine der Aussagen $a<b, b<a, a= b$\\
Diese Anordnung genügt folgenden Axiomen:
\begin{axiom}[Anordnungsaxiome]
    \theoremescape
    \begin{enumerate}[label=(II.\arabic*)]
        \item $a<b \land b < c \impl a < c$ (Transitivität)
        \item $a<b, c \in\realnumbers \impl a + c < b + c$
        \item $a<b, c > 0 \impl ac < bc$
    \end{enumerate}
\end{axiom}

\begin{notation}
    \theoremescape
    \begin{enumerate}[label=-]
        \item $a < b$: a ist (echt) kleiner als b
        \item $b > a$: b ist größer als a
        \item $a\leq b$: $a=b$ oder $a < b$
        \item $a\in\realnumbers$ ist positiv, wenn $a>0$; negativ, wenn $a <0$; nicht-negativ, wenn $a\geq 0$; nicht-positiv, wenn $a\leq 0$
    \end{enumerate}
\end{notation}

\begin{beispiel}
    $a<b\equivalent b - a > 0$
    \begin{proof}
        \begin{align*}
            a &<b\\
            \impl 0 = a + \pair{-a} &< b + \pair{-a} = b - a\\[10pt]
            b-a &>0\\
            \impl a &< a + \pair{b-a} = b\qedhere
        \end{align*}
    \end{proof}
\end{beispiel}

%%%%%%%%%%%%%%%%%%%%%%%%
% 7. November 2023
%%%%%%%%%%%%%%%%%%%%%%%%

\begin{satz}[Aus den Anordnungsaxiomen abgeleitete Regeln]
    \marginnote{[7. Nov]}
    \theoremescape
    \begin{enumerate}[label=(II.\arabic*)]
        \setcounter{enumi}{3}
        \item $a<b\equivalent b-a > 0$
        \item $a<0\equivalent -a > 0$ und $a>0\equivalent -a < 0$
        \item $a<b\equivalent -b < -a$
        \item $a<b \land c < d \equivalent a+c<b+d$
        \item $ab > 0 \equivalent \pair{a>0 \land b > 0}\lor\pair{a < 0 \land b < 0}$ und $ab < 0 \equivalent \pair{a>0 \land b < 0}\lor\pair{a < 0 \land b > 0}$
        \item $a\neq 0 \impl a^2 > 0$\quad(Insbesondere $1>0$)
        \item $a<b \land c<0 \impl ac > bc$
        \item $a>0 \equivalent \frac{1}{a}>0$
        \item $a^2 < b^2 \land a > 0 \land b > 0 \impl a < b$
    \end{enumerate}
    \newpage
    \begin{proof}
        \theoremescape
        \begin{enumerate}[label=(II.\arabic*)]
            \setcounter{enumi}{3}
            \item Sei $a<b \impl 0=a+(-a) \annot{<}{(II.2)} b + (-a) = b-a$.\\
            Ist $b-a>0 \annot{\impl}{(II.2)} a<a+(b-a)=b$
            \item Setze $b\definedas 0$ in (II.4) $\impl b-a=-a>0$.\\
            2ter Teil: Ersetze $a$ durch $-a$ in (II.5). ($a>0 \impl -a < 0 \equivalent -(-a)>0 \equivalent a >0$)
            \item (II.6) folgt aus (II.5), da $a<b\equivalent b-a>0 \equivalent (-a)-(-b) > 0\equivalent -b < -a$
            \item Sei $a<b \land c < d \annot{\impl}{(II.2)} a + c < b + c \land b + c < b + d \annot{\impl}{(II.1)} a+c < b + d$
            \item $a,b>0 \annot{\impl}{(II.3)} ab > 0\cdot b = 0$ und $a,b<0 \annot{\impl}{(II.5)} -a,-b>0 \impl (-a)(-b) > 0 \impl ab > 0$.\\
            Umkehrung: Sei $ab>0 \impl a\neq 0 \land b \neq 0$. Wäre $a>0 \land b < 0 \annot{\impl}{(II.5)} -b>0$. Wie gerade gezeigt folgt $a(-b) > 0 \impl -ab > 0 \annot{\impl}{(II.5)} ab < 0$ (Widerspruch zur Annahme).\\
            Genauso zeigt man, dass die Annahme $a<0 \land b > 0$ falsch ist.\\
            (Zweite Behauptung lässt sich analog zeigen).
            \item $a\neq 0 \equivalent a > 0 \lor a < 0 \annot{\impl}{(II.8)} a^2 = a \cdot a > 0$. Ferner ist $1\neq 0 \impl 1=1\cdot 1 > 0$
            \item Sei $c<0 \impl -c > 0$ und aus $a<b$ folgt $(-c)\cdot a < (-c)\cdot b \impl -c\cdot a < -c\cdot b \impl c\cdot b < c\cdot a$
            \item $a\cdot a^{-1} = 1 > 0$ (falls $a\neq 0$) $\annot{\impl}{(II.8)} a^{-1} > 0$ sofern $a>0$ ist und aus $a^{-1}>0$ folgt $a>0$
            \item Sei $a^{2}<b^2, a>0, b>0$. Angenommen $a<b$ ist falsch, d.h. $a\geq b \impl a^{2} \geq a \cdot a \geq a\cdot b\geq b\cdot b = b^{2}\impl a^{2}\geq b^{2}$ (Widerspruch)
        \end{enumerate}
    \end{proof}
\end{satz}

\newpage

\subsection{Das Vollständigkeitsaxiom}

\begin{axiom}[Vollständigkeitsaxiom]
    Jede nicht-leere Teilmenge $M\subseteq \realnumbers$, welche nach oben beschränkt ist, besitzt eine kleinste obere Schranke, genannt das Supremum von $M$.
\end{axiom}

\begin{notation}[Supremum]
    Das Supremum einer Menge $M$ schreiben wir als $\sup M$
\end{notation}

\begin{definition}[Beschränktheit von Mengen]
    Sei $M\subseteq \realnumbers, M\neq \emptyset$.
    \begin{enumerate}[label=(\roman*)]
        \item $M$ heißt \textbf{nach oben beschränkt}, falls ein $k\in\realnumbers$ existiert mit $\forall x\in M:x\leq k$.
        Jede solche Zahl $k$ heißt obere Schranke von $M$.
        \item $M$ heißt \textbf{nach unten beschränkt}, falls ein $k\in\realnumbers$ existiert mit $\forall x\in M:x\geq k$.
        Jede solche Zahl $k$ heißt untere Schranke von $M$.
        \item $M$ heißt \textbf{beschränkt}, falls ein $k\geq 0$ existiert mit $-k\leq x \leq k\quad \forall x\in M$
    \end{enumerate}
\end{definition}

\begin{definition}[Kleinste obere und größte untere Schranke]
    Eine Zahl $k\in\realnumbers$ heißt kleinste obere (größte untere) Schranke, falls
    \begin{enumerate}
        \item es eine obere (untere) Schranke ist und
        \item es keine kleinere obere (größere untere) Schranke für $M$ gibt
    \end{enumerate}
\end{definition}

\begin{folgerung}
    \theoremescape
    Allgemein gilt
    \begin{align*}
        x \leq k \equivalent -k \leq -x
    \end{align*}
    das heißt für eine Menge $M\neq\emptyset$ gilt
    \begin{center}
        $k$ ist eine obere Schranke für $M$\\ $\equivalent -k$ ist eine untere Schranke für $-M\definedas\set{-x|~x\in M}$
    \end{center}
    und
    \begin{center}
        $k$ ist kleinste obere Schranke für M\\ $\equivalent -k$ ist die größte untere Schranke für $-M$
    \end{center}
    Das heißt das Anordnungsaxiom ist äquivalent zum \textit{Anordnungsaxiom}$^{-1}$ (Jede nicht-leere Teilmenge $M\subseteq \realnumbers$, welche nach unten beschränkt ist, besitzt eine größte untere Schranke, genannt das Infimum von $M$. Wir schreiben $\inf M$).
\end{folgerung}

\begin{beispiel}
    \theoremescape
    \begin{align*}
        M &\definedas \interv{0,1} = \set{x|~0\leq x \leq 1}\\
        \sup M &= 1 \qquad \inf M = 0\nn
        A &\definedas \pair{0,1} = \set{x|~0< x < 1}\\
        \sup A &= 1 \qquad \inf A = 0\\
    \end{align*}
\end{beispiel}

\begin{notation}
    Sei $M\subseteq \realnumbers, M\neq \emptyset$\\
    Wir schreiben $\sup M < \infty$, falls $M$ nach oben beschränkt ist, andernfalls setzen wir
    \begin{align*}
        \sup M \definedas \infty
    \end{align*}
    Falls $M$ nach unten beschränkt ist, schreiben wir $\inf M > -\infty$, andernfalls setzen wir
    \begin{align*}
        \inf M \definedas -\infty
    \end{align*}
\end{notation}

\begin{satz}[Eigenschaften des Supremums]
    \label{satz:sup}
    Sei $M\subseteq \realnumbers$
    \begin{enumerate}[label=(\roman*)]
        \item Ist $\sup M < \infty$, so folgt $\forall \varepsilon > 0~\exists x\in M$ mit $\sup\pair{M} -\varepsilon < x$
        \item Ist $\sup M = \infty$, so gilt $\forall k\geq 0~\exists x\in M$ mit $x> k$
    \end{enumerate}
    \begin{proof}
        \theoremescape
        \begin{enumerate}
            \item Wir setzen $a\definedas \sup M$. Sei $a<\infty$. Wäre (i) falsch, so folgt $\exists \varepsilon>0~\forall x\in M:a-\varepsilon > x$.\\Das heißt $a-\varepsilon$ ist eine obere Schranke für $M$. Aber $a-\varepsilon < a$; (Widerspruch)
            \item Ist $a=\infty$, so hat $M$ keine obere Schranke. Nach Def. folgt für jedes $k\in\realnumbers$ existiert ein $x\in M: x > k$\qedhere
        \end{enumerate}
    \end{proof}
\end{satz}
\begin{satz}[Eigenschaften des Infimums]
    \label{satz:inf}
    Sei $M\subseteq \realnumbers$
    \begin{enumerate}[label=(\roman*)]
        \item Ist $\inf M > -\infty$, so folgt $\forall \varepsilon > 0~\exists x\in M$ mit $x < \inf\pair{M}+ \varepsilon$
        \item Ist $\inf M = -\infty$, so gilt $\forall k\geq 0~\exists x\in M$ mit $x< -k$
    \end{enumerate}
    \begin{proof}
        Wende Satz~\ref{satz:sup} auf $-M\definedas\set{-x|~x\in M}$ an und beachte $\sup\pair{-M} = \inf\pair{M}$
    \end{proof}
\end{satz}
\begin{definition}[Maximum und Minimum]
    Es sei $M\subseteq \realnumbers, M\neq\emptyset$\\
    $m\in M$ heißt größtes Element von $M$ (Maximum), geschrieben $\max M$, falls $x\leq m~\forall x\in M$.\\
    Entsprechend: $m\in M$ heißt kleinstes Element von $M$ (Minimum), geschrieben $\min M$, falls $x\geq m~\forall x\in M$
\end{definition}

\begin{beispiel}
    Sei $M$ beschränkt, $M\neq\emptyset$
    \begin{align*}
        M&\definedas\set{x|~0\leq x< 1}\\
        \sup M &= 1\\
        \inf M &= \min M = 0\\
        M&\text{ hat kein Maximum}
    \end{align*}
\end{beispiel}

\newpage


    \section{Die natürlichen Zahlen $\N$ und vollständige Induktion}
    \thispagestyle{pagenumberonly}
Frage: Was sind die natürlichen Zahlen? (zum Beispiel 1, $2\definedas 1+1$, 3$\definedas 2+1$, 4$\definedas 3+1$, usw.)

\subsection{Induktive Mengen}
\begin{definition}[Induktive Menge]
    Eine Teilmenge $A\subseteq \realnumbers$ heißt induktiv, falls
    \begin{enumerate}
        \item $1\in A$ und
        \item Ist $x\in A$ so ist auch $x+1\in A$
    \end{enumerate}
\end{definition}

\begin{beispiel}
    \begin{align*}
        \rinterv{0, \infty} &= \set{x|~x\geq 0}\text{ und}\\
        \rinterv{1, \infty} &= \set{x|~x\geq 1} \text{ sind induktiv}
    \end{align*}
\end{beispiel}

\begin{beobachtung}[Schnittmengen von induktiven Mengen]
    Ist $J\neq\emptyset$ Indexmenge und $A_j$ induktive Teilmenge von $\realnumbers$ für jedes $j\in J$.
    Dann folgt daraus: $A\definedas \bigcap_{j\in J}A_j$ ist induktiv.\\
    Anders formuliert: Beliebige Schnittmengen von induktiven Mengen sind induktiv.
    \begin{proof}
        \begin{align*}
            x\in A &\equivalent \forall j\in J: x \in A_j\\
            &\impl x+1 \in A_j \quad \forall j\in J\\
            &\impl x+1 \in \pair{\bigcap_{j\in J}A_j}=A\qedhere
        \end{align*}
    \end{proof}
\end{beobachtung}

\begin{definition}[Definition von $\naturalnumbers$]
    Sei $\bar{f}\definedas\set{A\subseteq \realnumbers|~A\text{ ist induktiv}}$.
    \begin{align*}
        \naturalnumbers &\definedas \text{kleinste induktive Teilmenge von }\realnumbers\\
        &\definedas \bigcap_{A\in \bar{f}}A
    \end{align*}
    d. h. $\naturalnumbers\subseteq A$, falls $A$ induktiv ist.
\end{definition}

\begin{satz}[Induktionsprinzip]
    \label{satz:induktionsprinzip}
    Ist $M\subseteq \naturalnumbers$ induktiv $\impl M = \naturalnumbers$
    \begin{proof}
        Nach Voraussetzung ist $M\subseteq \naturalnumbers$ und aus der Definition von $\naturalnumbers$ als Schnitt aller induktiven Teilmengen von $\realnumbers$ ist auch $\naturalnumbers\subseteq M \impl \naturalnumbers = M$.
    \end{proof}
\end{satz}

%%%%%%%%%%%%%%%%%%%%%%%%
% 9. November 2023
%%%%%%%%%%%%%%%%%%%%%%%%

\subsection{Vollständige Induktion}

\begin{satz}[Induktionsbeweis]
    Für jedes $n\in\naturalnumbers$ sei eine Aussage $B_n$ gegeben derart, dass folgendes gilt:
    \begin{enumerate}
        \item $B_1$ ist wahr.
        \item Aus der Annahme, dass $B_n$ ($n\in\naturalnumbers$) wahr ist folgt, dass $B_{n+1}$ wahr ist.
    \end{enumerate}
    Dann ist $B_n$ wahr $\forall n\in\naturalnumbers$.
    \begin{proof}
        Definiere: $M\definedas\set{n\in\naturalnumbers|~B_n\text{ ist wahr}} \subseteq \naturalnumbers$.\\
        Zu zeigen: $M=\naturalnumbers$. Also reicht nach Satz~\ref{satz:induktionsprinzip} zu zeigen, dass $M$ induktiv ist.
        \begin{enumerate}
            \item $1\in\naturalnumbers$, da $B_1$ wahr ist.
            \item Ist $n\in M$ dann ist $B_n$ wahr $\impl$ $B_n+1$ ist wahr $\impl$ $n+1\in M$ $\impl$ $M$ ist induktiv $\impl$ $M=\naturalnumbers$ \qedhere
        \end{enumerate}
    \end{proof}
\end{satz}

\begin{bemerkung}[Starke Induktion]
    Die starke Induktion ist eine Variante der vollständigen Induktion. Sie ist definiert durch:
    \begin{enumerate}
        \item $B_1$ ist wahr.
        \item für alle $n\in\naturalnumbers$ gilt $B_1, B_2, B_3, \dots, B_n$ ist wahr $\impl$ $B_{n+1}$ ist wahr.
    \end{enumerate}
\end{bemerkung}

\begin{beispiel}
    Zu zeigen ist: $B_n: n < 2^{n}$
    \begin{proof}
        ~\\
        \begin{induktionsanfang}
            $B_1$ ist wahr, da $1<2^{1}=2$
        \end{induktionsanfang}
        \begin{induktionsschritt}
            Induktionsannahme $B_n$ ist wahr für ein $n=k$, d.h. $k<2^k$. Und Schluß zu zeigen: $k+1<2^{k+1} \impl 2^{k+1} = 2\cdot 2^{k} > 2k \geq k+1$\qedhere
        \end{induktionsschritt}
    \end{proof}
\end{beispiel}

\begin{uebung}
    Zeigen Sie, dass $2k\geq k+1$ für alle $k\in\naturalnumbers$ gilt.
\end{uebung}

\begin{beispiel}[Gaußsche Summenformel]
    Zu zeigen: $1+2+3+\dots+n= \frac{n\cdot(n+1)}{2}~\forall n\in\naturalnumbers$
    \begin{proof}
        $B_n: 1+2+\dots+n = \frac{n\cdot(n+1)}{2}$\\
        \begin{induktionsanfang}
            $B_1: 1=\frac{1\cdot(1+1)}{2}=\frac{2}{2}=1$
        \end{induktionsanfang}
        \begin{induktionsschritt}
            $B_n$ ist wahr für ein $k\in\naturalnumbers$, d.h. $1+2+\dots+k=\frac{k\cdot(k+1)}{2}$
            \begin{align*}
                1+2+\dots+k+(k+1) &= \frac{k\cdot(k+1)}{2}+(k+1)\\
                &= (k+1)\cdot\pair{\frac{k}{2}+1}\\
                &= \frac{(k+1)\cdot(k+2)}{2}
            \end{align*}
        \end{induktionsschritt}
        \noindent Also ist $B_n$ wahr für $n=k+1$. Nach dem Prinzip der vollständigen Induktion folgt $\forall n\in\naturalnumbers:~B_n$ ist wahr, d.h. $1+2+\dots+n=\frac{n\cdot(n+1)}{2}$\qedhere
    \end{proof}
\end{beispiel}

\begin{satz}[Eigenschaften von $\naturalnumbers$]
    \label{satz:n-eigenschaften}
    \theoremescape
    \begin{enumerate}
        \item $n\geq 1~\forall n\in\naturalnumbers$
        \item $n+m\in\naturalnumbers~\forall n,m\in\naturalnumbers$
        \item $n\cdot m\in\naturalnumbers~\forall n,m\in\naturalnumbers$
        \item Für $n\in\naturalnumbers$ gilt entweder $n=1$ oder $n-1\in\naturalnumbers$
        \item $(n-m) \in \naturalnumbers~\forall n,m\in\naturalnumbers$ mit $m<n$
    \end{enumerate}
    \begin{proof}[Beweis (1.)]
        ~\\
        \begin{induktionsanfang}
            $n\geq 1$ gilt für $n=1$
        \end{induktionsanfang}
        \begin{induktionsschritt}
            Nach Anfang: $n\geq 1$ ist wahr für ein $n=k$.\\
            Da $k+1>k\impl k+1>k\geq 1 \impl k+1\geq 1$\qedhere
        \end{induktionsschritt}
    \end{proof}
    \begin{proof}[Beweis (2.)]
        Fixiere $m\in\naturalnumbers$ für $n\in\naturalnumbers$. Behauptung: $B_n: m+n\in \naturalnumbers$\\
        \begin{induktionsanfang}
            $B_1: m+1 \in \naturalnumbers$, da $\naturalnumbers$ induktiv ist.
        \end{induktionsanfang}
        \begin{induktionsschritt}
            Nach Anfang: $B_n$ ist wahr für $n=k$, d.h. $(m+k)\in\naturalnumbers$. Somit für $k+1: m+(k+1) = (m+k)+1\in\naturalnumbers \impl$ $B_n$ ist wahr $\forall n\in\naturalnumbers$\qedhere
        \end{induktionsschritt}
    \end{proof}
    \begin{proof}[Beweis (3.)]
        Fixiere $m\in\naturalnumbers$ für $n\in\naturalnumbers$. Behauptung: $B_n: m\cdot n\in \naturalnumbers$\\
        \begin{induktionsanfang}
            $B_1: m\cdot 1=m \in \naturalnumbers$
        \end{induktionsanfang}
        \begin{induktionsschritt}
            Nach Anfang: $B_n$ ist wahr für $n=k$, d.h. $m\cdot(k+1) = mk + m \in\naturalnumbers$ gilt nach Satz 2.\qedhere
        \end{induktionsschritt}
    \end{proof}
    \begin{proof}[Beweis (4.)]
        $B_n: n= 1 \lor n-1\in\naturalnumbers$\\
        \begin{induktionsanfang}
            $B_1: 1=1$
        \end{induktionsanfang}
        \begin{induktionsschritt}
            Nehmen an, für ein $n=k$ ist $G_k$ wahr. Also ist entweder (a) $k=1$ oder (b) $(k-1)\in\naturalnumbers$.\\
            Für $n=k+1$ gilt dann im Fall (a): $(k+1)-1 = k\in\naturalnumbers$\\
            Im Fall (b): $(k+1)-1 = (k-1)+1$ und es gilt $(k-1)\in\naturalnumbers$. Daraus folgt, dass $(k-1)+1\in\naturalnumbers$, da $\naturalnumbers$ induktiv ist.\qedhere
        \end{induktionsschritt}
    \end{proof}
    \begin{proof}[Beweis (5.)]
        $B_n: n-m\in\naturalnumbers$ für jedes $m,n\in\naturalnumbers$ mit $m<n$\\
        \begin{induktionsanfang}
            $B_1$ leere Behauptung, da kein $m$ existiert, mit $m<1$ (nach (1.)).
        \end{induktionsanfang}
        \begin{induktionsschritt}
            $B_n$ wahr für ein $n=k$. Das heißt $k-1\in\naturalnumbers~\forall m\in\naturalnumbers$ mit $m<k$.\\
            Zu zeigen: $(k+1)-m\in\naturalnumbers~\forall m<k+1$. Ist $m=1\impl m-1 = 0$ und $(k+1)-m = k+1-m=k\in\naturalnumbers$.\\
            Ist $m>1 \annot{\impl}{(4.)} m-1\in\naturalnumbers$. Da $m<k+1 \impl m-1<k \impl (k+1)-m = k-(m-1)\in\naturalnumbers$ (nach Annahme).\qedhere
        \end{induktionsschritt}
    \end{proof}
\end{satz}

\begin{korollar}
    \label{korollar:4.2.7}
    Es gibt kein $n\in\naturalnumbers$ mit $0<n<1$. Ferner gilt $\forall n\in\naturalnumbers$ gibt es keine natürliche Zahl $m\in\naturalnumbers$ mit $n<m<n+1$ oder mit $n-1<m<n$.
\end{korollar}
\begin{uebung}
    Beweisen Sie dies.
\end{uebung}
\begin{notation}[Zahlenmengen]
    \begin{align*}
        \naturalnumbers_0 &\definedas \naturalnumbers\cup\set{0}\\
        \mathbb{-N} &\definedas \set{-n|~n\in\naturalnumbers}\\
        \mathbb{Z} &\definedas \naturalnumbers\cup\set{0}\cup\mathbb{-N} \tag{Menge der ganzen Zahlen}\\
        \mathbb{Q} &\definedas\set{\frac{p}{q}\middle|~p\in\mathbb{Z}, q\in\naturalnumbers} \tag{Menge der rationalen Zahlen}\\
        \realnumbers&\exclude\mathbb{Q}\tag{Irrationale Zahlen}
    \end{align*}
\end{notation}
\begin{bemerkung}
    Sei für $n\in\mathbb{Z}~B_n$ eine Aussage und $n_0\in\naturalnumbers$.\\
    Dann gilt $\pair{\forall n\geq n_0: B_n} \equivalent$ ($B_n$ ist wahr) $\land$ (Ist $B_n$ wahr für $n\geq n_0$ so ist auch $B_{n+1}$ wahr)
\end{bemerkung}

\begin{satz}
    $n,k\in\mathbb{Z} \impl a +b \in\mathbb{Z}$ und $a\cdot b \in\mathbb{Z}$
    \begin{proof}
        Folgt aus Satz~\ref{satz:n-eigenschaften} mit $-(-a) = a\quad a > 0 \impl -a < 0$
    \end{proof}
\end{satz}

\begin{satz}[Satz von Archimedes]
    $\forall x\in\realnumbers~\exists n\in\naturalnumbers: x < n$.
    (Das heißt $\naturalnumbers$ ist eine nach oben unbeschränkte Teilmenge von $\realnumbers$)
    \begin{proof}
        Angenommen die Aussage ist falsch. Das heißt $\exists x\in\realnumbers$ mit $n\leq x\quad\forall n\in\naturalnumbers$\\
        $\annot{\impl}{(Vollständ.)} a\definedas \sup\naturalnumbers$ existiert und $n<a~\forall n\in\naturalnumbers$\\
        Es gilt $a+1 > a \impl a - 1 < a$, das heißt $a-1$ ist keine obere Schranke für $\naturalnumbers$\\
        $\impl \exists n\in\naturalnumbers: a-1<n \impl a < n + 1 \in\naturalnumbers$. Widerspruch zu $a$ ist obere Schranke für $\naturalnumbers$.
    \end{proof}
\end{satz}

\begin{korollar}
    $\forall x\in\realnumbers~\exists n\in\naturalnumbers\colon -n<x$
    \begin{proof}
        Wende vorherigen Satz auf $-x$ an. $\impl \exists n\in\naturalnumbers\colon -x < n \impl x > -n$
    \end{proof}
\end{korollar}

\begin{satz}[Wohlordnungsprinzip für $\naturalnumbers$]
    Jede nichtleere Menge natürlicher Zahlen hat ein kleinstes Element.
    \begin{proof}
        Sei $M\subseteq \naturalnumbers, M\neq\emptyset$. Es gilt $\inf \naturalnumbers = 1 \impl a = \inf M \geq 1 > -\infty$. Zu zeigen: $a\in M$.\\
        Annahme: $a\not\in\naturalnumbers \impl a < m\quad\forall m \in\naturalnumbers$\\
        Satz~\ref{satz:inf} besagt, dass $\forall \varepsilon > 0~\exists m\in M: m<a+\varepsilon$.\\
        1) Wähle $\varepsilon = 1 \impl \exists m\in M: m< a + 1$.\\
        2) Wähle $\varepsilon = a -m: \impl \exists m'\in M: m'<a+\varepsilon =m$. Das heißt $m',m\in M$ mit $a<m'<m<a+1$\\
        $\impl 0 < m-m' < 1 \annot{\impl}{(5.)} m-m'\in\naturalnumbers$. Widerspruch zu Korollar~\ref{korollar:4.2.7}
    \end{proof}
\end{satz}

\newpage


    \section{Summe, Produkt, Wurzeln}
    %%%%%%%%%%%%%%%%%%%%%%%%
% 14. November 2023
%%%%%%%%%%%%%%%%%%%%%%%%

\thispagestyle{pagenumberonly}

\subsection{Summenzeichen, Produktzeichen}
\begin{definition}
    Seien $m\leq n$, $m,n\in\naturalnumbers_{0}$. Für jedes $k\in\naturalnumbers_0$, $m\leq k\leq n$, sei $a_k\in\realnumbers$.\\
    Dann setzt man:
    \begin{align*}
        \sum_{k=m}^{n}a_k &= a_m+a_{m+1}+a_{m+2}+\dots+a_{n}
        \intertext{und}
        \prod_{k=m}^{n} &= a_m\cdot a_{m+1}\cdot a_{m+2}\cdot \dots\cdot a_{n}
    \end{align*}
    Für $n\in\naturalnumbers_0$, $n<m$ setzt man $\prod_{k=m}^{n}a_k = 1$.
\end{definition}
\begin{definition}[Fakultät]
    Sei $n\in\naturalnumbers$, dann gilt:
    \begin{align*}
        n! &= 1 \cdot 2\cdot 3 \cdot \dots \cdot n
        \intertext{und wir definieren}
        0! &= 1
    \end{align*}
    Alternativ lässt sich rekursiv definieren:
    \begin{align*}
        0! &\definedas 1\\
        n! &= (n-1)! \cdot n
    \end{align*}
\end{definition}

\begin{satz} % Satz 3
    Die Anzahl aller möglichen Anordnungen einer $n$-elementigen Menge $\set{A_1, \dots, A_n}$ ist gleich $n!$.\\
    Wenn wir beispielsweise die Menge $\set{1,2,3}$ betrachten. Mögliche Anordnungen: $\set{1,2,3}$, $\set{1,3,2}$, $\set{2,1,3}$, $\set{2,3,1}$, $\set{3,1,2}$, $\set{3,2,1}$. Somit gibt es 6 Möglichkeiten, was $3!$ entspricht.
    \begin{proof}[Induktionsbeweis]
        ~\\
        \begin{induktionsanfang}
            $n=1$, es gibt eine Anordnung $\set{A_1}$ und es gilt $1! = 1$
        \end{induktionsanfang}
        \\
        \begin{induktionsschritt}
            Die Gesamtzahl aller Anordnungen von $\set{A_1, \dots, A_{n+1}}$ ist gleich
            \begin{align*}
                &(n+1)\cdot [\text{Gesamtzahl von Anordnungen von }\set{A_1, \dots, A_n}]\\
                \annot{=}{I-Ann} &(n+1) \cdot n! = (n+1)!\qedhere
            \end{align*}
        \end{induktionsschritt}
    \end{proof}
\end{satz}

\subsection{Binomischer Lehrsatz}
\begin{definition}[Binomialkoeffizient]
    Für $n,k\in\naturalnumbers_0$ setzt man:
    \begin{align*}
        \binom{n}{k} &\definedas \frac{n\cdot(n-1)\cdot\dots\cdot(n-k+1)}{k!} = \frac{n!}{k!\cdot(n-k)!} \tag{$n$ über $k$}
    \end{align*}
\end{definition}
\begin{bemerkung}[Spezielle Binomialkoeffizienten]
    $\binom{n}{0} = 1, \binom{n}{n} = 1, \binom{n}{k} = 0$ für $k>n$
\end{bemerkung}

\begin{satz}
    \label{satz:teilmengen-anzahl}
    Die Anzahl der $k$-elementigen Teilmengen einer $n$-elementigen Menge $\set{A_1, \dots, A_n}$ ist gleich $\binom{n}{k}$.
\end{satz}

\begin{hilfsatz}
    \label{hilfsatz:binom-add}
    $\forall k,n\in\naturalnumbers$ gilt $\binom{n}{k} = \binom{n-1}{k-1} + \binom{n-1}{k}$.
    \begin{proof}
        \begin{align*}
            \binom{n-1}{k} + \binom{n-1}{k-1} &= \frac{(n-1)!}{k!\cdot(n-1-k)!} + \frac{(n-1)!}{(k-1)!\cdot(n-1-k+1)!}\\
            &= \frac{(n-1)!\cdot(n-k)}{k!\cdot(n-k)!} + \frac{(n-1)!\cdot k}{k!\cdot(n-k)!}\\
            &= \frac{(n-1)!\cdot\interv{n-k+k}}{k!\cdot(n-k)!}\\
            &= \frac{n!}{k!\cdot(n-k)!} \annot{=}{Def.} \binom{n}{k}\qedhere
        \end{align*}
    \end{proof}
\end{hilfsatz}
\begin{proof}[Beweis von Satz~\ref{satz:teilmengen-anzahl} (Induktion nach $n$)]
    ~\\
    \begin{induktionsanfang}
        $n=1$, $\set{A_1}$. Wenn $k=0$, dann gibt es eine Möglichkeit und es gilt $\binom{1}{0} = 1$. Wenn $k=1$, gibt es auch eine Möglichkeit und es gilt $\binom{1}{1} = 1$.\\
    \end{induktionsanfang}
    \\
    \begin{induktionsschritt}
        $n\rightarrow n+1$\\
        Die Behauptung sei für $M_n=\set{A_1, \dots, A_n}$ schon bewiesen. Wir betrachten $M_n+1 = \set{A_1, \dots, A_{n+1}}$. Für $k=0$ und $k=n+1$ ist die Behauptung offensichtlich.\\
        Für $1\leq k \leq n$ gehört jede $k$-elementige Teilmenge von $M_{n+1}$ zu genau einer der folgenden Klassen:
        \begin{enumerate}
            \item $T_0$ besteht aus $k$-elementigen Teilmengen, die $A_{n+1}$ nicht enthalten.
            \item $T_1$ besteht aus denjenigen Teilmengen, die $A_{n+1}$ enthalten.
        \end{enumerate}
        \noindent In $T_0$ gibt es nach Induktionsannahme $\binom{n}{k}$ Elemente.\\
        In $T_1$ gibt es $\binom{n}{k-1}$ Elemente\footnotemark.\\
        Insgesamt:
        \begin{equation*}
            \binom{n}{k}+\binom{n}{k-1}\annot{=}{\ref{hilfsatz:binom-add}}\binom{n+1}{k}\qedhere
        \end{equation*}
    \end{induktionsschritt}
    \footnotetext{Wir wissen, dass $A_{n+1}$ bereits ein Element der Teilmenge ist. Damit müssen wir noch $k-1$ aus $n$ Elemente auswählen. Die Formel dafür folgt aus der Induktionsannahme}
\end{proof}

\begin{satz}[Binomischer Lehrsatz]
    \label{satz:binom-lehrsatz}
    Sei $x,y\in\realnumbers$ und $n\in\naturalnumbers$. Dann gilt:
    \begin{align*}
    (x+y)
        ^{n} &= \sum_{k=0}^{n} \binom{n}{k}\cdot x^{n-k}\cdot y^k
    \end{align*}
\end{satz}
\begin{beispiel}[Folgerung der binomischen Formel aus dem binomischen Lehrsatz]
    Es sei $n=2$. Es gilt $\binom{2}{0}=1$, $\binom{2}{1}=2$, $\binom{2}{2}=1$. Daraus folgt:
    \begin{align*}
    (x+y)
        ^{2} &= x^2+2xy+y^2
    \end{align*}
\end{beispiel}
\begin{proof}[Beweis von Satz~\ref{satz:binom-lehrsatz}]
    ~\\IA: $n=0$
    \begin{align*}
    (x+y)
        ^0&=1\\
        \sum_{k=0}^0\binom{0}{k}\cdot x^{k}\cdot y^{0-k} &= \binom{0}{k}\cdot 1\cdot 1 = 1
    \end{align*}
    Induktionsschritt: $n\rightarrow n+1$
    \begin{align*}
    (x+y)
        ^{n+1} &= (x+y)^n\cdot (x+y) = (x+y)^n \cdot x + (x+y)^n\cdot y\\
        (x+y)^n\cdot x &\annot{=}{I-An} \sum_{k=0}^{n}\binom{n}{k}\cdot x^{n-k}\cdot y^k\cdot x\\
        &=1\cdot x^{n+1} + \sum_{k=1}^{n} \binom{n}{k}\cdot x^{n+1-k}\cdot y^{k}\\
        (x+y)^n\cdot y &= \sum_{k=0}^{n}\binom{n}{k}\cdot x^{n-k}\cdot y^{k+1}\tag{$l\definedas k+1$}\\
        &= \sum_{l=1}^{n+1} \binom{n}{l-1}\cdot x^{n+1-l}\cdot y^{l}\\
        &= \sum_{k=1}^{n+1} \binom{n}{k+1}\cdot x^{n+1-k} \cdot y^{k}\\[10pt]
        \impl (x+y)^{n+1} &= x^{n+1} + \sum_{k=1}^{n} \interv{\binom{n}{k}+\binom{n}{k+1}} \cdot x^{n+1-k}\cdot y^k + y^{n+1}\\
        &\annot{=}{\ref{hilfsatz:binom-add}} \sum_{k=0}^{n+1} \binom{n+1}{k}\cdot x^{n+1-k}\cdot y^{k}\qedhere
    \end{align*}
\end{proof}

\begin{bemerkung}
    Sei $x>0$, dann gilt $(1+x)^n = 1+\underbrace{\binom{n}{1}x}_{n\cdot x} + \underbrace{\sum \dots}_{>0} > 1 + n\cdot x$
\end{bemerkung}

\subsection{Bernoullische Ungleichung}

\begin{satz}
    Es sei $n\in\naturalnumbers$ und $a\in\realnumbers$, $a > -1$. Dann gilt
    \begin{align*}
    (1+a)
        ^n \geq 1+na
    \end{align*}
    \begin{proof}
        Wir verwenden vollständige Induktion:\\
        \begin{induktionsanfang}
            $n=1 \impl 1+a = 1+a$
        \end{induktionsanfang}
        \\
        \begin{induktionsschritt}
            $n\rightarrow n+1$
            \begin{align*}
            (1+a)
                ^{n+1} = (1+a)^n\cdot (1+a) \annot{\geq}{I-Ann} (1+na)\cdot(1+a) = 1+na+a+na^2\geq 1+(n+1)\cdot a
            \end{align*}
        \end{induktionsschritt}
    \end{proof}
\end{satz}

%%%%%%%%%%%%%%%%%%%%%%%%
% 16. November 2023
%%%%%%%%%%%%%%%%%%%%%%%%

\newpage

\subsection{Wurzeln}

\begin{satz}
    Für jedes $c\in\realnumbers$, $c>0$, gibt es genau ein $x>0$, so dass $x^2 = c$ ist.
    \begin{proof}
        \textit{Eindeutigkeit}\\
        $x_1>0$, $x_2>0$: $\pair{x_1}^2 = \pair{x_2}^2 = c \impl 0 = (\pair{x_1}^2-\pair{x_2}^2) = (x_1-x_2) \cdot \underbrace{(x_1+x_2)}_{>0} \impl x_1 = x_2$\\
        \textit{Existenz}\\
        Wir definieren $M\definedas\set{z\in\realnumbers|~z\geq 0, z^2 \leq c}$. Dann gilt $0\in M \impl M\neq \emptyset$\\[10pt]
        $M$ ist beschränkt, weil $(1+c)^2=1+2c+c^2 > c$, \quad$z\in M \impl z < 1 + c$\\
        Somit $\exists \sup M$ und wir definieren $x\definedas \sup M$. Zu zeigen: $x^2 = c$\\[10pt]
        Wir nehmen an, dass $x^2<c$ und setzen $\varepsilon \definedas \min\set{1, \frac{c-x^2}{2x+1}} \impl 0 < \varepsilon \leq 1 \impl \varepsilon^2 < \varepsilon$\\
        $(x+\varepsilon)^2 = x^2 + 2\varepsilon x + \varepsilon^2 < x^2+\varepsilon\pair{2x+1}\leq x^2+c-x^2=c$\\
        $\impl x+\varepsilon\in M$ (Widerspruch) $\impl x^2 \geq c$\\[10pt]
        Wir nehmen an, dass $x^2 > c$, $\varepsilon \definedas\min\set{\frac{x^2-c}{2x}, \frac{x}{2}}$, $\varepsilon > 0$, $x-\varepsilon \geq x-\frac{x}{2}>0$\\
        $\pair{x-\varepsilon}^2 = x^2 - 2 x\varepsilon + \varepsilon^2 > x^2-2x\varepsilon \geq x^2-x^2+c\impl (x-\varepsilon)^2 > c \impl x\neq \sup M$ (Widerspruch)\\[10pt]
        $\impl x^2 = c$
    \end{proof}
\end{satz}

\begin{bemerkung}
    $x=\sqrt {c}$, $x=c^{\frac{1}{2}}$, $x$ ist die Quadratwurzel von $c$
\end{bemerkung}

\begin{satz}
    Für $n\in\naturalnumbers$ und für jedes $c\in\realnumbers$, $c\geq 0$ gibt es genau ein $x \geq 0$, $x\in\realnumbers$, so dass $x^n = c$.
    \begin{proof}
        \textit{Eindeutigkeit}\\
        $x_1>0$, $x_2>0$, $\pair{x_1}^n-\pair{x_2}^n = c$, $0=\pair{\pair{x_1}^n-\pair{x_2}^n} = \pair{x_1 - x_2}\cdot\pair{\sum_{k=0}^{n-1} \pair{x_1}^{n-k+1}\cdot \pair{x_2}^k}$\\
        $\impl x_1 = x_2$
    \end{proof}
    \noindent Die Existenz ist Aufgabe auf dem Übungsblatt.
\end{satz}

\begin{definition}[Spezielle Potenzen]
    $m,n\in\naturalnumbers$\quad $x^\frac{m}{n}\definedas \pair{x^\frac{1}{m}}^n$, $x^0 = 1$, $0^m = 0$ mit $m\neq 0$, $0^0 = 1$
\end{definition}

\subsection{Absolutbetrag}

\begin{definition}[Betrag]
    $\abs{a} \definedas \left\{ \begin{array}{lr}
                                    a  & a>0 \\
                                    0  & a=0 \\
                                    -a & a<0
    \end{array}\right.$
\end{definition}

\begin{satz}[Eigenschaften des Betrags]
    \theoremescape
    \begin{enumerate}[label=(\roman*)]
        \item $\abs{a} \geq 0$, $\abs{a} = 0 \equivalent a = 0$
        \item $\abs{\lambda\cdot a} = \abs{\lambda}\cdot\abs{a}$, $\forall \lambda, a \in \realnumbers$
        \item $\abs{a+b}\leq \abs{a}+\abs{b}$ (Dreiecksungleichung) %%% 5.14
    \end{enumerate}
    \begin{proof}[Beweis von (iii)]
        \begin{align*}
            \abs{a+b}^2 = &\pair{a+b}^2 = a^2 + 2ab + b^2\\
            \leq &\abs{a}^2 + 2\abs{a}\abs{b} + \abs{b}^2 = \pair{\abs{a}+\abs{b}}^2\\
            \impl &\abs{a+b}\leq\abs{a}+\abs{b}\qedhere
        \end{align*}
    \end{proof}
\end{satz}

\begin{definition}[Geometrische Betrachtung des Betrags]
    Man nennt $\abs{a-b}$ den Abstand zweier Punkte $a,b\in\realnumbers$ auf der Zahlengerade.
\end{definition}

\begin{satz}[Eigenschaften von Differenzen im Betrag]
    \label{satz:diff-abs}
    \theoremescape
    \begin{enumerate}[label=(\roman*)]
        \item $\abs{a-b}\geq 0$, $\abs{a-b} = 0\equivalent a = b$
        \item $\abs{a-b} = \abs{b-a}$
        \item $\abs{a-b}\leq \abs{a-c} + \abs{b-c}$ $\forall c\in\realnumbers$ %%% 5.15
    \end{enumerate}
    \begin{proof}[Beweis von (iii)]
        \begin{align*}
            \abs{a-b} &= \abs{a-c+c-b} \leq \abs{a-c}+\abs{c-b} = \abs{a-c}+\abs{b-c}\qedhere
        \end{align*}
    \end{proof}
\end{satz}

\begin{satz} %%% 5.16
    $\forall a,b\in\realnumbers$ gilt $\abs{\abs{a}-\abs{b}}\leq\abs{a-b}$
    \begin{proof}
        \begin{align*}
            \abs{a} = \abs{a-b+b} &\leq \abs{a-b} + \abs{b}\\
            \impl \abs{a}-\abs{b} &\leq \abs{a-b}\\
            \abs{b}-\abs{a} &\leq \abs{b-a} = \abs{a-b}\\[10pt]
            \impl \abs{a-b}&\geq\abs{\abs{a}-\abs{b}}\qedhere
        \end{align*}
    \end{proof}
\end{satz}

\begin{folgerung}
    \theoremescape
    \begin{enumerate}[label=(\roman*)]
        \item $\abs{a-b}\geq \abs{a}-\abs{b}$
        \item $\abs{a+b} = \abs{a-\pair{-b}} \geq \abs{a} - \abs{-b}$
    \end{enumerate}
\end{folgerung}

\begin{bemerkung}
    Durch Induktion leitet man her, dass:
    \begin{align*}
        \abs{\sum_{i=1}^{n} a_i} &\leq \sum_{i=1}^{n} \abs{a_i}\quad a_i \in\realnumbers
    \end{align*}
\end{bemerkung}

\newpage


    \section{Folgen und Grenzwerte}
    \subsection{Konvergenz}
\thispagestyle{pagenumberonly}

\begin{definition}[Reelle Folge]
    Eine reelle Folge ist eine Abbildung $\N\fromto\R$. Alternative Notation: $\pair{a_n}_{n\in\N}$, $\pair{a_1, a_2, \dots}$
\end{definition}

\begin{beispiel}
    $a_n = \frac{1}{n}$, $n\geq 1$\quad $\pair{1,\frac{1}{2},\frac{1}{3}, \dots}$
\end{beispiel}

\begin{definition}[Konvergenzkriterium]
    Sei $\pair{a_n}_{n\in\N}$ eine Folge reeller Zahlen. Die Folge heißt konvergent gegen $a\in\R$, falls gilt:
    \begin{align*}
        \forall \varepsilon > 0 ~\exists n_0\in\N\colon \abs{a_n-a} &< \varepsilon\text{ für alle $n>n_0$}
        \intertext{$a$ heißt Grenzwert von $\pair{a_n}_{n\in\N}$ und man schreibt}
        \lim_{n\fromto \infty} a_n &= a
    \end{align*}
\end{definition}

\begin{beispiel}[Nachweis von Konvergenz]
    Es sei
    \begin{align*}
        \pair{a_n}_{n\in\N}&=1+\frac{(-1)^n}{2n}
        \intertext{wir definieren $\varepsilon$ und $n_0$}
        \varepsilon > 0,&\quad n_0 = \frac{1}{2\varepsilon}\\
        \intertext{und wenden das Konvergenzkriterium an}
        \abs{a_n-1} = \abs{1+\frac{(-1)^n}{2n}-1} &= \abs{\frac{(-1)^n}{2n}} < \frac{\abs{(-1)^n}}{\abs{\frac{2}{2\varepsilon}}} = \varepsilon\\
        \impl \lim_{n\fromto\infty} a_n &=1
    \end{align*}
\end{beispiel}

\begin{bemerkung}[\anf{Fast alle Elemente}]
    Wir sagen, dass fast alle Element der Folge $(a_n)_{n\in\N}$ eine Eigenschaft (E) haben, wenn es höchstens eine endliche Anzahl von $a_n$ existiert, die die Eigenschaft (E) nicht erfüllen.
\end{bemerkung}

\begin{definition}[Alternativ formuliertes Konvergenzkriterium]
    Sei $(a_n)_{n\in\N}$ eine Folge, dann heißt diese Folge konvergent gegen $a\in\R$, wenn $\forall \varepsilon > 0$ und für fast alle $a_n$ gilt: $\abs{a_n-a}<\varepsilon$
\end{definition}

\begin{definition}[Nullfolge]
    Eine Folge, die gegen $0$ konvergiert, heißt Nullfolge.
\end{definition}

\begin{definition}[Divergenz]
    Eine nicht-konvergente Folge heißt divergent.
\end{definition}

\begin{definition}[$\varepsilon$-Umgebung]
    Für $\varepsilon > 0$, $a\in\R$ versteht man unter der $\varepsilon$-Umgebung von $a$ das Intervall $\left]a-\varepsilon, a+\varepsilon\right[$
\end{definition}

\subsection{Geometrische Bedeutung der Konvergenz}

\begin{visualisierung}
    ~
    \begin{figure}[H]
        \centering
        \begin{tikzpicture}
            \draw (0,0) -- (10,0);
            \draw (5,-0.25) node[below] {$a$} -- (5,0.25);
            \draw (3.8,-0.4) -- (3.9, -0.4) node[below] {$a-\varepsilon$}  -- (4,-0.4) -- (4,0.4) -- (3.8, 0.4);
            \draw (6.2,-0.4) -- (6.1, -0.4) node[below] {$a+\varepsilon$}  -- (6,-0.4) -- (6,0.4) -- (6.2, 0.4);
            \foreach \x in {1, 2.2, 3, 3.9, 4.3, 4.6, 4.8, 4.9, 4.94, 4.96, 4.98, 5.005, 5.01, 5.04, 5.1, 5.35, 5.85, 6.4}
            \draw (\x,-0.15) -- (\x,0.15);
        \end{tikzpicture}
        \caption{Geometrische Darstellung einer\\ konvergenten Folge und ihrer $\varepsilon$-Umgebung}
    \end{figure}
\end{visualisierung}

%%%%%%%%%%%%%%%%%%%%%%%%
% 21. November 2023
%%%%%%%%%%%%%%%%%%%%%%%%

\subsection{Eigenschaften von Folgen und Konvergenzen}

\begin{definition}[Beschränktheit von Folgen]
    \marginnote{[21. Nov]}
    Eine Folge $(a_n)_n$ heißt nach oben beschränkt, falls es ein $k\in\R$ gibt mit
    \begin{align*}
        \forall n\in \N\colon a_n \leq k\tag{$k$ ist obere Schranke für $(a_n)_n$}
    \end{align*}
    Beschränktheit nach unten wird analog definiert.\\
    $(a_n)_n$ heißt beschränkt, falls ein $k\geq 0$ existiert mit
    \begin{align*}
        \forall n\in\N\colon -k \leq a_n \leq k\quad \text{bzw.}\quad \forall n\in\N\colon \abs{a_n} \leq k
    \end{align*}
\end{definition}

\begin{satz}
    \label{satz:konv-folg-beschr}
    Jede konvergente Folge $(a_n)_{n\in\N}$ ist beschränkt.
    \begin{proof}
        Sei $a$ der Grenzwert von $(a_n)_n$~$\pair{\lim_{n\fromto\infty} a_n = a}$. Das heißt für $\varepsilon = 1$
        \begin{align*}
            \impl \exists N\in\N\colon \abs{a_n-a} &< 1\quad &\forall n\geq N\\[8pt]
            \abs{a_n} = \abs{a_n-a+a} \leq \abs{a_n-a} + \abs{a} &< 1+\abs{a} \quad &\forall n\geq N
            \intertext{Setze $k\definedas \max\pair{\abs{a_1}, \abs{a_2}, \dots, \abs{a_{N-1}}, 1+\abs{a}} \geq 0$}
            \impl \abs{a_n} &\leq k \quad&\forall n\in\N &\qedhere
        \end{align*}
    \end{proof}
\end{satz}

\begin{satz}[Eindeutigkeit des Limes]
    Der Limes einer konvergenten Folge $(a_n)_n$ ist eindeutig.
    \begin{proof}
        Annahme: $(a_n)_n\fromto a$ und $(a_n)_n\fromto b$ mit  $a\neq b$. Dann gilt \OBDA, dass $a<b$. Setzen $\varepsilon = \frac{b-a}{2} > 0$.
        \begin{align*}
            \impl &\exists N_1 \in\N\colon \abs{a_n-a} < \varepsilon \quad \forall n\geq N_1\\
            \impl &\exists N_2 \in\N\colon \abs{a_n-b} < \varepsilon \quad \forall n\geq N_2\\[10pt]
            &N\definedas \max\pair{N_1, N_2}\\
            \impl &\forall n\geq N\colon a_n < a+\varepsilon = a + \frac{b-a}{2} = \frac{b+a}{2} = b-\varepsilon\\
            &\forall n\geq N\colon a_n > b-\varepsilon = \frac{b+a}{2}\\
            \impl &a_n <\frac{b+a}{2} < a_n \quad \forall n\geq N\qquad\text{ (Widerspruch)}\qedhere
        \end{align*}
    \end{proof}
\end{satz}

\begin{satz}[Eigenschaften von konvergenten Folgen]
    \label{satz:konvergenzsaetze}
    Seien $(x_n)_n, (y_n)_n$ konvergente reelle Folgen mit $x_n\fromto a$, $y_n\fromto b$ für $n\fromto\infty$. Dann gilt:
    \begin{enumerate}[label=(\alph*)]
        \item $x_n+y_n\fromto a+b$\quad ($\lim\pair{x_n+y_n}=\lim x_n + \lim y_n$).
        \item $x_n\cdot y_n \fromto a\cdot b$\quad ($\lim\pair{x_n\cdot y_n} = \lim x_n \cdot \lim y_n$).
        \item $\lambda\cdot y_n \fromto \lambda\cdot b$\quad ($\lim\pair{\lambda\cdot y_n} = \lambda \cdot \lim y_n$). ($\lambda\in\R$)
        \item Ist $b\neq 0$, so ist $y_n \neq 0$ für fast alle $n\in\N$ und $\frac{x_n}{y_n}$ ist für fast alle $n$ definiert und $\lim\pair{\frac{x_n}{y_n}} = \frac{\lim x_n}{\lim y_n} = \frac{a}{b}$.
        \item $\abs{x_n} \fromto \abs{a}$\quad ($\lim \abs{x_n} = \abs{\lim x_n}$)
        \item Ist $x_n \leq y_n$ für fast alle $n\in\N$ $\impl a = \lim x_n \leq b = \lim y_n$.
    \end{enumerate}

    \begin{proof}
        \theoremescape
        \begin{enumerate}[label=(\alph*)]
            \item Nach Satz~\ref{satz:diff-abs} gilt
            \begin{align*}
                \abs{x_n+y_n - (a+b)} = \abs{x_n - a + (y_n-b)} &\leq \abs{x_n - a} + \abs{y_n-b}
                \intertext{Aufgrund der Konvergenz der Folgen gilt}
                \forall \varepsilon > 0~\exists N_1\colon \abs{x_n - a} &< \frac{\varepsilon}{2} \quad \forall n\geq N_1\\
                \forall \varepsilon > 0~\exists N_2\colon \abs{y_n - b} &< \frac{\varepsilon}{2} \quad \forall n\geq N_2\\
                \intertext{Wir wählen $N=\max\pair{N_1, N_2}$}
                \impl \abs{\pair{x_n+y_n}-(a+b)} \leq \abs{x_n - a} + \abs{y_n-b} &\leq \frac{\varepsilon}{2} + \frac{\varepsilon}{2} = \varepsilon\quad \forall n\geq N
            \end{align*}
            \item Umformen, um eine passende Ungleichung zu erreichen
            \begin{align*}
                x_n \cdot y_n - ab &= \pair{x_n-a+a} \cdot y_n - ab\\
                &= \pair{x_n-a}\cdot y_n + a \cdot y_n - ab\\
                &= \pair{x_n-a}\cdot y_n + a\cdot \pair{y_n-b}\\[10pt]
                \impl \abs{x_n\cdot y_n - ab} &= \abs{\pair{x_n-a}\cdot y_n + a\cdot \pair{y_n-b}}\\
                &\leq \abs{x_n-a}\cdot\abs{y_n}+\abs{a}\cdot\abs{y_n-b}
                \intertext{Nach Satz~\ref{satz:konv-folg-beschr} ist $y_n$ beschränkt}
                \exists k\geq 0\colon \abs{y_n} &\leq k \quad\forall n\in\N
                \intertext{Nach der Konvergenz der Folgen gilt außerdem}
                \forall \varepsilon >0~\exists N_1, N_2\colon \abs{x_n-a} &< \frac{\varepsilon}{2(k+1)}\quad\forall n\geq N_1\\
                \abs{y_n-b} &< \frac{\varepsilon}{2(\abs{a}+1)}\quad\forall n\geq N_2\\[10pt]
                \impl \forall n\geq \max\pair{N_1, N_2}\colon \abs{x_n\cdot y_n-ab} &\leq \frac{\varepsilon}{2(k+1)}\cdot k + \abs{a}\cdot\frac{\varepsilon}{2(\abs{a}+1)} \leq \varepsilon
            \end{align*}
            \item Setze $y_n=\lambda \fromto \lambda$ und verwende (b)
            \item Wir wählen $\varepsilon\definedas\frac{\abs{b}}{2}$
            \begin{align*}
                \exists N\in\N\colon \abs{y_n-b} &< \varepsilon = \frac{\abs{b}}{2}\quad\forall n\geq \N.\\[10pt]
                \forall n\geq N\colon\quad\abs{y_n} &= \abs{y_n-b+b} = \abs{b+(y_n-b)}\\
                &\geq \abs{b} - \abs{y_n-b} > \abs{b} - \frac{\abs{b}}{2} + \frac{\abs{b}}{2}\\
                &> 0
            \end{align*}
            $\impl y_n \neq 0$ für fast alle $n$ und somit $\frac{x_n}{y_n}$ wohldefiniert für fast alle $n\in\N$
            \begin{align*}
                \intertext{Wir betrachten den Spezialfall $x_n = 1$}
                \abs{\frac{1}{y_n} - \frac{1}{b}} &= \abs{\frac{b-y_n}{y_n\cdot b}} = \frac{\abs{b-y_n}}{\abs{y_n}\cdot\abs{b}}.\\
                \intertext{Wir wissen schon, dass $\exists N_1\colon \abs{y_n} \geq \frac{\abs{b}}{2} \quad\forall n\geq N_1$}
                \impl \forall n\geq N_1\colon \abs{\frac{1}{y_n}-\frac{1}{b}} &\leq \frac{\abs{y_n-b}}{\abs{y_n}\cdot\abs{b}} \leq \frac{2\cdot\abs{y_n-b}}{\abs{b}\cdot \abs{b}}\\
                \forall \varepsilon > 0~\exists N_2\colon \abs{y_n-b} &< \frac{\abs{b}^2}{2}\cdot\varepsilon\\
                \impl \forall n\geq \max\pair{N_1, N_2}\colon \abs{\frac{1}{y_n}-\frac{1}{b}} &< \frac{2}{\abs{b}^2} \cdot \frac{\abs{b}^2}{2} \cdot \varepsilon = \varepsilon
                \intertext{Jetzt wenden wir (b) an und dann gilt}
                \lim_{n\fromto \infty} \frac{x_n}{y_n} = \lim_{n\fromto \infty} \pair{x_n\cdot\frac{1}{y_n}} &= \lim x_n \cdot \lim\frac{1}{y_n} = a\cdot \frac{1}{b} = \frac{a}{b}
            \end{align*}
            \item Wir zeigen mit der Umkehrung der Dreiecksungleichung
            \begin{align*}
                \abs{\abs{x_n}-\abs{a}} \annot[{&}]{\leq}{Dreiecks.} \abs{x_n-a}\\
                \impl \abs{x_n} &\fromto \abs{a}
            \end{align*}
            \item Angenommen $a > b$. Wir wählen $\varepsilon=\frac{a-b}{2}$.
            \begin{align*}
                \impl \exists N_1\colon \abs{x_n-a} &< \varepsilon \quad\forall n\geq N_1\\
                \impl \exists N_2\colon \abs{y_n-b} &< \varepsilon \quad\forall n\geq N_2\\
                \intertext{Für $n\geq \max\pair{N_1,N_2}$ folgt}
                x_n &> a-\varepsilon = a + \frac{a-b}{2}\\
                &= \frac{a+b}{2} = b+\frac{a-b}{2} = b + \varepsilon\\
                &> y_n\\[10pt]
                \impl x_n &> y_n \tag{Widerspruch, Annahme falsch}\\
                \impl a &\leq b
            \end{align*}
        \end{enumerate}
    \end{proof}
\end{satz}

\newpage

\begin{satz}[Sandwich Satz]
    \label{satz:sandwich}
    Seien $(a_n)_n$, $(b_n)_n$, $(c_n)_n$ Folgen mit $\lim a_n = a$, $\lim \pair{b_n-a_n} = 0$ und $a_n \leq c_n \leq b_n$ für fast alle $n\in\N$.\\
    Dann folgt, dass $(c_n)_n$ und $(b_n)_n$ jeweils gegen $a$ konvergieren.
    \begin{proof}
        ~\\
        1. Schritt: $\lim b_n = a$.\\[10pt]
        Sei $\varepsilon > 0$. Da $a_n\fromto a$, $\exists N_1\colon \abs{a_n-a} < \frac{\varepsilon}{2}\quad\forall n\geq N_1$\\
        $0\leq b_n-a_n$ ist Nullfolge $\impl \exists N_2\colon \abs{b_n-a_n} < \frac{\varepsilon}{2}\quad\forall n\geq N_2$\\
        \begin{align*}
            \abs{b_n-a} &= \abs{b_n-a_n+a_n-a}\\
            &\leq \abs{b_n - a_n} + \abs{a_n-a}\\
            &\leq \frac{\varepsilon}{2} + \frac{\varepsilon}{2} = \varepsilon\quad \forall n\geq \max\pair{N_1, N_2}
        \end{align*}
        \noindent 2. Schritt: $\lim c_n = a$.
        \begin{align*}
            \lim a_n &= a = \lim b_n\\
            \impl \forall\varepsilon > 0~\exists N_1, N_2\colon \abs{a_n-a} &< \varepsilon \quad\forall n\geq N_1\\
            \text{ und } \abs{b_n-a} &< \varepsilon \quad\forall n\geq N_2\\[10pt]
            a-\varepsilon < a_n &< a + \varepsilon\quad\forall n\geq N_1\\
            a-\varepsilon < b_n &< a + \varepsilon\quad\forall n\geq N_2\\
            \intertext{Auch gilt $a_n \leq c_n \leq b_n~\forall n\geq N_3$ und wir definieren $N\definedas\max\pair{N_1, N_2, N_3}$}
            \impl a-\varepsilon < a_n \leq c_n &\leq b_n < a + \varepsilon\quad\forall n\geq N
        \end{align*}
        Das heißt $\abs{c_n-a} < \varepsilon~\forall n\geq N$. Das heißt $\lim c_n = a$\qedhere
    \end{proof}
\end{satz}

\subsection{Monotone Konvergenz}

Wir sind bisher immer davon ausgegangen, dass wir den Grenzwert einer Folge bereits kennen. Das folgende Unterkapitel beschäftigt sich damit, die Konvergenz einer Folge nachzuweisen, wenn deren Grenzwert nicht bekannt ist.

\begin{definition}[Monotonie]
    Eine reelle Folge $(a_n)_n$ heißt
    \begin{enumerate}[label=(\roman*)]
        \item monoton wachsend, wenn $a_n \leq a_{n+1}\quad\forall n\in\N$
        \item streng monoton wachsend, wenn $a_n < a_{n+1}\quad\forall n\in\N$
        \item monoton fallend, wenn $a_n \geq a_{n+1}\quad\forall n\in\N$
        \item streng monoton fallend, wenn $a_n > a_{n+1}\quad\forall n\in\N$
    \end{enumerate}
    \noindent Wir nennen $(a_n)_n$ (streng) monoton, falls sie (streng) monoton wachsend oder fallend ist.
\end{definition}

\begin{satz}[Monotone Konvergenz]
    \label{satz:monoton-konv}
    Eine monoton wachsende Folge $(a_n)_n$ konvergiert genau dann, wenn $(a_n)_n$ nach oben beschränkt ist.\\
    Und eine monoton fallende Folge $(a_n)_n$ konvergiert genau dann, wenn sie nach unten beschränkt ist.
\end{satz}

\newpage

\begin{lemma}[Hilfaussage für monotone Konvergenz]
    ~\label{lemma:hilf-monoton-konv}
    Jede nach oben (bzw. unten) beschränkte Folge besitzt eine kleinste obere (bzw. größte untere) Schranke.
    \begin{proof}
        Sei $(a_n)_n$ nach oben beschränkt und $S\definedas\set{c\in\R|~a_n \leq c~\forall n} \neq \emptyset\footnote{Folgt aus der Beschränktheit}.$ Dann ist $S$ nach unten beschränkt, da $\forall c\in S\colon a_1 \leq c$.
        \begin{align*}
            &\impl a\definedas \inf S \text{ existiert}
            \intertext{Behauptung: $a$ ist obere Schranke für $(a_n)_n$, das heißt $a\in S$. Annahme: $a\notin S$ ($a$ ist keine obere Schranke)}
            &\impl \exists n_0 \in\N\colon a_{n_0} > a
            \intertext{Wir setzen $\varepsilon\definedas a_{n_0} - a > 0$}
            &\impl \exists c\in S\colon c < a + \varepsilon = a + a_{n_0} - a = a_{n_0}
        \end{align*}
        Widerspruch zu $c\in S$. Das heißt $a$ ist obere Schranke für $(a_n)_n$.
    \end{proof}
\end{lemma}

\begin{proof}[Beweis von Satz~\ref{satz:monoton-konv}]
    Angenommen $(a_n)_n$ ist monoton wachsend und nach oben beschränkt.
    \begin{align*}
        &a\definedas\sup_{n\in\N} a_n\\
        \impl &a_n \leq a\quad\forall n\in\N
        \intertext{Sei $\varepsilon > 0 \impl a-\varepsilon$ keine obere Schranke für $(a_n)_n$ mehr.}
        \impl &\exists N\in\N\colon a_N \geq a-\varepsilon\\
        \intertext{Sei $n\geq N$. Dann gilt:}
        &a_N \leq a_{N+1} \leq a_{N+2} \leq \dots \leq a_{N+k} = a_n\\
        \impl &\forall n\geq N\colon a-\varepsilon < a_n\\
        \impl &\abs{a_n-a} < \varepsilon\qedhere
    \end{align*}
\end{proof}

\begin{proof}[Alternativer Beweis]
    \footnote{Dieser Beweis bezieht sich laut Vorlesung auf Lemma 14, welches allerdings nicht existiert. Gemeint ist vermutlich Lemma 15 (hier Lemma~\ref{lemma:hilf-monoton-konv})}
    ~\\
    Sei $(a_n)_n$ eine Folge mit $a_n = f(a),~f:\N\fromto\R$
    \begin{align*}
        \impl \bild(f) &= f(\N) \text{ nach oben beschränkt }\\
        \impl a&\definedas\sup\pair{f\pair{\N}}\geq a_n \quad\forall n\in\N\qedhere
    \end{align*}
\end{proof}

\newpage

%%%%%%%%%%%%%%%%%%%%%%%%
% 23. November 2023
%%%%%%%%%%%%%%%%%%%%%%%%

\begin{beispiel}[Berechnen von $\sqrt {c}$ für $c>0$]
    \marginnote{[23. Nov]}
    \begin{align*}
        x^2 = c &\equivalent x = \frac{c}{x}\tag{$x >0$}\\
        &\equivalent x=\frac{1}{2}\pair{x+\frac{c}{x}}
    \end{align*}
    Folge $x_n$. Wählen $x_0 > 0$. Für $n\in\N$ sei $x_n=\frac{1}{2}\pair{x_{n-1}+\frac{c}{x_{n-1}}}$\\
    Behauptung 1: $x_n \geq \sqrt{c}\quad\forall n\in\N$
    \begin{proof}
        \begin{align*}
            x_1 &= \frac{1}{2}\pair{x_0+\frac{c}{x_0}}
        \end{align*}
        \begin{mdframed}
            Einschub: Arithmetisch-geometrisches Mittel (AGM)
            \begin{align*}
                0 \leq (x-y)^2 &= x^2-2xy+y^2 \equivalent xy \leq \frac{x^2+y^2}{2}
                \intertext{Wir setzen $x=\sqrt{a}$ und $y=\sqrt{b}$}
                \impl \forall a,b \geq 0\colon \sqrt {ab} &\leq \frac{a+b}{2}
            \end{align*}
        \end{mdframed}

        \begin{align*}
        (x_1)
            ^2 &= \pair{\frac{1}{2}\pair{x_0+\frac{c}{x_0}}}^2 \annot{\geq}{AGM} x_0\cdot\frac{c}{x_0} = c\\
            \impl x_1 &\geq \sqrt{c}\\
            \intertext{Zu zeigen: Falls $x_n \geq 0\impl x_{n+1}\geq \sqrt{c}$}
            \pair{x_{n+1}}^2 &= \pair{\frac{1}{2}\pair{x_n + \frac{c}{x_n}}}^2 \annot{\geq}{AGM} x_n \cdot \frac{c}{x_n} = c\qedhere
        \end{align*}
    \end{proof}
    \vspace{0.5cm}

    \noindent Behauptung 2: $(x_n)_n$ ist monoton fallend.
    \begin{proof}
        \begin{align*}
            x_{n+1} &= \frac{1}{2}\pair{x_n+\frac{c}{x_n}}\\
            (x_n)^2 &= \pair{\frac{1}{2}\pair{x_{n+1} + \frac{c}{x_{n+1}}}}^2 \geq c\tag{$n\geq 1$}\\
            \impl x_n &\geq \frac{c}{x_n}\\
            \impl x_{n+1} &= \frac{1}{2}\pair{x_n+\frac{c}{x_n}}\leq \frac{1}{2}\pair{x_n+x_n} = x_n\qedhere
        \end{align*}
    \end{proof}

    \noindent Nach dem Satz der monotonen Konvergenz (\ref{satz:monoton-konv}) existiert ein Grenzwert $x$ mit
    \begin{align*}
        x&\definedas \lim_{n\fromto\infty} x_n = \lim_{n\fromto\infty} x_{n-1}\\
        x &= \lim_{n\fromto\infty} x_n = \lim_{n\fromto\infty} \frac{1}{2}\pair{x_{n-1}+\frac{c}{x_{n-1}}} = \frac{1}{2}\pair{x+\frac{c}{x}}\\
        \equivalent x &= \sqrt{c}
    \end{align*}
\end{beispiel}

\begin{beispiel}[Harmonische Folge]
    Es sei $x_n = \frac{1}{n}\fromto 0$
    \begin{align*}
        \abs{\frac{1}{n}-0} &= \frac{1}{n}
        \intertext{geg. $\varepsilon > 0$ wähle $N\in\N$ mit $N>\frac{1}{\varepsilon} \equivalent \varepsilon > \frac{1}{N}$}
        \impl \forall n\geq N\colon \abs{\frac{1}{n}-0} &= \frac{1}{n} \leq \frac{1}{N}< \varepsilon
    \end{align*}
    Ähnlich funktioniert $x_n = \frac{1}{\sqrt {n}}$
\end{beispiel}

\begin{beispiel}
    \begin{align*}
        x_n&\definedas \frac{1+2+3+\dots+n}{n^2} \fromto \frac{1}{2}\text{ für }n\fromto\infty
        \intertext{Mit Gaußscher Summenformel ($1+2+3+\dots+n = \frac{n\cdot(n+1)}{2}$)}
        \impl x_n &= \frac{\frac{n\cdot(n+1)}{2}}{n^2} = \frac{n^2+n}{2n^2} = \frac{1+\frac{1}{n}}{2} \fromto \frac{1}{2}
    \end{align*}
\end{beispiel}

\begin{beispiel}[Geometrische Folge (1)]
    \label{beispiel:geometrische-folge}
    Sei $0\leq q< 1$. Dann gilt $x_n \definedas q^n \fromto 0$
    \begin{proof}
        Ist $q=0\impl x_n = 0^n = 0 \fromto 0$. Also sei $0<q<1$
        \begin{align*}
            \impl \frac{1}{q} &> 1\\
            h &\definedas\frac{1}{q}-1>0\\
            \impl \frac{1}{q} &= 1 + h\\
            \impl q &= \frac{1}{1+h}\\
            \impl x_n &= q^n = \pair{\frac{1}{1+h}}^n = \frac{1}{(1+h)^n}
            \intertext{Nach Bernoulli (\ref{satz:bernoulli-ungleichung}) gilt $(1+h)^n \geq 1+nh>nh$}
            \impl q^n &= \frac{1}{(1+h)^n}< \frac{1}{nh}\fromto 0 \text{ für } \ntoinf\\
            \impl \abs{q^n-0} &= q^n < \frac{1}{nh} \fromto 0 \text{ für } \ntoinf
            \intertext{Für $\varepsilon>0$ wähle $N\geq \frac{1}{\varepsilon\cdot h}\equivalent \frac{1}{N\cdot h}\leq \varepsilon$}
            \impl \forall n\geq N&\colon \abs{q^n-0} = q^n \leq \frac{1}{nh} \leq \frac{1}{Nh} < \varepsilon\qedhere
        \end{align*}
    \end{proof}
\end{beispiel}

\begin{uebung}[Geometrische Folge (2)]
    Es sei $-1<q<1$. Weisen Sie basierend auf Beispiel~\ref{beispiel:geometrische-folge} nach, dass dann $x_n \definedas q^n \fromto 0$ gilt.
\end{uebung}

\newpage

\begin{beispiel}
    Sei $a>0$, $x_n\definedas a^\frac{1}{n} = \sqrt[n]{a} \fromto 1$ für $\ntoinf$.\\
    \begin{proof}
        Wir unterscheiden in drei Fälle.
        \theoremescape
        \begin{enumerate}[label=\arabic*.]
            \item $a=1$
            \begin{align*}
                \impl x_n &= 1
                \intertext{\item $a> 1$}
                \impl a^\frac{1}{n} &> 1^\frac{1}{n}=1\\
                h_n &\definedas a^\frac{1}{n} - 1 > 0 \text{ und } 1+h_n = a^\frac{1}{n}\\
                a&=\pair{1+h_n}^n \geq 1+n\cdot h_n > n\cdot h_n\\
                \impl h_n &< \frac{a}{n}\\
                \abs{a^\frac{1}{n}-1} &= a^\frac{1}{n}-1 = h_n < \frac{a}{n}\\
                \impl \lim_{\ntoinf} a^\frac{1}{n}&= 1
                \intertext{\item $0<a<1$}
                b &\definedas \frac{1}{a} > 1\\
                \impl \lim_{\ntoinf} b^\frac{1}{n} &= 1\\
                \lim_{\ntoinf} b^\frac{1}{n} &=\lim_{\ntoinf} \frac{1}{a^\frac{1}{n}}\\
                \impl \lim_{\ntoinf} a^{\frac{1}{n}} &= 1\qedhere
            \end{align*}
        \end{enumerate}
    \end{proof}
\end{beispiel}

\begin{beispiel}
    Es sei $x_n = \sqrt[n]{n} = n^\frac{1}{n}\fromto 1$
    \begin{align*}
        n^{\frac{1}{n}} &> 1 \text{ für } n\geq 2\\
        h_n &\definedas n^{\frac{1}{n}} - 1\\
        \impl n = \pair{n^\frac{1}{n}}^n = \pair{1+h_n}^n \annot[{&}]{\geq}{\ref{satz:bernoulli-ungleichung}} 1+n\cdot h_n > n\cdot h_n\\
        \impl h_n &\leq \frac{n}{n} = 1
        \intertext{Wir wenden den Binomischen Lehrsatz (\ref{satz:binom-lehrsatz}) an}
        n = \pair{1+h_n}^n &= \sum_{k=0}^{n} \underbrace{\binom{n}{k}\pair{h_n}^k}_{\geq 0}\\
        &\geq \binom{n}{0}\cdot \pair{h_n}^0 \binom{n}{1}\cdot \pair{h_n}^1 + \binom{n}{2}\cdot \pair{h_n}^2\tag{$n\geq 2$}\\
        &= 1 + n\cdot h_n + \frac{n\cdot(n-1)}{2}\cdot(h_n)^2\\
        &> \frac{n\cdot (n-1)}{2}\cdot \pair{h_n}^2\\
        \impl h_n^2 &< \frac{2n}{n\cdot (n-1)} = \frac{2}{n-1}\\
        0 &< h_n < \sqrt{\frac{2}{n-1}} \fromto 0 \text{ für } \ntoinf\\
        \impl \lim h_n &= 0\\
        \equivalent \lim n^{\frac{1}{n}} &= 1
    \end{align*}
\end{beispiel}

\begin{definition}[Divergenz gegen unendlich]
    Eine reelle Folge $(x_n)_n$ strebt gegen unendlich, falls
    \begin{align*}
        &\forall k \geq 0~\exists N\in\N\colon x_n \geq k\quad \forall n> N\\[8pt]
        (\equivalent &\forall k \geq 0 \text{ ist } x_n < k \text{ für endlich viele } n\in\N)
    \end{align*}
    Die Folge $(x_n)_n$ strebt gegen $-\infty$, falls $(-x_n)_n$ gegen $\infty$ strebt.
\end{definition}

\begin{notation}
    Wenn die Folge $(x_n)_n$ gegen unendlich strebt, schreiben wir $x_n\fromto\infty$ oder $\lim_{n\fromto\infty} x_n = +\infty$.
\end{notation}

\begin{beispiel}[Monoton wachsende, divergente Folge]
    $x_n=n$ divergiert gegen $\infty$
\end{beispiel}


\begin{satz}[Eigenschaften von Kehrwerten von Folgen]
    \theoremescape
    \begin{enumerate}[label=(\alph*)]
        \item Falls $x_n\fromto \infty$ für $\ntoinf$, so folgt $\frac{1}{x_n}\fromto 0$ für $\ntoinf$.
        \item Ist $(x_n)_n$ eine Nullfolge mit $x_n > 0$ für fast alle $n\in\N$, so ist $\frac{1}{x_n}\fromto \infty$ für $\ntoinf$.\\
        Falls $x_n < 0$ für fast alle $n$, so folgt $\frac{1}{x_n} \fromto -\infty$ für $\ntoinf$.
    \end{enumerate}

    \begin{proof}[Beweis (a)]
        Sei $x_n\fromto \infty$
        \begin{align*}
            \impl \forall \varepsilon > 0~\exists N\in\N\colon x_n &> \frac{1}{\varepsilon}\quad\forall n \geq N\\
            \impl 0 < \frac{1}{x_n} &< \varepsilon\quad \forall n \geq N
        \end{align*}
        das heißt $\frac{1}{x_n}\fromto 0$.
    \end{proof}
\end{satz}

\begin{uebung}
    Weisen Sie den Teil (b) des vorherigen Satzes nach.
\end{uebung}

\begin{beispiel}
    $\frac{n}{2^n}\fromto 0$ für $\ntoinf$ (Sogar $\forall k\in\N\colon \frac{n^k}{2^n} \fromto 0$ für $\ntoinf$)

    \begin{proof}
        Zu zeigen: $x_n = \frac{2^n}{n}\fromto \infty$
        \begin{align*}
            2^n = (1+1)^n &= \sum_{k=0}^{n} \binom{n}{k}\geq \binom{n}{2} = \frac{n\cdot (n-1)}{2}\\
            &\impl x_n = \frac{2^n}{n}\geq \frac{\frac{n\cdot (n-1)}{2}}{n} = \frac{1}{2}\cdot (n-1)\\
            &\impl x_n \fromto \infty\\
            &\impl \frac{1}{x_n}\fromto 0\qedhere
        \end{align*}
    \end{proof}
\end{beispiel}

\begin{beispiel}[Die Eulersche Zahl $e$]
    \begin{align*}
        a_n &\definedas \pair{1+\frac{1}{n}}^n\quad b_n \definedas \pair{1 + \frac{1}{n}}^{n+1}\\
        \impl a_n &< b_n\quad\forall n \in\N
    \end{align*}
    Behauptung: $\forall n\in\N\colon a_n < a_{n+1}$ und $b_n > b_{n+1}$.
    \begin{proof}
        Sei $n\geq 2$

        \begin{align*}
            \frac{a_n}{a_{n-1}} &= \frac{\pair{1+\frac{1}{n}}^n}{\pair{1+\frac{1}{n-1}}^{n-1}} = \pair{1+\frac{1}{n-1}} \cdot \frac{\pair{1+\frac{1}{n}}^n}{\pair{1+\frac{1}{n-1}}^n}\\
            &= \frac{n}{n-1} \cdot \pair{\frac{\frac{n+1}{n}}{\frac{n}{n-1}}}^n = \frac{n}{n-1}\cdot \pair{\frac{(n+1)\cdot (n-1)}{n^2}}^n\\
            &= \frac{n}{n-1}\cdot \pair{\frac{n^2-1}{n^2}}^n = \frac{n}{n-1}\cdot\pair{1-\frac{1}{n^2}}^n\\
            \annot[{&}]{\geq}{\ref{satz:bernoulli-ungleichung}} \frac{n}{n-1}\cdot\pair{1-\frac{n}{n^2}} = \frac{n}{n-1}\cdot\pair{1-\frac{1}{n}} = 1\\
            \impl a_n &> a_{n-1}\quad\forall n\geq 2\\[10pt]
            \frac{b_{n-1}}{b_n} &= \frac{\pair{1+\frac{1}{n-1}}^n}{\pair{1+\frac{1}{n}}^{n+1}} = \pair{1+\frac{1}{n}}^{-1}\cdot\pair{\frac{1+\frac{1}{n-1}}{1+\frac{1}{n}}}^n\\
            &= \frac{n}{n+1}\cdot\pair{\frac{\frac{n}{n-1}}{\frac{n+1}{n}}}^n = \frac{n}{n+1}\cdot\pair{\frac{n^2}{(n+1)\cdot(n-1)}}^n\\
            &= \frac{n}{n+1}\cdot\pair{\frac{n^2-1+1}{n^2-1}}^n = \frac{n}{n+1}\cdot\pair{1+\frac{1}{n^2-1}}^n\\
            &> \frac{n}{n+1}\cdot\pair{1+\frac{1}{n^2}}^n \annot{\geq}{\ref{satz:bernoulli-ungleichung}} \frac{n}{n+1}\cdot\pair{1+\frac{n}{n^2}} = 1\\
            \impl b_n & < b_{n-1}\quad\forall n\in\N\\[10pt]
            \impl a_1 < a_2 < \dots < a_n &< b_n < \dots < b_2 < b_1
            \intertext{Mit dem Satz der monotonen Konvergenz (\ref{satz:monoton-konv}) folgt daraus}
            e &\definedas \lim_{n\fromto\infty} a_n = \lim_{n\fromto\infty} \pair{1+\frac{1}{n}}^n \text{ existiert } \\
            b &\definedas \lim_{n\fromto\infty} b_n \text{ existiert und sogar } b = e
            \intertext{Da $1<\frac{b_n}{a_n} = \frac{\pair{1+\frac{1}{n}}^{n-1}}{\pair{1+\frac{1}{n}}^n} = 1 + \frac{1}{n}\fromto 1$}
            \impl 1 = \lim_{n\fromto\infty} \frac{b_n}{a_n} &= \frac{\lim_{n\fromto\infty} b_n}{\lim_{n\fromto\infty} a_n} = \frac{b}{e}\\[10pt]
            \impl b &= e\qedhere\\[10pt]
            \pair{1+\frac{1}{n}}^n &< e < \pair{1+\frac{1}{n}}^{n+1}
        \end{align*}
    \end{proof}
\end{beispiel}

\newpage

\subsection{Häufungswerte und Teilfolgen}

Wir möchten eine Folge $(a_n)_n$ \anf{massieren}.
\begin{align*}
    a_n &= f(n), \quad f: \N \fromto \R
\end{align*}

\noindent Wir wollen die Folgeglieder umordnen oder auch beliebige weglassen. Wie machen wir das und wie lässt sich das ausdrücken?

\begin{definition}[Umordnung]
    Sei $(a_n)_n$ eine reelle Folge. Eine Umordnung ist gegeben durch eine Bijektion
    \begin{align*}
        \sigma: \N\fromto \N\\
        b_n \definedas a_{\sigma(n)} \tag{Umordnung von $(a_n)_n$}
    \end{align*}
\end{definition}

\begin{definition}[Ausdünnung]
    Es sei $\kappa: \N\fromto\N$ streng monoton wachsend
    \begin{align*}
        b_n \definedas a_{\kappa(n)}\tag{Teilfolge von $(a_n)_n$}
    \end{align*}
\end{definition}

\begin{satz}[Konvergenz von Teilfolgen und Umordnungen]
    \label{satz:konv-teilfolgen-umordnungen}
    Für jede konvergente reelle Folge $(a_n)_n$ konvergiert jede Umordnung und jede Teilfolge gegen den selben Grenzwert.

    \begin{proof}[Beweis\footnotemark.]
        \footnotetext{Nachtrag vom 28. November 2023.}
        Sei $(a_n)_n$ konvergent gegen $a$ und $\kappa: \N\fromto\N$ monoton wachsend ($\kappa(n+1) > \kappa(n)$).\\
        $\impl \kappa(j) \geq j\quad\forall j \in\N$\footnotemark
        \footnotetext{Lässt sich per Induktion nachweisen}
        \begin{align*}
            b_j &\definedas a_{\kappa(j)}
            \intertext{$a_n\fromto a$, das heißt}
            \forall \varepsilon > 0~\exists N\in\N&\colon a-\varepsilon < a_n < a + \varepsilon\quad\forall n \geq N\\
            \impl \forall \varepsilon > 0~\exists N\in\N&\colon a-\varepsilon < a_{\kappa(j)} < a + \varepsilon\quad\forall j \geq N\\[10pt]
            \impl \lim_{j\fromto\infty} a_{\kappa(j)} &= a\quad\text{ d.h. } \lim_{j\fromto\infty} b_j = a
        \end{align*}
        Für Umordnung: Sei $\sigma: \N\fromto\N$ Bijektion.
        \begin{align*}
            b_j &\definedas a_{\sigma(j)}\\
            \intertext{Wir haben $\forall \varepsilon > 0~\exists N\in\N\colon a-\varepsilon < a_n < a+\varepsilon$ und betrachten das Urbild}
            A&\definedas \sigma^{-1}\pair{\set{1,2,3, \dots, N}}\subseteq \N\\
            \intertext{$A$ hat endlich viele Elemente}
            L&\definedas \max\pair{\sigma^{-1}(1), \sigma^{-1}(2), \dots, \sigma^{-1}(N)}\\
            j \geq L &\impl \sigma(j) \geq N\\
            &\impl \forall j\geq L\colon a-\varepsilon < a_{\sigma(j)} < a + \varepsilon\\
            &\impl \lim_{j\fromto\infty} a_{\sigma(j)} = a\qedhere
        \end{align*}
    \end{proof}
\end{satz}

\begin{uebung}
    Weisen Sie nach, dass sich der Grenzwert einer Folge nicht verändert, wenn man endlich viele Elemente ändert.
\end{uebung}

\begin{definition}[Häufungswert]
    Sei $(a_n)_n$ reelle Folge. Eine reelle Zahl $a$ heißt Häufungswert (oder Häufungspunkt) von $a_n$, falls $\forall \varepsilon >0$ unendlich viele $a_n$ in $\pair{a-\varepsilon, a+\varepsilon}$ liegen. Das heißt:
    \begin{align*}
        a-\varepsilon < a_n < a+ \varepsilon\text{ für unendlich viele $n$}
    \end{align*}
    Das heißt $\forall L\in\N~\exists n> L\colon a-\varepsilon < a_n < a+\varepsilon$.
\end{definition}

\begin{beispiel}
    \theoremescape
    \begin{enumerate}
        \item $a_n=\frac{1}{n}$ hat Häufungswert 0.
        \item $a_n=(-1)^n$ hat Häufungswerte $1$, $-1$.
        \item $a_n=n$ hat keinen Häufungswert.
        \item $a_n=(-1)^n+\frac{1}{n}$ hat Häufungswerte 1, -1.
    \end{enumerate}
\end{beispiel}

\begin{satz}[Häufungswertkriterium über Teilfolgen]
    \label{satz:haeufungswert-teilfolge}
    Eine reelle Zahl $a$ ist genau dann Häufungswert einer Folge $(a_n)_n$, wenn eine Teilfolge von $a_n$ existiert, die gegen $a$ konvergiert.

    \begin{proof}
        \anf{$\impl$}: $a$ sei Häufungswert von $(a_n)_n$. Das heißt
        \begin{align*}
            \forall \varepsilon > 0~\forall L\in\N~\exists n>L\colon a-\varepsilon < a_n < a+\varepsilon
        \end{align*}
        \begin{enumerate}[label=\arabic*)]
            \item Wir wählen $\varepsilon = 1 \impl \exists n_1\in\N\colon a-1 < a_{n_1} < a+1$
            \item Wir wählen $\varepsilon = \frac{1}{2}$, $L=n_1 + 1 \impl \exists n_2 > L > n_1\colon a-\frac{1}{2} < a_{n_2} < a + \frac{1}{2}$
            \item Wir wählen $\varepsilon = \frac{1}{3} \impl \exists n_3 > n_2\colon a - \frac{1}{3} < a_{n_3} < a + \frac{1}{3}$
            \item[$j$)] Wir wählen $\varepsilon = \frac{1}{j+1} \impl \exists n_{j+1} > n_j\colon a-\frac{1}{j+1} < a_{n_{j+1}} < a + \frac{1}{j+1}\qquad \impl n_j < n_{j+1}\quad\forall j \in \N$
        \end{enumerate}
        \noindent Wir definieren $\kappa\pair{j} \definedas n_j$ und $b_j \definedas a_{\kappa(j)}$ als eine Teilfolge von $(a_n)_n$. Es gilt
        \begin{align*}
            a-\frac{1}{j} < b_j < a + \frac{1}{j}\quad\forall j\in\N
        \end{align*}
        und nach Satz~\ref{satz:sandwich} konvergiert $(b_j)_j$ gegen $a$.\qedhere
    \end{proof}
    \begin{uebung}
        Beweisen Sie mittels Konvergenzkriterien und der Definition von Häufungswerten die Rückrichtung des vorherigen Satzes.
    \end{uebung}
\end{satz}

\newpage

%%%%%%%%%%%%%%%%%%%%%%%%
% 28. November 2023
%%%%%%%%%%%%%%%%%%%%%%%%

\subsection{Größter und kleinster Häufungswert - Limes superior und Limes inferior}
\marginnote{[28. Nov]}

Es sei $(a_n)_{n}$ eine beschränkte reelle Folge.
\begin{align*}
    x_n \definedas \sup_{l\geq n} a_l
\end{align*}
ist monoton fallend, weil
\begin{align*}
    \sup_{l\geq n} a_l = \max(a_n, \sup_{l\geq n+1} a_l) \geq \sup_{l\geq n+1} a_l = x_{n+1}
\end{align*}
$(x_n)_{n}$ ist monoton fallend und nach unten beschränkt.
\begin{align*}
    \annot{\impl}{\ref{satz:monoton-konv}} x = \lim_{\ntoinf} x_{n} = \lim_{n\fromto\infty} \sup_{l\geq n} a_l = \inf \sup_{l\geq n} a_l
\end{align*}
\noindent existiert.\\[10pt]
Genauso: $y_n\definedas \inf_{l\geq n} a_l \impl y_n < y_{n+1}$ und $y_n$ ist nach oben beschränkt und damit existiert:
\begin{align*}
    y = \lim_{\ntoinf} y_n = \lim_{n\fromto\infty} \inf_{l\geq n} a_l = \sup \inf_{l\geq n} a_j
\end{align*}

\begin{definition}[Limes superior und inferior] % Definition 1
    Sei $(a_n)_{n}$ eine beschränkte reelle Folge.
    \begin{align*}
        \limsup_{n\fromto\infty} a_n \definedas \lim_{n\fromto\infty} \sup_{l\geq n} a_l = \inf \sup_{l\geq n} a_l \tag{Limes superior}\\
        \liminf_{n\fromto\infty} a_n \definedas \lim_{n\fromto\infty} \inf_{l\geq n} a_l = \sup \inf_{l\geq n} a_l \tag{Limes inferior}
    \end{align*}
    Damit gilt außerdem
    \begin{align*}
        y_n < x_n\quad \inf_{l\geq n} a_l &\leq \sup_{l\geq n} a_l\\
        \impl \liminf_{n\fromto\infty} a_n &\leq \limsup_{n\fromto\infty} a_n
        \intertext{und}
        \limsup_{n\fromto\infty} \pair{-a_n} &= -\liminf_{n\fromto\infty} a_n\\
        \liminf_{n\fromto\infty} \pair{-a_n} &= -\limsup_{n\fromto\infty} a_n
    \end{align*}
\end{definition}

\begin{beispiel}
    \begin{align*}
        a_n = (-1)^n \impl \liminf_{n\fromto\infty} a_n &= -1\\
        \limsup_{n\fromto\infty} a_n &= 1\\[10pt]
        \limsup_{n\fromto\infty} (-a_n) = - \liminf_{n\fromto\infty} a_n\\
        \liminf_{n\fromto\infty} (-a_n) = -\limsup_{n\fromto\infty} a_n
    \end{align*}
\end{beispiel}
\newpage

\begin{lemma}[Charakterisierung von $\limsup$ und $\liminf$] % Lemma 2
    \label{lemma:limsup-charak}
    Sei $(a_n)_{n}$ beschränkte reelle Folge. Dann gilt:
    \begin{align*}
        a^{*} = \limsup a_n\quad\equivalent\quad \forall \varepsilon > 0~
        &
        \begin{array}{l}
            \text{ist } a_n < a^{*} + \varepsilon\text{ für fast alle $n$} \\
            \text{und } a_n > a^{*} -\varepsilon\text{ für unendlich viele $n$}
        \end{array}
        \\[10pt]
        a_{*} = \liminf a_n\quad\equivalent\quad \forall \varepsilon > 0~
        &
        \begin{array}{l}
            \text{ist } a_n > a_{*} - \varepsilon\text{ für fast alle $n$} \\
            \text{und } a_n < a_{*} + \varepsilon\text{ für unendlich viele $n$}
        \end{array}
    \end{align*}
    \begin{proof}[Beweis für ersten Teil des Lemmas, zweiter analog.]
        \anf{$\impl$}:
        \begin{align*}
            a^{*} = \limsup_{n\fromto\infty} a_n &= \inf_{n\in\N} \sup_{l\geq n} a_l\\
            \intertext{Angenommen $\exists \varepsilon > 0\colon a_l \geq a^{*} +\varepsilon$ für unendlich viele $l$}
            \impl \forall n\in\N\colon \sup_{l\geq n} a_l &\geq a^{*} + \varepsilon\\
            \impl a^{*} = \inf_{n\in\N} \sup_{l\geq n} a_l &\geq a^{*} + \varepsilon\quad\text{ (Widerspruch)}
            \intertext{Angenommen $\exists \varepsilon_0 > 0\colon a_n \leq a^{*} -\varepsilon_0$ für fast alle $n$}
            \impl \sup_{l\geq n} a_l &\leq a^{*} - \varepsilon_0\\
            \impl a^{*} = \lim_{n\fromto\infty} \sup_{l\geq n} a_l &\leq a^{*} - \varepsilon_0\quad\text{(Widerspruch)}
        \end{align*}
        \anf{$\Leftarrow$}: Sei $a^{*}\in\R$
        \begin{align*}
            \forall \varepsilon > 0\colon a_n &< a^{*} + \varepsilon\text{ für fast alle $n$}\\
            a_n &> a^{*} - \varepsilon\text{ für unendlich viele $n$}\\
            \impl \exists k\in\N\colon a_l &< a^{*} + \varepsilon\quad\forall l\geq k\\[10pt]
            \impl \forall n\geq k\colon \sup_{l\geq n} a_l &\leq a^{*} + \varepsilon\\
            \sup_{l\geq n} a_l &> a^{*} - \varepsilon\\
            \impl \forall n \geq k\colon a^{*} - \varepsilon &< \sup_{l\geq n} a_l \leq a^{*} + \varepsilon\\
            \impl a^{*} - \varepsilon \leq \lim_{n\fromto\infty} \sup_{l\geq n} a_l &\leq a^{*} + \varepsilon\quad \forall \varepsilon > 0\\
            \impl \limsup_{n\fromto\infty} a_n &= a^{*}\qedhere
        \end{align*}
    \end{proof}
\end{lemma}

\begin{satz}[Eigenschaften von $\limsup$ und $\liminf$] % Satz 3
    \label{satz:limsup-haeufungspunkt}
    Sei $(a_n)_n$ eine beschränkte reelle Folge und $H(a_n)$ die Menge der Häufungspunkte von $a_n$. Dann gilt
    \begin{align*}
        \limsup a_n, \liminf a_n \in H(a_n)\tag{1}
    \end{align*}
    Insbesondere ist $H(a_n) \neq \emptyset$. Ferner ist
    \begin{align*}
        \forall x \in H(a_n)\colon \liminf a_n \leq x \leq \limsup a_{n}\tag{2}
    \end{align*}

    \begin{proof}[Beweis von (1)]
        \begin{align*}
            a^{*} &\definedas\limsup a_n\\
            \annot{\impl}{\ref{lemma:limsup-charak}} \forall\varepsilon > 0\colon a_n &< a^* + \varepsilon\text{ für fast alle } n\\
            a_n &> a^*-\varepsilon\text{ für unendlich viele }n\\[10pt]
            \impl \forall\varepsilon > 0\colon a^* - \varepsilon &< a_n < a^* + \varepsilon\text{ für unendlich viele } n\\
            \impl &a^*\text{ ist Häufungswert von } (a_n)_n\qedhere
        \end{align*}
    \end{proof}
    \noindent Analog lässt sich der Beweis auch für $\liminf a_n$ führen.

    \begin{proof}[Beweis von (2)]
        Sei $a$ Häufungswert von $(a_n)_n$.
        Annahme: $a>a^*\definedas\limsup a_n$:
        \begin{align*}
            &\impl a-\varepsilon < a_n < a + \varepsilon\text{ für unendlich viele } n
            \intertext{Wähle $\varepsilon = \frac{a-a^*}{2}$}
            &\impl a - \frac{a-a^*}{2} < a_n < a + \frac{a-a^*}{2}\\
            &\impl a_n > a - \frac{a-a^*}{2} = \frac{a+a^*}{2} = a^* + \varepsilon\text{ für unendlich viele }n
            \intertext{Widerspruch zu Lemma~\ref{lemma:limsup-charak}}
            &\impl a \leq \limsup a_n\qedhere
        \end{align*}
    \end{proof}
    \noindent Mit $-a_n$ lässt sich analog zeigen, dass $a\geq \liminf a_n\quad\forall$ Häufungspunkte $a$ von $(a_n)_n$.
\end{satz}

\begin{notation}[Limes superior/inferior von unbeschränkten Folgen]
    Es sei $(a_n)_n$ nicht nach oben beschränkt. Dann setzten wir
    \begin{align*}
        \limsup a_n &\definedas +\infty
    \end{align*}
    Ist $(a_n)_n$ nicht nach unten beschränkt. Dann setzen wir
    \begin{align*}
        \liminf a_n &\definedas -\infty
    \end{align*}
\end{notation}

\begin{korollar}[Konvergenzkriterium nach limes superior und limes inferior]
    Eine reelle Folge $(a_n)_n$ konvergiert genau dann, wenn $(a_n)_n$ beschränkt ist und $\limsup a_n = \liminf a_n$.
    \begin{proof}
        \anf{$\impl$}:\\
        Jede konvergente Folge ist beschränkt. $(a_n)_n$ konvergiert gegen $a$ genau dann, wenn $H(a_n) = \set{a}$\\
        $\impl \liminf a_n = a = \limsup a_n$\\[10pt]
        \anf{$\Leftarrow$}: Sei $\liminf a_n = \limsup a_n = a$
        \begin{align*}
            \annot{\impl}{\ref{lemma:limsup-charak}} \forall \varepsilon > 0\colon a_n &< a+\varepsilon \text{ für fast alle $n$}\\
            \text{ und } a_n &> a-\varepsilon \text{ für fast alle } n\\
            \impl \lim a_n &= a\qedhere
        \end{align*}
    \end{proof}
\end{korollar}

\begin{uebung}
    Zeigen Sie: Wenn $\liminf$ und $\limsup$ als reelle Zahlen existieren, dann ist die Folge beschränkt.
\end{uebung}

\begin{satz}[Satz von Bolzano-Weierstraß] % Satz 5
    \label{satz:bolzano-weierstrass}
    Jede beschränkte reelle Folge $(a_n)_n$ besitzt mindestens einen Häufungspunkt.

    \begin{proof}
        $a^* \definedas \limsup a_n$ ist ein Häufungspunkt von $(a_n)_n$ nach Satz~\ref{satz:limsup-haeufungspunkt}.
    \end{proof}
\end{satz}

\begin{korollar}
    \label{korollar:beschr-konv-teilfolge}
    Jede beschränkte reelle Folge $(a_n)_n$ hat eine konvergente Teilfolge.
    \begin{proof}
        Nach Satz~\ref{satz:bolzano-weierstrass} ist $H(a_n) \neq\emptyset$ und nach Satz~\ref{satz:haeufungswert-teilfolge} gibt es zu jedem Häufungspunkt eine konvergente Teilfolge.
    \end{proof}
\end{korollar}

\vfill

\subsection{Das Konvergenzkriterium von Cauchy}

\begin{definition}[Cauchy-Folge]
    Eine reelle Folge $(a_n)_n$ heißt Cauchy-Folge, falls
    \begin{align*}
        &\forall\varepsilon > 0~\exists N\in\N\colon \abs{a_n - a_m} < \varepsilon\quad\forall n,m \geq N\\
        (\equivalent &\forall \varepsilon > 0~\exists N\in\N\colon \abs{a_n - a_m} < \varepsilon\quad\forall n\geq m\geq N)
    \end{align*}
\end{definition}

\begin{lemma} % Lemma 2
    \label{lemma:konv-cauchy}
    Jede konvergente reelle Folge $(a_n)_n$ ist eine Cauchy-Folge.
    \begin{proof}
        Es sei $(a_n)_n\fromto a$ eine reelle Folge.
        \begin{align*}
            \forall \varepsilon > 0~\exists N\in\N\colon &\abs{a_n-a} < \frac{\varepsilon}{2}\quad\forall n\geq N\\
            \impl\text{ Sei }n,m\geq N \impl &\abs{a_n-a_m} = \abs{a_n-a+a-a_m}\\
            \leq &\abs{a_n-a} + \abs{a-a_m} < \frac{\varepsilon}{2} + \frac{\varepsilon}{2} = \varepsilon\qedhere
        \end{align*}
    \end{proof}
\end{lemma}

\begin{lemma} % Lemma 3
    \label{lemma:beschr-cauchy}
    Jede reelle Cauchy-Folge ist beschränkt.
    \begin{proof}
        Es sei $\varepsilon = 1$
        \begin{align*}
            \impl \exists N\colon \abs{a_n-a_m} &< 1\quad \forall n,m\geq N\\
            \impl \forall n,m\geq N\colon\abs{a_n} &= \abs{a_n-a_m+a_m}\\
            &\leq \abs{a_n-a_m} + \abs{a_m}\\
            &< 1 + \abs{a_m}
            \intertext{Da wir $m=N$ wählen können, gilt}
            \impl \forall n\geq N\colon \abs{a_n} &< 1 + \abs{a_N}
            \intertext{Es gibt eine Schranke ab dem $N$-ten Folgenglied und damit gilt}
            \impl \abs{a_n} &\leq \max\pair{\abs{a_1}, \abs{a_2},\dots, \abs{a_{N-1}}, 1+\abs{a_N}}\qedhere
        \end{align*}
    \end{proof}
\end{lemma}

\vfill

\newpage

\begin{lemma} % Lemma 4
    \label{lemma:cauchy-konv-teilfolge}
    Eine reelle Cauchy-Folge $(a_n)_n$ konvergiert genau dann, wenn sie eine konvergente Teilfolge hat.

    %%%%%%%%%%%%%%%%%%%%%%%%
    % 30. November 2023
    %%%%%%%%%%%%%%%%%%%%%%%%

    \begin{proof}
        \marginnote{[30. Nov]}
        ~\\
        \anf{$\impl$}: Klar, weil nach Satz~\ref{satz:konv-teilfolgen-umordnungen} jede Teilfolge einer konvergenten Folge konvergiert.\\[10pt]
        \anf{$\Leftarrow$}: Sei $(b_j)_j$ eine Teilfolge von $(a_n)_n$ mit $b_j = a_{n_j}$ und $n_1 < n_2 < \dots < n_j < n_{j+1} < \dots$\\
        Es sei $a\definedas \lim_{j\fromto\infty} b_j$. Behauptung: $\lim_{n\fromto\infty} a_n = a$\\[10pt]
        \noindent Wir wissen:
        \begin{align*}
            \forall\varepsilon>0~\exists N_1 \in\N&\colon \abs{a-a_{n_j}} < \frac{\varepsilon}{2}\quad\forall j\geq N_1\\
            \exists N_2 \in\N&\colon \abs{a_n-a_m} < \frac{\varepsilon}{2}\quad\forall m\geq n\geq N_2\\[10pt]
            \impl \abs{a_n-a} &= \abs{a_n-a_{n_j} + a_{n_j} - a}\\
            &\leq \abs{a_n-a_{n_j}} + \abs{a_{n_j}-a}
            \intertext{Wir wählen $j\geq\max\pair{N_1,N_2}$}
            \impl \abs{a_n-a} &\leq \abs{a_n-a_{n_j}} + \abs{a_{n_j}-a}\\
            &<\frac{\varepsilon}{2} + \frac{\varepsilon}{2} = \varepsilon \quad\forall n\geq N\\
            \impl \lim_{n\fromto\infty} a_n &= a\qedhere
        \end{align*}
    \end{proof}
\end{lemma}

\begin{satz} % Satz 5
    \label{satz:jede-konv-cauchy}
    Jede reelle Folge konvergiert genau dann, wenn sie eine Cauchy-Folge ist.

    \begin{proof}
        \anf{$\impl$}: Folgt direkt aus Lemma~\ref{lemma:konv-cauchy}\\[10pt]
        \anf{$\Leftarrow$}: Sei $(a_n)_{n\in\N}$ eine Cauchy-Folge $\annot{\impl}{Lemma~\ref{lemma:beschr-cauchy}} (a_n)_n$ ist beschränkt $\annot{\impl}{Satz~\ref{korollar:beschr-konv-teilfolge}} (a_n)_n$ hat eine konvergente Teilfolge $\annot{\impl}{Lemma~\ref{lemma:cauchy-konv-teilfolge}} (a_n)_n$ ist konvergent
    \end{proof}
\end{satz}

\newpage


    \section{Dichtheit von $\Q$ in $\R$}
    \subsection{Dichtheit im Allgemeinen}
\thispagestyle{pagenumberonly}

Wir kennen bereits folgende Mengen:

\begin{align*}
    \Q&\definedas\set{\frac{m}{n}\middle|~m\in\Z \land n\in\N}\tag{Rationale Zahlen}\\
    \Q&\subseteq \R\quad \Q\neq\R\text{, da } \sqrt{2}\notin \Q\\
    \R&\exclude\Q \tag{Irrationale Zahlen}
\end{align*}

\begin{bemerkung}[Größenvergleich der irrationalen und rationalen Zahlen]
    $\R\exclude\Q$ ist sehr viel größer als $\Q$. (Wird später noch behandelt)
\end{bemerkung}

\begin{definition}[Dichte Teilmenge] % Definition 1
    Sei $A\subseteq B (\subseteq \R)$. $A$ heißt dicht in $B$, falls
    \begin{align*}
        \forall b\in B~\exists (a_n)_{n\in\N}, (a_n \in A~\forall n\in\N)\text{ mit }\lim_{n\fromto\infty} a_n = b
    \end{align*}
    Das heißt für jedes $b\in B$ existiert eine Folge in $A$, die gegen $b$ konvergiert.
\end{definition}

\begin{notation}
    Statt $\forall n\in\N\colon a_n \in A$ schreiben wir auch $(a_n)_n\subseteq A$.
\end{notation}

\subsection{Dichtheits-Begriff für rationale und reelle Zahlen}
Wir wollen nun zeigen, dass $\Q$ dicht in $\R$ ist. Ziel:
\begin{align*}
    \forall x\in\R~\exists (a_n)_{n\in\N}, a_n\in\Q\text{ mit }\lim_{n\fromto\infty} a_n =x
\end{align*}

\begin{lemma}[Zwischenwerte von reellen Zahlen] % Lemma 2
    \label{lemma:zwisch-reelle-zahlen}
    \theoremescape
    \begin{enumerate}[label=(\roman*)]
        \item $\forall x,y\in\R$ mit $y-x>1~\exists m\in\Z$ mit $x < m < y$
        \item $\forall x,y\in\R$ mit $y>x~\exists q\in\Q\colon x<q<y$
    \end{enumerate}
    \begin{proof}
        \theoremescape
        \begin{enumerate}[label=(\roman*)]
            \item Sei $y>x+1$. Wir definieren
            \begin{align*}
                A&\definedas\set{p\in\Z|~p>x} \subseteq \Z
                \intertext{$A$ ist nach unten beschränkt und nach Satz~\ref{satz:von-archimedes} gilt $A\neq \emptyset$. Nach Satz~\ref{satz:wohlordnungsprinzip} erweitert auf $\Z$ folgt}
                \impl m&\definedas\min(A)\text{ existiert,}\quad m\in A\\
                \impl m-1&\notin A,\quad m-1\in \Z\\
                \impl x &< m,\quad m-1 \leq x\\
                \impl m&\leq x+1< x+(y-x) = y\\
                \impl x&<m<y
            \end{align*}
            \item Sei $y>x\equivalent y-x>0$
            \begin{align*}
                \annot{\impl}{\ref{satz:von-archimedes}} \exists n\in\N\colon n\cdot\pair{y-x}&>1\\
                \annot{\impl}{(i)} \exists m\in \Z\colon nx &< m < ny\\
                \equivalent x &< \frac{m}{n}< y\\
                \text{Wähle }q&=\frac{m}{n}\qedhere
            \end{align*}
        \end{enumerate}
    \end{proof}
\end{lemma}

\begin{folgerung}[Dichtheit von $\Q$ in $\R$]
    Anwendung: $\forall x\in\R~\exists (a_n)_n\in\Q$ mit $q_n\fromto x$
    \begin{proof}
        Für $n\in\N$ wähle $y=x+\frac{1}{n}$
        \begin{align*}
            \annot{\impl}{Lemma~\ref{lemma:zwisch-reelle-zahlen}} \exists q_n\in\Q\colon x &< q_n < y = x + \frac{1}{n}\\
            x &< q_n < x + \frac{1}{n}\quad (\fromto x)
            \intertext{Nach dem Sandwich-Satz~(\ref{satz:sandwich}) gilt}
            \impl q_n &\fromto x\qedhere
        \end{align*}
    \end{proof}
\end{folgerung}

\begin{lemma} % Lemma 3
    $\forall x\in\R~\forall n\in\N~\exists m_n\in\Z\colon$
    \begin{align*}
        \frac{m_n-1}{n} \leq x < \frac{m_n}{n}
    \end{align*}
    \begin{uebung}
        Beweisen Sie das vorherige Lemma.\\
        \textit{Hinweis}: Betrachte $A_n\definedas\set{p\in\Z|~p>nx}\subseteq\Z$. Behauptung: $m_n\definedas \min A_n$ ermöglicht den Beweis.
    \end{uebung}
\end{lemma}

\begin{bemerkung}[Definition von irrationalen Exponenten über Folgen. Siehe Walter: Analysis 1, Kapitel 3.8 und 4.8]
    Sei $a > 0$
    \begin{align*}
        \impl\forall n\in\N\colon a^{\frac{1}{n}} = \sqrt[n]{a}\text{ existiert}
        \intertext{Da $a^n \definedas \prod_{j=1}^n a\quad a^{-n} \definedas \frac{1}{a^n}\quad a^0 \definedas 1$ für $n\in\N$ gilt}
        a^m\text{ definiert }\quad\forall m\in\Z
    \end{align*}
    Definiere: $m\in\Z$, $n\in\N$, $a^\frac{m}{n}\definedas \pair{a^\frac{1}{m}}^m$\\
    Überprüfe Wohldefiniertheit. Das heißt ist $q=\frac{m_1}{n_1} = \frac{m_2}{n_2}$ muss gelten
    \begin{align*}
        \pair{a^\frac{1}{n}}^{m_2} = a^\frac{m_1}{n_1} = a^\frac{m_2}{n_2} = \pair{a^\frac{1}{n_2}}^{m_2}
    \end{align*}
    Wir haben also eine Definition für rationale Exponenten. Um auch irrationale Exponenten abbilden zu können, gehen wir wie folgt vor. Für $x\in\R$ wähle $(q_n)_n\subseteq\Q$, $q_n\fromto x$ und definiere
    \begin{align*}
        a^x \definedas\lim_{n\fromto\infty} a^{q_n}
    \end{align*}
    Überprüfe: $a^x\cdot a^y = a^{x+y}$, $a^{x}b^x = (ab)^x$, $a^{-x} = \frac{1}{a^x}$ usw.
\end{bemerkung}

\newpage


    \section{[*] Reihen (und Konvergenz von Reihen)}
    \thispagestyle{pagenumberonly}

    Bedeutung von endlichen Summen ist klar.\\
    Frage: Gegeben eine reelle Folge $a_n$. Was ist $\pair{a_1 + a_2 + a_3 + \dots = \sum_{n=1}^{\infty} a_n}$?

    \subsection{[*] Konvergenz-Kriterien für Reihen}

    \begin{definition}[Reihen als Partialsummen] % Definition 1
        Das Symbol
        \begin{align*}
            \sum_{n=1}^{\infty} a_n\tag{Sei $a_n$ eine reelle Folge}
        \end{align*}
        wird folgendermaßen verwendet:

        \begin{enumerate}[label=\alph*)]
            \item Es steht für die Folge der Partialsummen:
            \begin{align*}
                s_n&\definedas \sum_{j=1}^{n} a_j\quad\forall n\in\N
            \end{align*}
            \item Die Reihe $\sum_{n=1}^{\infty} a_n$ konvergiert, falls der Grenzwert der Partialsummen $\lim_{n\fromto\infty} s_n$ existiert.\\
            Wir setzen
            \begin{align*}
                \sum_{n=1}^{\infty} a_n &\definedas \lim_{n\fromto\infty} s_n
            \end{align*}
            \item Konvergiert $(s_n)_n$ nicht, so heißt $\sum_{n=1}^{\infty} a_n$ divergent. Falls $s_n$ bestimmt divergiert so setzen wir
            \begin{align*}
                \sum_{n=1}^{\infty} a_n &\definedas \infty \tag{Wenn $\lim_{n\fromto\infty} s_n = \infty$}\\
                \sum_{n=1}^{\infty} a_n &\definedas -\infty \tag{Wenn $\lim_{n\fromto\infty} s_n = -\infty$}
            \end{align*}
        \end{enumerate}
    \end{definition}

    \begin{satz}[Monotone Konvergenz für Reihen] % Satz 2
        \label{satz:mont-konv-reihen}
        Sei $a_n$ eine reelle Folge mit $\forall n\colon a_n\geq 0$. Dann konvergiert die Reihe $\sum_{j=1}^{\infty} a_j$ genau dann, wenn die Folge der Partialsummen $s_n$ nach oben beschränkt ist.

        \begin{proof}
            Betrachte
            \begin{align*}
                s_{n+1} &= \sum_{j=1}^{n+1} a_j = \sum_{j=1}^n a_j + a_{n+1} \geq \sum_{j=1}^{n} a_j = s_n
            \end{align*}
            und wende Satz~\ref{satz:monoton-konv} an.
        \end{proof}
    \end{satz}

    \begin{korollar} % Korollar 3
        Für eine Reihe $\sum_{n=1}^{\infty} a_n$ mit $a_n\geq 0$ gilt entweder
        \begin{align*}
            \sum_{n=1}^{\infty} a_n < \infty\quad\text{oder}\quad\sum_{n=1}^{\infty} a_n = \infty
        \end{align*}

        \begin{proof}
            Folgt direkt aus Satz~\ref{satz:mont-konv-reihen}.
        \end{proof}
    \end{korollar}

    \begin{bemerkung}
        Oft hat man Reihen der Form
        \begin{align*}
            \sum_{n=0}^{\infty} a_n &= a_0+a_1+a_2+\dots\\
            s_n&\definedas \sum_{j=0}^{n} a_j \tag{$n\in\N_0$}
            \intertext{oder}
            s_n &\definedas a_0 + \sum_{j=1}^{\infty} a_j\tag{$n\in\N$}
            \intertext{Sofern ein Limes existiert, gilt dann}
            \sum_{n=0}^{\infty} a_n &\definedas \lim s_n\\[10pt]
            \intertext{Allgemein für $v\in\Z\quad a_v, a_{v+1}, a_{v+2}, \dots$}
            \sum_{n=v}^{\infty} a_n &= a_v + a_{v+1} + \dots\\
            s_n &\definedas \sum_{j=v}^{n} a_j\text{ def. Folge $(s_n)_{n\geq v}$}
        \end{align*}
    \end{bemerkung}

    \begin{beispiel}[Geometrische Folge und Reihe]
        \footnote{Wir setzen $0^0 = 1$}
        \begin{align*}
            q\neq 1 \impl \sum_{j=0}^{n} q^j = \frac{1-q^{n+1}}{1-q}\quad\forall n\in\N_0\\
            \text{Ist } \abs{q} < 1\colon \sum_{n=0}^{\infty} q^n\text{ konvergiert und } \sum_{n=0}^{\infty} q^n = \frac{1}{1-q}\tag{geometrische Reihe}
        \end{align*}
        Zum Beispiel $q=\frac{1}{2}$
        \begin{align*}
            \sum_{j=0}^{n} q^j &= \frac{1-q^{n+1}}{1-q}\\
            \equivalent \pair{1-q}\cdot \sum_{j=0}^{n} q^j &= 1-q^{n+1}\\
            \frac{1}{2}\sum_{j=0}^{n} \pair{\frac{1}{2}}^j &= 1 - \pair{\frac{1}{2}}^{n+1}\\
            \impl 1 - \pair{\frac{1}{2}}^{n+1} &= \sum_{j=0}^{n} \frac{1}{2}\cdot\pair{\frac{1}{2}}^j\\
            &= \sum_{j=0}^{n} \pair{\frac{1}{2}}^{j+1} = \sum_{j=1}^{n+1} \pair{\frac{1}{2}}^j
        \end{align*}
        Das heißt es sollte gelten
        \begin{align*}
            \sum_{j=1}^{n} \pair{\frac{1}{2}}^j &= 1-\pair{\frac{1}{2}}^n\quad \forall n\in\N
        \end{align*}

        \noindent Dass dieser Zusammenhang gelten muss, lässt sich einfach veranschaulichen. Die linke Seite der Gleichung kann als Summe über Teilflächen des Einheitsquadrats\footnote{Original: \anf{Kuchen}} visualisiert werden.\\
        Erst wird eine Hälfte, dann ein Viertel, dann ein Achtel (usw.) des Quadrats hinzugefügt. Der zurückbleibende Flächeninhalt ist immer genauso groß wie das zuletzt hinzugefügte Stück. Dieser Term wird durch den rechten Teil der Gleichung beschrieben.

        \begin{proof}[Beweis der Reihenformel]
            \begin{align*}
                s_n &\definedas \sum_{j=0}^{n} q^n\\
                q\cdot s_n &= q\cdot \sum_{j=0}^{n} q^j = \sum_{j=0}^{n} q^{j+1}\\
                &= \sum_{j=1}^{n+1} q^j\tag{Indexshift}\\
                \impl (1-q)\cdot s_n = s_n - q\cdot s_n &= \sum_{j=0}^{n}  q^j - \sum_{j=1}^{n+1} q^j\tag{Reißverschlusssumme}\\
                = q^0 - q^{n+1} &= 1 - q^{n+1}\\
                \impl s_n &= \frac{1-q^{n+1}}{1-q}\qedhere
            \end{align*}
        \end{proof}

        %%%%%%%%%%%%%%%%%%%%%%%%
        % 05. Dezember 2023
        %%%%%%%%%%%%%%%%%%%%%%%%

        \begin{proof}[Beweis der Konvergenz für $\abs{q} < 1$]
            \marginnote{[5. Dez]}
            \begin{align*}
                s_n\definedas \sum_{j=0}^{n} q^j &= \frac{1-q^{n+1}}{1-q}\\
                \lim_{n\fromto\infty} q^n &= 0 = \lim_{n\fromto\infty} q^{n+1}\tag{Weil $\abs{q} < 1$}\\
                \annot{\impl}{\ref{satz:konvergenzsaetze}} \lim s_n &= \frac{1-\lim_{n\fromto\infty} q^{n+1}}{1-q} = \frac{1-0}{1-q} = \frac{1}{1-q}\qedhere
            \end{align*}
        \end{proof}
        \begin{bemerkung}
            Ist $\abs{q}\geq 1$, dann ist $1+\underbrace{q}_{\geq 1}+\underbrace{q^2}_{\geq 1}+\underbrace{q^3}_{\geq 1}+\dots + \underbrace{q^n}_{\geq 1} \geq 1+n\fromto\infty$.
        \end{bemerkung}
    \end{beispiel}

    % TODO: Wahrscheinlichkeitstheoretische Anschauung
    \newpage

    \begin{beispiel}[Harmonische Reihe]
        \begin{align*}
            s_n &\definedas \sum_{j=1}^{n} \frac{1}{j}
        \end{align*}
        $(s_n)_n$ ist monoton wachsend, aber nicht nach oben beschränkt. Das heißt $\lim_{n\fromto\infty} s_n = \infty \impl \sum_{n=1}^{\infty} \frac{1}{n} = \infty$.

        \begin{proof}
            \begin{align*}
                s_{2n} - s_n &= \sum_{j=n+1}^{2n} \frac{1}{j} \geq \sum_{j=n+1}^{n} \frac{1}{2n} = n \cdot\frac{1}{2n} = \frac{1}{2}\\
                \impl s_2 - s_1 &\geq \frac{1}{2}\\
                \impl s_2 &\geq s_1 + \frac{1}{2} = 1 + \frac{1}{2} > \frac{1}{2}\\
                s_4 - s_2 &\geq \frac{1}{2}\\
                \impl s_4 &\geq s_2 + \frac{1}{2} > \frac{1}{2} + \frac{1}{2} = 1\\
                s_8 - s_4 &\geq \frac{1}{2}\\
                \impl s_8 &\geq s_4 + \frac{1}{2} \geq \frac{3}{2}\\
                \annot{\impl}{Induktion} s_{\pair{2^j}} &> \frac{j}{2}\quad\forall j\in\N
            \end{align*}
            \noindent Also ist $s_{\pair{2^j}}$ nicht nach oben beschränkt $\impl$ $(s_n)_n$ nicht nach oben beschränkt.
            \begin{align*}
                \annot{\impl}{\ref{satz:monoton-konv}} \sum_{n=1}^{\infty} \frac{1}{n} = \lim_{n\fromto\infty} s_n &= +\infty\qedhere
            \end{align*}
        \end{proof}
    \end{beispiel}

    \begin{satz} % Satz 6
        Seien $\sum_{n=1}^{\infty} a_n$, $\sum_{n=1}^{\infty} b_n$ konvergente Reihen. Dann ist
        \begin{align*}
            \forall\lambda, \mu \in\R\colon \sum_{n=1}^{\infty} \pair{\lambda\cdot a_n + \mu\cdot b_n}
        \end{align*}
        konvergent und es gilt
        \begin{align*}
            \sum_{n=1}^{\infty} \pair{\lambda\cdot a_n + \mu\cdot b_n} = \lambda \cdot \sum_{n=1}^{\infty} a_n + \mu\cdot \sum_{n=1}^{\infty} b_n
        \end{align*}
        \begin{proof}
            \begin{align*}
                s_n &\definedas \sum_{j=1}^{n} a_n \quad t_n \definedas \sum_{j=1}^{n} b_n\\
                d_n &\definedas \sum_{j=1}^{n} \pair{\lambda\cdot a_n + \mu\cdot b_n} =   \lambda \cdot \sum_{n=1}^{n} a_n + \mu\cdot \sum_{n=1}^{n} b_n\\
                &= \lambda\cdot s_n + \mu\cdot t_n \fromto \lambda\cdot s + \mu\cdot t\tag{Mit $s$ und $t$ als Limes}
            \end{align*}
        \end{proof}
    \end{satz}

    \newpage

    \begin{satz}[Majoranten-Kriterium] % Satz 7
        \label{satz:majoranten-kriterium}
        Gegeben zwei Folgen $0\leq a_n \leq b_n~\forall n\in\N$. Konvergiert
        \begin{align*}
            \sum_{n=1}^{\infty} b_n\quad\text{so konvergiert auch}\quad\sum_{n=1}^{\infty} a_n
        \end{align*}
        und es gilt
        \begin{align*}
            0\leq \sum_{n=1}^{\infty} a_n \leq \sum_{n=1}^{\infty} b_n
        \end{align*}
        \begin{proof}
            \begin{align*}
                &s_n = \sum_{j=1}^{n} a_j\quad t_n = \sum_{j=1}^{n} b_j\\
                \impl &s_{n+1} = s_n + a_{n+1} \geq s_n\\
                &t_{n+1} = t_n + b_{n+1} \geq t_n\\
                \impl &(s_n)_n,~(t_n)_n\text{ sind monoton wachsend}\\
                \intertext{Mit der Konvergenz von $(t_n)_n$ und $(t_n)_n$ monoton wachsend folgt mit Satz~\ref{satz:mont-konv-reihen}}
                \impl &(t_n)_n\text{ ist beschränkt}\\
                \impl &\exists M\geq 0\colon t_n \leq M\quad\forall n\in\N\\
                0\leq a_n \leq b_n \impl &0\leq s_n \leq t_n \leq M \quad\forall n\in\N\\
                \impl &(s_n)_n\text{ ist nach oben beschränkt und monoton wachsend}\\
                \annot{\impl}{\ref{satz:mont-konv-reihen}} &\lim_{n\fromto\infty} s_n\text{ existiert }\\
                \quad s &\definedas \lim_{n\fromto\infty} s_n \leq \lim_{n\fromto\infty} t_n = \sum_{n=1}^{\infty} b_n\qedhere\\
                \intertext{Außerdem gilt:}
                t_n - s_n &= \sum_{j=1}^{n} b_j - \sum_{j=1}^{n} a_j = \sum_{j=1}^{n} n \pair{b_j-a_j} \geq 0
            \end{align*}
        \end{proof}
    \end{satz}

    \begin{satz}[Minoranten-Kriterium] % Satz 8
        Sei $0\leq b_n \leq a_n~\forall n\in\N$ und
        \begin{align*}
            \sum_{n=1}^{\infty} b_n &= \infty\\
            \impl \sum_{n=1}^{\infty} a_n&\text{ divergiert auch bestimmt gegen }\infty
        \end{align*}
        \begin{proof}
            \begin{align*}
                t_n &= \sum_{j=1}^{n} b_n \quad s_n = \sum_{j=1}^{n} a_n
                \intertext{Analog zum Beweis des Majoranten-Kriteriums gilt:}
                (s_n)_n,~(t_n)_n&\text{ sind monoton wachsend und }t_n \leq s_n \quad\forall n\in\N
                \intertext{Dann lässt sich folgern}
                \sum_{n=1}^{\infty} b_n = \infty &\equivalent (t_n)_n \text{ wächst über alle Grenzen}\\
                &\impl (s_n)_n\text{ wächst über alle Grenzen}\\
                &\impl \lim_{n\fromto\infty} s_n = \infty = \sum_{n=1}^{\infty} a_n\qedhere
            \end{align*}
        \end{proof}
    \end{satz}

    \begin{beispiel}[Anwendung des Minoranten-Kriteriums]
        Es sei
        \begin{align*}
            a_n\geq \frac{c}{n}\quad\forall n\in\N\tag{$c>0$}
            \intertext{Nach Minorantenkriterium und der Divergenz der harmonischen Reihe gilt}
            \impl \sum_{n=1}^{\infty} a_n = \infty
        \end{align*}
    \end{beispiel}
    \begin{beispiel}[Anwendung des Majoranten-Kriteriums]
        Es sei wieder $c>0$. Dann folgt aus
        \begin{align*}
            0\leq a_n \leq c\cdot q^n\land 0\leq q < 1
            \intertext{nach Majoranten-Kriterium und dem Konvergenzkriterium der geometrischen Reihe, dass}
            \impl \sum_{n=1}^{\infty} a_n\text{ konvergiert}\tag{$b_n=c\cdot q^n$}
        \end{align*}
    \end{beispiel}

    \begin{bemerkung}[Abgeschwächtes Majoranten-Kriterium]
        Die Konvergenz/Divergenz von Reihen (und Folgen) ändert sich nicht, wenn man endlich viele Summanden (Folgeglieder) abändert.\\
        Für das Majoranten-Kriterium reicht also, dass $0\leq a_n \leq b_n$ für fast alle $n\in\N$, damit
        \begin{align*}
            \sum_{n=1}^{\infty} b_n\text{ konvergiert }\impl \sum_{n=1}^{\infty} a_n\text{ konvergiert}
        \end{align*}
        Das gleiche gilt analog für das Minoranten-Kriterium
    \end{bemerkung}

    \newpage

    \begin{satz}[Cauchyscher Verdichtungssatz] % Satz 9
        \label{satz:cauchy-verdichtung}
        Sei $(a_n)_n$ eine monoton fallende Nullfolge. Dann gilt
        \begin{align*}
            &\sum_{n=0}^{\infty} a_n\text{ konvergiert}\\
            \equivalent &\sum_{n=0}^{\infty} 2^n\cdot a_{\pair{2^n}}\text{ konvergiert}\tag{Verdichtete Reihe}
        \end{align*}
        \begin{proof}
            \anf{$\Leftarrow$} 1. Schritt. Zu zeigen: $a_n\geq 0\quad\forall n\in\N$.
            \begin{align*}
                a_n&\geq a_{n+1} \geq a_{n+2} \geq \dots \geq a_{n+l} \fromto 0\text{ für } l\fromto\infty\\
                \impl a_n &\geq 0\quad\forall n\in\N
            \end{align*}
            2. Schritt:
            \begin{align*}
                s_n &\definedas \sum_{j=0}^{n} a_j\\
                t_n &\definedas \sum_{\nu=0}^{n} 2^{\nu} \cdot a_{\pair{2^\nu}}
                \intertext{Jedes $\overline{n}\in\N$ können wir eindeutig schreiben als $\overline{n}=2^\nu + l$ mit $\nu\in\N_0,~0\leq l < 2^\nu$~\footnotemark.\endgraf \noindent Sei $1\leq n< 2^k$ für ein $k\in\N_0$}
                s_n &= \sum_{j=0}^{n} a_j = a_0 + \sum_{j=1}^{n} a_j\\
                &\leq a_0 + \sum_{j=1}^{2^k-1} a_j = a_0 + \sum_{\nu=0}^{k-1} \pair{\sum_{l=0}^{2^\nu - 1} \underbrace{a_{\pair{2^\nu + l}}}_{\leq a_{\pair{2^\nu}}}}\\
                &\leq a_0 + \sum_{\nu=0}^{k-1} \pair{\sum_{l=0}^{2^\nu-1} a_{\pair{2^\nu}}}\\
                &= a_0 + \sum_{\nu = 0}^{k-1} 2^{\nu}\cdot a_{\pair{2^\nu}}\\
                &= a_0 + t_{k-1}\\[10pt]
                \impl &\forall n < 2^k\text{ gilt }s_n\leq a_0 + t_{k-1}
            \end{align*}
            \footnotetext{Es lässt sich zeigen, dass $l$ und $\nu$ in diesem Fall eindeutig sind.}
            Angenommen
            \begin{align*}
                \sum_{\nu = 0}^{\infty} 2^{\nu} \cdot a_{2^\nu}\text{ konvergent} &\equivalent \lim_{n\fromto\infty} t_n = t\text{ existiert}\\
                &\impl s_n \leq a_0 + \lim_{k\fromto\infty} t_{k-1} = a_0 + t\quad\forall n\in\N
                \intertext{Somit ist $a_0 + l$ eine obere Schranke von $(s_n)_n$. Da $s_n\leq s_{n+1} \annot{\impl}{\ref{satz:monoton-konv}} \lim_{n\fromto\infty} s_n$ existiert}
                \impl \sum_{n=0}^{\infty} a_n&\text{ konvergent}
            \end{align*}
            \anf{$\impl$} Sei $n\geq 2^k$
            \begin{align*}
                s_n &= \sum_{j=0}^{n} a_j \geq \sum_{j=0}^{2^k} a_j\\
                &= \sum_{\nu=0}^{k} \pair{\sum_{l=0}^{2^\nu-1} \underbrace{a_{\pair{2^\nu + l}}}_{\geq a_{\pair{2^\nu+1}}}}\\
                &\geq \sum_{\nu=0}^{k} \pair{\sum_{l=0}^{2^\nu-1} a_{\pair{2^\nu+1}}} = \sum_{\nu=0}^{k} 2^{\nu}\cdot a_{\pair{2^\nu}} + 1\\
                &= \sum_{\nu = 1}^{k+1} 2^{\nu-1} a_{\pair{2^\nu}} = \frac{1}{2}\sum_{\nu=1}^{k+1} 2^{\nu} a_{\pair{2^\nu}}\\
                &= \frac{1}{2}\pair{\sum_{\nu=0}^{k+1} 2^\nu a_{\pair{2^\nu}} - a_0}\\
                &= t_{k+1} - a_0\\
                \impl t_k \leq t_{k+1} &\leq s_n + a_0\quad\forall 2^k \geq n\\
                \impl t_k &\leq \lim_{n\fromto\infty} s_n + a_0 = s + a_0 < \infty\\
                \text{sofern} \sum_{n=0}^{\infty} a_0\text{ konvergiert}
            \end{align*}
            Da $t_k < t_{k+1}$ konvergiert, konvergiert auch $\sum_{\nu=0}^{\infty} 2^\nu \cdot a_{2^\nu}$
        \end{proof}
    \end{satz}

    \begin{beispiel}[Cauchyscher Verdichtungssatz als Konvergenzkriterium]
        Es sei
        \begin{align*}
            a_n &= \frac{1}{n^\alpha}\\
            2^{n} \cdot a_{\pair{2^\nu}} &= \frac{2^n}{(2^n)^\alpha} = \frac{2^n}{2^{n\cdot\alpha}}\\
            &= 2^{n-n\cdot\alpha} = 2^{(1-\alpha)\cdot n} = \pair{2^{1-\alpha}}^n = q^n
            \intertext{Damit $q^n$ und damit auch $(a_n)_n$ konvergiert muss wie bereits gezeigt gelten}
            q &\definedas 2^{1-\alpha} < 1 \equivalent 1-\alpha < 0 \equivalent \alpha > 1
            \intertext{Es sei}
            a_n &= \frac{1}{n}\\
            2^n \cdot a_{\pair{2^\nu}} &= 2^n \cdot \frac{1}{2^n} = 1\impl (a_n)_n\text{ konvergiert}
        \end{align*}
    \end{beispiel}

    %%%%%%%%%%%%%%%%%%%%%%%%
    % 7. Dezember 2023
    %%%%%%%%%%%%%%%%%%%%%%%%

    \begin{definition}[Alternierende Reihe] % Definition 10
        \marginnote{[7. Dez]}
        Sei $(a_n)_n$ eine reelle Folge nicht-negativer reeller Zahlen. Dann heißt
        \begin{align*}
            \sum_{n=1}^{\infty} (-1)^{n+1} \cdot a_n = a_1 - a_2 + a_3 - a_4 \dots
        \end{align*}
        alternierende Reihe. Alternativ $a_n\geq 0$, $n\in\N_0$.
        \begin{align*}
            \sum_{n=0}^{\infty} (-1)^n\cdot a_n = a_0 - a_1 + a_2 - a_3 \dots
        \end{align*}
    \end{definition}

    \begin{satz}[Leibniz-Kriterium] % Satz 11
        Sei $(a_n)_n$ eine monoton fallende Nullfolge. Dann konvergiert
        \begin{align*}
            \sum_{n=1}^{\infty} (-1)^{n+1} \cdot a_n
        \end{align*}

        \begin{proof}
            Idee: Wir unterscheiden zwischen geraden und ungeraden $n$.
            \begin{align*}
                s_n &\definedas \sum_{j=1}^{n} (-1)^{j+1} \cdot a_j\\
                s_{2(n+1)} &= \sum_{j=1}^{2n+2} (-1)^{j+1} \cdot a_j\\
                &= \sum_{j=1}^{2n} (-1)^{j+1} \cdot a_j + (-1)^{(2n+1)+1} \cdot a_{2n+1} + (-1)^{(2n+2)+1} \cdot a_{2n+2}\\
                &= s_{2n} + \underbrace{a_{2n+1} - a_{2n+2}}_{\geq 0}\\
                &\geq s_{2n}
                \intertext{Also ist $(s_{2n})_n$ monoton wachsend}
                s_{2(n+1)+1} &= s_{2n+3} = \sum_{j=1}^{2n+3} (-1)^{j+1} \cdot a_j\\
                &= s_{2n+1} + (-1)^{(2n+2)+1} \cdot a_{2n+2} + (-1)^{(2n+3)+1} \cdot a_{2n+3}\\
                &= s_{2n+1} \underbrace{- a_{2n+2} + a_{2n+3}}_{\leq 0}\\
                &\leq s_{2n+1}
            \end{align*}
            Also ist $(s_{2n+1})_n$ ist monoton fallend und
            \begin{align*}
                s_{2n+1} - s_{2n} &= \sum_{j=1}^{2n+1} (-1)^{j+1} \cdot a_j - \sum_{j=1}^{2n} (-1)^{j+1}\cdot a_j\\
                &= (-1)^{(2n+1)+1} \cdot a_{2n+1} = a_{2n+1} \geq 0
                \intertext{$\impl \abs{s_{2n+1}-s_{2n}} = a_{2n+1}$ und $s_{2n+1} \geq s_{2n}$}
                0 \leq a_1 - a_2 &= s_2 \leq s_{2n} \leq s_{2n+1} \leq s_1 = a_1\\
                \annot{\impl}{\ref{satz:monoton-konv}} s_g &\definedas \lim_{n\fromto\infty} s_{2n}\text{ existiert}\\
                s_u &\definedas \lim_{n\fromto\infty} s_{2n-1}\text{ existiert}\\
                \text{und } s_g &= s_u\\
                \impl (s_n)_n&\text{ konvergiert gegen $s=s_g=s_u$}\qedhere
            \end{align*}
        \end{proof}
        \begin{uebung}
            Weisen Sie nach, dass eine Folge konvergiert, wenn die Teilfolgen der geraden und der ungeraden Folgeglieder konvergieren und die Differenz eine Nullfolge ist.
        \end{uebung}
    \end{satz}

    \begin{beispiel}
        \begin{align*}
            &\sum_{n=1}^{\infty} \pair{-1}^{n+1} \cdot \frac{1}{\sqrt{n}}\text{ konvergiert}\\
            \text{aber } &\sum_{n=1}^{\infty} \frac{1}{\sqrt {n}} \text{ divergiert}\\
            &\sum_{n=1}^{\infty} (-1)^{n} \cdot \frac{1}{n}\text{ konvergiert}\\
            &\sum_{n=2}^{\infty} (-1)^n \cdot \frac{1}{\log\pair{n}}\text{ konvergiert}
        \end{align*}
    \end{beispiel}

    \begin{satz}[Cauchy-Kriterium] % Satz 12
        \label{satz:cauchy-kriterium}
        Sei $(a_n)_n$ eine Folge reeller Zahlen. Dann konvergiert
        \begin{align*}
            \sum_{n=1}^{\infty} a_n
        \end{align*}
        genau dann, wenn
        \begin{align*}
            \forall \varepsilon > 0~\exists N_{\varepsilon}\in\N\colon \abs{\sum_{j=n+1}^{m} a_j} <\varepsilon\quad\forall m>n\geq N_{\varepsilon}
        \end{align*}
        \begin{proof}
            \begin{align*}
                s_n &\definedas \sum_{j=1}^{n} a_j\\
                \intertext{$(s_n)_n$ konvergiert nach Satz~\ref{satz:jede-konv-cauchy} genau dann, wenn es eine Cauchy-Folge ist. Das heißt}
                \forall\varepsilon > 0~\exists N_{\varepsilon}\in\N\colon &\abs{s_{m} - s_{n}} < \varepsilon\quad\forall m> n \geq N_{\varepsilon}\\
                \abs{s_m - s_n} &= \sum_{j=1}^{m} a_j = \sum_{j=1}^{n} a_j\\
                &= \sum_{j=n+1}^{m} a_j\qedhere
            \end{align*}
        \end{proof}
    \end{satz}

    \begin{korollar} % Korollar 12
        \label{korollar:folge-von-reihe-nullfolge}
        Ist die reelle Reihe $\sum_{n=1}^{\infty} a_n$ konvergent, so ist $(a_n)_n$ eine Nullfolge.
        \begin{proof}
            Nach Satz~\ref{satz:cauchy-kriterium} $\impl$
            \begin{align*}
                \forall \varepsilon > 0~\exists N_{\varepsilon}\in\N\colon &\abs{\sum_{j=n+1}^{n+p} a_j} < \varepsilon\quad\forall n\geq N_{\varepsilon}, p\in\N\\
                \intertext{Wähle $p=1$}
                \sum_{j=n+1}^{n+1} a_{j} &= a_{j+1}\\
                \impl \abs{a_{n+1}} &< \varepsilon\quad \forall n\geq N_{\varepsilon}\\
                \impl \lim_{\ntoinf} a_{n+1} &= 0\\
                \impl \lim_{\ntoinf} a_{n} &= 0\qedhere
            \end{align*}
        \end{proof}
    \end{korollar}

    \subsection{[*] Absolut konvergente Reihen und Umordnungen}

    \begin{definition}[Absolute Konvergenz] % Definition 1
        Eine Reihe$\sum_{n=1}^{\infty} a_n$ heißt absolut konvergent, falls
        \begin{align*}
            \sum_{n=1}^{\infty} \abs{a_n}
        \end{align*}
        konvergiert. Das heißt falls
        \begin{align*}
            \sum_{n=1}^{\infty} \abs{a_n} < \infty
        \end{align*}
    \end{definition}

    \begin{satz}[Absolute Konvergenz als Konvergenzkriterium] % Satz 2
        \label{satz:absolut-konvergenz-konvergenkriterium}
        Ist eine Folge $\sum_{n}^{\infty} a_n$ absolut konvergent, so ist sie auch konvergent und
        \begin{align*}
            \abs{\sum_{n=1}^{\infty} a_n} \leq \sum_{n=1}^{\infty} \abs{a_n}
        \end{align*}

        \begin{proof}
            Wir haben für $m>n$
            \begin{align*}
                \abs{\sum_{j=n+1}^{m} a_j} \leq \sum_{j=n+1}^{m} \abs{a_j}
            \end{align*}
            Wir nehmen an, dass $\sum_{n=1}^{\infty} \abs{a_n}$ konvergiert. Das heißt Cauchy ist erfüllt.
            \begin{align*}
                \impl \forall \varepsilon > 0~\exists N_{\varepsilon}\colon \sum_{j=n+1}^{m} \abs{a_j} &< \varepsilon \quad\forall m> n\geq N_{\varepsilon}\\
                \impl \abs{\sum_{j=n+1}^{m} a_j} &< \varepsilon\quad\forall m>n\geq N_{\varepsilon}\\
                \impl \sum_{j=1}^{\infty} a_j&\text{ konvergiert}\qedhere
            \end{align*}
        \end{proof}
    \end{satz}

    \begin{beispiel}[Konvergenz ohne absolute Konvergenz]
        \begin{align*}
            \sum_{n=1}^{\infty} (-1)^{n+1}\cdot \frac{1}{n}
        \end{align*}
        ist konvergent, aber nicht absolut konvergent.
    \end{beispiel}

    \begin{definition}[Majorante] % Def 3
        Die Reihe
        \begin{align*}
            \sum_{n=1}^{\infty} c_n\quad c_n\geq 0~\forall n\in\N
        \end{align*}
        ist eine Majorante der Reihe $\sum_{n}^{\infty} a_n$ falls
        \begin{align*}
            \abs{a_n} &\leq c_n\text{ für fast alle }n\in\N
            \intertext{Das heißt}
            \exists n_0\in\N\colon \abs{a_n} &< c_n\quad\forall n\geq n_0
        \end{align*}
    \end{definition}

    \begin{satz}[Majoranten-Konvergenz-Kriterium für Reihen] % Satz 4
        \label{satz:majorante-reihen}
        Hat die Reihe $\sum_{n=1}^{\infty} a_n$ eine konvergente Majorante $\sum_{n=1}^{\infty} c_n$, so ist diese Reihe absolut konvergent und somit auch konvergent.
        \begin{proof}
            Folgt aus Satz~\ref{satz:absolut-konvergenz-konvergenkriterium} und Satz~\ref{satz:majoranten-kriterium}.
        \end{proof}
    \end{satz}

    \begin{satz}[Quotientenkriterium] % Satz 5
        \label{satz:quotientenkriterium}
        Sei
        \begin{align*}
            s_n &\definedas \sum_{n=0}^{\infty} a_n
        \end{align*}
        eine Reihe mit $a_n\neq 0$. Ferner gebe es ein $0\leq q < 1$ so dass
        \begin{align*}
            \frac{\abs{a_{n+1}}}{\abs{a_n}} \leq q\text{ für fast alle } n\in\N\\
            \impl \sum_{n=0}^{\infty} a_n\text{ absolut konvergent}
        \end{align*}

        \begin{proof}
            Wir haben $n_0\in\N$
            \begin{align*}
                \frac{\abs{a_{n+1}}}{\abs{a_n}} &\leq q\quad\forall n\geq n_0\\[10pt]
                \abs{a_{n+1}} \leq q\cdot\abs{a_n} &\leq q^2 \cdot \abs{a_{n-1}} \leq q^3 \cdot \abs{a_{n-2}}\\[10pt]
                \leq \dots &\leq q^{p+1} \cdot\abs{a_{n_0}} = q^{n-n_0+1} \cdot\abs{a_{n_0}}\tag{$p\in\N$, $n=n_0+p$}\\[10pt]
                &= q^{n+1} \cdot q^{-n_0} \cdot\abs{a_{n_0}}\\[10pt]
                \impl \abs{a_n} &\leq \underbrace{q^n \cdot K}_{\definedasbackwards c_n}\tag{$K \definedas q^{-n_0} \cdot \abs{a_{n_0}}$}\\
                \impl \abs{a_{n_0}} &\leq c_n\quad\forall n\geq n_0\\
                \intertext{und}
                \sum_{n=0}^{\infty} c_n &= \sum_{n=0}^{\infty} K\cdot q^{n} = K\cdot \sum_{n=0}^{\infty} q^{n} < \infty
                \intertext{Da $0< q < 1$}
                \impl \sum_{n=c}^{\infty} &a_n\text{ hat die konvergente Majorante } K \cdot \sum_{n=c}^{\infty} q^{n}
            \end{align*}
            Damit lässt sich aus Satz~\ref{satz:majorante-reihen} folgern, dass die Reihe konvergiert.
        \end{proof}
    \end{satz}

    \begin{bemerkung}[Quotientenkriterium über $\limsup$ und $\liminf$]
        \theoremescape
        \begin{enumerate}[label=(\roman*)]
            \item Ist $\limsup_{n\fromto\infty} \frac{\abs{a_{n+1}}}{\abs{a_n}} < 1 \equivalent$ Quotientenkriterium
            \item Ist $\liminf_{n\fromto\infty} \frac{\abs{a_{n+1}}}{\abs{a_n}} > 1\impl \sum_{n=0}^{\infty} a_n$ divergent
            \item Ist $\limsup_{n\fromto\infty} \frac{\abs{a_{n+1}}}{\abs{a_n}} = 1\impl \text{Keine Aussage über absolute Konvergenz von } \sum_{n=0}^{\infty} a_n$ möglich
        \end{enumerate}
        \begin{proof}[Beweis (ii).]
            \begin{align*}
                \text{Ist } \overline{q}\definedas\liminf_{n\fromto\infty} \frac{\abs{a_{n+1}}}{\abs{a_n}} &> 1\\[8pt]
                \impl \forall \varepsilon > 0~\exists n_0\colon \frac{\abs{a_{n+1}}}{\abs{a_n}} \geq \overline{q} - \varepsilon = \frac{\overline{q}+1}{2} &\geq q > 1\tag{Wähle $\varepsilon=\frac{\overline{q}-1}{2} > 0$}\\[8pt]
                \impl \frac{\abs{a_n+1}}{\abs{a_n}} &\geq q > 1\quad\forall n\geq n_0
                \intertext{Wir wenden ein ähnliches Prinzip wie im vorherigen Beweis an}
                \impl \abs{a_{n+1}} &\geq q^{n+1} \cdot q^{-n_0}\cdot\abs{a_{n_0}}\\[8pt]
                \impl \abs{a_n} &\geq q^n\cdot K\tag{$K=q^{-n_0}\cdot\abs{a_{n_0}}$}\\[8pt]
                \impl \sum a_j\text{ divergiert nach}&\text{ Korollar~\ref{korollar:folge-von-reihe-nullfolge}}\qedhere
            \end{align*}
        \end{proof}
    \end{bemerkung}

    \begin{beispiel}[Divergenz bei nicht-eindeutigem Quotientenkriterium]
        $a_n \definedas \frac{1}{n}$
        \begin{align*}
            \frac{\abs{a_{n+1}}}{\abs{a_n}} = \frac{a_{n+1}}{a_n} &= \frac{n}{n+1} = 1 + \frac{1}{n} \fromto 1
            \intertext{Und}
            \sum_{n=1}^{\infty} \frac{1}{n}&\text{ divergiert (Harmonische Reihe)}
        \end{align*}
    \end{beispiel}

    \begin{beispiel}[Konvergenz bei nicht-eindeutigem Quotientenkriterium]
        $a_n \definedas \frac{1}{n^2}$
        \begin{align*}
            \frac{a_{n+1}}{a_n} = \frac{n^2}{(n+1)^2} &= \pair{\frac{n}{n+1}}^2 = \pair{1-\frac{1}{n+1}}^2 \fromto 1
            \intertext{Aber}
            \sum_{n=1}^{\infty} &a_n
            \intertext{konvergiert absolut:}
            \pair{1-\frac{1}{n}}^2 &= 1 - \frac{2}{n+1}+ \pair{\frac{1}{n+1}}^2\\
            &= 1 - \frac{2}{n+1}\cdot\pair{1-\frac{1}{2(n+1)}}\\
            &\leq 1- \frac{s-\delta}{n+1}\tag{Für $\delta>0$ und fast alle $n$}
        \end{align*}
    \end{beispiel}

    \newpage

    \begin{beispiel}[Eulersche Zahl über Reihendarstellung]
        Die Reihe
        \begin{align*}
            \sum_{n=0}^{\infty} \frac{1}{n!}
        \end{align*}
        ist absolut konvergent.
        \begin{align*}
            a_n &= \frac{1}{n!}\\
            \frac{a_{n+1}}{a_n} &= \frac{n!}{(n+1)!} = \frac{1}{n+1}\fromto 0
        \end{align*}
        Behauptung:
        \begin{align*}
            e'\definedas\sum_{n=0}^{\infty} \frac{1}{n!} &= e\tag{Eulersche Zahl}
        \end{align*}
        \begin{proof}
            Wir wenden die bereits gezeigt Formel für $e$ an:
            \begin{align*}
                e &= \lim_{\ntoinf} \pair{1+\frac{1}{n}}^n\\[10pt]
                \pair{1+\frac{1}{n}}^n &= \sum_{k=0}^{n}\binom{n}{k} \pair{\frac{1}{n}}^k\\
                &= \sum_{k=0}^{n} \frac{n\cdot(n-1)\cdot(n-k+1)}{k!\cdot n^k}\\
                &= \sum_{k=0}^{n} \frac{1}{k!} \cdot \prod_{j=0}^{k-1} \frac{n-j}{n}\\
                &\leq \sum_{k=0}^{n} \frac{1}{k!} \leq \sum_{k=0}^{\infty} \frac{1}{k!}\\[10pt]
                \impl e &\leq \sum_{n=0}^{\infty} \frac{1}{n!}
                \intertext{Noch zu zeigen: $e'\leq e$}
                \pair{1+\frac{1}{n}}^n &= \sum_{k=0}^{n} \frac{1}{k!} \cdot \prod_{j=0}^{k-1} 1-\frac{j}{n}\\
                &\geq \sum_{k=0}^{m} \frac{1}{k!}\cdot \prod_{j=0}^{k-1} \pair{1-\frac{j}{n}}\tag{$\forall n>m$}\\
                \intertext{Wir halten $m\in\N$ fest}
                \impl e &= \lim_{\ntoinf} \pair{1+\frac{1}{n}}^n \geq \lim_{\ntoinf} \sum_{k=0}^{m} \frac{1}{k!}\cdot \prod_{j=0}^{k-1} \pair{1-\frac{j}{n}}\\
                &= \sum_{k=0}^{m} \frac{1}{k!} \cdot \prod_{j=0}^{k-1} \underbrace{\lim_{\ntoinf} \pair{1-\frac{j}{n}}}_{=1}\\
                &= \sum_{n=0}^{m} \frac{1}{k!}\tag{$\forall m\in\N$}\\
                \impl e&\geq \lim_{\ntoinf} \sum_{k=0}^{m} \frac{1}{k!} = \sum_{k=0}^{\infty} \frac{1}{k!} = e\\
                \impl e &\leq e' \land e'\leq e\\
                \impl e &= e'\qedhere
            \end{align*}
        \end{proof}

        %%%%%%%%%%%%%%%%%%%%%%%%
        % 12. Dezember 2023
        %%%%%%%%%%%%%%%%%%%%%%%%

        \begin{bemerkung}[Ännäherung von $e$ über Reihen]
            \marginnote{[12. Dez]}
            \begin{align*}
                s_n &\definedas \sum_{k=0}^{n} \frac{1}{k!}\\
                r_{n,p} &\definedas \sum_{k=n+1}^{n+p} \frac{1}{k!}\\
                r_{n,p} &= \sum_{k=n+1}^{n+p} \frac{1}{k!} = \frac{1}{(n+1)!} + \frac{1}{(n+2)!} + \dots + \frac{1}{(n+p)!}\\
                &= \frac{1}{(n+1)!}\cdot\interv{1+\frac{1}{n+2}+\frac{1}{(n+2)\cdot(n+3)} + \dots + \frac{1}{(n+2)\cdot(n+3)\cdot\ldots\cdot(n+p)}}\\
                &> \frac{1}{(n+1)!}
                \intertext{Wir betrachten den zweiten Faktor}
                &1+\frac{1}{n+2}+\frac{1}{(n+2)\cdot(n+3)} + \dots + \frac{1}{(n+2)\cdot(n+3)\cdot\ldots\cdot(n+p)}\\
                &< 1 + \frac{1}{2} + \frac{1}{2\cdot 3}+\dots + \frac{1}{2\cdot 3 \cdot\ldots\cdot p}\\
                &= \sum_{k=0}^{p} \frac{1}{k!} - 1 < \sum_{k=0}^{\infty} \frac{1}{k!} - 1 = e-1
                \intertext{Wir kombinieren die Abschätzung über beide Faktoren und erhalten}
                \impl &\frac{1}{(n+1)!} < r_{n,p} < \frac{e-1}{(n+1)!}\\
                \equivalent &\frac{1}{(n+1)!} < s_{n+p} - s_{n} < \frac{e-1}{(n+1)!}\\
                &s_{n+p} - s_{n} \fromto e - s_n\text{ für }p\fromto\infty\\
                \impl &\frac{1}{(n+1)!} \leq e - s_n \leq \frac{e-1}{(n+1)!}
                \intertext{Wir erhalten also ein Verfahren, um einen Näherungswert für $e$ zu bestimmen}
                &2,5 \leq e \leq 3\quad(n=1)\\
                &2,66 \leq e \leq 2,8\quad(n=2)\\
                \vdots\\
                &e = 2,71828182\ldots
            \end{align*}
        \end{bemerkung}
    \end{beispiel}


    \begin{bemerkung}[Ausblick: Definition von Exponentialfunktionen über Reihen]
        \begin{align*}
            a_n &\definedas\sum_{n=0}^{\infty} \frac{x^n}{n!}\\
            \frac{\abs{a_{n+1}}}{\abs{a_n}} &= \abs{\frac{x^{n+1}\cdot n!}{(n+1)!\cdot x^n}} = \frac{\abs{x}}{n+1}\fromto 0
        \end{align*}
        Diese Konvergenz werden wir in einem späteren Kapitel nutzen, um die Exponentialfunktion über
        \begin{align*}
            \exp(x) &= \sum_{k=0}^{\infty} \frac{x^n}{n!} = e^x
        \end{align*}
        zu definieren.
    \end{bemerkung}

    \begin{satz}[Nach Lambert 1707] % Satz 7
        Die eulersche Zahl $e$ ist irrational.
        \begin{proof}
            Wir haben
            \begin{align*}
                \frac{1}{(n+1)!} &\leq e-s_n \leq \frac{e-1}{(n+1)!} < \frac{2}{(n+1)!}
                \intertext{Angenommen $e=\frac{p}{q}$\quad$p,q\in\N$. Wir nehmen $p=q\cdot m$}
                \impl \frac{1}{(q+1)!} &\leq \frac{p}{q} - s_q < \frac{2}{(q+1)!}\\
                0 &< \frac{1}{q+1} \leq \frac{p}{q}\cdot q! - q!\cdot s_q < \frac{2}{q+1}<1
                \intertext{Es lässt sich zeigen, dass der dritte Term eine ganze Zahl ist:}
                \frac{p}{q}\cdot q! &= p\cdot(q-1)!\in\N\\
                q!\cdot s_q &= q! \cdot \sum_{k=0}^{q} \frac{1}{k!}\in\N
            \end{align*}
            Damit ergibt sich ein Widerspruch, weil keine ganze Zahl zwischen $0$ und $1$ liegt.
        \end{proof}
    \end{satz}

    \newpage

    \begin{satz}[Wurzelkriterium] % Satz 8
        \label{satz:wurzelkriterium}
        Sei $(a_n)_n$ eine reelle Folge.
        \begin{enumerate}[label=(\roman*)]
            \item Ist $\limsup_{\ntoinf} \sqrt[n]{\abs{a_n}} < 1$, dann konvergiert $\sum_{n=0}^{\infty} a_n$ absolut.
            \item Ist $\limsup_{\ntoinf} \sqrt[n]{\abs{a_n}} > 1$, dann ist $\sum_{n=0}^{\infty} a_n$ divergent.
            \item Ist $\limsup_{\ntoinf} \sqrt[n]{\abs{a_n}} = 1$, so ist keine Aussage möglich.
        \end{enumerate}
        \begin{proof}[Beweis (iii)]
            \begin{align*}
                a_n &\definedas \frac{1}{n^p}\\
                \sqrt[n]{\abs{a_n}} &= \pair{\frac{1}{\sqrt[n]{n}}}^p \fromto \pair{\frac{1}{1}}^p = 1\tag{$n\fromto\infty$}\\
                \sum_{n=1}^{\infty} \frac{1}{n^p}&\text{ divergiert für $p=1$ und konvergiert für $p>1$}\\
                \impl &\text{ keine Aussage möglich}\qedhere
            \end{align*}
        \end{proof}
        \begin{proof}[Beweis (i)]
            Sei
            \begin{align*}
                \hat{q}&\definedas \limsup \sqrt[n]{\abs{a_n}} < 1
                \intertext{Wähle $\varepsilon\definedas \frac{1-\hat{q}}{2} > 0$}
                \annot{\impl}{\ref{lemma:limsup-charak}} \sqrt[n]{\abs{a_n}} &< \hat{q} + \varepsilon = \frac{1+\hat{q}}{2}\text{ für fast alle $n$}\\
                \impl \exists n_{0}\in\N\colon \sqrt[n]{\abs{a_n}} &\leq q\quad\forall n\geq n_0\\
                \equivalent \abs{a_n} &\leq q^n\quad\forall n\geq n_0\\
                \intertext{Das heißt $\sum_{n=0}^{\infty} q^n$ ist eine konvergente Majorante für $\sum a_n$}
                \impl \sum_{n=0}^{\infty} a_n&\text{ ist absolut konvergent}\qedhere
            \end{align*}
        \end{proof}
        \begin{proof}[Beweis (ii)]
            Sei
            \begin{align*}
                \hat{q}&\definedas \limsup_{\ntoinf} \sqrt[n]{\abs{a_n}} > 1\\
                \varepsilon &\definedas \frac{\hat{q}-1}{2}>0\\
                q&\definedas \hat{q}-\varepsilon = \frac{1+\hat{q}}{2} > 1\\
                \impl \sqrt[n]{\abs{a_n}} &> \hat{q}-\varepsilon \definedasbackwards q > 1\tag{für unendlich viele $n$}\\
                \impl \abs{a_n} &> q^n\tag{für unendlich viele $n$}\\
                \impl (a_n)_n&\text{ keine Nullfolge}
            \end{align*}
            Somit konvergiert $\sum a_n$ nicht.
        \end{proof}
    \end{satz}

    \newpage

    \begin{definition}[Umordnung von Reihen] % Definition 9
        Seien
        \begin{align*}
            \sum_{n=0}^{\infty} a_n\quad \sum_{n=0}^{\infty} b_n
        \end{align*}
        Reihen mit Gliedern $a_n, b_n\in\R$, $n\in\N_0$. Wir nennen $\sum_{n=0}^{\infty} b_n$ eine Umordnung von $\sum_{n=0}^{\infty} a_n$, falls eine Bijektion
        \begin{align*}
            \sigma: \N_0 \fromto \N_0
        \end{align*}
        existiert mit $b_n = a_{\sigma(n)}~\forall n\in\N_0$.\\
        Ähnlich für Reihen
        \begin{align*}
            \sum_{n=1}^{\infty} a_n\quad \sum_{n=1}^{\infty} b_n
        \end{align*}
        $\sum_{n=1}^{\infty} b_n$ Umordnung von $\sum_{n=1}^{\infty} a_n$, falls Bijektion $\sigma: \N \fromto\N$ existiert mit $b_n = a_{\sigma(n)}$.
    \end{definition}

    \begin{definition}[Unbedingte Konvergenz] % Definition 10
        Eine Reihe $\sum_{n=0}^{\infty} a_n$ heißt unbedingt konvergent, falls jede Umordnung $\sum_{n=0}^{\infty} b_n$ von dieser Reihe ebenfalls konvergiert und die selbe Summe hat.\\
        Andernfalls heißt $\sum_{n=0}^{\infty} a_n$ bedingt konvergent.
    \end{definition}

    \begin{satz}[Direcklet 1837] % Satz 11
        Eine Reihe $\sum_{n=0}^{\infty} a_n$, $a_n\in\R$ ist absolut konvergent genau dann, wenn sie unbedingt konvergiert.
        \begin{proof}
            \anf{$\impl$} Sei
            \begin{align*}
                \sum_{n=0}^{\infty} a_n\text{ absolut konvergent}\\
                \impl \sum_{n=0}^{\infty} \abs{a_n}\text{ konvergent}\\
                \equivalent \sum_{n=0}^{\infty} \abs{a_n} < \infty
                \intertext{Wir wenden das Cauchy-Kriterium an}
                \forall \varepsilon>0~\exists N\in\N\colon \sum_{j=n+1}^{n+p} \abs{a_j} < \varepsilon\quad\forall n\geq N, p\in\N\\
                \impl \sum_{j=n+1}^{\infty}\abs{a_j} = \lim_{p\fromto\infty} \sum_{j=n+1}^{n+p} \abs{a_j} \leq \varepsilon\quad\forall n\geq N\tag{3}
                \intertext{Wir definieren}
                s_n \definedas \sum_{j=0}^{\infty} a_j\quad \sum_{n=0}^{\infty} b_n\text{ Umordnung von } s_n\\
                b_n = a_{\sigma(n)}\quad \sigma: \N_0 \fromto\N_0\text{ Bijektion}\\
                t_n \definedas \sum_{j=0}^{n} b_j
                \intertext{Wir wissen $s_n\fromto s$. Zu zeigen: $t_n\fromto s$}
                \set{1,2,\dots, N} \subseteq \set{\sigma(1), \sigma(2), \dots, \sigma(M)}\tag{Nehmen $M\in\N$}\\
                \intertext{Ist dann $n\geq M$, dann ist}
                \set{a_1, a_2, a_3, \dots, a_N}\subseteq \set{b_1, b_2, \dots b_M} = \set{a_{\sigma(1)}, a_{\sigma(2)}, \dots, a_{\sigma(M)}}\\
                \impl\text{ Alle Glieder $a_1, \dots, a_n$ in der Summe $s_n$ treten in $t_n = t_1 + t_2 + \dots + t_n$ auf}\\
                \impl\text{ Diese Terme heben sich in $s_n-t_n$ gegenseitig auf, sofern $n\geq M$ ist}\\
                \impl \abs{s_n-t_n} \leq \sum_{j\geq N+1, j=\sigma{k}\text{ für ein }k\in\set{1,\dots, M}}^{} \abs{a_j}\\
                \leq \sum_{j=??}^{\infty} \abs{a_j} \leq \varepsilon\\
                \impl s_n - t_n \fromto 0\text{ für }\ntoinf
            \end{align*}
            Da $(s_n)_n$ gegen $s$ konvergiert, konvergiert auch $(t_n)_n$ gegen $s$. Somit konvergiert $\sum_{n=0}^{\infty} b_n$ und hat die selbe Summe wie $\sum_{n=0}^{\infty} a_n$.\\[10pt]
            \anf{$\Leftarrow$} Angenommen $\sum_{n=0}^{\infty} a_n$ ist unbedingt konvergent, aber nicht absolut konvergent.
            \begin{align*}
                p_n &\definedas (a_n)_{+} \definedas \max\pair{0, a_n}\\
                q_n &\definedas (a_n)_{-} \definedas \max\pair{0,-a_n} = -\min\pair{0,a_n}\\
                \impl \abs{a_n} &= p_n + q_n\quad\forall n\in\N_0\\
                \intertext{Wir haben $\sum_{n=0}^{\infty} \abs{a_n}$ konvergiert, aber $\sum_{n=0}^{\infty} p_n = \infty$}
            \end{align*}
            Behauptung: $\sum_{n=0}^{\infty} p_n = \infty$ und $\sum_{n=0}^{\infty} q_n = \infty$. Angenommen $0\leq \sum_{n=0}^{\infty} p_n < \infty$.
            \begin{align*}
                \impl \sum_{n=0}^{\infty} \pair{p_n - a_n}\text{ konvergiert}\\
                \impl \sum_{n=0}^{\infty} a_n < \infty
                \intertext{Da $\abs{a_n} = p_n + q_n$}
                \impl \sum_{n=0}^{\infty} \abs{a_n} = \sum_{n=0}^{\infty} \pair{p_n+q_n} = \sum_{n=0}^{\infty} p_n + \sum_{n=0}^{\infty} q_n < \infty\\
                \text{Widerspruch zu } \sum_{n=0}^{\infty} b_n\text{ ist nicht absolut konvergent}
            \end{align*}
            Jetzt setze $r_0=0$ und bestimme induktiv $(r_n)_n~r_n < r_{n+1}$ mit $p_0 + p_1 + \dots + p_{r_n} > n + q_0 + q_1 + \dots + q_n~\forall n\in\N$.
            $r_q\definedas$ kleinste natürliche Zahl mit $p_0 + p_q + \dots + p_{r_1} > 1 + q_? + q_1$.\\
            $r_2\definedas$ kleinste natürliche Zahl $\geq r_1 + 1$: $p_0 + \dots + p_{r_2} > 2 + q_0 + q_1 + q_2$.\\
            Machen induktiv weiter: Gegeben $r_n$ wähle $r_{n+1} =$ kleinste natürliche Zahl $\geq r_n + 1$ mit $p_0 + p_1 + \dots + p_{r_{n+1}} > n + q_0 + q_1 + \dots + q_n$.\\[10pt]
            Umordnung $p_0 - q_0 + p_1 + \dots + p_{r_1} - q_1 + p_{r_{1+1}} + \dots + p_{r_2} - q_2 + p_{r_{2+1}} + \dots + p_{r_3} - q_3 + \dots$. Diese Umordnung divergiert gegen $+\infty$.
        \end{proof}
    \end{satz}

    \newpage

    %%%%%%%%%%%%%%%%%%%%%%%%
    % 14. Dezember 2023
    %%%%%%%%%%%%%%%%%%%%%%%%

    \begin{satz}[Nach Riemann 1854]
        \marginnote{[14. Dez]}
        Ist $\sum_n a_n$ konvergent, aber nicht absolut konvergent. Dann gibt es zu jedem $c\in\R$ eine Umordnung $\sum_{n}^{} b_n$ von $\sum_{n}^{} a_n$ so, dass $\sum_n b_n$ konvergiert und den Wert $c$ hat ($\sum_{n} b_n = c$).

        \begin{proof}
        (Später)
        \end{proof}
        % Skizze zu Riemann-Satz
    \end{satz}

    \newpage


    \section{[*] $\R^d$, Konvergenz im $\R^d$, die komplexen Zahlen $\C$ und der Raum $\C^d$}

    \subsection{[*] Der Raum $\R^d$ und Normen}

    \thispagestyle{pagenumberonly}

    \begin{definition}
        \begin{align*}
            \R^d &\definedas\text{ Vektorraum der reellen $d$-Tupel}\\
            &=\set{\pair{x_1, x_2, \dots, x_d} ~\middle|~ x_j\in\R,~ j=1,\dots,d }\\
        \end{align*}
    \end{definition}

    \begin{notation}[Linearkombination von Vektoren]
        Es sei
        \begin{align*}
            x&= \pair{x_1, \dots, x_d}\\
            y&= \pair{y_1, \dots, y_d}
            \intertext{und $\alpha, \beta \in \R$, dann gilt}
            \alpha x + \beta y &\definedas \pair{\alpha x_1 + \beta y_1, \alpha x_2 + \beta y_2, \dots, \alpha x_d + \beta y_d}
        \end{align*}
    \end{notation}

    \begin{notation}[Vektorschreibweise]
        \begin{align*}
            x = \begin{pmatrix}
                    x_1 \\ \vdots \\ x_d
            \end{pmatrix}
        \end{align*}
    \end{notation}

    \horizontalline
    Im $\R$ haben wir Konvergenz über den Betrag definiert. Im $\R^d$ benötigen wir daher ein ähnliches Konzept. Wir betrachten dafür zunächst einige Beispiele solcher \textit{Normen}.

    \begin{beispiel}[Euklidische Länge eines Vektors]
        Für $d=2$ gilt für die euklidische Länge $\norm{x}$ eines Vektors $x\in\R^2$
        \begin{align*}
            \norm{x}^2 &= (x_1)^2 + (x_2)^2\\
            \norm{x} &= \sqrt{(x_1)^2 + (x_2)^2}
        \end{align*}
        Allgemein gilt
        \begin{align*}
            \norm{x} &\definedas \sqrt{\pair{\sum_{j=1}^{d} (x_j) ^2}}\tag{$x\in\R^d$}
            \intertext{Wir schreiben auch}
            \norm{x}_2 &\definedas \sqrt{\pair{\sum_{j=1}^{d} (x_j) ^2}}\tag{$x\in\R^d$}
        \end{align*}
    \end{beispiel}

    \begin{beispiel}[Andere Normen]
        Neben der euklidischen lassen sich noch weitere Normen definieren. Zum Beispiel
        \begin{align*}
            \norm{x}_1 &\definedas \sum_{j=1}^{d} \abs{x_j} \tag{Manhattan-Norm}\\
            \norm{x}_{\infty} &\definedas \max_{1\leq j \leq d} \abs{x_j}\tag{Maximums-Norm}
        \end{align*}
    \end{beispiel}

    \begin{definition}[Norm] % Definition 1
        Eine Norm auf $\R^d$ (oder einem reellen Vektorraum) ist eine Abbildung
        \begin{align*}
            \norm{.}: \R^d\fromto \R
        \end{align*}
        mit folgenden Eigenschaften:
        \begin{enumerate}[label=\alph*)]
            \item $\norm{x} \geq 0~\forall x\in\R^d$ sowie $\norm{x} = 0\impl x=0$
            \item $\forall\lambda\in\R,~x\in\R^d\colon \norm{\lambda x} = \abs{\lambda}\cdot\norm{x}$
            \item $\forall x,y\in\R^d\colon\norm{x+y} \leq \norm{x} + \norm{y}$\quad\text{(Dreiecksungleichung)}
        \end{enumerate}
    \end{definition}

    \begin{bemerkung}
        Dass a), b) und c) für $\norm{\cdot}_1$ und $\norm{\cdot}_{\infty}$ gelten, ist einfach zu zeigen. Außerdem sind a) und b) für $\norm{\cdot}_2$ einfach zu zeigen, c) ist tricky. (Übung)
    \end{bemerkung}

    \begin{definition}[Äquivalenz von Normen]
        2 Normen $\norm{\cdot}_a$ und $\norm{\cdot}_b$ sind äquivalent, falls
        \begin{align*}
            \exists c_1, c_2\in\R\colon c_1\cdot\norm{x}_a \leq \norm{x}_b \leq c_2 \norm{x}_a\quad\forall x\in V
        \end{align*}
    \end{definition}

    \begin{beispiel}[Äquivalenz von euklidischer und Maximums-Norm]
        \label{beispiel:norm-equiv}
        Wir zeigen, dass $\norm{\cdot}_{\infty}$ und $\norm{\cdot}_{2}$ äquivalent sind.
        \begin{align*}
            \abs{x_k} &\leq \sqrt{\sum_{j=1}^{d} \abs{x_j}^2} = \norm{x}_2\tag{$1\leq k \leq d$}\\
            \impl \norm{x}_{\infty} &= \max_{k=1,\dots, d} \abs{x_k} \leq \norm{x}_2\tag{1}
            \intertext{Umgekehrt}
            \norm{x}_2^2 = \sum_{j=1}^{d} \abs{x_j}^2 &\leq \sum_{j=1}^{d} \pair{\max_{k=1,\dots, d}\pair{\abs{x_k}}}^2 = d \cdot \norm{x}_{\infty}^2\\
            \impl \norm{x}_2 &\leq \sqrt {d}\cdot \norm{x}_{\infty}\tag{2}
            \intertext{Mit (1) und (2) gilt dann}
            \impl \frac{1}{\sqrt{d}}\cdot \norm{x}_2 &\leq\norm{x}_{\infty} \leq \norm{x}_2
        \end{align*}
    \end{beispiel}

    \begin{uebung}
        Weisen Sie die Äquivalenz von $\norm{x}_1$ und $\norm{x}_\infty$ analog zu Beispiel~\ref{beispiel:norm-equiv} nach.
    \end{uebung}

    \subsection{[*] Konvergenz im $\R^d$}

    \begin{bemerkung}[Abstand zwischen 2 Vektoren im $\R^d$]
        Zu jeder Norm auf $\R^d$ (oder reellen Vektorräumen) definieren wir den Abstand von 2 Vektoren $x,y\in\R^d$ als $\norm{x-y}$.
        \begin{align*}
            d(x,y) &\definedas \norm{x-y}
            \intertext{Mit $z$ als weiterem Vektor gilt}
            \norm{x-y} &= \norm{x-z+z-y}\\
            &\leq \norm{x-z} + \norm{z-y}\\
            \norm{x-y} &= \norm{-(y-x)} = \abs{-1}\cdot \norm{y-x} = \norm{y-x}
        \end{align*}
    \end{bemerkung}

    \begin{folgerung}[Mehrdimensionale $\varepsilon$-Umgebung]
        Bisher basierte unser Konvergenz-Begriff auf dem Abstand von $(x_n)_n$ und $x$ und somit auf einer eindimensionalen $\varepsilon$-Umgebung.\\
        Wir verallgemeinern dieses Konzept für eine offene Kugel im $\R^d$ mit $x\in\R^d$
        \begin{align*}
            B_\varepsilon(x) \definedas \set{y\in\R^d\middle |~ \norm{x-y}_2 < \varepsilon}
        \end{align*}
    \end{folgerung}

    \begin{visualisierung}[Zweidimensionale $\varepsilon$-Umgebung]
        Wir formen die Bedingung um
        \begin{align*}
            \norm{x-y}_2 &< \varepsilon\\
            \impl \sqrt{(x_1-y_1)^2 + (x_2 - y_2)^2} &< \varepsilon\\
            \impl \pair{x_1-y_1}^2 + \pair{x_2-y_2}^2 &< \varepsilon^2
        \end{align*}
        In der Visualisierung erhalten wir einen offenen Ball um $x$ mit Radius $\varepsilon$.

        \begin{figure}[H]
            \centering
            \begin{tikzpicture}
                \draw[fill=\rgbcolor{230}{230}{230}, dashed] (0,0) circle (0.75cm);
                \draw[fill=\rgbcolor{230}{230}{230}, dashed] (2,1) circle (0.75cm);
                \draw[->] (-3,0) -- (3,0);
                \draw[->] (0,-3) -- (0,3);
                \draw[->] (0,0) node {$\times$} -- (0.52, 0.52) node[below, pos=0.8] {$\varepsilon$};
                \draw[->] (2,1) node {$\times$} -- (2, 1.75) node[right, pos=0.5] {$\varepsilon$};
                \draw (-0.25, 0.35) -- (-1.5, 0.35) node[left] {$B_{\varepsilon}\pair{(0,0)}$};
                \draw (1.75, 1.35) -- (-1.5, 1.35) node[left] {$B_{\varepsilon}\pair{(x_1,x_2)}$};
                \node at (-0.75,-0.75) {1.};
                \node at (2.75,0.25) {2.};
                \node at (3,-2) {\begin{tabular}{l}
                                     1. $x=(0,0)$ \\ 2. $x=(x_1, x_2)$
                \end{tabular}};
            \end{tikzpicture}
            \caption{Zweidimensionale $\varepsilon$-Umgebungen}
        \end{figure}
    \end{visualisierung}

    \begin{definition}[Konvergenz im $\R^d$] % Definition 2
        Sei $(x_n)_n$ eine Folge in $\R^d$ mit $x_n\in\R^d~\forall n$. Dann konvergiert $(x_n)_n$ gegen $x\in\R^d$, falls
        \begin{alignat*}{2}
            \forall\varepsilon > 0~\exists N\in\N\colon& \norm{x_1-x}_2 <\varepsilon\quad&&\forall n\geq N
            \intertext{Das heißt}
            \forall\varepsilon > 0~\exists N\in\N\colon& x_n\in B_{\varepsilon}(x)\quad&&\forall n\geq N
        \end{alignat*}
    \end{definition}

    \begin{beobachtung}[Konvergenz in $\R$ vs $\R^d$]
        Es sei ${x_n}\in\R^d$ mit $x_n = \pair{x_n^1, x_n^2, \dots, x_n^d}$\footnote{Koordinaten von $(x_n)_n$}. Damit ergeben sich die Folgen $(x_n^1)_n$, $(x_n^2)_n$, \dots, $(x_n^d)_n$ in $\R$.
        Angenommen $(x_n)_n$ konvergiert gegen $x\in\R^d$. Dann konvergieren alle Folgen $(x_n^l)_n$ in $\R$ und $\lim_{\ntoinf} x_n^l = x^l$

        \begin{proof}
            \begin{align*}
                \abs{x_n^{l} - x^{l}}^2 &\leq \sum_{j=1}^{d} \abs{x_n^{j}-x^{j}}^2 = \norm{x_n -x}_2^2\quad \forall l=1,\dots, d\\
                \impl \abs{x_n^l-x^l} &\leq \norm{x_n-x}_2 \fromto 0\text{ für } n\fromto\infty
            \end{align*}
            Also konvergiert $(x_n^l)$ gegen $x^l$ in $\R$ für alle $l=1,\dots,d$.\qedhere
        \end{proof}
    \end{beobachtung}

    \noindent Es gilt sogar die Umkehrung:

    \begin{satz}[Konvergenz im $\R^d$]
        Eine Folge $(x_n)_n$ in $\R^d$ konvergiert genau dann, wenn jede der Koordinatenfolge $(x_n^l)_n$ in $\R$ konvergiert $\forall l=1,\dots, d$.

        \begin{proof}
            ~\\\anf{$\impl$} Gerade gezeigt.\\
            \anf{$\Leftarrow$} Angenommen
            \begin{align*}
                \exists x^l &\definedas \lim_{\ntoinf} x^l_n\quad\forall l=1,\dots, d\\
                \impl \abs{x_n^l - x^l} &\fromto 0 \text{ für }\ntoinf\quad\forall l=1,\dots, d\colon\\
                \intertext{Das heißt}
                \forall\varepsilon > 0~\exists N_l\in\N\colon \abs{x_n^l - x^l} &< \frac{\varepsilon}{\sqrt{d}}\quad\forall n\geq N_{l}
                \intertext{Wir definieren}
                N&\definedas\max\pair{N_1, N_2, \dots, N_d}
                \intertext{Dann gilt $\forall n>N$}
                \norm{x_n-x}^2_2 &= \sum_{j=1}^{d} \abs{x_n^j - x^j}^2 < \sum_{j=1}^{d} \pair{\frac{\varepsilon}{\sqrt{d}}}^2\\
                &= d\cdot\frac{\varepsilon^2}{d} = \varepsilon^2\\
                \impl \norm{x_n-x}_2 &< \varepsilon\quad\forall n\geq N
                \intertext{Wir definieren den Grenzwert}
                x&\definedas (x^1, x^2, \dots, x^d)\quad x^j\definedas \lim_{\ntoinf} x_n^j\qedhere
            \end{align*}
        \end{proof}
    \end{satz}

    \newpage

    %%%%%%%%%%%%%%%%%%%%%%%%
    % 19. Dezember 2023
    %%%%%%%%%%%%%%%%%%%%%%%%

    \begin{bemerkung}
        \marginnote{[19. Dez]}
        Sei $(a_n)_n\in\R^d$ mit $a_n = \pair{a_n^1, a_n^2, \dots, a_n^d}$. Reihe $\sum_{n=1}^{\infty} a_n = ??$. Wir betrachten die Partialsummen
        \begin{align*}
            s_n &\definedas \sum_{j=1}^{n} a_j
        \end{align*}
    \end{bemerkung}

    \begin{definition}
        $\sum_{n} a_n$ konvergiert, falls $(s_n)_n$ im $\R^d$ konvergiert
    \end{definition}

    \begin{satz}[Cauchy-Kriterium für Konvergenz im $\R^d$]
        Eine Folge $(x_n)_n$ ist genau dann im $\R^d$ konvergent, wenn $(x_n)_n$ eine Cauchy-Folge ist. Das heißt
        \begin{align*}
            \forall\varepsilon > 0~\exists N\in\N\colon \norm{x_n-x_m}_2 < \varepsilon\quad\forall n,m\geq N
        \end{align*}

        \begin{proof}
            \anf{$\impl$} Wie im Fall $\R$.\\
            \anf{$\Leftarrow$} Sei $(x_n)_n$ Cauchy-Folge.
            \begin{align*}
                \impl\text{ Alle Koordinaten } &(x_n^j)_n\text{ sind Cauchy-Folgen in }\R\text{, weil}\\
                \abs{x_n^j-x_m^j} &= \sqrt {\abs{x_n^j-x_m^j}^2} \leq \sqrt {\sum_{l=1}^{d} \abs{x_n^l - x_m^l}^2}\\
                &= \norm{x_1-x_m}_2\\
                \impl x_n &\definedas \lim_{\ntoinf} x_n^j\text{ existiert}\\
                \impl x &= \lim_{\ntoinf} x_n\text{ existiert}\qedhere
            \end{align*}
        \end{proof}

        \begin{proof}[2. Beweis]
            Wir wollen Satz~\ref{satz:bolzano-weierstrass} im $\R^d$ zeigen:
            Jede beschränkte Folge $(x_n)_n$ in $\R^d$ besitzt eine konvergente Teilfolge. ($(x_n)_n$ ist beschränkt, wenn $\exists R\geq 0\colon \norm{x_n}_2 \leq R~\forall n\in\N$ .Bzw., wenn $x_n\in\set{y\in\R^d \middle|~ \norm{y}_2 \leq R}$ (abgeschlossene Kugel mit Radius $R$ um 0).).\\
            \begin{align*}
                \impl \forall j=1,\dots, d\colon (x_n^j)_n\text{ beschränkte Folge im }\R^d\\
                \impl \exists\text{ Teilfolge } (x_n^1)_k\text{ von } (x_n^1)_n\text{, welche in }\R\text{ konvergiert}\\
                \impl \text{ Ausdünnung }\sigma: \N\fromto\N\text{ mit } \sigma(k) < \sigma(k+1)\\
                n_k \definedas \sigma(k)\quad\text{Existenz des Grenzwert} x_1 \definedas \lim_{\ntoinf} x_{\sigma(k)}
                \intertext{Genauso für $\lim (x^2_n)_n$}
                \annot{\impl}{\ref{satz:bolzano-weierstrass}} \exists \kappa\colon \N\fromto\N\colon (x^2_{\kappa(k)})_k\text{ konvergent}
                \intertext{Catch: $(x^1_{\sigma(k)}, x^2_{\sigma(k)})_k$ ist im Allgemeinen keine Teilfolge von $(x^1_n, x^2_n)_n$. Lösung betrachte}
                \pair{x_{\sigma(k)}}_k\text{ Teilfolge von }(x_n)_n\\
                = \pair{x^1_{\sigma(k)}, x^2_{\sigma(k)}, x^3_{\sigma(k)}, \dots, x^d_{\sigma(k)}, }
                \intertext{Mache mit $(x^2_{\sigma(k)})_k$ weiter}
                \annot{\impl}{\ref{satz:bolzano-weierstrass}} \exists\text{ konvergente Teilfolge von } (x^2_{\sigma(k)})_k\\
                \impl\text{ Ausdünnung } \sigma_2: \N\fromto\N\\
                \impl \pair{x^2_{\sigma(\sigma_2(k))}}_k\text{ konvergent}\\
                x_2 \definedas \sigma \circ \sigma_{2}\\
                \impl\text{ Neue Teilfolge } (x_{\kappa_{2}(k)}) = (x^1_{\kappa_{2}(k)},x^2_{\kappa_{2}(k)},x^3_{\kappa_{2}(k)},\dots, x^d_{\kappa_{2}(k)})
                \intertext{mit $\lim_{k\fromto\infty} x^1_{\kappa_{2}(k)}$ und $\lim_{k\fromto\infty} x^2_{\kappa_{2}(k)}$ existent}
            \end{align*}
            Mache induktiv so weiter, nach maximal $d$ Schritten sind wir fertig.\\
            Jede Cauchy-Folge in $\R^d$ ist beschränkt.
            \begin{proof}
                \begin{align*}
                    \varepsilon = 1\\
                    \impl \exists N\colon\norm{x_n-x_m}_2 \leq 1\quad\forall n,m\geq N\\
                    \impl \norm{x_n}_2 = \norm{x_n-x_N+x_N}_2 \leq \norm{x_n-x_N}_2 + \norm{x}_2\quad\forall n\geq N\\
                    < 1 + \norm{x_n}\\
                    \impl \norm{x}_2 \leq \max\pair{\norm{x_1}_2, \norm{x_2}_2,\dots\norm{x_{N-1}}_2, 1+\norm{x_N}_2}
                \end{align*}
            \end{proof}
            Auch: Eine Cauchy-Folge im $\R^d$ ist genau dann konvergent, wenn sie eine konvergente Teilfolge besitzt. (Beweis wie im Fall $d=1$).\\
            Alle bisherigen Konvergenz-Sätze, welche nicht die Anordnung im $\R$ benötigen, übertragen sich auf $\R^d$.\\
            Insbesondere: Reihe $\sum_{n=1}^{\infty} a_n$ ist absolut konvergent, falls $\sum_{n=1}^{\infty} \norm{a_n}_2 < \infty$.\\
            Ist eine Reihe $\sum_{n=1}^{\infty} a_n$ in $\R^d$ konvergent, so konvergieren alle Umordnungen gegen den selben Wert. Und es gilt die Umkehrung! (Satz von Direcklet in $\R^d$)
        \end{proof}
    \end{satz}

    \subsection{[*] Die Komplexen Zahlen}

    Was sind die komplexen Zahlen?\\
    $x^2+1$ ist nie Null $\forall x\in\R$. Wir definieren eine Zahl $i$ mit $i^2=-1$.\\
    Schreiben $x+iy$ mit $x,y\in\R$.
    \begin{align*}
    (x_1 + y_1\cdot i)
        \cdot (x_2 + y_2\cdot i) &= x_1\cdot x_2 + x_1\cdot i\cdot y_2 + i\cdot y_1 \cdot x_2 + (i\cdot y_1)\cdot(i\cdot y_2)\\
        &= x_1\cdot x_2 - y_1\cdot y_2 + \pair{x_1\cdot y_2 + y_1\cdot x_2}\cdot i\\
        (x_1+y_1\cdot i) + (x_2 + y_2\cdot i) &= x_1 + x_2 + \pair{y_1+y_2}\cdot i
    \end{align*}
    Betrachte $\R^2 = \set{(x,y) \middle|~ x,y\in\R}$.
    % Visualisierung: Eukldische Ebene
    \begin{align*}
        z &= (x,y)\\
        z_1 + z_2 &\definedas (x_1, y_1) + (x_2,y_2)\\
        &\definedas (x_1 + x_2, y_1 + y_2)
        \intertext{Wir definieren eine \anf{seltsame} Multiplikation}
        z_1 \cdot z_2 &\definedas (x_1, y_1)\cdot (x_2, y_2)\\
        &= \definedas \pair{x_1 x_2 - y_1 y_2, x_1 y_2 + x_2 y_1}
        \intertext{Wir definieren den Betrag}
        \abs{z} &\definedas\text{ Länge des Vektors } (x,y) = \sqrt {x^2+y^2}\\
        \intertext{Wir nennen}
        x&= \text{ Realteil von } z\\
        y&= \text{ Imaginärteil von } z\\
    \end{align*}

    Man rechnet einfach nach, dass Multiplikation und Addition Assoziativität, Kommutativität und Distributivität erfüllen.
    \begin{align*}
    (1,0)
        \cdot (1,0) &= (1,0)\\
        (0,1) \cdot (0,1) &= (0-1,0) = -(1,0)\\
        z\cdot (1,0) &= (x,y)\cdot(1,0)= (x,y) = z
        \intertext{in $\R^2$}
        (x,y) &= x\cdot e_1 + y \cdot e_2\tag{$e_1 = (1,0)$, $e_2 = (0,1)$}\\
        (e_1)^2 &= e_1\quad (e_2)^2 = -e_1
    \end{align*}
    Wir schreiben $1$ für $e_1$ und $i$ für $e_2$.
    \begin{align*}
        z &= (x,y) = x\cdot e_1 + y\cdot e_2\\
        &= x\cdot 1 + y\cdot i\\
        &= x + y\cdot i
    \end{align*}
    $\C = \R^2$ mit obiger Multiplikation und Addition. Wir identifizieren $\R$ mit $\R\times\set{0}$ als Teilmenge von $\C$.\\
    Was ist das Inverse von $z=x+y\cdot i$?
    \begin{definition}[Komplexe Konjugation]
        $\overline{z} \definedas \overline{x+yi} \definedas x-yi$
    \end{definition}

    \begin{align*}
        z\cdot\overline{z} &= (x+yi) \cdot (x-yi)\\
        &= (x^2-(yi)^2) = x^2+y^2 = \abs{z}^2\\
        \impl \abs{z} &= \sqrt{z\cdot\overline{z}}\\
        \frac{1}{z} &= \frac{\overline{z}}{z\cdot\overline{z}} = \frac{z}{\abs{z}^2} = \frac{x-yi}{x^2+y^2} = \frac{x}{x^2+y^2} - \frac{y}{x^2+y^2}\cdot i
    \end{align*}

    \newpage

    %%%%%%%%%%%%%%%%%%%%%%%%
    % 21. Dezember 2023
    %%%%%%%%%%%%%%%%%%%%%%%%

    \marginnote{[21. Dez]}

    Wie $\R^d$ wollen wir auch $\C^d$ definieren.

    \begin{definition}[Der Raum $\C^d$]
        \begin{align*}
            \C^d &\definedas \set{\pair{z_1, \dots, z_d} ~\middle | ~ z_j \in\C}\\
            u &= \pair{u_1, \dots, u_d} \in\R^d\\
            w &= \pair{w_1, \dots, w_d} \in\R^d\\
            u+w &= \pair{u_1+w_1,\dots,u_d+w_d}\\
            \lambda\cdot u &= \pair{\lambda u_1,\dots,\lambda u_d}\tag{$\lambda\in\C$}
        \end{align*}
        $\C^d$ ist ein komplexer Vektorraum.

        \begin{align*}
            \abs{u} &\definedas \sqrt{\pair{\sum_{j=1}^{d} \abs{u_j}^2}}\tag{Euklidische Länge}\\
            u &= \pair{u_1, \dots, u_d}\\
            &= \pair{x_1+y_1i, x_2 + y_2 i,\dots, x_d + y_d i}\\
            &= \pair{x_1, x_2,\dots,x_d} + \pair{y_1,y_2,\dots,y_d}i\\
            ???
        \end{align*}
    \end{definition}

    \begin{definition}[]
        \begin{align*}
            \overline{u} &\definedas x - yi\quad (u=x+iy, x,y\in\R^d)\\
            \C^d &= \R^{2d}\\[10pt]
            u &= x + yi\\
            x &= \frac{1}{2}\pair{u+\overline{u}}\quad y = \frac{1}{2i}\pair{u-\overline{u}}
        \end{align*}
    \end{definition}

    Zu ?? im $\R^d$
    \begin{align*}
        \norm{x}_2 &= \sqrt{\pair{\sum_{j=1}^{d} \abs{x_j}^2}} \geq c\\
        \norm{\lambda x}_2 &= \sqrt{\pair{\sum_{j=1}^{d} \abs{\lambda x_j}^2}} = \abs{\lambda}\norm{x}_2
    \end{align*}

    Wann gilt die Dreiecksungleichung?\\
    Auf $\R^d$ gibt es ein Skalarprodukt.\\
    \begin{align*}
        \sprod{x,y} &\definedas \sum_{j=1}^{d} x_j\cdot y_j\tag{$x,y\in\R^d$}\\
        \impl \sprod{x,x} &\definedas \sum_{j=1}^{d} \pair{x_j}^2 = \norm{x}^2_2\\
        \norm{x}_2 &= \sqrt {\sprod{x,x}}
        \intertext{Auch}
        \sprod{x_1+x_2,y} &= \sprod{x_1,y} + \sprod{x_2,y}\\
        \sprod{x,y_1+y_2} &= \sprod{x,y_1} + \sprod{x,y_2}\\
        \sprod{\alpha x,y} &= \alpha \sprod{x,y}\\
        \sprod{x,\alpha y} &= \alpha \sprod{x,y}
    \end{align*}

    \begin{uebung}
        Rechnen Sie die vorherigen Eigenschaften des Skalarprodukts nach.
    \end{uebung}

    \begin{align*}
        \norm{x}_2^2 &= \sprod{x,x}\quad\forall x\in\R^d
        \intertext{Es sei $t\in\R$}
        \norm{x+ty}^2_2 &= \sprod{x+ty, x+ty}\\
        &= \sprod{x,x+ty} + \sprod{ty,x+ty}\\
        &= \sprod{x,x} + \sprod{x+ty} + \sprod{ty,x} + \sprod{ty+ty}\\
        &= t\sprod{x,y}\\
        &= \sprod{x,x} + 2t\sprod{x,y} + t^2\sprod{y,y}
        \intertext{Genauso}
        \norm{x-ty}_2^2 &= \sprod{x-ty, x-ty}\\
        &= \sprod{x,x} - 2t\sprod{x,y} + t^2\sprod{y,y}\\
        &= \norm{x}_2^2 t^2 - 2\sprod{x,y}t + \norm{x}_2^2
        \intertext{Sei $y\neq 0$}
        &= \norm{x}_2^2 \cdot\pair{t^2 - \frac{2\sprod{x,y}}{\norm{y}^2} + \pair{\frac{\sprod{x,y}}{\norm{y}^2}}^2 - \pair{\frac{\sprod{x,y}}{\norm{y}^2}}^2} + \norm{x}_2^2\\
        \intertext{Wir erhalten im Sinne eines Polynoms $a = \norm{y}_2^2\quad b= \sprod{x,y}\quad c= \norm{x}_2^2$}
        &= a\cdot\pair{t^2 - \frac{2b}{a}t + \pair{\frac{b}{a}}^2} + c\\
        &= a\cdot\pair{t-b}^2 - \frac{b^2}{a}+c
        \intertext{Da wir am Anfang der Rechnung eine Norm verwendet haben, muss der Ausdruck nicht-negativ sein}
        a\cdot\pair{t-b}^2 - \frac{b^2}{a}+c &\geq 0\quad\forall t\in\R
        \intertext{Das heißt wir müssen haben}
        \frac{b^2}{a} + c &\geq 0\\
        \equivalent b^2 \leq ac\\
        \equivalent \sprod{x,y}^2 &\leq \norm{y}_2^2 \cdot\norm{x}_2^2\\
        \equivalent \abs{\sprod{x,y}} &\leq \norm{y}_2 \norm{x}_2\tag{Cauchy-Schwarzer-Ungl.}
    \end{align*}

    Und ist $y \neq 0$ und gilt $\abs{\prod{x+y}} = \norm{x}_2 \norm{y}_2$. Dann sind $x$ und $y$ linear abhängig. Das heißt
    \begin{align*}
        \exists t\in\R \text{ mit } x = ty
    \end{align*}
    \begin{proof}
        Dann gilt
        \begin{align*}
            b^2 &= ac\quad \frac{b^2}{a} ac = 0\\
            \impl 0 &\leq \norm{x-ty}^2_2 = a\abs{t-b}^2\\
            &= 0 \text{ für $t=b$ }\\
            \impl \norm{x-ty}_2 &= 0 \text{ für } t=b\\
            \impl x-ty &= 0\\
            \impl x&= ty\qedhere
        \end{align*}
    \end{proof}
    Erkentnis: Aus Cauchy-Schwarzer folgt die Dreiecksungleichung für die Euklidische Norm.

    \begin{align*}
        \norm{x+y}_2^2 &= \sprod{x+y,x+y}\\
        &= \sprod{x,x+y}+ \sprod{y,x+y}\\
        &= \sprod{x,x} + 2 \sprod{x,y} + \sprod{y,y}\\
        &= \norm{x}_2^2 + 2\sprod{x,y} + \norm{y}_2^2\\
        &\leq \norm{x}_2^2 + 2\abs{\sprod{x,y}} + \norm{y}_2^2\\
        &\leq \norm{x}_2^2 + 2\norm{x}_2 \norm{y}_2 + \norm{y}^2\\
        &= \pair{\norm{x}_2 + \norm{y}_2}^2\\
        \impl \norm{x+y}_2 &\leq \norm{x}_2 + \norm{y}_2\tag{Dreiecksungleichung für Eukl. Norm}
    \end{align*}

    \noindent Frage: Was passiert, wenn $\norm{x+y}_2 = \norm{x}_2 + \norm{y}_2$?\\
    (Später)

    \newpage


    \section{Polynome}
    \subsection{Reelle Polynome}
\thispagestyle{pagenumberonly}

\begin{definition}[Reelles Polynom]
    Es sei $a_0,a_1, \dots, a_n\in\realnumbers$,~$a_n \neq 0,~\in\realnumbers$. Dann ist
    \begin{align*}
        P(x) &\definedas a_0 + a_1 \cdot x + a_2 \cdot x^2 + \dots + a_n \cdot x^n
    \end{align*}
    ein reelles Polynom vom Grad $n$ mit $Grad(P) = n$.
\end{definition}

\begin{definition}[Nullpolynom]
    $P(x) = 0$ ist das Nullpolynom. Ein Polynom $P$ ist nicht-trivial, wenn es nicht das Nullpolynom ist.
\end{definition}

\begin{bemerkung}[Analytische Polynome]
    Ähnlich zu reellen Polynomen lassen sich auch analytische Polynome definieren. Es sei $a_0, a_1, \dots, a_n \in\complexnumbers,~a_0\neq 0,~z\in\complexnumbers$. Dann ist
    \begin{align*}
        P(z) &\definedas a_0 + a_1\cdot z + a_2\cdot z^2 + \dots + a_n\cdot z^n
    \end{align*}
    eine analytisches Polynom mit $Grad(P) = n$.
\end{bemerkung}

\begin{definition}[Grad von speziellen Polynomen]
    Wir definieren den Grad von konstanten Polynomen der Form $P(x) = a_0$ als 0 und den Grad des Nullpolynoms als $-1$.
\end{definition}

\begin{satz}[Eigenschaften von Polynomen]
    \theoremescape
    \label{satz:eigenschaften-polynome}
    \begin{enumerate}[label=(\roman*)]
        \item Sei $P$ ein Polynom mit $Grad(P) = n$ und $\lambda\in\realnumbers,~\lambda\neq 0$. Dann ist $Grad(\lambda P) = n$.
        \item Seien $P, Q$ nicht-triviale Polynome mit $Grad(P) = n$ und $Grad(Q)=m$. Dann gilt $PQ$ ist ein Polynom mit $Grad(PQ)=n+m$.
        \begin{proof}
            \begin{align*}
                P(x) &= \sum_{j=0}^{n} a_j x^j\\
                Q(x) &= \sum_{i=0}^{n} b_i x^i\\
                P(x) \cdot Q(x) &= \pair{\sum_{j=0}^{n} a_j x^j}\cdot \pair{\sum_{i=0}^{n} b_i x^i}\\
                &= \sum_{j=0}^{n} \sum_{l=0}^{m} a_{j} b_l\cdot x^{j+l}
                \intertext{Wir setzen $k= j+l\in\set{0,1,\dots,n+m}$}
                &= \sum_{k=0}^{n+m} \pair{\sum_{\substack{j=0\\ 0\leq k-j\leq m}}^{n} a_j b_{k-j}} x^k\\
                &= a_n b_m\cdot x^{n+m} + \textit{Terme niedrigerer Ordnung}\qedhere
            \end{align*}
        \end{proof}
        \item Entwicklung in einem anderen Punkt. Wir schreiben $x = \eta + \zeta$
        \begin{align*}
            P(x) &= \sum_{j=0}^{n} a_j x^j = \sum_{j=0}^{n} a_j \cdot\pair{\eta + \zeta}^j\\
            \annot[{&}]{=}{\ref{satz:binom-lehrsatz}} \sum_{j=0}^{n} a_j \cdot\pair{\sum_{l=0}^{j} \binom{j}{l}\cdot\eta^{l}\cdot\zeta^{j-l}}
            \intertext{Es gilt $\binom{j}{l}=0$, falls $l>j$. Also können wir auch schreiben}
            &= \sum_{j=0}^{n} a_j \cdot\pair{\sum_{l\geq 0} \binom{j}{l}\cdot\eta^{l}\cdot\zeta^{j-l}}\\
            &= \sum_{l\geq 0} \pair{\sum_{j=0}^{n} a_j\cdot \binom{j}{l}\cdot\zeta^{j-l}}\cdot\eta^l\\
            &= \sum_{l\geq 0} \underbrace{\pair{\sum_{j=0}^{n} \binom{j}{l}\cdot a_j\cdot\zeta^{j-l}}}_{\definedasbackwards b_l}\cdot\pair{x-\zeta}^l\\
            &= \sum_{l=0}^{n} b_l \cdot\pair{x-\zeta}^l
            \intertext{Wir betrachten $l=0$}
            b_0 &= \sum_{j=0}^{n} \binom{j}{0}\cdot a_j\cdot\zeta^{j} = \sum_{j=0}^{n} a_j \zeta^j = P(\zeta)\\
            \impl P(x) &= P(\zeta) + \sum_{l=1}^{n} b_l \cdot\pair{x-\zeta}^l\\
            &= P(\zeta) + (x-\zeta) \cdot \underbrace{\sum_{l=0}^{n-1} b_{l+1} \cdot \pair{x-\zeta}^l}_{\definedasbackwards Q(x-\zeta)}\\
            &= P(\zeta) + (x-\zeta)\cdot Q(x-\zeta)
        \end{align*}
        Dabei gilt außerdem $Grad(Q) = n-1$. Diesen Zusammenhang werden wir später in Satz~\ref{satz:nullstellen-polynome} verwenden, um eine Aussage über die Anzahl an Nullstellen eines Polynoms zu treffen.
    \end{enumerate}
\end{satz}

%%%%%%%%%%%%%%%%%%%%%%%%
% 9. Januar 2024
%%%%%%%%%%%%%%%%%%%%%%%%

\begin{satz}[Nullstellensatz für Polynome]
    \marginnote{[9. Jan]}
    \label{satz:nullstellen-polynome}
    Jedes nicht-triviale Polynom von Grad $n$ hat höchstens $n$ Nullstellen.

    \begin{proof}
        Induktion in $n$.~\\
        \begin{induktionsanfang}
            Für $n=0$ (konstantes Polynom) stimmt die Behauptung.
        \end{induktionsanfang}
        \begin{induktionsschritt}
            Angenommen die Behauptung stimmt für Polynom von Grad $n$. Sei $P$ Polynom von Grad $n+1$.\\
            1. Fall: $P$ hat keine Nullstelle $\impl$ Die Behauptung stimmt für $P$.\\
            2. Fall: $P(\zeta) = 0$ für ein $\zeta\in\realnumbers \annot{\impl}{\ref{satz:eigenschaften-polynome}} P(x) = (x-\zeta)\cdot Q\pair{x-\zeta}$. $Q$ ist Polynom von Grad $n$ mit höchstens $n$ Nullstellen. $\impl$ Anzahl der Nullstellen von $P$ ist $\leq n+1$.\qedhere
        \end{induktionsschritt}
    \end{proof}
\end{satz}

\begin{korollar}
    Sind $P,Q$ Polynome von Grad $\leq n$ und stimmen $P,Q$ an $n+1$ verschiedenen Stellen überein, so ist $P=Q$.

    \begin{proof}
        $h\definedas P-Q$ ist Polynom von Grad $\leq n$. Nach Satz~\ref{satz:nullstellen-polynome} hat $h$ damit höchstens $n$ Nullstellen, sofern $h$ nicht-trivial ist. Aber nach Voraussetzung existieren paarweise verschiedene $x_1,\dots, x_{n+1}$ mit
        \begin{alignat*}{2}
            P(x_j) &= Q(x_j)\quad&\forall j=1,\dots,n+1\\
            \impl h(x_j) &= 0\quad&\forall j=1,\dots,n+1
        \end{alignat*}
        Damit ist $h$ nach Satz~\ref{satz:nullstellen-polynome} trivial, das heißt $(\forall x\colon H(x) = 0) \impl (\forall x\colon P(x)=Q(x))$.
    \end{proof}
\end{korollar}

\begin{bemerkung}
    Der vorherige Beweis lässt sich auch auf analytische Polynome in den komplexen Zahlen übertragen.
\end{bemerkung}

\begin{korollar}[Koeffizientenvergleich]
    Zwei Polynome $P,Q$ sind genau dann gleich, wenn sie dieselben Koeffizienten haben.
\end{korollar}

\newpage


    \section{[*] Cauchyprodukt und Exponentialfunktionen}
    \thispagestyle{pagenumberonly}

    \subsection{[*] Cauchyprodukt}
    Frage: Gegeben Reihen $\sum_{n=0}^{\infty} a_n$, $\sum_{n=0}^{\infty} b_n$, beide konvergent. Wie kann man das Produkt $\pair{\sum_{n=0}^{\infty} a_n}\cdot\pair{\sum_{n=0}^{\infty} b_n}$ geschickt berechnen?\\
    \begin{align*}
        u &= \sum_{n=0}^{\infty} a_n = \lim_{n\fromto\infty} s_n\quad s_n \definedas \sum_{l=0}^{n} a_l\\
        v &= \sum_{n=0}^{\infty} b_n = \lim_{n\fromto\infty} t_n \quad t_n \definedas \sum_{j=0}^{n} b_j\\
        \impl u\cdot v &= \lim_{n\fromto\infty} s_n \cdot \lim_{n\fromto\infty} t_n = \lim_{n\fromto\infty} \pair{s_n\cdot t_n}\\[10pt]
        s_n \cdot t_n &= \sum_{l=0}^{n} a_l \cdot \sum_{j=0}^{n} b_j = \sum_{l=0}^{n} \sum_{j=0}^{n} a_l\cdot b_j
    \end{align*}
    Fakt: Indexmenge der Produkte $a_l b_j$ ist $\N_0 \times \N_0$.
    \begin{align*}
        = \set{(l,j):~l,j\in\N_0} &= \N_0\times\N_0
    \end{align*}
    Es gibt (viele) Bijektionen $\sigma: \N_0 \fromto \N_0\times\N_0$ zum Beispiel durch Schrägabzählen.\\
    Frage: Ist $\sigma: \N_0 \fromto \N_0\times\N_0$ eine beliebige Bijektion, gilt dann
    \begin{align*}
        \pair{\sum_{n=0}^{\infty} a_n}\cdot\pair{\sum_{n=0}^{\infty} b_n} &= \sum_{n=0}^{\infty} a_{\sigma_1(n)}\cdot b_{\sigma_2(n)}\\
        \sigma(n) &= \pair{\sigma_1(n), \sigma_2(n)}
    \end{align*}
    Antwort: Im Allgemeinen nein, aber okay, wenn $\sum_{n}^{} a_n$, $\sum_{n}^{} b_n$ absolut konvergent sind.

    \begin{satz} % Satz 1
        \label{satz:cauchyprodukt}
        Seien $\sum_{n=0}^{\infty} a_n$, $\sum_{n=0}^{\infty} b_n$ absolut konvergente Reihen. Dann gilt für jede Bijektion $\sigma: \N_0 \fromto \N_0\times\N_0$
        \begin{align*}
            \pair{\sum_{n=0}^{\infty} a_n} \cdot \sum_{n=0}^{\infty} b_n &= \sum_{n=0}^{\infty} a_{\sigma_1(n)} \cdot b_{\sigma_2(n)}
            \intertext{Das heißt mit $\sigma(n)=\pair{\sigma_1(n), \sigma_2(n)}$}
            c_n &\definedas a_{\sigma_1(n)}\cdot b_{\sigma_2(n)}\\
            u\cdot v &= \sum_{n=0}^{\infty} c_n
            \intertext{mit der Reihe $\sum_{n=0}^{\infty} c_n$ auch absolut konvergent}
        \end{align*}
        Ferner gilt
        \begin{align*}
            u\cdot v &= \sum_{l=0}^{\infty} \pair{\sum_{j=0}^{\infty} a_l\cdot b_j} \tag{Vertikal zuerst}\\
            &= \sum_{j=0}^{\infty} \pair{\sum_{l=0}^{\infty} a_l\cdot b_j}\tag{Horizontal zuerst}
        \end{align*}
        und
        \begin{align*}
            u\cdot v &= \sum_{k=0}^{\infty} \pair{\sum_{0\leq j,l~l+j=k}^{} a_l\cdot b_j} \tag{Schräg abzählen}
        \end{align*}

        \begin{align*}
            \sum_{k=0}^{L} \abs{a_{\sigma_1(k)}\cdot b_{\sigma_2(k)}} &\leq \sum_{0\leq l\leq M_L~0\leq j\leq M_L}^{} \abs{a_l\cdot b_j} = \pair{\sum_{l=0}^{M_L} \abs{a_l}} \cdot \pair{\sum_{j=0}^{M_L} \abs{b_j}}\\
            &\leq \pair{\sum_{l=0}^{\infty} \abs{a_l}} \cdot \pair{\sum_{j=0}^{\infty} \abs{b_j}}\\
            M_L &\definedas \max_{0\leq k\leq L}\pair{\max\pair{\sigma_1(k), \sigma_2(k)}}
        \end{align*}
        \marginnote{[11. Jan]}
        Insbesondere gilt
        \begin{align*}
            \pair{\sum_{j=0}^{\infty} a_j}\cdot\pair{\sum_{k=0}^{\infty} b_k} &= \sum_{n=0}^{\infty} \pair{\sum_{j,k\geq 0~j+k=n}^{} a_j\cdot b_k}\\
            &= \sum_{n=0}^{\infty} \pair{a_n b_0 + a_{n-1} b_1 + \dots + a_0 b_n}\tag{Cauchy-Produkt}
        \end{align*}

        %%%%%%%%%%%%%%%%%%%%%%%%
        % 11. Januar 2024
        %%%%%%%%%%%%%%%%%%%%%%%%

        \begin{proof}
            1. Schritt:
            \begin{align*}
                k' \definedas \sum_{j=0}^{\infty} \abs{a_j} &< \infty\\
                k'' \definedas \sum_{k=0}^{\infty} \abs{b_k} &< \infty\\
                n\in\N_0: \pair{\sum_{j=0}^{n} \abs{a_j}}\cdot\pair{\sum_{k=0}^{n} \abs{b_k}}\\
                &= \sum_{j=0}^{n} \sum_{k=0}^{n} \abs{a_j}\abs{b_k} \leq k'\cdot k''\quad\forall n\in\N_0
                \intertext{Sei}
                \sigma: \N_0\fromto\N\times\N
                \intertext{eine beliebige Bijektion. Behauptung 1: $ \sum_{n=0}^{\infty} c_n$ ist absolut konvergent, wobei}
                c_n &\definedas a_{\sigma_1(n)}b_{\sigma_2(n)} = a_j b_k\tag{$j=\sigma_1(n)$, $k=\sigma_2(n)$}
                \intertext{Sei $L\in\N_0$}
                M_L^1 &\definedas \max_{n=0,1,\dots, L}\pair{\sigma_1(n)}\\
                M_L^2 &\definedas \max_{n=0,1,\dots, L}\pair{\sigma_2(n)}\\
                M_L &\definedas \max\pair{M_L^1, M_L^2}\\[10pt]
                \impl \abs{c_n} &\leq \abs{a_j}\abs{b_n} \text{ für } 0\leq j \leq M_L^1, 0\leq n \leq M_L^2\\
                \sum_{n=0}^{L} \abs{c_n} &\leq \sum_{j=0}^{M_L} \sum_{k=0}^{M_L} \abs{a_j}\abs{b_k}\\
                &= \pair{\sum_{j=0}^{M_L} \abs{a_j}}\cdot\pair{\sum_{k=0}^{M_L} \abs{b_k}} \leq k' \cdot k'' < \infty\quad\forall L\in\N_0\\
                \impl \sum_{n=0}^{\infty} \abs{c_n} &= \sup_{L\in\N_0} \sum_{n=0}^{L} \abs{c_n} \leq k'\cdot k''<\infty
                \intertext{Schritt 2:}
                s&\definedas \sum_{n=0}^{\infty} c_n \in\R
                \intertext{unabhängig von der Reihenfolge, in der man die $c_n$ aufsummiert, wegen absoluter und damit unbedingter Konvergenz. Das heißt für jede Bijektion $\kappa: \N_0\fromto\N_0\times\N_0$ ist}
                s &= \sum_{n=0}^{\infty} a_{\kappa_1(n)}b_{\kappa_2(n)}\tag{$\sigma\circ\kappa^{-1}$ Bijektion}
                \intertext{Schritt 3: Wir zeigen, dass}
                s &= \pair{\sum_{n=0}^{\infty} a_n}\cdot\pair{\sum_{n=0}^{\infty} b_n}
                \intertext{Sei $\sigma:\N_0\fromto\N_0\times\N_0$ Bijektion durch Quadratabzählen. ($c_0=a_{0} b_0$, $c_1 = a_1 b_0$, $c_2=a_1, b_1$, $c_3=a_0 b_2$)}
                \text{Partialsumme } \sum_{k=0}^{n} c_k &= \sum_{k=0}^{n} a_{\sigma_1(k)} b_{\sigma_2(k)} \text{ mit $\sigma$ Quadratabzählen }
                \intertext{Wir betrachten die Folge}
                \sum_{j=0}^{L} \sum_{k=0}^{L} a_j b_k
                \intertext{Da $\sum_{j=0}^{L} \sum_{k=0}^{L} a_j b_k$ die Summe der geschlossenen Quadrate ist, ist diese eine Teilfolge von $\sum_{k=0}^{n} c_k$ und beide haben den gleichen Grenzwert.}
                \impl s=\sum_{k=0}^{n} c_k &= \lim_{L\fromto\infty} \sum_{j=0}^{L} \sum_{k=0}^{L} a_j b_k = \lim_{L\fromto\infty} \pair{\sum_{j=0}^{L} a_j}\cdot\pair{\sum_{k=0}^{L} b_k}\\
                &= \pair{\sum_{j=0}^{\infty} a_j}\cdot\pair{\sum_{k=0}^{\infty} b_k}
                \intertext{Schritt 4: Cauchy-Produkt}
                \sum_{n=0}^{L} \sum_{\substack{0\leq j,k\\ j+k=n}}^{} a_j b_k &= \sum_{n=0}^{L} \pair{\sum_{j=0}^{L} a_j b_{L-j}}
                \intertext{ist Teilfolge der Folge $\sum_{n}^{} c_n$ mittels Schrägabzählen}
                \impl \lim_{L\fromto\infty} \sum_{n=0}^{L} \sum_{j=0}^{L} a_j b_{L-j} &= \sum_{n=0}^{\infty} c_{\sigma(n)} = s\qedhere
            \end{align*}
        \end{proof}
    \end{satz}
    \vspace{0.5cm}

    \par\noindent\rule[0.25\baselineskip]{.37\textwidth}{0.4pt}\hfill Einschub: Abzählungen\hfill\rule[0.25\baselineskip]{.37\textwidth}{0.4pt}

    \begin{definition}[Unendliche Mengen]
        Es sei $A_n\definedas\set{1,2,\dots, n}$. Eine Menge $B$ ist unendlich groß, wenn $B\neq\emptyset$ und keine Bijektion $\kappa: A_n \fromto B$ für ein beliebiges $n$ existiert.
    \end{definition}

    \begin{beispiel}[Vergleich von Kardinalitäten unendlicher Mengen]
        Wir wollen zeigen, dass $\linterv{0,1}$ und $\interv{0,1}$ gleich groß sind. Wir können alle Zahlen auf sich selber abbilden außer der 1. Wir versuchen $1\mapsto\frac{1}{2}$, $\frac{1}{2}\mapsto\frac{1}{3}$, $\frac{1}{3}\mapsto\frac{1}{4}$, \dots.\\
        Damit können wir alle rationalen Zahlen, die sich als Bruch mit 1 im Zähler darstellen lassen, verschieben. Wir definieren:
        \begin{align*}
            \sigma: \interv{0,1}&\fromto\linterv{0,1}\\
            x&\mapsto
            \begin{cases}
                \frac{1}{n+1},\quad &x=\frac{1}{n} \text{ mit } n\in\N\\
                x,\quad &x\in\interv{0,1}\exclude(\bigcup_{n\in\N} \frac{1}{n})
            \end{cases}
        \end{align*}
    \end{beispiel}

    \begin{bemerkung}[Beispiel für eine Abzählung von $\N\times\N$]
        \label{bem:abzaehlen-nxn}
        Wir wollen eine bijektive Abbildung $\sigma: \N\fromto\N\times\N$ konstruieren.\\
        Level $l\in\N: A_l = \set{\pair{j,k}: j+k = l+1,~j,k\in\N}$ (schrägen Diagonalen).\\
        Anzahl Punkte in $\N\times\N$ auf Level $l\leq k$ mit
        \begin{align*}
            \sum_{l=1}^{k} l &= \frac{k(k+1)}{2}
            \intertext{Schreibe}
            n &= \frac{k(k+1)}{2} + r\quad r\in\set{0,1,2,\dots, k}, k\in\N
            \intertext{Das ist eine Eindeutige Zerlegung von $\N$. Definiere}
            \sigma{n} &= \pair{\sigma_1(n), \sigma_2(n)}\\
            &\definedas \pair{k-r, r}\\
            \sigma_1(n) &= k-r\\
            \sigma_2(n) &= r
        \end{align*}
    \end{bemerkung}

    \begin{uebung}
        Weisen Sie die Bijektivität der definierten Funktion $\sigma$ aus Bemerkung~\ref{bem:abzaehlen-nxn} nach.
    \end{uebung}

    \par\noindent\rule{\textwidth}{0.4pt}

    %%%%%%%%%%%%%%%%%%%%%%%%
    % 16. Januar 2024
    %%%%%%%%%%%%%%%%%%%%%%%%

    \begin{bemerkung}
        \marginnote{[16. Jan]}
        Letzter Schritt: Cauchy-Produkt:
        \begin{align*}
            \sum_{n=0}^{\infty} &= \sum_{n=0}^{\infty} \sum_{\nu=0}^{\infty} a_{\nu}\cdot b_{n-\nu}\\
            &= \pair{\sum_{j=0}^{n} a_j}\cdot\pair{\sum_{k=0}^{\infty} b_k}
        \end{align*}
    \end{bemerkung}

    \newpage

    \subsection{Exponentialfunktionen}

    \begin{definition}[Exponentialfunktion]
        Es sei $z\in\C$. Dann gilt
        \begin{align*}
            e^z = \exp(z)\definedas \sum_{n=0}^{\infty} \frac{z^n}{n!}
        \end{align*}
        Außerdem ist $z^0=1$.
    \end{definition}

    \begin{satz}[Eigenschaften der Exponentialfunktion]
        \theoremescape
        \begin{enumerate}[label=(\alph*)]
            \item Für alle $z\in\C$ konvergiert die obige Reihe absolut. (Wohldefiniertheit der $\exp$-Funktion)
            \item Es gilt $\exp(z_1)\cdot\exp(z_2) = \exp(z_1+z_2)$. Insbesondere ist $\exp(z)\neq 0~\forall z\in\C$ und $\exp(z)^{-1} = \exp(-z)$.
            \item $\conj{\exp(z)} = \exp(\conj{z})$
            \item $\abs{\exp(z)} = \exp(\Re(z))$
            \item $e^x = \exp(x) > 0\quad\forall x\in\R$
        \end{enumerate}

        \begin{proof}[Beweis (a)]
            Wir zeigen, dass die Reihe absolut konvergiert.
            \begin{align*}
                a_n &= \frac{1}{n!}\cdot z^n
                \intertext{Nach dem Quotientenkriterium (\ref{satz:quotientenkriterium})}
                \abs{\frac{a_{n+1}}{a_n}} &= \frac{z}{n+1}\fromto 0 \text{ für } n\fromto\infty\\
                \impl \sum_{n=0}^{\infty} a_n &= \sum_{n=0}^{\infty} \frac{z^n}{n!}\text{ konvergiert absolut}\qedhere
            \end{align*}
        \end{proof}
        \begin{proof}[Beweis (b)]
            \begin{align*}
                \exp(z_1) \cdot \exp(z_2) &= \pair{\sum_{j=0}^{\infty} \frac{z^j}{j!}} \cdot \pair{\sum_{k=0}^{\infty} \frac{z^k}{k!}} \annot{=}{\ref{satz:cauchyprodukt}} \sum_{n=0}^{\infty} \sum_{\nu=0}^{n} a_{\nu} b_{n-\nu}\\
                &= \sum_{n=0}^{\infty} \sum_{\nu=0}^{n} \frac{(z_1)^{\nu}}{\nu!}\cdot \frac{(z_2)^{n-\nu}}{(n-\nu)!}\\
                &= \sum_{n=0}^{\infty} \frac{1}{n!}\cdot \underbrace{\sum_{\nu=0}^{n} \frac{n!}{\nu!\cdot(n-\nu)!}\cdot (z_1)^{\nu}\cdot (z_2)^{n-\nu}}_{\text{Binomischer Lehrsatz}}\\
                &= \sum_{n=0}^{\infty} \frac{1}{n!}\cdot(z_1+z_2)^n = \exp(z_1 + z_2)\qedhere
                \intertext{Insbesondere}
                \exp(z)\cdot \exp(-z) &= \exp(z-z) = e^0 = 1\\
                \impl \Bigg\{
                \begin{split}
                    \exp(z), \exp(-z) &\neq 0 \quad \forall z\in\C\\
                    \exp(z)^{-1} &= \exp(-z)
                \end{split}
            \end{align*}
        \end{proof}
        \begin{proof}[Beweis (c)]
            \begin{align*}
                \conj{\exp(z)} &= \conj{\sum_{k=0}^{\infty} \frac{z^k}{k!}} = \sum_{k=0}^{\infty} \conj{\frac{z^k}{k!}} = \sum_{k=0}^{\infty} \frac{\conj{z^k}}{k!} = \sum_{k=0}^{\infty} \frac{\pair{\conj{z}}^k}{k!} = \exp(\conj{z})\qedhere
            \end{align*}
        \end{proof}
        \begin{proof}[Beweis (d)]
            \begin{align*}
                \abs{\exp(z)}^2 &= \conj{\exp(z)}\cdot \exp(z) = \exp(\conj{z})\cdot \exp(z)\\
                &= \exp(\conj{z} + z) = \exp(2\cdot \Re(z))\\
                &=\exp(\Re(z)+ \Re(z)) = \pair{\exp(\Re(z))}^2\\[10pt]
                \impl \abs{\exp(z)} &= \abs{\exp(\Re(z))} = \exp(\Re(z))\qedhere
            \end{align*}
        \end{proof}
        \begin{proof}[Beweis (e)]
            \begin{align*}
                \text{Ist } x \geq 0 &\impl \exp(x) = \sum_{n=0}^{\infty} \frac{x^n}{n!}= 1 + \sum_{n=1}^{\infty} \frac{x^n}{n!}\geq 1\\
                \text{Ist } x < 0 &\impl \exp(x) = \frac{1}{\exp(-x)} > 0\qedhere
            \end{align*}
        \end{proof}
    \end{satz}

    \begin{satz}[Definition von Sinus und Kosinus über Expontentialfunktionen]
        Für $\alpha\in\R$ ist $\abs{\exp(i\alpha)}=1$. Wir setzen
        \begin{align*}
            \cos(\alpha) \definedas \Re (e^{i\alpha}) &= \frac{1}{2} \cdot\pair{e^{i\alpha} + e^{-i\alpha}}\\
            \sin(\alpha) \definedas \Im (e^{i\alpha}) &= \frac{1}{2i}\cdot\pair{e^{i\alpha}-e^{-i\alpha}}
            \intertext{Dann haben wir}
            -1 \leq \cos \alpha &\leq 1\\
            -1 \leq \sin \alpha &\leq 1
            \intertext{Außerdem gilt $\forall \alpha\in\R$}
            \cos(\alpha)^2 + \sin(\alpha)^2 &= 1\\
            \cos(\alpha) + i\cdot\sin(\alpha) &= e^{i\alpha} \tag{Eulersche Gleichung}
        \end{align*}

        \begin{proof}
            Es sei $\alpha\in\R$.
            \begin{align*}
                \abs{\exp(i\alpha)}^2 &= \conj{\exp(i\alpha)}\cdot \exp(i\alpha) = \exp(-i\alpha)\cdot \exp(i\alpha) = \exp(0) = 1\\[10pt]
                \Re(e^{i\alpha}) &= \frac{1}{2}\cdot\pair{e^{i\alpha} + \conj{e^{i\alpha}}} = \frac{1}{2}\cdot\pair{e^{i\alpha} + e^{-i\alpha}} \definedasbackwards \cos \alpha\\
                \Im(e^{i\alpha}) &= \frac{1}{2i}\cdot\pair{e^{i\alpha} - \conj{e^{i\alpha}}}= \frac{1}{2i}\cdot\pair{e^{i\alpha} - e^{-i\alpha}}\definedasbackwards \sin \alpha\\[10pt]
                \impl \abs{\exp(i\alpha)}^2 &= \pair{\Re(\exp(i\alpha))}^2 + \pair{\Im(\exp(i\alpha))}^2\\
                &= \pair{\cos (\alpha)}^2 + \pair{\sin (\alpha)}^2\\[10pt]
                \exp(i\alpha) &= \Re (\exp(i\alpha)) + i\cdot \Im (\exp(i\alpha)) = \cos \alpha + i \cdot \sin \alpha\qedhere
            \end{align*}
        \end{proof}
    \end{satz}

    \newpage


    \section{[*] Potenzreihen}
    \imaginarysubsection{Potenzreihen}
    \thispagestyle{pagenumberonly}

    Wir untersuchen Reihen der Form $ \sum_{n=0}^{\infty} a_n\cdot z^n$ oder $ \sum_{n=0}^{\infty} a_n \cdot \pair{z-z_0}^n$, $z_0\in\C$ fest, $z\in\C$ oder $\R$, $a_n\in\C$.\\
    Partialsummen:
    \begin{align*}
        s_n(z) &\definedas \sum_{j=0}^{n} a_j\cdot z^j
    \end{align*}
    Frage: Konvergenz?
    \begin{beispiel}
        \begin{align*}
            \exp(z) &\definedas \sum_{n=0}^{\infty} \frac{1}{n!} z^n\\
            &= \sum_{n=0}^{\infty} n! \cdot z^n \text{ konvergiert nur für } z=0
        \end{align*}
    \end{beispiel}

    \begin{definition}[Konvergenzradius]
        \begin{align*}
            R &\definedas \sup \set{\abs{z}: z\in\C \text{ und } \sum_{n=0}^{\infty} a_n \cdot z^n \text{ konvergent } }
        \end{align*}
        $R$ ist der Konvergenzradius.
    \end{definition}

    \begin{satz}
        Die Potenzreihe $ \sum_{n=0}^{\infty} a_n z^n$ konvergiert absolut für jedes $z$ in der Kreisscheibe
        \begin{align*}
            B_R (a) &= \set{z\in\C: \abs{z} < R}
        \end{align*}
        Für jedes $\abs{z} > R$ divergiert $ \sum_{n=0}^{\infty} a_n \cdot z^n$.
        \begin{proof}
            Sei $z_1\neq 0$. $ \sum_{n=0}^{\infty} a_n \cdot z^n$ konvergiert.
            \begin{align*}
                \impl (a_n \cdot z^n) \text{ Nullfolge }\\
                \impl \text{ ist beschränkt }\\
                \impl K\definedas \sup_{n\in\N} \set{a_n z^n} < \infty
                \intertext{Sei $0 < r < \abs{z_1}$, $0 < \theta \definedas \frac{r}{\abs{z_1}} < 1$}
                z &\leq \conj{B_r (a)} = \set{\abs{z} \leq r}\\[10pt]
                \abs{a_n z^n} &= \abs{a_n}\abs{z^n} = \abs{a_n} \cdot \abs{z}^n &= \abs{a_n} \cdot \abs{z}^n \frac{\abs{z}}{\abs{z_1}} ^n\\
                &\leq k \theta^n\quad \forall n \geq 0\\
                \impl \sum_{n=0}^{\infty} k \theta^n \text{ konvergente Majorante für } \sum_{n=0}^{\infty} a_n z^n \text{ sofern } 0 \leq z \leq r < \abs{z}\\
                \impl \sum_{n}^{} a_n z^n \text{ konvergiert absolut }\\[12pt]
                \impl (1) \sum_{n=0}^{\infty} a_n z^n \text{ konvergiert für alle } \abs{z} \leq r\\
                \impl (2) \text{ Angenommen } \sum_{n}^{} a_n z^n \text{ konvergiert für ein } \abs{z} > R\\
                (2) \impl \text{ Widerspruch zu Definition von } R
            \end{align*}
        \end{proof}
    \end{satz}

    \begin{bemerkung}
        Konvergiere $ \sum_{n}^{} a_n z^n$ und $ \sum_{n}^{} b_n z^n$ für $\abs{z} < R$
        \begin{align*}
            \impl \sum_{n=0}^{\infty} \pair{\lambda a_n + \mu b_n} z^n &= \lambda \sum_{n=0}^{\infty} a_n z^n + \mu \sum_{n=0}^{\infty} b_n z^n
        \end{align*}
    \end{bemerkung}

    \begin{bemerkung}
        Konvergieren $ \sum_{n=0}^{\infty} a_n z^n$, $ \sum_{n=0}^{\infty} b_n z^n$ auf $B_R (0)$
        \begin{align*}
            \impl \pair{\sum_{n=0}^{\infty} a_n z^n} \cdot \pair{\sum_{n=0}^{\infty} b_n z^n} = \sum_{n=0}^{\infty} \pair{\sum_{\nu=0}^{\infty} a_{\nu} b_{n-\nu}} z^n \tag{Cauchy-Produkt}
            \intertext{Sei $0 < r < R$ für $\abs{z} \leq r$}
        \end{align*}
    \end{bemerkung}

    \begin{bemerkung}
        Wir können auch Potenzreihen der Form
        \begin{align*}
            \sum_{n=0}^{\infty} a_n z^n, \sum_{n=0}^{\infty} a_n \cdot \pair{z-z_0}^n \text{ mit } a_n\in\R^d \text{ oder } \C^d
        \end{align*}
        betrachten. Wir setzen
        \begin{align*}
            R &= \sup \set{\abs{z}: z\in\C und \sum_{n=0}^{\infty} a_n z^n \text{ konvergent } }
        \end{align*}
    \end{bemerkung}

    \begin{satz} % Satz 2
        Die Potenzreihe $ \sum_{n=0}^{\infty} a_n z^n$ konvergiert absolut $\forall z\in B_R (0)$ und divergiert für $\abs{z} > R$.

        \begin{proof}
            Abschreiben des Beweises von Satz 1.
        \end{proof}
    \end{satz}

    \begin{lemma}
        Konvergiert $ \sum_{n=0}^{\infty} a_n z^n$ für ein $z=z_1 \neq 0$ und ist $0 < r < \abs{z_r}$. So ist $ \sum_{n=0}^{\infty} a_n z^n$ auf $B_r (0) = \set{\abs{z} \leq r}$ beschränkt.\\
        Das heißt $\exists M = M_r \geq 0$ mit $\abs{\sum_{n=0}^{\infty} a_n z^n} \leq M_r~\forall \abs{z} \leq r$

        \begin{proof}
            \begin{align*}
                \theta &\definedas \frac{r}{\abs{z_1}} < 1
                \intertext{Da $ \sum_{n=0}^{\infty}  a_n z_1^n$ konvergiert ist}
                (a_n z_n^n)_n \text{ Nullfolge also beschränkt }
                \impl K &= \sup_{n\in\N_0} \abs{a_n z_1^n} = \sup_{n\in\N} \abs{a_n} \abs{z_1}^n < \infty\\
                \intertext{Ist $\abs{z} < r$}
                \abs{a_n z^n} &= \abs{a_n} \abs{z}^n = \abs{a_n} \abs{z_1}^n \abs{\frac{\abs{z}}{\abs{z_1}}}^n\\
                \impl \abs{\sum_{n=0}^{\infty} a_n z^n} &\leq \sum_{n=0}^{\infty} \abs{a_n z}^n \leq \sum_{n=0}^{\infty} k \theta^n = \frac{k}{1-\theta} < \infty\quad \forall \abs{z} \leq r
            \end{align*}
        \end{proof}
    \end{lemma}

    \begin{lemma} % Lemma 4
        \label{lemma:temp-4}
        Annahmen wie bei vorherigem Lemma.
        \begin{align*}
            \impl \text{ Für alle } 0 &< r < \abs{z_1}\quad\forall k\in\N_0\\
            \text{ existiert } M &= M_k, r\geq 0
            \intertext{mit}
            \abs{\sum_{n=k+1}^{\infty} a_n z^n} &\leq M_{k,r} \abs{z}^{k+1}\quad\forall \abs{z} \leq r
        \end{align*}

        \begin{proof}
            \begin{align*}
                \sum_{n=k+1}^{\infty} a_n z^n \text{ konvergiert auch }\\
                \impl \sum_{n=k+1}^{\infty} a_n z^{n-(k+1)} &= z^{-(k+1)} \sum_{n=k+1}^{\infty} a_n z^n \text{ konvergiert }\\
                \impl \text{ verschobene Reihe } \sum_{n=k+1}^{\infty} a_n z^{n-(k+1)} \text{ konvergiert }\\
                \annot{\impl}{Lemma 3} \text{ Für } 0 < r < z_1  \text{ existiert ein } M = M_{k,r} \geq 0
                \intertext{sodass}
                \abs{\sum_{n=k+1}^{\infty} a_n z^{n-(k+1)}} &\leq M_r\quad\forall \abs{z} \leq r < \abs{z_1}\\
                \text{ (linke Seite) } &= \abs{z^{-(k+1)} \sum_{n=k+1}^{\infty} a_n z^n}\\
                &= \frac{1}{\abs{z}^{k+1}} \abs{\sum_{n=k+1}^{\infty} a_n z^n}\\
                \impl \abs{\sum_{n=k+1}^{\infty} a_n z^n} &\leq M_{k,r} \abs{z}^{k+1}
            \end{align*}
        \end{proof}
    \end{lemma}

    \begin{anwendung}
        \begin{align*}
            \sum_{n=0}^{\infty} a_n z^n &= \underbrace{\sum_{n=0}^{k} a_n z^n}_{s_k (z)} + \underbrace{\sum_{n=k+1}^{\infty} a_n z^n}_{Fehler}\\
            \abs{Fehler} &\leq M_{k,r} \abs{z}^{k+1} \leq M_{k,r} \theta^{k+1} \tag{$\theta = \frac{r}{\abs{z_1}}$}
        \end{align*}
    \end{anwendung}

    \newpage

    %%%%%%%%%%%%%%%%%%%%%%%%
    % 18. Januar 2024
    %%%%%%%%%%%%%%%%%%%%%%%%

    Bessere Version von Lemma~\ref{lemma:temp-4}:

    \begin{align*}
        \varphi(z) &= \sum_{n=0}^{\infty} a_n z^n\\
        \abs{\varphi(z) - \sum_{n=k}^{\infty} a_n z^n} &= \abs{\sum_{n=0}^{k} a_n z^n} \leq M_{r,z} \cdot \abs{z}^{k+1}\quad 0< r < \abs{z_1}
        \intertext{Sei $ \sum_{n=0}^{\infty} a_n z^n$ konvergent für ein $z=z_1\neq 0$}
        \impl \forall~ 0< r < \abs{z_1}\colon \text{ existiert }  M_{r,z_1} &> 0
        \intertext{sodass}
        \forall k\in\N_0\colon \abs{\sum_{n=0}^{\infty} a_n z^n} &\leq M_{r,z_1} \abs{\frac{z}{z_1}}^{k+1}\quad\forall \abs{z} \leq r\\
        \abs{\frac{z}{z_1}} &\leq \frac{\abs{z}}{\abs{z_1}} \leq \frac{r}{\abs{z_1}} = \theta < 1
    \end{align*}

    \begin{proof}
        \begin{align*}
            \sum_{n\geq 0}^{} a_n z_1^n \text{ konvergiert }\\
            \impl \pair{a_n z_1^n}_n \text{ ist Nullfolge }\\
            k &\definedas \sup_{n\geq 0} \abs{a_n z_1^n} &< \infty\\
            \impl \abs{a_n z^n} &= \abs{a_n z_1^n \pair{\frac{z}{z_1}}^n} = \abs{a_n z_1^n} \abs{\frac{z}{z_1}}^n\\
            \abs{\frac{z}{z_1}} &\leq \frac{r}{\abs{z_1}} = \theta < 1\\
            \impl \abs{\sum_{n=k+1}^{\infty} a_n z^n} &= \sum_{n=k+1}^{\infty} \abs{a_n z^n}\\
            &\leq \sum_{n=k+1}^{\infty} K \cdot \abs{\frac{z}{z_1}}^n\\
            &= k\cdot \abs{\frac{z}{z_1}}^{k+1} \cdot \sum_{n=0}^{k} \abs{\frac{z}{z_1}}^n\\
            &= \frac{k}{1-\theta} \abs{\frac{z}{z_1}}^{k+1} = \frac{K}{1-\frac{r}{??}}\cdot ??
        \end{align*}
    \end{proof}

    \begin{satz} % Satz 5
        \marginnote{[18. Jan]}

        Sei $ \sum_{n=0}^{\infty}$ eine Potenzreihe, die für ein $z=z_1\neq 0$ konvergiert.\\
        $(z_j)_j$: Folge $0< \abs{z_j} < \abs{z_1}$, $z_j \fromto 0$, $j\fromto\infty$ mit
        \begin{align*}
            \sum_{n=0}^{\infty} a_n (z_j)^n &= 0\quad\forall j\in\N\\
            \impl a_n &= 0\quad\forall n\in\N_0
        \end{align*}
        \begin{proof}
            Angenommen nicht alle $a_n=0$
            \begin{align*}
                \impl B &\definedas \set{n\in\N_0: a_n \neq 0} \neq \emptyset\\
                \impl B \text{ hat ein kleinstes Element, nennen wir } n_0\\
                \impl a_0 &= a_1 = \dots = a_{n_0-1} = 0\quad a_{n_0}\neq 0\\
                f(z) &= \sum_{n=0}^{\infty} a_n z^n = \sum_{n=n_0}^{\infty} a_n z^n = a_{n_0} z^{n_0} + \sum_{n=n_0 + 1}^{\infty} a_n z^n\\
                f(z_1) &= 0\quad\forall j\quad z_j \neq 0\quad z_j\fromto 0\\
                \impl \abs{a_{n_0} (z_j)^{n_0}} &= \abs{- \sum_{n=n_0 + 1}^{\infty} a_n z_j^n} \leq M_{r, z_1} \pair{\frac{\abs{z_j}}{\abs{z_1}}}^{n+1}\\
                \impl \abs{a_{n_0}} \leq M_{r,z} \abs{z_1}^{-(n_0+1)}\quad \abs{z_j} \fromto 0\quad j\fromto\infty\\
                \impl a_{n_0} &= 0\tag{Widerspruch}
            \end{align*}
        \end{proof}
    \end{satz}

    \begin{satz} % Satz 6
        Sei $ \sum_{n=0}^{\infty} a_n z^{n}$, $ \sum_{n=0}^{\infty} b_n z^n$ welche für ein $z=w\neq 0$ konvergieren.\\
        $(z_j)_j$ Folge in $\C$, $z_1\neq 0$, $\forall i, z_i \fromto 0$\\
        mit $ \sum_{n=0}^{\infty} a_n z^n = \sum_{n=0}^{\infty} b_n z_j^n$ für fast alle $z_j$.\\
        Dann ist $a_n = b_n~\forall n\in\N_0$.
        \begin{proof}
            \begin{align*}
                c_n &= a_n - b_n
                \intertext{\OBDA sind alle $\abs{z_1} < \abs{w}$}
                \impl h(z) &\definedas \sum_{n=0}^{\infty} c_n z^n \text{ konvergiert für } z=w\neq 0\\
                \text{ und } h(z_j) &= 0\quad \forall j\\
                \annot{\impl}{Satz 5} c_n &= 0\quad \forall n\in\N_0 \equivalent a_n = b_n \quad\forall n\in\N_0
            \end{align*}
        \end{proof}
    \end{satz}

    \begin{bemerkung}
        Hatten geg. $ \sum_{n=0}^{\infty} a_n z^n$. Dann ist der Konvergenzradius
        \begin{align*}
            R &= \sup \set{\abs{z} : z\in\C \text{ und } \sum_{n=0}^{\infty} a_n z^n \text{ konvergent } }
        \end{align*}
    \end{bemerkung}

    \begin{satz} % Satz 7
        \begin{align*}
            R &= \frac{1}{\limsup_{n\fromto\infty}\sqrt[n]{\abs{a_n}}}
        \end{align*}

        \begin{proof}
            Schritt 1: Zu zeigen: Für $\abs{z} < R$ konvergiert die Potenzreihe.
            \begin{align*}
                M &= \limsup_{n\fromto\infty}\sqrt[n]{\abs{a_n}}\\
                \impl \forall \varepsilon > 0~\exists N \colon \sqrt[n]{\abs{a_n}} &\leq M+\varepsilon\\
                \impl \forall \varepsilon > 0 \text{ ist } \sqrt[n]{\abs{a_n}} &> M- \varepsilon \text{ für fast alle } n
            \end{align*}
            Schritt 2: Zu zeigen: Für $\abs{z} > R$ konvergiert die Potenzreihe nicht. (Übung)
        \end{proof}
    \end{satz}

    \begin{korollar}
        Die Potenzreihe $ \sum_{n=0}^{\infty} a_n z^n$ und $ \sum_{n=1}^{\infty} n a_n z^{n-1}$ haben den gleichen Konvergenzradius.
        \begin{proof}
            Folgt mit vorherigem Satz und $\sqrt[n]{n}\fromto 1$ für $n\fromto\infty$.
        \end{proof}
    \end{korollar}

    \newpage


    \section{Stetige Funktionen einer reellen (oder komplexen) Variablen}
    Um zu kennzeichnen, dass Aussagen sowohl auf $\R$ als auch auf $\C$ valide sind, stehe ab sofort $\K$ stellvertretend für $\R$ oder $\C$ (Körper).\\
Es sei $D\subseteq \R$ (oder $D\subseteq \C$, wenn wir eine komplexe Funktion betrachten) und $D\neq\emptyset$.

\subsection{Das $\varepsilon$-$\delta$-Kriterium}
\thispagestyle{pagenumberonly}

\begin{definition}[$\varepsilon-\delta$ Definition von Stetigkeit]
    Sei $f: D \fromto \K$. Die Funktion $f$ heißt stetig in $x_0$, falls
    \begin{align*}
        \forall \varepsilon > 0~\exists \delta > 0\colon \abs{f(x)-f(x_0)} < \varepsilon \text{ für alle } x\in D \text{ mit } \abs{x-x_0} < \delta
    \end{align*}
    $f$ ist stetig in $D$, falls es in jedem Punkt $x_0\in D$ stetig ist.
\end{definition}

\begin{bemerkung}
    \theoremescape
    \begin{enumerate}[label=\arabic*)]
        \item Wir können Stetigkeit auch etwas anschaulicher betrachten: $x$ ist der Input, $f(x)$ der Output. $f$ ist genau dann stetig in $x_0$, wenn der Output $f(x)$ $\varepsilon$-nahe bei $f(x_0)$ ist, sofern Input $x$ $\delta$-nahe bei $x_0$ ist.
        \item Eine Funktion $f: D\fromto\K^d$ ist stetig in $x_0\in D$, falls
        \begin{align*}
            \forall \varepsilon>0~\exists\delta>0\colon \underbrace{\norm{f(x)-f(x_0)}}_{\text{Norm}} &< \varepsilon\qquad \forall x\in D\colon \abs{x-x_0} <\delta
            \intertext{oder $f:D\fromto V$ (dabei sei $V$ Vektorraum mit Norm $\norm{\cdot}$) ist stetig in $x_0$, falls}
            \forall \varepsilon>0~\exists\delta>0\colon \norm{f(x)-f(x_0)} &< \varepsilon\qquad \forall x\in D\colon \abs{x-x_0} <\delta
        \end{align*}
        \item Stetigkeit ist eine lokale Eigenschaft.
    \end{enumerate}
\end{bemerkung}

\begin{beispiel}[Dirichlet-Funktion]
    Die Funktion
    \begin{align*}
        f(x) &\definedas \begin{cases}
                             1\quad &x\in\Q\\
                             0\quad &x\in\R\exclude\Q
        \end{cases}
    \end{align*}
    ist nirgends stetig auf $\R$.
\end{beispiel}
\begin{beispiel}
    Die Funktion
    \begin{align*}
        f(x) &\definedas \begin{cases}
                             x\quad &x\in\Q\\
                             0\quad &x\in\R\exclude\Q
        \end{cases}
    \end{align*}
    ist nur stetig in $x_0=0$.
\end{beispiel}
\begin{beispiel}
    \begin{align*}
        f(x) &\definedas \begin{cases}
                             \sin\pair{\frac{1}{x}}\quad &x\neq 0\\
                             0\quad &x = 0
        \end{cases}
    \end{align*}
    ist nicht stetig in 0.
\end{beispiel}

\newpage

\begin{lemma} % Lemma 2
    \label{lemma:abschaetzung-stetigkeit}
    Ist $f:D\fromto\K$ (oder $\K^d$) stetig in $x_0\in D$. Dann gilt
    \begin{enumerate}[label=\arabic*.]
        \item $\exists \delta > 0$ mit $\abs{f(x)}\leq 1+\abs{f(x_0)}$\quad$\forall x\in D\colon \abs{x-x_0} < \delta$
        \item Ist $f(x_0) \neq 0$ dann existiert ein $\delta>0$ sodass
        \begin{align*}
            \frac{1}{2}\abs{f(x_0)} &\leq \abs{f(x)} \leq \frac{3}{2} \abs{f(x_0)} \text{ für } x\in D \text{ mit } \abs{x-x_0} < \delta
        \end{align*}
    \end{enumerate}

    \begin{proof}[Beweis (1.)]
        Wähle $\varepsilon = 1$
        \begin{align*}
            \impl \exists\delta > 0\colon &\abs{f(x)-f(x_0)} < \varepsilon\tag{$x\in D,~\abs{x-x_0} < \delta$}\\[10pt]
            \impl \abs{f(x)} = &\abs{f(x)-f(x_0)+f(x_0)}\\
            \leq &\underbrace{\abs{f(x)-f(x_0)}}_{< 1} + \abs{f(x_0)}\leq 1 + \abs{f(x_0)}\qedhere
        \end{align*}
    \end{proof}
    \begin{proof}[Beweis (2.)]
        Ist $f(x_0) \neq 0$, wähle $\varepsilon = \frac{1}{2}\cdot\abs{f(x_0)} > 0$, $\delta > 0$.
        \begin{align*}
            \abs{f(x)-f(x_0)} &< \varepsilon = \frac{1}{2} \abs{f(x_0)} \tag{$x\in D, \abs{x-x_0} < \delta$}\\[10pt]
            \impl \abs{f(x)} &= \abs{f(x_0)+f(x)-f(x_0)}\\
            &\leq \abs{f(x_0)} + \abs{f(x) - f(x_0)}\\
            &\leq \abs{f(x_0)}+\frac{1}{2}\abs{f(x_0)} = \frac{3}{2}\abs{f(x_0)}\\[10pt]
            \abs{f(x)} &= \abs{f(x_0)+f(x)-f(x_0)}\\
            &\geq \abs{f(x_0)} - \abs{f(x)-f(x_0)}\\
            &\geq \abs{f(x_0)} - \frac{1}{2} \abs{f(x_0)} = \frac{1}{2}\abs{f(x_0)}\qedhere
        \end{align*}
    \end{proof}
\end{lemma}

\begin{notation}
    Seien
    \begin{align*}
        f&: D_f \fromto \K\\
        g&: D_g \fromto \K
    \end{align*}
    Wir definieren
    \begin{align*}
        D_{f+g} &\definedas D_f \cap D_g\\
        D_{\frac{f}{g}} &\definedas D_{f+g} \exclude\set{x\in D_g: g(x)=0}
    \end{align*}
    und setzen
    \begin{align*}
        f + g: D_{f+g} &\fromto \K,\quad x \mapsto  f(x) + g(x)\\
        f \cdot g: D_{f+g} &\fromto \K,\quad x \mapsto  f(x) \cdot g(x)\\
        \frac{f}{g}: D_{\frac{f}{g}} &\fromto \K,\quad x \mapsto \frac{f(x)}{g(x)}
    \end{align*}
\end{notation}

\begin{bemerkung}
    $f+g$ geht auch für $f: D_f \fromto \K^d$, $g: D_g \fromto \K^d$. $f\cdot g$ und $\frac{f}{g}$ gehen auch für $f: D_f \fromto \K^d$, $g: D_g \fromto \K$.
\end{bemerkung}

\newpage

\begin{satz} % Satz 3
    \label{satz:stetigkeit-arithmetik}
    Seien $f, g: D\fromto \K$ stetig in $x_0\in D$. Dann gilt
    \begin{enumerate}[label=\arabic*.]
        \item $f+g: D\fromto\K,~x\mapsto f(x)+g(x)$ ist stetig in $x_0$.
        \item $f\cdot g: D\fromto\K,~x\mapsto f(x)\cdot g(x)$ ist stetig in $x_0$.
        \item Ist $g(x_0)\neq 0$, dann ist $\frac{f}{g}: D_{\frac{f}{g}}\fromto\K,~x\mapsto \frac{f(x)}{g(x)}$ stetig in $x_0$.
    \end{enumerate}

    \begin{proof}[Beweis (1.)]
        \begin{align*}
            \abs{f(x)+g(x)-(f(x_0)+g(x_0))} &\leq \abs{f(x)-f(x_0)} + \abs{g(x)-g(x_0)}
            \intertext{Nach der Stetigkeit der beiden einzelnen Funktionen gilt}
            \forall\varepsilon>0~\exists \delta_1, \delta_2 > 0\colon \abs{f(x)-f(x_0)} &< \frac{\varepsilon}{2}\tag{$x\in D, \abs{x-x_0} < \delta_1$}\\
            \abs{g(x)-g(x_0)} &< \frac{\varepsilon}{2}\tag{$x\in D, \abs{x-x_0} < \delta_2$}
            \intertext{Ist $\delta\definedas\min \set{\delta_1, \delta_2} > 0$}
            \impl \forall\varepsilon>0\colon \abs{f(x)+g(x)-(f(x_0)+g(x_0))} &< \frac{\varepsilon}{2} + \frac{\varepsilon}{2} = \varepsilon\tag{$x\in D, \abs{x-x_0} < \delta$}
        \end{align*}
    \end{proof}

    \begin{proof}[Beweis (2.)]
        Aus Lemma~\ref{lemma:abschaetzung-stetigkeit} folgt
        \begin{align*}
            \exists \delta_0 > 0\colon \abs{g(x)} &\leq 1 + \abs{g(x_0)} \qquad\forall x\in D\colon \abs{x-x_0} < \delta_0
            \intertext{Mit der Stetigkeit von $f$ und $g$ in $x_0$ gilt}
            \forall\varepsilon >0~\exists \delta_1, \delta_2 > 0\colon \abs{f(x)-f(x_0)} &< \frac{\varepsilon}{2\cdot(1+\abs{g(x_0)})}\tag{$x\in D, \abs{x-x_0} < \delta_1$}\\
            \abs{g(x)-g(x_0)} &< \frac{\varepsilon}{2\cdot(1+\abs{f(x)})}\tag{$x\in D, \abs{x-x_0} < \delta_2$}
            \intertext{Wir definieren $\delta\definedas \min(\delta_0, \delta_1, \delta_2)$}
            \impl \abs{f(x)\cdot g(x)-(f(x_0)\cdot g(x_0))} &\leq \abs{f(x)-f(x_0)}\cdot\abs{g(x)} + \abs{f(x_0)}\cdot\abs{g(x)-g(x_0)}\\
            &\leq \frac{\varepsilon}{2\cdot(1+\abs{g(x_0)})}\cdot \abs{g(x)} + \frac{\varepsilon}{2\cdot(1+\abs{f(x)})}\cdot \abs{f(x_0)}\\
            &\leq \frac{\varepsilon}{2\cdot(1+\abs{g(x_0)})}\cdot \pair{1+\abs{g(x_0)}} + \frac{\varepsilon}{2\cdot(1+\abs{f(x)})}\cdot \pair{1+\abs{f(x)}}\\
            &\leq \frac{\varepsilon}{2} + \frac{\varepsilon}{2} = \varepsilon\tag{$x\in D, \abs{x-x_0} < \delta$}
        \end{align*}
    \end{proof}

    \begin{proof}[Beweis (3.)]
        Es reicht aus, zu zeigen, dass $\frac{1}{g(x)}$ in $x_0$ stetig ist. Dann können wir (2.) anwenden. Wir wählen den Ansatz
        \begin{align*}
            \abs{\frac{1}{g(x)} - \frac{1}{g(x_0)}} &= \frac{\abs{g(x_0)-g(x)}}{\abs{g(x)\cdot g(x_0)}}
        \end{align*}
        und schätzen den Term geschickt ab.
    \end{proof}
    \begin{uebung}
        Beweisen Sie Teil (3.) des vorherigen Satzes mit einer geschickten Abschätzung.
    \end{uebung}
\end{satz}

%%%%%%%%%%%%%%%%%%%%%%%%
% 23. Januar 2023
%%%%%%%%%%%%%%%%%%%%%%%%

\subsection{Stetige Funktionen und Folgenkriterium}

\begin{satz}[Stetigkeit von Polynomen] % Satz 3
    \marginnote{[23. Jan]}
    \label{stetigkeit:polynome}
    Für eine stetige Funktion $f$ und eine Zahl $a$, ist auch $a\cdot f$ stetig. Außerdem ist nach Satz~\ref{satz:stetigkeit-arithmetik} $x^2$ bzw. $z^2$ stetig. Es lässt sich induktiv zeigen, dass $x^n$ bzw. $z^n$ damit auch stetig sein müssen.\\
    Daraus folgt, dass alle Polynome (in $\K$) stetig auf ganz $\K$ sind. Außerdem ist für $P,Q$ Polynome mit $Q(x)\neq 0$ auch die Funktion
    \begin{align*}
        R(x) &= \frac{P(x)}{Q(x)}
    \end{align*}
    stetig auf $\K\exclude\set{x\in\K: Q(x) = 0}$.
\end{satz}

\begin{korollar} % Korollar 4
    Alle Polynome (reell oder komplex) sind stetig und alle rationalen Funktionen $R(x) = \frac{P(x)}{Q(x)}$ sind stetig auf $D_R \definedas \K\exclude\set{x\in\K: Q(x) = 0}$.
\end{korollar}

\begin{uebung}[Stetigkeit von Exponentialfunktionen]
    Es gilt
    \begin{align*}
        e^x-e^0 &= e^x - 1 = \sum_{n=0}^{\infty} \frac{x^n}{n!}-1 = \sum_{n=1}^{\infty} \frac{x^n}{n!}
    \end{align*}
    Weisen Sie basierend auf dieser Gleichung und den Abschätzungen für Potenzreihen die Stetigkeit der Funktion $e^x$ zunächst im Punkt $x_0 = 0$ und dann auf ganz $\R$ nach.
\end{uebung}

\begin{satz}[Folgenkriterium für Stetigkeit] % Satz 5
    \label{satz:stetigkeit-folgenkriterium}
    Sei $f: D\fromto\K$ (oder $\K^d$) mit $D\subseteq\K$. Dann gilt $f$ ist genau dann stetig in $x_0$, wenn für alle Folgen $(x_n)_n \subseteq D$ mit $x_n \fromto x_0$ folgt, dass $f(x_n) \fromto f(x_0)$ für $\ntoinf$.

    \begin{proof}
        \anf{$\impl$}: $f$ ist stetig in $x_0$. Dann gilt
        \begin{align*}
            \forall\varepsilon > 0~\exists\delta>0\colon \abs{f(x)-f(x_0)} &< \varepsilon\tag{$x\in D$, $\abs{x-x_0}<\delta$}
            \intertext{Sei $(x_n)_n\subseteq D$ Folge mit $x_n\fromto x_0$. Das heißt}
            \forall \hat{\varepsilon} > 0~\exists N\in\N\colon \abs{x_n-x_0} &< \hat{\varepsilon}\quad\forall n\geq N\\
            \intertext{Gegeben $\varepsilon$ verwenden wir die Stetigkeits-Eigenschaften}
            \exists \delta>0\colon \abs{f(x)-f(x_0)} &< \varepsilon\quad \forall x\in D\colon \abs{x-x_0} < \delta\\
            \intertext{nehmen $\hat{\varepsilon} = \delta$}
            \impl \exists N\in\N\colon \abs{x_n-x_0} &< \delta\quad\forall n\geq N\\
            \impl \abs{f(x_n)-f(x_0)} &< \varepsilon\quad\forall n\geq N_0
        \end{align*}
        Mit der Definition der Konvergenz folgt dann $f(x_n) \fromto f(x_0)$.\\[10pt]
        \anf{$\Leftarrow$}: Wir verwenden Kontraposition. Wenn $f$ nicht stetig in $x_0$ ist, dann
        \begin{align*}
            \impl \exists \varepsilon > 0~\forall \delta > 0~\exists x\in D\colon& \pair{\abs{x-x_0} < \delta} \land \pair{\abs{f(x)-f(x_0)} \geq \varepsilon > 0}
            \intertext{Wählen $\delta = \frac{1}{n}$, $n\in\N$}
            \impl \fa n\in\N \ex x_n\in D\colon& \abs{x_n-x_0} < \frac{1}{n} \text{ und } \abs{f(x_n)-f(x_0)} \geq \varepsilon
        \end{align*}
        Das heißt wir haben ein $(x_n)_n\subseteq D, x_n \fromto x_0$ und $f(x_n)$ konvergiert nicht gegen $f(x_0)$.
    \end{proof}
\end{satz}

\begin{satz}[Stetigkeit unter Verkettung] % Satz 6
    \label{satz:verkettung-stetigkeit}
    Sei $f: D_f \fromto\R$, $D_g\subseteq\R$, $g: D_g\fromto\R$ und $f(D_f)\subseteq D_g$. Sind $f$ stetig in $x_0$ und $g$ stetig in $y_0\definedas f(x_0)\in D_g$. Dann ist $g\circ f: D_f\fromto\R$ stetig in $x_0$.

    \begin{proof}
        Sei $(x_n)_n\subseteq D_f$ mit $x_n\fromto x_0$. Dann folgt aus Satz~\ref{satz:stetigkeit-folgenkriterium}, dass $y_n \definedas f(x_n) \fromto f(x_0) = y_0$. Wenn wir den Satz noch ein zweites Mal anwenden, muss gelten, dass $g(y_n)\fromto g(y_0)$. Also gilt für alle Folgen $(x_n)_n\subseteq D_f$, dass
        \begin{align*}
        (g\circ f)
            \of{x_n} = g(f(x_n)) = g(y_n) \fromto g(y_0) = g(f(x_0)) = (g\circ f)\of{x_0}
        \end{align*}
        Nach Satz~\ref{satz:stetigkeit-folgenkriterium} ist damit $g\circ f$ stetig in $x_0$.
    \end{proof}
\end{satz}

\begin{uebung}
    Zeigen Sie Satz~\ref{satz:verkettung-stetigkeit} mit dem $\varepsilon$-$\delta$-Kriterium der Stetigkeit.
\end{uebung}

\newpage


    \section{Der Zwischenwertsatz}
    \setcounter{subsection}{1}
\thispagestyle{pagenumberonly}

Stetigkeit sagt uns, dass die Werte, die eine Funktion für nahe $x_1$ und $x_2$ annimmt, nah beieinander liegen. Können wir also davon ausgehen, dass eine beliebige stetige Funktion, die zwei unterschiedliche Werte hat, auch alle Werte dazwischen annimmt?

\begin{beispiel}[Fehlende Zwischenwerte für nicht-reelle stetige Funktionen]
    Die Funktion
    \begin{minipage}{.5\textwidth}
        \begin{align*}
            f: \Q &\fromto \R\\
            x &\mapsto
            \begin{cases}
                1\quad &x^2 > 2\\
                -1\quad &x^2 < 2
            \end{cases}
        \end{align*}
    \end{minipage}
    \begin{minipage}{.5\textwidth}
        \centering
        \begin{figure}[H]
            \begin{tikzpicture}
                \draw[->] (-3,0) -- (3,0);
                \draw[->] (0,-1.25) -- (0,1.25);
                \draw (-3, 0.6) -- (-1, 0.6);
                \draw (1, 0.6) -- (3, 0.6);
                \draw (-1, -0.6) -- (1, -0.6);
                \draw[dotted] (-1, 0.6) -- (-1, -0.6) node[above] {\footnotesize$-\sqrt{2}~~$};
                \draw[dotted] (1, 0.6) -- (1, -0.6) node[above] {\footnotesize$\sqrt{2}$};
            \end{tikzpicture}
        \end{figure}
    \end{minipage}
    nimmt nur die Werte $-1$ und $1$ an, aber ist eine stetige Funktion auf $\Q$, weil $\sqrt{2}\not\in\Q$. (Übung)
\end{beispiel}

\begin{satz}[Zwischenwertsatz] % Satz 1
    \label{satz:zwischenwertsatz}
    Sei $f: \interv{a,b}\fromto\R$ stetig und $f(a)\neq f(b)$. Dann gibt es zu jedem $c$ zwischen $f(a)$ und $f(b)$ (d.h. $f(a) < c < f(b)$ oder $f(b) < c < f(a)$) ein $\zeta\in\interv{a,b}$ mit $f(\zeta) = c$.

    \begin{proof}
        O.B.d.A. sei $f(a) < f(b)$ (sonst ersetzen wir $f$ durch $-f$).\\[10pt]
        Schritt 1: Sei $f(a) < 0 < f(b)$ und $c=0$. Wir setzen $M\definedas\set{x\in\interv{a,b}: f(x) < 0}$. Dann ist $a\in M \impl M\neq\emptyset$. Wir setzen $\zeta\definedas \sup M\in\R$.\\
        Angenommen $f(\zeta) \neq 0$. Dann gilt indirekt nach Lemma~\ref{lemma:abschaetzung-stetigkeit}\footnote{Die gleiche Aussage lässt sich auch direkt über das $\varepsilon$-$\delta$-Kriterium und Stetigkeit zeigen}
        \begin{align*}
            \impl \exists \delta > 0~\forall x\in\pair{\zeta-\delta, \zeta+\delta} \cap\interv{a,b} \text{ hat } f(x) \text{ das gleiche Vorzeichen wie } f(\zeta)\tag{1}
        \end{align*}
        1. Fall: $f(\zeta)>0$.
        \begin{align*}
            \annot{\impl}{(1)} \exists \delta > 0\colon f(x) &> 0\quad\forall x\in\interv{a,b}\cap\pair{\zeta-\delta, \zeta+\delta}\\
            \impl f(x) &> 0\quad\fa \zeta-\frac{\delta}{2} \leq x \leq \min\set{b, \zeta+\delta}
            \intertext{Da $\zeta$ eine obere Schranke für $M$ ist, muss dann auch $\zeta-\frac{\delta}{2}$ eine obere Schranke für $M$ sein, weil die Werte von $f$ dazwischen positiv bleiben.}
            \impl \zeta-\frac{\delta}{2} &\text{ ist kleinere obere Schranke für } M \text{ als } \zeta\tag{Widerspruch}
        \end{align*}
        2. Fall: $f(\zeta) < 0$.
        \begin{align*}
            \annot[{&}]{\impl}{(1)} \exists\delta > 0\colon f(x) < 0 \quad\forall x\in\interv{a,b}\cap \pair{\zeta-\delta, \zeta+\delta}
            \intertext{Wir können \OBDA annehmen, dass $\zeta+\delta \leq b$}
            &\impl f\of{\zeta+\frac{\delta}{2}} < 0 \impl \zeta+\frac{\delta}{2}\in M\\
            &\impl \zeta \text{ keine obere Schranke in } M
        \end{align*}
        Damit ist $f\of{\zeta} = 0$.\\[10pt]
        Schritt 2: Wenn wir ein $c'\neq 0$ zwischen $f(a)$ und $f(b)$ wählen wollen, können wir unsere Funktion einfach um $c'$ verschieben und dann gilt nach Schritt 1, dass $f(\zeta) - c' = 0 \impl f(\zeta) = c'$.
    \end{proof}
\end{satz}

\begin{bemerkung}
    Dieser Satz funktioniert nicht auf $\C$, da wir die Anordnung der reellen Zahlen verwenden und funktioniert nach dem vorherigen Beispiel nicht auf unvollständigen Mengen.
\end{bemerkung}

\newpage


    \section{Der Satz von Weierstraß}
    \thispagestyle{pagenumberonly}

\subsection{Beschränkte, abgeschlossene und kompakte Mengen}

\begin{definition}[Beschränkte Mengen] % Definition 1
    Eine Menge $A\subseteq\R$ ist beschränkt, falls
    \begin{align*}
        \exists R>0\colon &A\subseteq\pair{-R, R}
        \intertext{bzw. eine Menge $A\sbset\C$ ist beschränkt, falls}
        \exists R>0\colon &A\subseteq B_R(0) \definedas\set{z\in\C: \abs{z} < R}
    \end{align*}
\end{definition}

\begin{definition}[Abgeschlossene Mengen]
    Eine Menge $A$ ist abgeschlossen, falls für jede konvergente Folge $(x_n)_n\subseteq A$ gilt, dass $\lim_{n\fromto\infty} x_n \in A$.
\end{definition}

\begin{beispiel}
    $\interv{a,b}$ ist abgeschlossen und beschränkt für $-\infty < a < b < \infty$. $\pair{0,1}$ ist beschränkt, aber nicht abgeschlossen.
\end{beispiel}

\begin{definition}[Kompakte Mengen] % Definition 2
    $K\subseteq\R$ (oder $\C$) ist kompakt, falls jede Folge $(x_n)_n\subseteq K$ eine in $K$ konvergente Teilfolge besitzt. Das heißt
    \begin{enumerate}[label=\arabic*.]
        \item $(x_n)_n$ hat konvergente Teilfolge $(x_{n_j})_j$ und
        \item Der Grenzwert der Teilfolge $\biglim{j\fromto\infty} x_{n_j} \in K$
    \end{enumerate}
\end{definition}

\begin{satz} % Satz 3
    \label{satz:kompaktheit}
    $K\subseteq\K$ ist genau dann kompakt, wenn $K$ abgeschlossen und beschränkt ist.

    \begin{proof}
        \anf{$\Leftarrow$}: $K$ ist abgeschlossen und beschränkt. Und $(x_n)_n\subseteq K$ sei beliebige Folge in $K$. Dann ist $(x_n)_n$ beschränkte Folge. Nach Korollar~\ref{korollar:beschr-konv-teilfolge} existiert eine konvergente Teilfolge $(x_{n_j})_j$ von $(x_n)_n$. Da $K$ abgeschlossen ist, gilt
        \begin{align*}
            \lim_{j\toinf} x_{n_j}\in K
        \end{align*}
        \anf{$\impl$}: Sei $K$ kompakt. Dann ist $K$ auch abgeschlossen. (Folgt aus Definition)\\
        Wenn $K$ nicht beschränkt wäre, existiert für jedes $n\in\N$ ein $x_n\in K$ mit $\abs{x_n} > n$. Die Folge $(x_n)_n$ hat dann aber keine konvergente Teilfolge.
    \end{proof}
\end{satz}

\subsection{Kompaktheit unter Abbildungen}

\begin{satz} % Satz 4
    \label{satz:kompaktheit-unter-abbildung}
    Sei $K\subseteq\K$ kompakt und $f: K\fromto\K$ stetig. Dann ist $f(K) = \set{f(x): x\in K}$ kompakt.
    \begin{proof}
        Sei $(y_n)_n$ Folge in $f(K)$, also $(y_n)_n\sbset\bild\of{f}$. Dann folgt
        \begin{align*}
            \impl \fa n\in\N\ex& x_n\in K\colon f(x_n) = y_n
            \intertext{Nach Kompaktheit von $K$ existiert eine Teilfolge $(x_{n_j})_j$ von $(x_n)_n$ welche gegen ein $x_0\in K$ konvergiert}
            x_{n_j} &\fromto x_0\in K
            \intertext{Wir definieren}
            (y_{n_j})_j &\definedas f(x_{n_j})
            \intertext{Nach Satz~\ref{satz:stetigkeit-folgenkriterium} und der Stetigkeit von $f$ gilt}
            y_{n_j} = f(x_{n_j}) &\fromto f(x_0) \definedasbackwards y_0
        \end{align*}
        Das heißt Teilfolge $(y_{n_j})_j$ konvergiert und $y_0=f(x_0) \in f(K)$. Nach Def. ist $f(K)$ kompakt.
    \end{proof}
\end{satz}

%%%%%%%%%%%%%%%%%%%%%%%%
% 25. Januar 2023
%%%%%%%%%%%%%%%%%%%%%%%%

\begin{korollar} % Korollar 5
    \label{korollar:stetigkeit-unter-abbildung}
    \marginnote{[25. Jan]}
    Ist $f: K\fromto\K$ stetig und ist $K$ kompakt, so ist $f(K)$ abgeschlossen und beschränkt.
    \begin{proof}
        Folgt direkt aus Satz~\ref{satz:kompaktheit} und Satz~\ref{satz:kompaktheit-unter-abbildung}.
    \end{proof}
\end{korollar}

\begin{beispiel}
    $f: \R \fromto\R,~x\mapsto \frac{1}{1+x^2}$ stetig. Aber $\bild(f) = f(\R) = \rinterv{0,1}$ nicht kompakt.
\end{beispiel}

\subsection{Der Hauptlehrsatz von Weierstraß}

\begin{satz}[Weierstraß' Hauptlehrsatz] % Satz 6
    \label{satz:weierstrass-maximum-minimum}
    Ist $K\neq \emptyset$ kompakt und $f: K\fromto\R$ stetig so nimmt $f$ sein Maximum und Minimum an. Das heißt
    \begin{align*}
        \exists \underline{x}\in K,\overline{x}\in K\colon f(\underline{x}) \leq f(x) \leq f(\overline{x}) \quad\forall x\in K
    \end{align*}
    Das heißt es gilt
    \begin{align*}
        f(\underline{x}) = \inf_{x\in K} f(x)\quad f(\overline{x}) = \sup_{x\in K} f(x)
    \end{align*}
    \begin{proof}
        Nach Korollar~\ref{korollar:stetigkeit-unter-abbildung} $\impl f(K)$ ist beschränkt und abgeschlossene Teilmenge von $\R$.\\
        \begin{align*}
            \impl M&\definedas \sup_{x\in K} f(x) \in \R\\
            m&\definedas \inf_{x\in K} f(x) \in \R\\
            \impl m&\leq f(x)\leq M\quad\forall x\in K
        \end{align*}
        Schritt 1: $\exists \overline{x}\in K\colon f(\overline{x}) = M$.
        \begin{proof}
            Da $M=\sup_{x\in K} f(x)$ gilt
            \begin{align*}
                \forall \varepsilon >0~\exists x\in K\colon f(x) &> M-\varepsilon
                \intertext{Wähle $\varepsilon=\frac{1}{n}$}
                \impl \ex x_n\in K\colon f(x_n) &> M - \frac{1}{n}\\
                \impl \ex \text{maximierende Folge } (x_n)_n &\sbset K \text{ mit } \underbrace{M-\frac{1}{n}}_{\fromto M} < f(x_n) \leq M
                \intertext{Das heißt nach Satz~\ref{satz:sandwich} geht $f(x_n)\fromto M$ für $\ntoinf$. Da $K$ kompakt ist, hat $(x_n)_n$ eine konvergente Teilfolge $(x_{n_j})_j$ mit Grenzwert $\overline{x}\definedas \biglim{j\toinf} x_{n_j} \in K$}
                \impl M - \frac{1}{n_j} \leq f(x_{n_j}) &< M\\
                \impl M \leq f(\overline{x}) &< M\\
                \impl \ex \overline{x}\in K\colon f(\overline{x}) &= M\qedhere
            \end{align*}
        \end{proof}
        \noindent Schritt 2: Entweder minimierende Folge $(y_n)_n$ mit $m\leq f(y_n) < m+\frac{1}{n}$. Dann Teilfolge, etc.\\
        Oder: Wende Schritt 1 auf $h: K\fromto\R,~x\fromto -f(x)$ an.
    \end{proof}

    \begin{beispiel}
        Die Funktion $f: \R\fromto\R$ mit
        \begin{align*}
            f(x) &= \frac{1}{x^2+1}\quad\impl\quad 0 < f(x)\leq 1~\fa x\in\R
        \end{align*}
        ist auf $\R$ (nicht kompakt) definiert und nimmt ihr Maximum, aber nicht ihr Minimum an.
    \end{beispiel}
\end{satz}

\newpage


    \section{Grenzwerte von Funktionen}
    \subsection{Definition und Grenzwertsätze}
\thispagestyle{pagenumberonly}

\begin{definition}[Häufungspunkte]
    $A\subseteq\K$ hat den Häufungspunkt $x_0\in\K$, falls
    \begin{align*}
        \forall\varepsilon > 0~\exists x\in A\exclude\set{x_0}\colon \abs{x-x_0} < \varepsilon
    \end{align*}
\end{definition}

\begin{definition}[Diskrete Punkte]
    Für $A\subseteq\K$ ist $x_0\in A$ ein diskreter Punkt, falls
    \begin{align*}
        \exists \varepsilon > 0\colon \abs{x-x_0} \geq \varepsilon\quad\forall x\in A\exclude\set{x_0}
    \end{align*}
\end{definition}

\begin{bemerkung}
    \theoremescape
    \begin{enumerate}[label=(\roman*)]
        \item Häufungspunkte müssen keine Elemente von $A$ sein.
        \item $x_0$ ist genau dann ein Häufungspunkt von $A$, wenn $\exists$ Folge $(x_n)_n\subseteq A\exclude\set{x_0}$ mit $x_n\fromto x_0$ für $n\fromto\infty$ (weil $0<\abs{x_n-x_0}\fromto 0$)
    \end{enumerate}
\end{bemerkung}

\begin{definition}[Grenzwerte von Funktionen]
    Sei $f: D\fromto\R$ (oder $\R^d$) und $x_0$ Häufungspunkt von $D$. Wir sagen $f(x)$ strebt gegen $a$ bei Annäherung von $x$ gegen $x_0$ -- geschrieben $f(x)\fromto a$ für $x\fromto x_0$ oder $\biglim{x\fromto x_0} f(x) = a$ -- falls
    \begin{align*}
        \forall \varepsilon > 0~\exists \delta > 0\colon \abs{f(x)-a} < \varepsilon \text{ für alle } x\in D,~0 < \abs{x-x_0} < \delta
    \end{align*}
    Wir sagen $a$ ist der Limes oder Grenzwert von $f(x)$ oder $f(x)$ konvergiert gegen $a$ für $x\fromto x_0$.
\end{definition}

\begin{bemerkung}
    Es gilt $f(x)\fromto a$ für $x\fromto x_0$ genau dann, wenn
    \begin{align*}
        \forall\varepsilon > 0~\exists \delta > 0\colon \abs{f(x)-a} &< \varepsilon\quad\fa x\in D\cap \dot{B}_{\delta}(x_0)\\
        \dot{B}_{\delta}(x_0) &\definedas \set{x~\middle|~0 < \abs{x-x_0} < \delta}\\
        \dot{B}_{\delta}(x_0) &= B_{\delta}(x_0) \exclude\set{x_0}\tag{punktierter $\delta$-Ball um $x_0$}
    \end{align*}
\end{bemerkung}

\begin{satz}
    \label{satz:funktionen-grenzwerte-folgenkrit}
    Sei $f: D\fromto\R$ (oder $\R^d$) und $x_0$ Häufungspunkt von $D$. Dann gilt $\biglim{x\fromto x_0} f(x) = a$ genau dann, wenn für jede Folge $(x_n)_n\subseteq D\exclude\set{x_0}$ mit $x_n\fromto x_0$ folgt $ \biglim{n\fromto\infty} f(x_n) = a$.
    \begin{proof}
        \anf{$\impl$}: Klar nach Definition. (Selber machen)\\
        \anf{$\Leftarrow$}: Kontraposition. Angenommen $ \biglim{n\fromto\infty} f(x) \neq a$.
        \begin{align*}
            \impl \exists\varepsilon > 0~\forall \delta > 0&\colon \abs{f(x)-a} \geq \varepsilon \text{ für ein } x\in D\exclude\set{x_0}\colon \abs{x-x_0} < \delta
            \intertext{Nehmen $\delta = \frac{1}{n}$}
            \impl \exists \text{ Folge } &(x_n)_n\subseteq D\exclude\set{x_0} \text{ mit } x_n\fromto x_0 \text{ und } \abs{f(x_n)-a} \geq \varepsilon\\
            \impl f(x_n) &\text{ konvergiert nicht gegen } a\qedhere
        \end{align*}
    \end{proof}
\end{satz}

\begin{beispiel}
    Für $D = \R\exclude\set{1}$, $f(x) = \frac{x^2-1}{x-1}$ gilt $\biglim{x\fromto 1} f(x) = 2$\\
    \begin{align*}
        f(x) = \frac{x^2-1}{x-1} = \frac{(x+1)(x-1)}{x-1} = x + 1 \fromto 2 \text{ für } x\fromto 1
    \end{align*}
\end{beispiel}

\newpage

\begin{beispiel}[Ausblick: Differenzierbarkeit und Ableitung]
    $f: \interv{0,T}\fromto \R^3$ Kurve.
    \begin{align*}
        \varphi(h) &= \frac{f(t+h)-f(t)}{h}\tag{$h\neq 0$}
        \intertext{Falls $\biglim{h\fromto 0} \varphi(h)$ existiert, nennen wir $f$ in $t$ differenzierbar und definieren die Ableitung}
        f'(t) &\definedas \lim_{h\fromto 0} \varphi(h)
    \end{align*}
\end{beispiel}

\begin{lemma}[Zerlegung von Grenzwerten im $\R^d$] % Lemma 4
    Für $f: D\fromto\R^d$, $f=(f_1, \dots, f_d)$, $x_0$ Häufungspunkt von $D$, $a=(a_1, \dots, a_d)$ gilt
    \begin{align*}
        \lim_{x\fromto x_0} f(x) = a\quad\equivalent\quad \lim_{n\fromto x_0} f_j(x) = a_j~~\forall 1\leq j\leq d
    \end{align*}

    \begin{proof}
        Sei $\norm{f(x)-a}$ Euklidischer Abstand von $f(x)$ zu $a$.
        \begin{align*}
            \norm{f(x)-a} &= \sqrt{\sum_{j=1}^{d} \pair{f_j(x) - a_j}^2}\\
            \impl \forall 1\leq j\leq d\colon \abs{f_j(x) - a_j} &\leq \norm{f(x) - a} \leq \sqrt{d}\cdot\max_{1\leq l\leq d} \abs{f_l(x) - a_l}\qedhere
        \end{align*}
    \end{proof}
\end{lemma}

\begin{satz} % Satz 5
    \label{satz:funktionen-grenzwerte-arithmetik}
    Es sei $D\subseteq\R$ und $f,g: D\fromto \C$. Gilt $\biglim{x\fromto x_0} f(x) = a$, $\biglim{x\fromto x_0} g(x) = b$ so folgt
    \begin{enumerate}[label=(\alph*)]
        \item $\biglim{x\fromto x_0} (\lambda f(x) + \mu g(x)) = \lambda a + \mu b\quad(\lambda, \mu\in \C)$
        \item $\biglim{x\fromto x_0} f(x)\cdot g(x) = a\cdot b$
        \item Ist $b \neq 0$ so gilt $\biglim{x\fromto x_0} \frac{f(x)}{g(x)} = \frac{a}{b}$
        \item Sind $f,g: D\fromto \R^d$ so gilt $\biglim{x\fromto x_0} (\lambda f(x) + \mu g(x)) = \lambda a + \mu b\quad(\lambda, \mu\in \R)$
    \end{enumerate}
    \begin{proof}[Beweis (c)]
        Wegen (b) reicht es zu zeigen, dass $\frac{1}{g(x)} \fromto \frac{1}{b}$ für $x\fromto x_0$
        \begin{align*}
            \abs{\frac{1}{g(x)} - \frac{1}{b}} &= \frac{\abs{b-g(x)}}{\abs{g(x)}\cdot\abs{b}}
            \intertext{Sei $\varepsilon >0$ beliebig $\impl \exists\delta > 0$ sodass für $x\in D$, $0 < \abs{x-x_0} < \delta$ auch $\abs{g(x)-b} < \min\set{\frac{\abs{b}}{2}, \frac{\abs{b}^2}{2}\cdot\varepsilon}$. Für diese $x$ gilt}
            \abs{g(x)} = \abs{b+g(x)-b} &\geq \abs{b}-\abs{g(x)-b} > \abs{b} - \frac{\abs{b}}{2} = \frac{\abs{b}}{2}
            \intertext{und somit auch}
            \abs{\frac{1}{g(x)} - \frac{1}{b}} = \frac{\abs{g(x)-b}}{\abs{g(x)}\cdot\abs{b}} &\leq \frac{2}{\abs{b}^2}\cdot\abs{g(x)-b} < \frac{2}{\abs{b}^2}\cdot \frac{\abs{b}^2}{2}\cdot\varepsilon = \varepsilon\qedhere
        \end{align*}
    \end{proof}
\end{satz}

\begin{uebung}
    Beweisen Sie die übrigen Aussagen des vorherigen Satzes.
\end{uebung}

\newpage

\begin{satz}[Cauchykriterium für Existenz von $\biglim{x\fromto x_0} f(x)$] % Satz 6
    \label{satz:funktionen-grenzwerte-cauchy}
    Sei $x_0$ Häufungspunkt von $D$ und $f: D\fromto \R$ (oder $\R^d$). Dann gilt, dass $\biglim{x\fromto x_0} f(x)$ genau dann existiert, wenn
    \begin{align*}
        \forall \varepsilon > 0~\exists\delta>0 \text{ sodass für } x,y\in D \text{ mit } 0 < \abs{x-x_0} < \delta, 0 < \abs{y-x_0} < \delta\\
        \abs{f(x)-f(y)} < \varepsilon \text{ ist }
    \end{align*}

    %%%%%%%%%%%%%%%%%%%%%%%%
    % 30. Januar 2023
    %%%%%%%%%%%%%%%%%%%%%%%%

    \begin{proof}
        \marginnote{[30. Jan]}
        \anf{$\impl$}: Wir haben $\ex a\in\R$ (oder $\R^d$) sodass
        \begin{align*}
            \fa \varepsilon > 0\ex \delta > 0\colon \abs{f(x)-a} &< \frac{\varepsilon}{2} \text{ für } 0 < \abs{x-x_0} < \delta\\
            \impl \abs{f(x)-f(y)} &= \abs{f(x)-a + a - f(y)}\\
            &\leq \abs{f(x)-a} + \abs{a-f(y)}\\
            &< \frac{\varepsilon}{2} + \frac{\varepsilon}{2} = \varepsilon
        \end{align*}
        \anf{$\Leftarrow$}: Wir müssen zeigen, dass für jede Folge $(x_n)_n\sbset D$, $x_n \neq x_0$, $x_n\fromto x_0$ folgt $f(x_0)\fromto a$ für $\ntoinf$. 1. Schritt: Sei $(x_n)_n\sbset D$, $x_n\fromto x_0$, $x_n\neq x_0$. Haben
        \begin{align*}
            \fa\varepsilon > 0\ex\delta > 0\colon \abs{f(x)-f(y)} &< \varepsilon\quad\fa 0 < \abs{x-x_0} < \delta \land 0<\abs{y-x_0} < \delta\\
            \intertext{Da $x_n\fromto x_0$}
            \impl \ex N\in\N\colon \abs{x_n - x_0} &< \delta\quad\fa n\geq N\\
            \impl \fa n,m\geq N\colon \abs{f(x_n)-f(x_m)} &< \varepsilon\\
            \impl (f(x_n))_n \text{ ist } & \text{eine Cauchy-Folge}\\
            \impl \lim_{\ntoinf} f(x_n) &\definedasbackwards a \text{ existiert}
            \intertext{2. Schritt: $a$ ist unabhängig von der gewählten Folge $(x_n)_n$. Sei $(y_n)_n\sbset D$, $y_n\fromto x_0$, $y_n\neq x_0$}
            \impl b&\definedas \lim_{\ntoinf} f(y_n) \text{ existiert auch nach Schritt 1}
            \intertext{Warum ist $a=b$? Wir basteln eine neue Folge $x_1, y_1, x_2, y_2, \dots, x_n, y_n, \dots$}
            z_{2n} &\definedas y_n\\
            z_{2n+1} &\definedas x_{n}\\
            \impl z_n &\fromto x_0\\
            \annot{\impl}{Schritt 1} c &\definedas \lim_{\ntoinf} f(z_n) \text{ existiert}
            \intertext{Teilfolgen konvergieren auch gegen $c$}
            c &= \lim_{\ntoinf} f(z_{2n}) = \lim_{\ntoinf} f(y_n) = b\\
            &= \lim_{\ntoinf} f(z_{2n+1}) = \lim_{\ntoinf} f(x_n) = a\\
            \impl a &= b
        \end{align*}
        Das heißt für jede Folge $(x_n)_n\sbset D, x_n\fromto x_0$ konvergiert $f(x_0)$ gegen ein eindeutiges $a$.
    \end{proof}
\end{satz}

\subsection{Links-/Rechtsseitige Grenzwerte und Verhalten gegen $\infty$}

\begin{definition}[Links- und rechtsseitige Grenzwerte] % Definition 7
    Sei $D\sbset\R$, $f: D \fromto \R$ (oder $\R^d$), $x_0$ Häufungspunkt in $D$. Dann heißt $a$ rechtsseitiger Grenzwert von $f$ in $x_0$, falls
    \begin{align*}
        \fa\varepsilon > 0\ex\delta > 0\colon \abs{f(x)-a} < \varepsilon\quad \text{ für } 0 < x-x_0 < \delta,~x\in D
    \end{align*}
    Wir schreiben $f(x + 0) = \biglim{x\fromto x_0^+} f(x) = \biglim{x\searrow x_0} f(x)$.\\
    $a$ heißt linksseitiger Grenzwert, falls
    \begin{align*}
        \fa\varepsilon > 0\ex\delta > 0\colon \abs{f(x)-a} < \varepsilon\quad \text{ für } -\delta < x-x_0 < 0,~x\in D
    \end{align*}
    Wir schreiben $f(x_0) - 0 = \biglim{x\fromto x_0^-} f(x) = \biglim{x\nearrow x_0} f(x)$.
    \horizontalline
    Wir sagen $f$ hat Grenzwert $a$ für $x\fromto\infty$, falls $D$ nach oben unbeschränkt ist und
    \begin{align*}
        \fa\varepsilon > 0\ex k\colon \abs{f(x)-a} < \varepsilon\quad\fa x\in D, x>k
    \end{align*}
    Wir schreiben $a=\biglim{x\fromto\infty} f(x)$. Das gleiche funktioniert ähnlich für $\biglim{x\fromto -\infty} f(x)$. (Betrachten Sie $\biglim{x\toinf} h(x)$ mit $h(x) = f(-x)$)
\end{definition}

\begin{satz} % Satz 8
    \label{satz:equiv-stetigkeit-grenzwerte}
    Sei $D\sbset \R$, $f: D\fromto \R$ (oder $\R^d$), $x_0$ HP von $D$. Dann gilt
    \begin{align*}
        f \text{ ist stetig in } x_0 &\equivalent \lim_{x\fromto x_0} f(x) = f(x_0)\\
        &\equivalent \lim_{x\nearrow x_0} f(x) = \lim_{x\searrow x_0} f(x) = f(x_0)
    \end{align*}
\end{satz}

\begin{uebung}
    Beweisen Sie Satz~\ref{satz:equiv-stetigkeit-grenzwerte}.
\end{uebung}

\newpage


    \section{[*] Gleichmäßige Stetigkeit und gleichmäßge Konvergenz}

    \subsection{Gleichmäßige und Lipschitz-Stetigkeit}
    \thispagestyle{pagenumberonly}

    \begin{definition}[Gleichmäßige Stetigkeit] % Def 1
        Sei $f: D\fromto \R$ (oder $\R^d$) und $D\sbset\K$. $f$ heißt gleichmäßig stetig auf $D$, falls
        \begin{align*}
            \fa\varepsilon > 0\ex\delta\colon \abs{f(x)-f(y)} < \varepsilon\quad\fa x,y\in D \text{ mit } \abs{x-y} < \delta
        \end{align*}
    \end{definition}

    \begin{bemerkung}
        Gleichmäßige Stetigkeit ist nach der Definition eine strengere Eigenschaft als Stetigkeit auf $D$. Das heißt jede gleichmäßig stetige Funktion ist auch stetig, aber nicht umgekehrt.
    \end{bemerkung}

    \begin{beispiel}
        \begin{align*}
            f: \R\fromto\R,~x\mapsto\frac{1}{1+x^2}
        \end{align*}
        ist gleichmäßig stetig. (Übung)
    \end{beispiel}
    \begin{beispiel}
        \begin{align*}
            f: \rinterv{0,1}\fromto \R,~x\mapsto \frac{1}{x}
        \end{align*}
        ist stetig, aber nicht gleichmäßig stetig.
        \begin{proof}
            Für $0 < x < y = 2x$ gilt
            \begin{align*}
                \abs{f(x)-f(y)} &= \abs{\frac{1}{x} - \frac{1}{y}} = \frac{\abs{y-x}}{xy} = \frac{1}{y}\geq 1\qedhere
            \end{align*}
        \end{proof}
    \end{beispiel}

    \begin{definition}[Lipschitz-Stetigkeit]
        Eine Funktion $f: D\fromto\R$ (oder $\R^d$) heißt Lipschitz-stetig, falls
        \begin{align*}
            \ex L\geq 0\colon \abs{f(x)-f(y)} \leq L\cdot\abs{x-y}\quad\forall x,y\in D
        \end{align*}
        Jede Lipschitz-stetige Funktion ist gleichmäßig stetig. $(\delta = \frac{\varepsilon}{L})$
    \end{definition}

    \begin{satz}[Heine, 1872] % Satz 3
        \label{satz:17-3}
        Sei $K\sbset\R$ kompakt und $f: K\fromto\R$ (oder $\R^d$) stetig. Dann ist $f$ gleichmäßig stetig.
        \begin{proof}
            Angenommen $f$ ist nicht gleichmäßig stetig.
            \begin{align*}
                \impl\ex\varepsilon > 0\fa \delta > 0\ex x,y\in &K\colon\abs{x-y} < \delta \text{ und } \abs{f(x)-f(y)} > \varepsilon
                \intertext{Wähle $\delta = \frac{1}{n}$}
                \impl\ex x_n, y_n\sbset K\colon \abs{x_n- y_n} &< \frac{1}{n} \text{ aber } \abs{f(x_n)-f(y_n)} \geq \varepsilon > 0\\
                \impl x_n - y_n &\fromto 0 \text{ für } n\fromto\infty
                \intertext{Da $K$ kompakt $\ex$Konvergente TF $(y_{n_l})_l$ von $(y_n)_n$ nach Satz~\ref{satz:bolzano-weierstrass}}
                y &\definedas \lim_{l\toinf} (y_{n_l}) \text{ existiert in } K
                \intertext{Für eine Teilfolge $(x_{n_l})_l$ von $(x_n)_n$ gilt}
                \abs{x_{n_l} - y} &= \abs{x_{n_l} - y_{n_l} + y_{n_l} - y}\\
                &\leq \underbrace{\abs{x_{n_l} - y_{n_l}}}_{<\frac{1}{n}\fromto 0} + \underbrace{\abs{y_{n_l} - y}}_{\fromto 0}\fromto 0\\
                \impl \abs{f(x_{n_l}) - f(y_{n_l})} &\geq \varepsilon > 0
                \intertext{Aber}
                \abs{f(x_{n_l}) - f(y_{n_l})} &= \abs{f(x_{n_l}) - f(y) + f(y) - f(y_{n_l})}\\
                &\leq \underbrace{\abs{f(x_{n_l}) - f(y)}}_{\fromto 0} + \underbrace{\abs{f(y)-f(y_{n_l})}}_{\fromto 0}
            \end{align*}
            Damit ergibt sich ein Widerspruch zur Stetigkeit von $f$ und $f$ ist damit gleichmäßig stetig.
        \end{proof}
    \end{satz}

    \subsection{[*] Punktweise und gleichmäßige Konvergenz von Funktionenfolgen}

    Wir betrachten Folgen von Funktionen. $f_n: D\fromto \R$ (oder $\R^d$) $\leadsto$ Folge $(f_n)_n$ von Funktionen.

    \begin{definition}[Punktweise Konvergenz] % Def 4
        Eine Funktionenfolge $(f_n)_n$, $f_n: D\fromto\R$ (oder $\R^d$) konvergiert punktweise falls
        \begin{align*}
            \lim_{\ntoinf} f_n(x) \text{ existiert für jedes } x\in D
        \end{align*}
        Das heißt $(f_n(x))_n$ ist eine konvergente Folge $\fa x\in\R$. Dann definieren wir
        \begin{align*}
            f(x) \definedas \lim_{\ntoinf} f_n(x)
        \end{align*}
        eine Funktion $f: D\fromto \R$ (oder $\R^d$). Und sagen $f$ ist der punktweise Limes der Funktionenfolge $f_n(x) \fromto f(x)~\fa x\in D$.
    \end{definition}

    \begin{beispiel}
        Die Funktion
        \begin{align*}
            f_n(x) &= x^n\quad 0 \leq x \leq 1
            \intertext{konvergiert punktweise gegen}
            f(x) &= \begin{cases}
                        0\quad &0 \leq x < 1\\
                        1\quad &x = 1
            \end{cases}
        \end{align*}
    \end{beispiel}

    \begin{beispiel}
        Die Funktion
        \begin{align*}
            f_n(x) &= x^{\frac{1}{n}}\quad 0 \leq x \leq 1
            \intertext{ist stetig und punktweise konvergent gegen}
            f(x) &= \begin{cases}
                        0\quad&x = 0\\
                        1\quad&0 < x \leq 1
            \end{cases}
        \end{align*}
    \end{beispiel}

    \begin{beispiel}
        Die Funktion
        \begin{align*}
            f_n(x) &= \pair{1-x^2}^{\frac{n}{2}}\quad -1 \leq x \leq 1
            \intertext{ist stetig und punktweise konvergent gegen}
            f(x) &= \begin{cases}
                        1\quad&x = 0\\
                        0\quad&0 < \abs{x}\leq 1
            \end{cases}
        \end{align*}
    \end{beispiel}

    \begin{definition}[Gleichmäßige Konvergenz - Weierstraß 1841] % Definition 5
        $D\sbset\R$, Funktionenfolge $f_n: D\fromto\R$ (oder $\R^d$). $(f_n)_n$ konvergiert gleichmäßig gegen $f: D\fromto\R$ (oder $\R^d$) falls
        \begin{align*}
            \fa\varepsilon > 0\ex N\in \N \text{ mit } \abs{f_n(x) - f(x)} < \varepsilon\quad\fa n\geq N, x\in D
        \end{align*}
    \end{definition}

    \begin{bemerkung}
        Also gilt bei gleichmäßiger Konvergenz
        \begin{align*}
            \sup_{x\in D} \abs{f_n(x) - f(x)} &\leq \varepsilon\\
            \impl \lim_{\ntoinf} \sup_{x\in D}\abs{f_n(x) - f(x))} &= 0\\
            \equivalent \limsup_{\ntoinf} \pair{\abs{f_n(x) - f(x)}} &= 0
        \end{align*}
    \end{bemerkung}

    %%%%%%%%%%%%%%%%%%%%%%%%
    % 01. Februar 2024
    %%%%%%%%%%%%%%%%%%%%%%%%

    \begin{notation}[Supremumsnorm]
        \marginnote{[01. Feb]}
        Es sei $f: D\fromto \R$ (oder $\R^d$, $\C$). Dann schreiben wir
        \begin{align*}
            \norm{f}_{\infty} &\definedas \norm{f}_{D,\infty} = \norm{f}_{?}\\
            &= \sup_{x\in D} \abs{f(x)}
        \end{align*}
        Norm auf dem Vektorraum der beschränkten Funktionen auf $D$.
    \end{notation}

    \begin{satz}[Cauchy-Kriterium für gleichmäßige Konvergenz]
        Es sei $(f_n)_n$, $f_n: D\fromto \R$ (oder $\R^d$). Dann konvergiert $(f_n)_n$ genau dann gleichmäßig gegen $f$, wenn
        \begin{align*}
            \fa\varepsilon > 0\ex N\in\N\colon \abs{f_n(x) - f_m(x)} < \varepsilon\quad\forall x\in D, n,m\geq N
        \end{align*}
        \begin{proof}
            \anf{$\impl$}: $f(x) = \biglim{\ntoinf} f_n(x)$ existiert $\fa x\in D$.
            \begin{align*}
                \impl \abs{f_n(x) - f_m(x)} \leq \underbrace{\abs{f_n(x) - f(x)}}_{<\frac{\varepsilon}{2}} + \underbrace{\abs{f(x) - f_n(x)}}_{<\frac{\varepsilon}{2}} < \varepsilon\quad\fa n,m\geq N
            \end{align*}
            unabhängig von $x\in D$.\\
            \anf{$\Leftarrow$}: Für $x\in D$ ist $(f_n(x))_n$ eine Cauchy-Folge. Und $f(x) = \biglim{\ntoinf} f_n(x)$ existiert $\fa x\in D$.
            \begin{align*}
                \abs{f_n(x) - f(x)} &= \lim_{\ntoinf} \abs{f_n(x) - f_m(x)} < \varepsilon\quad n\geq N
                \intertext{Sei $\varepsilon > 0$}
                \impl \ex N\in\N\colon \abs{f_n(x) - f_m(x)} &< \varepsilon\quad\fa n,m\geq N\\
                \impl \abs{f_n(x) - f(x)} &= \lim_{\ntoinf} \abs{f_n(x) - f_m(x)} < \varepsilon\quad\fa n\geq N\\
                \impl \sup_{x\in D} \abs{f_n(x) - f(x)} &\leq \varepsilon\quad\fa n\geq N\\
                \impl (f_n)_n &\text{ geht gleichmäíg gegen } f\qedhere
            \end{align*}
        \end{proof}
    \end{satz}

    \begin{satz}[Weierstraß 1861] % Satz 7
        \label{satz:17-7}
        Seien $f_n: D \fromto \R$ (oder $\R^d$, $\C$) stetige Funktionen, welche gleichmäßig gegen eine Funktion $f$ konvergieren. Dann ist $f$ stetig!
        \begin{proof}
            Geg. $x_0\in D, x\in D$.
            \begin{align*}
                \abs{f(x) - f(x_0)} &= \abs{f(x) - f_n(x) + f_n(x) - f(x_0)}\\
                &\leq \abs{f(x) - f_n(x)} + \abs{f_n(x) - f_n(x_0)} + \abs{f_n(x) - f(x_0)}
                \intertext{$\frac{\varepsilon}{3}$-Trick}
                \fa\varepsilon > 0\ex N\in \N\colon \abs{f_n(y) - f(y)} < \frac{\varepsilon}{3}\quad\forall y\in D, n\geq N
                \intertext{Wir fixieren $n=N$. Dann ist $f_n$ stetig}
                \impl \text{ Geg. }\varepsilon > 0\ex \delta > 0\colon \abs{f_n(x) - f_n(x)} &< \frac{\varepsilon}{3}\quad \text{ für } \abs{x-x_0} < \delta\\
                \intertext{Für $x\in D$, $\abs{x-x_0} < \delta$ gilt}
                \abs{f(x) - f(x_0)} \leq \abs{f(x) - f_n(x)} + \abs{f_n(x) - f_n(x)} + \abs{f_n(x) - f(x_0)} &< \frac{\varepsilon}{3} + \frac{\varepsilon}{3} + \frac{\varepsilon}{3} = \varepsilon\\
                \impl f & \text{ ist stetig } \qedhere
            \end{align*}
        \end{proof}
    \end{satz}

    \begin{satz}[Weierstraß' M-Test] % Satz 8
        \label{satz:17-8}
        Eine Reihe $ \sum_{n=0}^{\infty} f_n$ von Funktionen $f_n: D\fromto \R$ (oder $\R^d$) konvergiert gleichmäßig, wenn sie eine konvergente Majorante hat, das heißt $\ex M_n\geq 0, N_0\in \N$ mit
        \begin{align*}
            \abs{f_n(x)} &\leq M_n\quad\forall x\in D, n\geq N_0
            \intertext{und}
            \sum_{n=0}^{\infty} N_n &< \infty
        \end{align*}
        \begin{proof}
            Partialsummen
            \begin{align*}
                s_n(x) &\definedas \sum_{j=0}^{n}  f_j(x)
                \intertext{Wir betrachten $n,m\geq N_0$}
                \abs{s_n(x) - s_m(x)} &= \abs{\sum_{j=m+1}^{n} f_j(x)} \leq \sum_{j=m+1}^{n} \underbrace{\abs{f_j(x)}}_{\leq M_j}\\
                &\leq \sum_{j=m+1}^{n} M_j\\
                \impl \abs{s_n(x) - s_m(x)} &\leq \underbrace{\sum_{j=m+1}^{n} M_j}_{\fromto 0}
                \intertext{Haben}
                \impl \sup_{n\geq m} \sup_{x\in D} \abs{s_n(x) - s_m(x)} &\fromto 0 \text{ für } m\fromto\infty\\
                \equivalent s_{n} \text{ konvergiert gleichmäßig auf } &D
                \intertext{$s_n(x)$ stetig, da endliche Summen von stetigen Funktionen stetig sind}
                \impl s(x) &= \lim_{\ntoinf} s_n(x) \text{ ist stetig nach Satz~\ref{satz:17-7}}\qedhere
            \end{align*}
        \end{proof}
    \end{satz}

    \begin{anwendung}[Potenzreihen]
        Satz~\ref{satz:17-7} und Satz~\ref{satz:17-8} gelten auch für Funktionen $f_n: D\fromto \C$, $D\sbset \C$. Potenzreihe
        \begin{align*}
            f(x) &= \sum_{n=0}^{\infty} a_n x^n\\
            s_n(x) &= \sum_{j=0}^{n} \underbrace{a_{j} x^j}_{\text{Polynom, daher stetig}}
            \intertext{Wir wollen Weierstraß' M-Test anwenden. Sei $R > 0$ Konvergenzradius der Potenzreihe}
            \impl \forall \abs{z} < R \text{ existiert } f(z) &= \sum_{n=0}^{\infty} a_n z^n
            \intertext{Geg. $\delta > 0, R-\delta > 0$ sei $z_1\in \C$}
            \abs{z_1} &= R - \frac{\delta}{2} < R
            \intertext{Aus der Verbesserung von Lemma~\ref{lemma:temp-4} folgt}
            \ex M \geq 0\colon \abs{\sum_{n=k+1}^{\infty} a_n z^n} &\leq M\cdot \abs{\frac{z}{z_1}}^{k+1}\quad\fa \abs{z}< \abs{z_1} = R - \frac{\delta}{2}\\
            \abs{z} &\leq R- \delta\\
            \impl \frac{\abs{z}}{\abs{z_1}} \leq \frac{R-\delta}{R-\frac{\delta}{2}} = q < 1\\
            \impl \abs{\sum_{n=k+1}^{\infty} a_n z^n} \leq M \cdot q^{k+1}\\
            \impl \abs{s(z) - s_k(z)} &= \abs{\sum_{n=k+1}^{\infty} a_n z^n}\\
            &\leq M \cdot q^{k+1} \fromto 0 \text{ für } k\fromto \infty\\
            \sup_{\abs{z} < R - \delta} \abs{s(z) - s_k(z)} &\leq M \cdot q^{k+1} \fromto 0 \text{ für } k\fromto\infty
            \intertext{Partialsummen}
            s_k(z) &= \sum_{n=0}^{k} a_n z^n \text{ konvergiert gleichmäßig gegen } s(z) \text{ für alle } \abs{z} \leq R-\delta
        \end{align*}
        liefert Stetigkeit.
    \end{anwendung}

    \begin{satz}[Weierstraß] % Satz 9
        \label{satz:17-9}
        Sei $a <b$, $f: \interv{a,b}\fromto \R$ stetig. Dann gilt es existiert eine Folge von Polynomen $(P_n)_n$, welche gleichmäßig auf $\interv{a,b}$ gegen $f$ konvergiert. Das heißt $\norm{f-P_n}_{\infty}  = \biglim{\ntoinf} \sup_{a \leq x \leq b} \abs{f(x) - P_n(x)} = 0$

        \begin{proof}
            O.B.d.A. $a=0$, $b=1$ -- sonst ist $g: \interv{0,1} \fromto \interv{a,b}, x\mapsto (b-a)x + a$ stetig und bijektiv und wir haben $h: \interv{0,1} \fromto \R, x\mapsto h(x) = f(g(x))$.\\
            Def. Bernstein Polynome
            \begin{align*}
                P_n(x) &\definedas \sum_{k=0}^{n} \binom{n}{k} x^k \cdot\pair{1-x}^{n-k} \cdot f\of{\frac{k}{n}}\tag{$n\in\N$}
                \intertext{Haben}
                a)&\quad\sum_{k=0}^{n} \binom{n}{k}x^k \cdot \pair{1-x}^{n-k} = \pair{x+(1-x)}^n = 1\tag{Bernoulli}\\
                b)&\quad \sum_{k=0}^{n} k\binom{n}{k}x^k\cdot (1-x)^{n-k} = n\cdot x\\
                c)&\quad \sum_{k=0}^{n} k\cdot(k-1)\binom{n}{k}x^k \cdot\pair{1-x}^{n-k} = n\cdot(n-1) \cdot x^2\\
                b)&\quad k\binom{n}{k} = k\cdot\frac{n!}{k!\cdot(n-k)!} = n\cdot\frac{(n-1)!}{(k-1)!(n+1-(k-1))!} = n\binom{n-1}{k-1}\\
                \impl \sum_{k=0}^{n} k\binom{n}{k}x^k \cdot (1-x)^{n-k} &= \sum_{k=1}^{n} n \binom{n-1}{k-1} x^k \cdot (1-x)^{n-k}\\
                &= n\cdot x\cdot \sum_{k=1}^{n} \binom{n-1}{k-1} x^{k-1} (1-x)^{n-1-(k-1)}\\
                &= \sum_{j=0}^{n-1} \binom{n-1}{j} x^j \cdot (1-x)^{n-1-j} = nx = 1\\
                c)&\quad \text{ Bernoulli: } k\cdot(k+1) \binom{n}{k} = n\cdot (n-1) \binom{n-2}{k-2}\tag{$k\geq 2$}\\
                \sum_{k=0}^{n} k(k+1) \binom{n}{k} x^k (1-x)^{n-k} &= n(n-1) x^{k-2} \underbrace{\sum_{k=2}^{n} \binom{n-2}{k-2} x^{k-2} (1-x)^{n-2-(k-2)}}_{=(1+(1-x))^{k-2} = 1}
            \end{align*}

            \begin{align*}
                \abs{f(x) - P_n(x)} &= \abs{\sum_{k=0}^{n} \binom{n}{k} x^k\pair{1-x}^{n-k}f(x) - \sum_{k=0}^{n}\binom{n}{k}x^k (1-x)^{n-k} f(\frac{n}{k}) }\\
                &\leq \sum_{k=0}^{n} \binom{k}{k} x^k (1-k)^{n-k} \cdot \abs{f(x)- f\of{\frac{n}{k}}}
                \intertext{Nach dem Satz von Heine ist $f$ stetig auf $\interv{0,1}$ das heißt}
                \fa\varepsilon > 0\ex\delta > 0\colon\abs{f(x)-f(y)} < \varepsilon \text{ für } x,y\in\interv{0,1}, \abs{x-y} < \delta
                \intertext{Teilen Summen in 2 Gebiete auf}
                A_1 &\definedas \set{0 \leq k \leq n: \abs{x-\frac{k}{n}} < \delta}\\
                A_2 &\definedas \set{0 \leq k \leq n: \abs{x-\frac{k}{n}} \geq \delta}\\
                \impl \abs{f(x) - P_n(x)} \leq \underbrace{\sum_{k<A_1}^{} \binom{n}{k} x^k (1-k)^{n-k} \abs{f(x) - f(\frac{k}{n})}}_{\definedasbackwards \delta_1}\\
                + \underbrace{\sum_{k<A_2}^{} \binom{n}{k} x^k (1-k)^{n-k} \abs{f(x) - f(\frac{k}{n})}}_{\definedasbackwards \delta_2}\\
                s_1 < \varepsilon \sum_{k < A1}^{} \binom{n}{k} x^k (1-x)^{n-k} < \varepsilon\\
                \intertext{$S2$}
                \abs{x-\frac{k}{n}} < \delta\quad\equivalent\quad \frac{\abs{x-\frac{k}{n}}}{\delta^2} \geq 1
                \intertext{mit}
                \abs{f(x)-f(\frac{k}{n})} \leq \abs{f(x)} + \abs{f(\frac{k}{n})} \leq 2\cdot\sup_{0\leq x \leq 1} (f(x)) < \infty\\
                \impl s_2 &\leq 2 \cdot\norm{f}_{\infty} \cdot \sum_{k=0}^{n} \binom{n}{k} x^k (1-x)^{n-k} \frac{\abs{x-\frac{k}{n}}}{\delta^2}\\
                &\leq 2 \norm{f}_{\infty} \sum_{k=0}^{n} \binom{n}{k} x^k (1-x)^{n-k} \frac{\abs{x-\frac{k}{n}}}{\delta^2}\\
                &= \frac{2\norm{f}}{\delta^2 \cdot n^2} \sum_{k=0}^{n} \binom{n}{k} x^k (1-x)^{n-k} \abs{nx -k}^2\\
                \impl \sum_{k=0}^{n} \binom{n}{k} x^k (1-x)^{n-k} (k^2)\\
                \vdots\quad ???\\
                \vdots\quad ???\\
                \impl \abs{f(x)-P_n(x)} \leq \varepsilon + \frac{\norm{f}_{\infty}}{2\delta^2 n}\\
                \impl \sup_{0\leq x \leq 1} \abs{f(x) - P_n(x)} \leq \varepsilon + \frac{\norm{f}_{\infty}}{2\delta^2 n}\\
                \impl \limsup_{\ntoinf} \norm{f-P_n}_{\infty} \leq \varepsilon\quad\fa \varepsilon > 0\\
                ????
            \end{align*}
        \end{proof}
    \end{satz}

    \newpage

    %%%%%%%%%%%%%%%%%%%%%%%%
    % 06. Februar 2024
    %%%%%%%%%%%%%%%%%%%%%%%%


    \section{[*] Ableitung (engl. Differention)}
    \thispagestyle{pagenumberonly}

    \subsection{Ableitung als Grenzwert}

    \begin{definition} % Definition 1
        \marginnote{[06. Feb]}
        Es seien $D = \pair{a,b}\sbset\R$, $x\in D$ und $f: D\fromto \R$. $f$ heißt im Punkt $x$ von rechts differenzierbar, falls
        \begin{align*}
            f'_+ &\definedas \lim_{h\fromto 0_+} \frac{f(x+h)-f(x)}{h}
            \intertext{existiert. $f$ ist von links differenzierbar, falls}
            f'_- &\definedas \lim_{h\fromto 0_-} \frac{f(x+h)-f(x)}{h}
        \end{align*}
        existiert. $f$ ist im Punkt $x$ differenzierbar, falls $f'_+$ und $f'_-$ existieren und $f'_+ = f'_-$. Das ist äquivalent zu der Existenz von
        \begin{align*}
            \lim_{\substack{h\fromto 0\\ h\neq 0}} \frac{f(x+h)-f(x)}{h}
        \end{align*}
    \end{definition}

    \begin{satz} % Satz 2
        \label{satz:18-2}
        Sei $a\in\pair{c,d}\sbset\R$. Die Funktion $f: \pair{c,d}\fromto\R$ ist genau dann im Punkt $a$ differenzierbar, wenn $\ex C\in\R$ mit $f(x) = f(a)+C\cdot\pair{x-a} + \varphi\of{x}$ wobei
        \begin{align*}
            \lim_{x\fromto a} \frac{\varphi\of{x}}{x-a} = 0
        \end{align*}
        \begin{proof}
            \anf{$\impl$}: Sei $f$ differenzierbar. Wir wählen $\varphi\of{x} \definedas f(x)-f(a) - f'(a)\cdot\pair{x-a}$ und $C=f'\of{a}$.
            \begin{align*}
                \lim_{x\fromto a} \frac{\varphi\of{x}}{x-a} &= \lim_{x\fromto a} \underbrace{\frac{f(x)-f(a)}{x-a}}_{\fromto f'(a)} - f'(a) = 0
            \end{align*}
            \anf{$\Leftarrow$}: Es gilt $f(x) = f(a)+C\cdot\pair{x-a} + \varphi\of{x}$ und $\frac{\varphi(x)}{x-a}\fromto 0$ für $x\fromto a$
            \begin{align*}
                &\impl \frac{f(x)-f(a)}{x-a} = C + \frac{\varphi(x)}{x-a}\\
                &\impl \abs{\frac{f(x)-f(a)}{x-a} - C} \fromto 0\\
                &\impl \frac{f(x)-f(a)}{x-a}\fromto C\\
                &\impl f \text{ ist in $a$ differenzierbar}\qedhere
            \end{align*}
        \end{proof}
    \end{satz}

    \begin{korollar}
        \theoremescape
        \label{korollar:abschaetzungen-ableitung}
        \begin{enumerate}[label=(\roman*)]
            \item Wenn $f$ im Punkt $a$ differenzierbar ist, dann ist $f$ im Punkt $a$ auch stetig.
            \item Sei $f'(a) \neq 0$. Dann gilt
            \begin{align*}
                \exists h_0~\forall h=x-a \text{ mit } \abs{h} < h_0,~h\neq 0\colon \abs{f(x)-f(a)}\geq\frac{1}{2}\abs{f'(a)}\cdot\abs{x-a}
            \end{align*}
            \newpage
            \item $\exists h_0$, so dass für alle $x$ mit $\abs{x-a} < h_0,~x\neq a$ gilt
            \begin{enumerate}[label=(\arabic*)]
                \item \fixedspace{6cm}{$\abs{f(x)-f(a)} \leq 2\abs{f'(a)}\cdot\abs{x-a}$} für $f'(a)\neq 0$
                \item \fixedspace{6cm}{$\abs{f(x)-f(a)} \leq \varepsilon\cdot\abs{x-a}$} für $f'(a) = 0$ \quad ($\forall\varepsilon > 0$, $h_0$ ist von $\varepsilon$ abhängig)
            \end{enumerate}
        \end{enumerate}

        \begin{proof}[Beweis (i)]
            Für $x\fromto a$ gilt
            \begin{align*}
                f(x)-f(a) &= C\cdot\underbrace{\pair{x-a}}_{\fromto 0} + \underbrace{\frac{\varphi(x)}{x-a}}_{\fromto 0}\cdot\underbrace{\pair{x-a}}_{\fromto 0}\fromto 0\qedhere
            \end{align*}
        \end{proof}

        \begin{proof}[Beweis (ii)]
            Für $x\fromto a$ gilt
            \begin{align*}
                f(x) &= f(a) + f'(a)\cdot\pair{x-a} + \frac{\varphi(x)}{x-a}\cdot\pair{x-a}\\
                \abs{f(x)-f(a)} &\geq \abs{f'(a)}\cdot\abs{x-a} - \underbrace{\abs{\frac{\varphi(x)}{x-a}}}_{\fromto 0}\cdot\abs{x-a}\geq \frac{1}{2}\cdot\abs{f'(a)}\cdot\abs{x-a}\qedhere
            \end{align*}
        \end{proof}
        \begin{proof}[Beweis (iii)]
        (1)
            Für $x\fromto a$ gilt
            \begin{align*}
                \abs{f(x)-f(a)} &\leq \abs{f'(a)}\cdot\abs{x-a} + \underbrace{\abs{\frac{\varphi(x)}{x-a}}}_{\fromto 0} \cdot \abs{x-a} \leq 2\abs{f'(a)}\cdot\abs{x-a}
                \intertext{(2)}
                \abs{f(x)-f(a)} &= \abs{\underbrace{f'(a)}_{\fromto 0} \cdot \pair{x-a} + \underbrace{\frac{\varphi(x)}{x-a}}_{\fromto 0}\cdot\pair{x-a}} \leq \varepsilon\cdot\abs{x-a}\qedhere
            \end{align*}
        \end{proof}
    \end{korollar}

    \subsection{Ableitungsregeln}

    \begin{satz} % Satz 3
        \label{satz:ableitungsregeln}
        Seien $f, g: (a,b)\fromto \R$ differenzierbar, $\lambda\in\R$, $x\in\pair{a,b}$. Dann gilt
        \begin{enumerate}[label=(\roman*)]
            \item $\pair{f+g}'\of{x} = f'(x)+g'(x)$
            \item $\pair{\lambda\cdot f}'\of{x} = \lambda\cdot f'(x)$
            \item $\pair{f\cdot g}'\of{x} = f'(x)\cdot g(x) + g'(x)\cdot f(x)$\quad\quad(Produktregel)
        \end{enumerate}
        \begin{proof}[Beweis (iii)]
            \begin{align*}
                \pair{f+g}'\of{x} &= \lim_{h\fromto 0} \frac{f(x+h)g(x+h) - f(x)g(x)}{h}\\
                &= \lim_{h\fromto 0} \frac{f(x+h)g(x+h) - f(x+h)g(x) + f(x+h)g(x) - f(x)g(x)}{h}\\
                &= \lim_{h\fromto 0} f(x+h)\cdot \underbrace{\frac{g(x+h)-g(x)}{h}}_{\fromto g'(x)} + \lim_{h\fromto 0} g(x) \cdot \underbrace{\frac{f(x+h)-f(x)}{h}}_{\fromto f'(x)}\\
                &= f(x) \cdot g'(x) + g(x)\cdot f'(x)\qedhere
            \end{align*}
        \end{proof}
    \end{satz}

    \begin{uebung}
        Beweisen Sie die verbleibenden Aussagen des vorherigen Satzes.
    \end{uebung}

    \begin{satz}[Kettenregel] % Satz 4
        \label{satz:kettenregel}
        Seien $f: \pair{a,b}\fromto \R$ und $g: (c,d)\fromto \R$ Funktionen mit $f\interv{\pair{a,b}} \sbset \pair{c,d}$. Die Funktion $f$ sei im Punkt $x\in\pair{a,b}$ differenzierbar und $g$ sei im Punkt $y\definedas f(x)$ differenzierbar. Dann gilt
        \begin{align*}
            \pair{g\circ f}'(x) = g'(f(x)) \cdot f'(x)
        \end{align*}
        \begin{proof}
            Wir definieren $F(x) \definedas g(f(x))$ und unterscheiden zwei Fälle.\\
            \textsc{Fall 1.} $f'(x) \neq 0$. Nach Korollar~\ref{korollar:abschaetzungen-ableitung} gilt
            \begin{align*}
                \ex h_0, \abs{h} < h_0\colon \abs{f(x+h)-f(x)} &\geq \frac{1}{2}\abs{f'(x)}\abs{h} \neq 0\\
                \lim_{h\fromto 0} \frac{F(x+h)-F(x)}{h} &= \lim_{h\fromto 0} \frac{\pair{F(x+h) - F(x)}\cdot\pair{f(x+h)-f(x)}}{\pair{f(x+h)-f(x)}\cdot h}\\
                &= \lim_{h\fromto 0} \frac{g(f(x+h))-g(f(x))}{f(x+h)-f(x)} \cdot \underbrace{\frac{f(x+h)-f(x)}{h}}_{\fromto f'(x)}
                \intertext{$f(x)=y$ und $f(x+h) = y + \Delta y$}
                \lim_{h\fromto 0} \frac{g(f(x+h))-g(f(x))}{f(x+h)-f(x)} &= \lim_{h\fromto 0} \frac{g(y+\Delta y) - g(y)}{\Delta y}\tag{$\Delta y \neq 0$}\\
                &= \lim_{\Delta y \fromto 0} \frac{g(y+\Delta y) - g(y)}{\Delta y}\\
                &= g'(y) = g'(f(x))\\
                \impl \frac{F(x+h)-F(x)}{h} &\fromto g'(f(x)) \cdot f'(x)
                \intertext{\textsc{Fall 2.} $f'(x) = 0$. Dann gilt nach Korollar~\ref{korollar:abschaetzungen-ableitung}}
                \abs{\frac{g(f(x+h)) - g(f(x))}{h}} &\leq \frac{c\cdot\abs{f(x+h)-f(x)}}{h}\\
                &\leq \frac{c\cdot\varepsilon\cdot \abs{x+h-x}}{h}\\
                &= c\cdot\varepsilon\\
                \intertext{mit $c=\max\set{2\cdot\abs{g'(f(x))}, 1}$ und $\varepsilon$ beliebig klein}
                \impl \lim_{h\fromto 0} \frac{F(x+h)-F(x)}{h} = 0 &= g'\of{f\of{x}} \cdot 0 = g'\of{f\of{x}}\cdot f'\of{x}\\
                \intertext{Damit folgt insgesamt}
                \pair{g\circ f}'(x) &= g'(f(x))\cdot f'(x)\qedhere
            \end{align*}
        \end{proof}
    \end{satz}

    \begin{satz}[Quotientenregel] % Satz 5
        \label{satz:quotient-ableitung}
        Für zwei differenzierbare Funktionen $u,v$ gilt
        \begin{align*}
            \pair{\frac{v}{u}}' = \frac{v'\cdot u - u'\cdot v}{u^2}
        \end{align*}
        \begin{proof}
            \begin{align*}
                \pair{\frac{v}{u}}' &= \pair{v\cdot\frac{1}{u}}' = v'\cdot \frac{1}{u} + v\cdot\frac{1}{u^2}\cdot u'\cdot (-1) = \frac{v'u - u'v}{u^2}\qedhere
            \end{align*}
        \end{proof}
    \end{satz}

    \begin{beispiel}[Ableitung der Exponentialfunktion]
        Es sei $a\in\C$, $x\in\R$. Wir wollen $e^{ax}$ ableiten
        \begin{align*}
            \pair{e^{ax}}' &= \lim_{h\fromto 0} \frac{e^{a(x+h)} - e^{ax}}{h} = \lim_{h\fromto 0}e^{ax} \cdot \frac{e^{ah}-1}{h} = e^{ax}\cdot \lim_{h\fromto 0} \frac{e^{ah}-1}{h}\\
            e^{ah} &= \sum_{n=0}^{\infty} \frac{(ah)^n}{n!} = 1 + ah + (ah)^2\cdot\sum_{n=2}^{\infty} \frac{(ah)^{n-2}}{n!}
            \intertext{Wir schätzen den letzten Teil der Gleichung mit $\abs{h} < \frac{1}{\abs{a}}$ ab}
            \abs{ \sum_{n=2}^{\infty} \frac{(ah)^{n-2}}{n!}} &\leq \sum_{n=2}^{\infty} \frac{1}{n!} < e\\
            \impl \pair{e^{ax}}' &= e^{ax}\cdot \lim_{h\fromto 0} \frac{1+ah+(ah)^2\cdot e - 1}{h} = a\cdot e^{ax}\tag{$a\in\C$}
        \end{align*}
    \end{beispiel}

    %%%%%%%%%%%%%%%%%%%%%%%%
    % 08. Februar 2024
    %%%%%%%%%%%%%%%%%%%%%%%%

    \begin{beispiel}[Ableitung von $\sin$ und $\cos$]
        \marginnote{[08. Feb]}
        \begin{align*}
            \pair{\sin x}' &= \pair{\Im e^{ix}}' = \Im \pair{e^{ix}}' = \Im\pair{i\cdot e^{ix}} = \Im\of{i\cdot\pair{\cos x + i\cdot \sin x}} = \cos x
            \intertext{Analog lässt sich zeigen, dass gilt}
            \pair{\cos x}' &= -\sin x
        \end{align*}
    \end{beispiel}
    \begin{beispiel}[Ableitung von $\tan$]
        \begin{align*}
            \pair{\tan x}' &= \pair{\frac{\sin x}{\cos x}}' = \frac{\cos x \cdot \cos x -\pair{-\sin x \cdot \sin x}}{\cos^2 x} = \frac{\cos^2 x + \sin^2 x}{\cos^2 x} = \frac{1}{\cos^2 x}
        \end{align*}
    \end{beispiel}

    \begin{satz}[Ableitung der Umkehrfunktion] % Satz 6
        \label{satz:ableitung-umkehrfunktion}
        Sei $f: \pair{a,b} \fromto \pair{c,d}$ eine bijektive Abbildung, die im Punkt $x\in\pair{a,b}$ differenzierbar ist. Dann ist die Funktion $f^{-1}$ an der Stelle $y=f\of{x}$ differenzierbar und es gilt
        \begin{align*}
            \pair{f^{-1}}'\of{y} = \frac{1}{f'\of{x}}
        \end{align*}
        \begin{proof}
            \begin{align*}
                \pair{f^{-1}}'\of{y} &= \lim_{h\fromto 0} \frac{\overbrace{f^{-1}\of{y+h}}^{\definedasbackwards \xi} - f^{-1}\of{y}}{h} = \lim_{h\fromto 0}\frac{\xi - x}{f\of{\xi} - f\of{x}}\\
                &= \lim_{h\fromto 0} \frac{1}{\frac{f\of{\xi} - f\of{x}}{\xi-x}} = \frac{1}{\lim_{h\fromto 0} \frac{f\of{\xi} - f\of{x}}{\xi-x}}
                \intertext{Wegen der Monotonie und Stetigkeit von $f$ können wir umformen zu}
                &= \frac{1}{\lim_{\xi\fromto x} \frac{f\of{\xi} - f\of{x}}{\xi-x}} = \frac{1}{f'\of{x}}\qedhere
            \end{align*}
        \end{proof}
    \end{satz}

    \begin{beispiel}[Ableitung des Logarithmus]
        Wir definieren $f\definedas e^x$. Damit gilt
        \begin{align*}
            \log y &= f^{-1}\of{y}\\
            \log\pair{e^x} &= x\\
            \pair{\log y}' &= \frac{1}{\pair{e^x}'_{\lvert y=e^x}} = \frac{1}{\pair{e^x}_{\lvert y=e^x}} = \frac{1}{y}
        \end{align*}
    \end{beispiel}

    \begin{bemerkung}[Ableitung von Potenzen]
        \footnote{In VL erst im nächsten Unterkapitel behandelt}
        \begin{align*}
            \pair{x^{\alpha}}' &= \pair{e^{\log x^{\alpha}}} = \pair{e^{\alpha\log x}}'
            \intertext{Nach Satz~\ref{satz:kettenregel}}
            &= e^{\alpha \log x} \cdot \alpha \cdot \pair{\log x}' = \frac{\alpha x^{\alpha}}{x} = \alpha \cdot x^{\alpha-1}
        \end{align*}
    \end{bemerkung}

    \begin{beispiel}[Ableitung von $\arccos$]
        Es sei $y\in\pair{-1, 1}$. Dann gilt
        \begin{align*}
            \pair{\arccos y}' &= \frac{1}{\pair{\cos x}'_{\lvert \cos x = y}} = \frac{1}{-\sin\of{\arccos\of{y}}}\\
            &= \frac{1}{\pm\sqrt{1-\cos^2\of{\arccos y}}} = \pm \frac{1}{\sqrt{1-y^2}}
        \end{align*}
    \end{beispiel}

    \subsection{[*] Lokale Extrema und Mittelwertsätze}

    \begin{definition}[Lokales Maximum und Minimum] % Definition 7
        Sei $f: \pair{a,b} \fromto\R$. Man sagt $f$ habe in $x\in\pair{a,b}$ ein lokales Maximum (Minimum), wenn ein $\varepsilon > 0$ existiert, so dass
        \begin{align*}
            f\of{x} \underset{(\leq)}{\geq} f\of{\xi}\quad \forall \xi\in\pair{x-\varepsilon, x+\varepsilon}
        \end{align*}
    \end{definition}

    \begin{satz}[Ableitung bei lokalen Extrema]
        \label{satz:ableitung-extrem}
        Sei $f: \pair{a,b} \fromto \R$ differenzierbar und $x\in\pair{a,b}$ ein lokales Extremum. Dann gilt $f'\of{x} = 0$.
        \begin{proof}
            $\ex \varepsilon > 0$, $f\of{\xi} \leq f\of{x}$, $\xi\in\pair{x-\varepsilon, x+\varepsilon}$
            \begin{align*}
                f'_{+}\of{x} &= \lim_{\substack{\xi\fromto x\\ \xi > x}} \frac{f\of{\xi} - f\of{x}}{\xi - x} \leq 0\\
                f'_{-}\of{x} &= \lim_{\substack{\xi\fromto x\\ \xi < x}} \frac{f\of{\xi} - f\of{x}}{\xi - x} \geq 0\\
                \impl f'_{+} &= f'_{-} = 0\qedhere
            \end{align*}
        \end{proof}
    \end{satz}

    \begin{bemerkung}
        Die Umkehrung gilt nicht. Wir betrachten $f: x\mapsto x^3$ mit $f'(0) = 3x^2 = 0$, aber die Funktion hat an der Stelle $x=0$ kein lokales Maximum oder Minimum.
    \end{bemerkung}

    \begin{bemerkung}
        Sei $f$ auf $\interv{a,b}$ stetig und auf $\pair{a,b}$ differenzierbar. Dann kann das Maximum/Minimum der Funktion auch auf den Intervall-Grenzen $a$ und $b$ liegen, obwohl die Ableitung für diese nicht bestimmbar ist.
    \end{bemerkung}

    \begin{satz}[Satz von Rolle]
        \label{satz:von-rolle}
        Sei $a < b$, $f: \interv{a,b} \fromto \R$ eine stetige Funktion mit $f(a) = f(b)$. Die Funktion $f$ sei in $\pair{a,b}$ differenzierbar. Dann $\ex\xi\in\pair{a,b}$ mit $f'\of{\xi} = 0$.
        \begin{proof}
            ~\\
            \textsc{Fall 1.} $f$ ist eine konstante Funktion. Dann gilt $f' = 0$.\\
            \textsc{Fall 2.} $f$ ist keine konstante Funktion. Das heißt nach Satz~\ref{satz:weierstrass-maximum-minimum} $\exists x$ mit $f\of{x} \neq f\of{a}$. Wenn $f(x) > f(a)$, dann existiert ein lokales Maximum bei $x_0\in\pair{a,b}$ und wenn $f(x) < f(a)$, dann existiert ein lokales Minimum bei $x_0\in\pair{a,b}$. Damit ist $f'\of{x_0} = 0$.
        \end{proof}
    \end{satz}

    \newpage

    \begin{satz}[Mittelwertsatz]
        \label{satz:mittelwertsatz}
        Sei $a < b$, $f:\interv{a,b}\fromto\R$ stetig und in $\pair{a,b}$ differenzierbar. Dann gilt
        \begin{align*}
            \ex\xi\in\pair{a,b}\colon f\of{b} - f\of{a} = f'\of{\xi}\cdot\pair{b-a}
        \end{align*}
        \begin{proof}
            \begin{align*}
                F(x) &\definedas f(x) - \frac{f\of{b}-f\of{a}}{b-a}\cdot\pair{x-a}\\
                F(a) &= f(a) - \frac{f(b)-f(a)}{b-a}\cdot 0 = f(a)\\
                F(b) &= f(b) - \frac{f(b)-f(a)}{b-a}\cdot\pair{b-a} = f\of{a}\\
                \impl F(a) &= F(b)
                \intertext{Damit gilt nach Satz~\ref{satz:von-rolle}}
                \impl \ex\xi \text{ mit } F'\of{\xi} &= 0\\
                \impl f'\of{\xi} - \frac{f\of{b} - f\of{a}}{b-a} &= 0\\
                \impl f(b) - f(a) &= f'\of{\xi}\cdot\pair{b-a}\qedhere
            \end{align*}
        \end{proof}
    \end{satz}

    \begin{visualisierung}[Geometrische Anschauung des Mittelwertsatzes]
        Der Mittelwertsatz sagt aus, dass es einen Punkt auf jeder differenzierbaren Funktion gibt, dessen Tangente parallel zu einer affin-linearen Funktion durch $f(a)$ und $f(b)$ läuft.
        \begin{figure}[H]
            \centering
            \begin{tikzpicture}
                \draw[->] (0, 0) -- (4, 0);
                \draw[->] (0, 0) -- (0, 4);
                \draw (0, 0.1) -- (0, -0.1) node[below] {$a$};
                \draw (4.21*.6, 0.1) -- (4.21*.6, -0.1) node[below] {$\xi$};
                \draw (5*.6, 0.1) -- (5*.6, -0.1) node[below] {$b$};
                \fill (0,0) circle[radius=1.5pt];
                \fill (5*.6,6*0.6*0.6) circle[radius=1.5pt];
                \draw[scale=0.6, domain=0:5, smooth, variable=\x] plot ({\x}, {(-43/120 *\x*\x*\x*\x+71/20*\x*\x*\x-1337/120*\x*\x+259/20*\x)*0.6});
                \draw[scale=0.6, domain=-1:6, smooth, variable=\x] plot ({\x}, {(\x*6/5)*0.6});
                \draw[scale=0.6, domain=-1:6, smooth, variable=\x, dashed] plot ({\x}, {(\x*6/5+4.35)*0.6});
            \end{tikzpicture}
            \caption{Tangente an der Stelle $x=\xi$\\parallel zur Geraden durch die beiden Punkte}
        \end{figure}
    \end{visualisierung}

    \begin{korollar} % Korollar 10
        Es gilt genau dann $f'(x) = 0$ für alle $x\in\pair{a,b}$, wenn $f(x)$ eine konstante Funktion in $\interv{a,b}$ ist.
        \begin{proof}
            \begin{align*}
                f(x_1) - f(x_2) \annot[{&}]{=}{\ref{satz:mittelwertsatz}} f'\of{\xi}\cdot\pair{x_1 - x_2}\\
                &= 0\cdot\pair{x_1 - x_2}\\
                \impl f(x_1) &= f(x_2)\qedhere
            \end{align*}
        \end{proof}
    \end{korollar}

    \begin{satz} % Satz 11
        \label{satz:18-11}
        Seien $f$ und $g$ stetig auf $\interv{a,b}$ und auf $\pair{a,b}$ differenzierbar. Dann gilt
        \begin{align*}
            \ex\xi\in\pair{a,b}\colon \interv{f(b)-f(a)}\cdot g'\of{\xi} &= \interv{g(b)-g(a)}\cdot f'\of{\xi}
        \end{align*}
        \begin{proof}
            \begin{align*}
                h(t) &\definedas \interv{f(b)-f(a)}\cdot g(t) - \interv{g(b)-g(a)}\cdot f(t)\\
                \impl h(a) &= h(b)\\
                \annot{\impl}{\ref{satz:mittelwertsatz}} \ex \xi \in\pair{a,b} &\text{ mit } h'\of{\xi} = 0\\
                \impl \interv{f(b)-f(a)}\cdot g'\of{\xi} &- \interv{g(b)-g(a)}\cdot f'\of{\xi} = 0\\
                \impl \interv{f(b)-f(a)}\cdot g'\of{\xi} &= \interv{g(b)-g(a)}\cdot f'\of{\xi}\qedhere
            \end{align*}
        \end{proof}
    \end{satz}

    \begin{notation}[Ableitungen höherer Ordnung] % Definition 11
        Wir definieren für die zweite Ableitung
        \begin{align*}
            f'' &= \pair{f'}'
            \intertext{und allgemein}
            f^{(n)} &= \pair{f^{(n-1)}}'
        \end{align*}
    \end{notation}

    \begin{satz}[Der Taylorsche Satz] % Satz 12
        \label{satz:taylor}
        Es sei $f$ eine reelle Funktion auf $\interv{a,b}$ und $f^{(n-1)}$ sei stetig auf $\interv{a,b}$ und $f^{(n)}$ existiere auf $\pair{a,b}$. Sei $\pair{\alpha,\beta}\subseteq \interv{a,b}$ und
        \begin{align*}
            P_{n-1}(t) &= \sum_{k=0}^{n-1} \frac{f^{(k)}\of{\alpha}}{k!}\cdot\pair{t-\alpha}^k\tag{\footnotemark}
            \intertext{Dann $\ex\xi\in\pair{\alpha,\beta}$ so dass}
            f\of{\beta} &= P_{n-1}\of{\beta} + \frac{f^{(n)}\of{\xi}}{n!}\cdot\pair{\beta-\alpha}^n
        \end{align*}

        \footnotetext{In manchen Lehrbüchern wird auch $T_n\of{f,\alpha}$ als alternative Schreibweise zu $P_{n}\of{t}$ verwendet.}

        %%%%%%%%%%%%%%%%%%%%%%%%
        % 13. Februar 2023
        %%%%%%%%%%%%%%%%%%%%%%%%

        \begin{proof}
            \marginnote{[13. Feb]}
            \begin{align*}
                M&\definedas \frac{f(\beta)-P_{n-1}\of{\beta}}{\pair{\beta-\alpha}^n}\quad\in\R\\
                g(t) &\definedas f(t)-P_{n-1}\of{t} - M \cdot \pair{t-\alpha}^n\quad \alpha\leq t\leq\beta\\
                g\of{\alpha} &= f\of{\alpha} - \underbrace{0 - f\of{\alpha}}_{P_{n-1}\of{\alpha}} - 0 = 0\\
                g'\of{\alpha} &= f'\of{\alpha} - f'\of{\alpha} - 0 = 0\\
                \vdots\\
                g^{(n-1)}\of{\alpha} &= 0\\[10pt]
                g\of{\beta} &= f\of{\beta} - P_{n-1}\of{\beta} - \frac{f\of{\beta} - P_{n-1}\of{\beta}}{\pair{\beta-\alpha}^n}\cdot \pair{\beta-\alpha}^n = 0
                \intertext{1. Schritt}
                g\of{\alpha} &= 0,~g\of{\beta} = 0
                \intertext{Nach Satz~\ref{satz:von-rolle}}
                \impl \ex x_1\colon g'(x_1) &= 0
                \intertext{2. Schritt}
                g'\of{\alpha} &= 0,~g'\of{x_1} = 0\\
                \impl \ex x_2\in\pair{\alpha, x_1}\colon g''(x_2) &= 0\\[10pt]
                g''(\alpha) &= 0,~g''\of{x_2} = 0\\
                \vdots\\
                \impl\ex x_n \in\pair{\alpha, x_{n-1}}\colon g^{(n)}\of{x_n} &= 0\tag{$\xi\definedas x_n$}\\
                f^{(n)}\of{\xi} - 0 - M\cdot n! &= 0\\
                M = \frac{f^{(n)}\of{\xi}}{n!} &= \frac{f\of{\beta}-P_{n-1}\of{\beta}}{\pair{\beta-\alpha}^n}\\
                \equivalent f\of{\beta} - P_{n-1}\of{\beta} &= \frac{f^{(n)}\of{\xi}}{n!}\cdot\pair{\beta-\alpha}^n\\
                f\of{\beta} &= P_{n-1}\of{\beta} + \frac{f^{(n)}\of{\beta}}{n!}\pair{\beta-\alpha}^n\qedhere
            \end{align*}
        \end{proof}
    \end{satz}

    \newpage

    \begin{bemerkung}
        Der Taylorsche Satz ist eine Verallgemeinerung des Mittelwertsatzes für höhere Ableitungen.
    \end{bemerkung}

    \begin{korollar}[Monotonie] % Korollar 14
        \label{korollar:monotonie}
        Sei $f$ auf $\pair{a,b}$ differenzierbar mit $f' > 0$ ($f' < 0$). Dann ist $f$ streng monoton wachsend (fallend) auf $\pair{a,b}$.
        \begin{proof}
            Seien $x_1, x_2\in\pair{a,b}$ mit $x_2 > x_1$. Nach Satz~\ref{satz:mittelwertsatz} gilt
            \begin{align*}
                f(x_2) - f(x_1) &= \underbrace{f'(\xi)}_{> 0}\cdot\underbrace{\pair{x_2 - x_1}}_{> 0} > 0\qedhere
            \end{align*}
            Der Beweis für fallende Funktionen funktioniert analog.
        \end{proof}
    \end{korollar}

    \begin{bemerkung}[Über Max und Min]
        Es seien $f$ differenzierbare Funktion und $x_0$ lokales Extremum von $f$ und sei $f''\of{x_0}$ existent und positiv (negativ). Dann ist $x_0$ ein lokales Minimum (Maximum).

        \begin{proof}
            \begin{align*}
                f''\of{x_0} &= \lim_{\xi\fromto x_0} \frac{f'\of{\xi} - f'\of{x_0}}{\xi-x_0} > 0\\
                &\impl \ex\varepsilon > 0\colon \frac{f'\of{\xi} - f'\of{x_0}}{\xi-x_0} > 0\quad\forall\xi, 0 < \abs{x_0-\xi} < \varepsilon\\
                &\impl \begin{cases}
                           f'\of{\xi} < 0 \text{ für } \xi < x_0~\leadsto \text{ Funktion fällt}\\
                           f'\of{\xi} > 0 \text{ für } \xi > x_0~\leadsto \text{ Funktion steigt}
                \end{cases}
            \end{align*}
            Da die Funktion vor $x_0$ fällt und danach steigt, ist $x_0$ ein lokales Minimum.
        \end{proof}
    \end{bemerkung}

    \newpage

    \subsection{[*] Die Regel von l'Hospital}

    \begin{satz}[Regel von de l'Hospital] % Satz 16
        \label{satz:l-hospital}
        Seien $f$ und $g$ differenzierbar in $\pair{a,b}$. Sei ferner $g'\of{x} \neq 0$ für alle $x\in\pair{a,b}$ und es gelte
        \begin{align*}
            \frac{f'(x)}{g'(x)} \fromto A \text{ für } x \fromto a\tag{1}
            \intertext{Außerdem gelte}
            f(x) \fromto 0 \text{ und } g(x)\fromto 0 \text{ für } x\fromto a\tag{2}
            \intertext{\underline{oder}}
            g(x)\fromto \infty \text{ für } x\fromto a\tag{2}
            \intertext{Dann gilt}
            \frac{f(x)}{g(x)}\fromto A
        \end{align*}
        Die analoge Behauptung war für $x\fromto b$ oder für $g(x)\fromto -\infty$.

        \begin{bemerkung}
            $g'(a) = 0$ ist bei der Anwendung des Satzes erlaubt.
        \end{bemerkung}

        \begin{proof}
            Sei $A < \infty$
            \begin{align*}
                \impl\ex q > A
                \intertext{Sei $A < r < q$}
                \impl \frac{f'(x)}{g'(x)} < r \text{ für } a < x < a + \varepsilon_0
                \intertext{Nach Satz~\ref{satz:18-11} gilt mit $a < x < y < a + \varepsilon_0$}
                \frac{f(x)-f(y)}{g(x)-g(y)} = \frac{f'(\xi)}{g'(\xi)}  < r
                \intertext{\textsc{Fall 1.} $f(x)\fromto 0, g(x)\fromto 0$ für $x\fromto a$}
                \impl g(y) \neq 0, \text{ weil } g(a) = 0,~g'(a)\neq 0\\
                \lim_{x\fromto a} \frac{\overbrace{f(x)}_{\fromto 0}-f(y)}{\underbrace{g(x)}_{\fromto 0}-g(y)} &= \lim_{x\fromto a} \frac{f(y)}{g(y)} \leq r < q\\
                \impl \fa a < y < a + \varepsilon_0\colon \frac{f(y)}{g(y)} &< q\\
                \impl \lim_{y\fromto a} \frac{f(y)}{g(y)} &< q
                \intertext{\textsc{Fall 1.} $g(x)\fromto\infty$ für $x\fromto a$. Dann gilt mit $a < x < y < a + \varepsilon_0$}
                \frac{f(x)-f(y)}{g(x)-g(y)} &< r\\
                \impl \frac{\frac{f(x)}{g(x)}- \frac{f(y)}{g(x)}}{1 - \frac{g(y)}{g(x)}} &< r\\
                \impl \frac{\frac{f(x)}{g(x)}}{1} &\leq r < q\\
                \impl\fa q > A\colon \frac{f(x)}{g(x)} < q
            \end{align*}
            Sei $A > -\infty$
            \begin{align*}
                \impl\ex p < A, p < r < A\\
                \frac{f(x)-f(y)}{g(x)-g(y)} &= \frac{f'\of{\xi}}{g'\of{xi}} &> r > p\quad a < x < y < a + \varepsilon_0
                \intertext{Mit der gleichen Argumentation wie davor ergibt sich}
                \frac{f(x)}{g(x)} &> p\\
                \impl \frac{f(x)}{g(x)} &\fromto A\qedhere
            \end{align*}
        \end{proof}
    \end{satz}

    \begin{beispiel}
        \begin{align*}
            \lim_{x\fromto 1_-} \interv{\ln x \cdot \ln\of{1-x}} &= \lim_{x\fromto 1_-} \frac{\ln\pair{1-x}}{\frac{1}{\ln\of{x}}}\\
            &= \lim_{x\fromto 1_-} \frac{\frac{1}{1-x}\cdot\pair{-1}}{-\frac{1}{\ln^2\of{x}} \cdot \frac{1}{x}}\\
            &= \lim_{x\fromto 1_-} \frac{x\cdot\ln^2\of{x}}{1-x}\\
            &= \lim_{x\fromto 1_-} \frac{2\ln\of{x}\cdot \frac{1}{x}}{-1} = 0
        \end{align*}
    \end{beispiel}

    \newpage


    \section{Konvexität}
    %%%%%%%%%%%%%%%%%%%%%%%%
% 15. Februar 2024
%%%%%%%%%%%%%%%%%%%%%%%%

\thispagestyle{pagenumberonly}

\subsection{Konvexe und konkave Funktionen}

\begin{skizze}[Konvexe Funktion]
    \marginnote{[15. Feb]}
    Wähle ein $\lambda\in\pair{0,1}$ und formuliere die Interpolation $r=\lambda x + \pair{1-\lambda}\cdot y$.
    \begin{figure}[H]
        \centering
        \begin{tikzpicture}
            \draw[->] (-1, 0) -- (3, 0);
            \draw[->] (0, -1) -- (0, 4);
            \draw (-0.95*0.5, 0.1) -- (-0.95*0.5, -0.1) node[below] {$x$};
            \draw (5.2*0.5, 0.1) -- (5.2*0.5, -0.1) node[below] {$y$};

            \fill (-0.95*.5,0.7875*.5) circle[radius=1.5pt] node[left] {$f(x)$};
            \fill (5.2*.5,5.4*.5) circle[radius=1.5pt] node[right] {$f(y)$};
            \draw[domain=-2:6, smooth, variable=\x] plot ({0.5*\x}, {(0.2*(\x-0.25)^2+0.5)*0.5}) node[anchor=east] {$f$};
            \draw[domain=-0.95:5.2, smooth, variable=\x, dashed] plot ({0.5*\x}, {(0.75*\x+1.5)*0.5});
        \end{tikzpicture}
        \caption{Konvexe Funktion mit eingezeichneter Sekante}
    \end{figure}
    \noindent Wir erhalten die Sekantengleichung $\lambda f(x) +\pair{1-\lambda}\cdot f(y)$. Die Funktion ist konvex, wenn sie unter der Sekante verläuft. Das heißt
    \begin{align*}
        \lambda f(x) +\pair{1-\lambda} f(y) \geq f\of{x+\pair{1-\lambda}\cdot y}\quad\forall\lambda\in\pair{0,1} \text{ und } x,y\in D
    \end{align*}
\end{skizze}

\begin{definition}[Konvexität/Konkavität]
    Sei $f: D\fromto \R$ wobei $D\subseteq \R$ ein Intervall. Die Funktion $f$ heißt konvex (konkav), falls für alle $x,y\in D$ und alle $\lambda \in\pair{0,1}$ gilt
    \begin{align*}
        f\of{\lambda x + \pair{1-\lambda}\cdot y} \underset{(\geq)}{\leq} \lambda f(x) + \pair{1-\lambda}\cdot f(y)
    \end{align*}
\end{definition}

\begin{beispiel}
    Wir betrachten $f: \R\fromto\R$ mit $x\mapsto x^2$. Sei $x,y\in\R$ und \OBDA sei $y\geq x$, $\lambda\in\pair{0,1}$. Dann ist zu zeigen
    \begin{align*}
        f\of{\lambda x + \pair{1-\lambda}\cdot y} &\leq \lambda f\of{x} + \pair{1-\lambda}\cdot  f\of{y}\\
        \equivalent \pair{\lambda x+\pair{1-\lambda}\cdot y}^2 &\leq \lambda x^{2} + \pair{1-\lambda}\cdot y^2\\
        \equivalent \lambda^2 x^2 + 2\lambda\pair{1-\lambda}\cdot xy + \pair{1-\lambda}^{2}\cdot y^2 & \leq \lambda x^2 + \pair{1-\lambda}\cdot y^2\\
        \equivalent \lambda\cdot\pair{\lambda -1}\cdot x^2 + \pair{1-\lambda}\cdot\pair{1-\lambda-1}\cdot y^2 + \lambda\pair{1-\lambda}\cdot 2xy &\leq 0\\
        \equivalent \lambda\pair{\lambda -1}\cdot\interv{x^2+y^2-2xy} &\leq 0\\
        \equivalent x^2+y^2-2xy &\geq 0\\
        \equivalent \pair{x-y}^2 &\geq 0
    \end{align*}
\end{beispiel}

\hfill

\begin{satz}[Konvexität und zweite Ableitung] % Satz 1
    Sei $D\subseteq\R$ ein offenes Intervall und $f: D\fromto\R$ eine auf $D$ zweimal differenzierbare Abbildung. Dann gilt
    \begin{align*}
        f \text{ ist konvex }\quad \equivalent\quad \forall x\in D\colon f''(x)\geq 0
    \end{align*}
    \newpage
    \begin{proof}
        \anf{$\Leftarrow$}: Da $f'' > 0$ auf $D$, ist $f'$ auf $D$ monoton wachsend. Sei $x,y\in D$, \OBDA sei $y>x$ und $\lambda\in\pair{0,1}$, $r\definedas\lambda x + \pair{1-\lambda}\cdot y$.
        Es gilt nach Konstruktion, dass $x < r < y$. Wir untersuchen die Intervalle $D_1\definedas\pair{x,r}$ und $D_{2}\definedas\pair{r,y}$. Nach Satz~\ref{satz:mittelwertsatz} gilt $\exists \xi_1\in D_1,\xi_2\in D_2, \xi_1 < \xi_2$ sodass
        \begin{align*}
            \frac{f(r)-f(x)}{r-x} = f'\of{\xi_1} &\leq f'\of{\xi_2} = \frac{f(y)-f(r)}{y-r}
            \intertext{Beachte $r-x = \lambda x + \pair{1-\lambda}\cdot y - x = \pair{1-\lambda}\pair{y-x}$ und $y-r = \lambda\pair{y-x}$}
            \impl \frac{f(r)-f(x)}{\pair{1-\lambda}\pair{y-x}} &\leq \frac{f(y)-f(r)}{\lambda\pair{y-x}}\\
            \impl f(r) &\leq \pair{1-\lambda}\cdot f(y) + \lambda f(x)\\
            \impl f(\lambda x + \pair{1-\lambda}\cdot y) &\leq \pair{1-\lambda}\cdot f(y) + \lambda f(x)\\
            \impl f &\text{ ist konvex}
            \intertext{\anf{$\impl$}: Wir führen einen Beweis mit Kontraposition. Angenommen es existiert ein $x_0\in D$ so, dass $f''(x_0) < 0$. Wir definieren die Hilfsfunktion}
            \varphi\of{x} &= f(x) + c\cdot\pair{x-x_0}\\
            \impl \varphi'\of{x} &= f'\of{x} + c\\
            \impl \varphi'\of{x_0} &= f'\of{x_0} + c
            \intertext{Wähle $c=-f'\of{x_0}$}
            \impl \varphi'\of{x_0} &= 0
            \intertext{Es gilt $\varphi''\of{x_0} = f''\of{x_0} < 0$}
            \impl \varphi \text{ hat in } &x_0\in D \text{ ein isoliertes Maximum}\\
            \impl \exists t > 0\colon \varphi\of{x_0-t} &< \varphi\of{x_0} \land \varphi\of{x_0+t} < \varphi\of{x_0}\\
            \impl f(x_0) = \varphi\of{x_0} &> \frac{1}{2}\pair{\varphi\of{x_0+t} + \varphi\of{x_0-t}}\\
            \impl f(x_0) &> \frac{1}{2}\pair{f(x_0+t)+ct + f(x_0-t)-ct}\\
            \impl f(x_0) &> \frac{1}{2}\pair{f(x_0+t)+f(x_0-t)}
            \intertext{Wir wählen $y=x_0+t$, $x=x_0-t$, $\lambda = \frac{1}{2}$}
            \impl f(\lambda x + \pair{1-\lambda}\cdot y) &> \frac{1}{2}f(x) + \pair{1-\lambda}f(y)\\
            \impl f &\text{ ist nicht konvex}\qedhere
        \end{align*}
    \end{proof}
\end{satz}

\begin{beispiel}
    Wir betrachten $f: \pair{0,\infty}\fromto\R$, $x\mapsto x^r$ für $r\in\R$
    \begin{align*}
        f'\of{x} &= r\cdot x^{r-1}\\
        f''(x) &= r\cdot\pair{r-1}\cdot x^{r-2}
    \end{align*}
    Für $r\geq 1$ und $r\leq 0$ ist $f$ auf $\pair{0,\infty}$ konvex, für $r\in\pair{0,1}$ konkav.
\end{beispiel}

\begin{beispiel}
    Die Funktion $f: \R\fromto\R$, $x\mapsto e^{\lambda x}$
    \begin{align*}
        f''(x) &= \lambda^2\cdot e^{\lambda x} > 0
    \end{align*}
    ist konvex auf $\R$.
\end{beispiel}

\begin{beispiel}
    Die Funktion $f: \pair{0, \infty}\fromto\R$, $x\mapsto\ln\of{x}$
    \begin{align*}
        f''(x) &= \frac{-1}{x^2} < 0
    \end{align*}
    ist konkav auf $\pair{0, \infty}$.
\end{beispiel}

\begin{beispiel}
    Die Funktion $f: \interv{-\frac{\pi}{2}, \frac{\pi}{2}}$, $x\mapsto \arctan\of{x}$
    \begin{align*}
        f'(x) &= \frac{1}{1+x^2} = \pair{1+x^2}^{-1}\\
        \impl f''(x) &= -\pair{1+x^2}^{-2}\cdot 2x
    \end{align*}
    ist konvex im Intervall $\interv{-\frac{\pi}{2}, 0}$ und konkav in $\interv{0, \frac{\pi}{2}}$.
\end{beispiel}

\begin{lemma} % Lemma 2
    \label{lemma:konvex-verkettung}
    Seien $D_1, D_2\sbset\R$ Intervalle und $f: D_1\fromto D_2$ und $g: D_2\fromto \R$. Dann gilt
    \begin{enumerate}[label=(\roman*)]
        \item Falls $f$ konvex, $g$ konvex und monoton wachsend $\impl g\circ f: D_1\fromto\R$ ist konvex
        \item Falls $f$ konkav, $g$ konvex und monoton fallend $\impl g \circ f: D_1\fromto \R$ ist konvex
    \end{enumerate}

    \begin{proof}[Beweis (i)]
        Es seien $x,y\in D_1$ mit $x < y$ und $\lambda\in\pair{0,1}$ dann ist
        \begin{align*}
            \pair{g\circ f}\of{\lambda x + \pair{1-\lambda}\cdot y} &= g\of{f\of{\lambda x + \pair{1-\lambda}\cdot y}}
            \intertext{Wegen der Konvexität von $f$ und der Monotonie von $g$ gilt}
            \impl \pair{g\circ f}\of{\lambda x + \pair{1-\lambda}\cdot y} &\leq g\of{\lambda f\of{x} + \pair{1-\lambda}\cdot f\of{y}}\\
            \impl \pair{g\circ f}\of{\lambda x + \pair{1-\lambda}\cdot y} &\leq \lambda g\of{f\of{x}} + \pair{1-\lambda}\cdot g\of{f\of{y}}\\
            \impl \pair{g\circ f}\of{\lambda x + \pair{1-\lambda}\cdot y} &\leq \lambda\pair{g\circ f}\of{x} + \pair{1-\lambda}\cdot\pair{g\circ f}\of{y}\qedhere
        \end{align*}
    \end{proof}
    \noindent Der Beweis von (ii) funktioniert analog.
\end{lemma}

\begin{lemma} % Lemma 3
    \label{lemma:konvex-umkehrabbildung}
    Es seien $D, B\sbset\R$ Intervalle und $f: D\fromto B$ bijektiv mit $f^{-1}: B\fromto D$ Umkehrabbildung. Dann gilt
    \begin{enumerate}[label=(\roman*)]
        \item $f$ ist monoton wachsend und konvex $\equivalent f^{-1}$ ist monoton wachsend und konkav
        \item $f$ ist monoton fallend und konvex $\equivalent f^{-1}$ ist monoton fallend und konvex
        \item $f$ ist monoton fallend und konkav $\equivalent f^{-1}$ ist monoton fallend und konkav
    \end{enumerate}
\end{lemma}

\begin{uebung}
    Beweisen Sie Lemma~\ref{lemma:konvex-umkehrabbildung}.
\end{uebung}

\newpage

\subsection{Ungleichungen von Jensen und Hölder}
\begin{satz}[Ungleichung von Jensen]
    \label{satz:ungleichung-jensen}
    Sei $f: D\fromto\R$ konvex (konkav), sowie $x_1,\dots, x_n\in D$ und $\lambda_1, \dots, \lambda_n\in\pair{0,1}$ mit $ \sum_{}^{} \lambda_i = 1$. Dann ist
    \begin{align*}
        f\of{\sum_{i=1}^{n} \lambda_i x_i}\underset{(\geq)}{\leq}~ \sum_{i=1}^{n} \lambda_i f\of{x_i}
    \end{align*}

    \begin{proof}
        Wir nutzen vollständige Induktion.\\
        \begin{induktionsanfang}
            $n=2$ gilt wegen Definition der Konvexität.
        \end{induktionsanfang}
        \begin{induktionsvoraussetzung}
            Es gelte die Behauptung für ein festes, aber beliebiges $n\in\N$ mit $n\geq 2$.
        \end{induktionsvoraussetzung}
        \begin{induktionsschritt}
            $n\fromto n+1$
            \begin{align*}
                f\of{\sum_{i=1}^{n+1} \lambda_i x_i} &= f\of{\underbrace{\pair{1-\lambda_{n+1}}}_{1-\theta}\cdot \underbrace{\sum_{i=1}^{n} \frac{\lambda_i}{1-\lambda_{n+1}}\cdot x_i}_{x} + \underbrace{\lambda_{n+1}}_{\theta}\cdot \underbrace{x_{n+1}}_{y}}
                \intertext{Nach der Konvexität von $f$ gilt damit}
                &\leq \pair{1-\lambda_{n+1}}\cdot f\of{\sum_{i=1}^{n} \frac{\lambda_i}{1-\lambda_{n+1}}\cdot x_i} + \lambda_{n+1} \cdot f\of{x_{n+1}}\\
                \annot[{&}]{\leq}{IV} \pair{1-\lambda_{n+1}}\cdot \pair{\sum_{i=1}^{n} \frac{\lambda_i}{1-\lambda_{n+1}}\cdot f\of{x_i}} + \lambda_{n+1}\cdot f(x_{n+1})\\
                &= \sum_{i=1}^{n+1} \lambda_i f\of{x_i}
            \end{align*}
            Nach Induktion folgt die Behauptung.\qedhere
        \end{induktionsschritt}
    \end{proof}
\end{satz}


\begin{korollar}[Ungleichung vom arithmetisch-geometrischen Mittel] % Korollar 1
    Es seien $x_1, \dots, x_n\in \R$. Dann gilt
    \begin{align*}
        \sqrt[n]{\prod_{i=1}^{n} x_i} &\leq \frac{1}{n}\cdot \sum_{i=1}^{n} x_i
    \end{align*}
    \begin{proof}
        $f: \pair{0,\infty}$, $x\mapsto \ln\of{x}$ ist konkav.
        \begin{align*}
            \annot{\impl}{\ref{satz:ungleichung-jensen}} \ln\of{ \sum_{i=1}^{n} \lambda_i x_i} &\geq \sum_{i=1}^{n} \lambda_i \ln\of{x_i}\\
            \impl \sum_{i=1}^{n} \lambda_i x_i &\geq \exp\of{\sum_{i=1}^{n} \lambda_i\cdot \ln\of{x_i}}\\
            &= \prod_{i=1}^{n}  e^{\ln\of{x_i^{\lambda_i}}}\\
            &= \prod_{i=1}^{n}  x_i^{\lambda_i}
            \intertext{Wähle $\lambda_i = \frac{1}{n}$}
            \impl \frac{1}{n}\cdot\sum_{i=1}^{n} x_i &\geq \sqrt[n]{\prod_{i=1}^{n} x_i}\qedhere
        \end{align*}
    \end{proof}
\end{korollar}

\begin{korollar}[Ungleichung von Hölder\footnote{Otto Hölder (1859 - 1937), deutscher Mathematiker, der durch seine Arbeit zur Hölder-Ungleichung, Hölder-Stetigkeit und Kompositionsreihen bekannt ist und der Schwiegervater der Enkelin Marius Sophus Lies.}] % Korollar 2
    \label{korollar:hoelder}
    Seien $p,q\in\pair{1,\infty}$ mit $\frac{1}{p} + \frac{1}{q} = 1$, $x = \pair{x_1, \dots, x_n}\in\C^n$, $y = \pair{y_1, \dots, y_n}\in\C^n$. Wir definieren die $p$-Norm
    \begin{align*}
        \norm{x}_p &\definedas \pair{\sum_{i=1}^{n} \abs{x_i}^p}^{\frac{1}{p}}
        \intertext{Dann gilt}
        \sum_{i=1}^{n} \abs{x_i y_i} &\leq \norm{x}_p \cdot \norm{y}_q
    \end{align*}
    \begin{proof}
        Da $\frac{1}{p} + \frac{1}{q} = 1$ und $\ln$ konvex gilt für beliebige $\zeta, \eta\in\R$
        \begin{align*}
            \ln\of{\frac{1}{p} + \frac{1}{q}} &\geq \frac{1}{p}\ln\of{\zeta} + \frac{1}{q}\ln\of{\eta}\\
            \impl \frac{\zeta}{p} + \frac{\eta}{q} &\geq \zeta^{\frac{1}{p}} \eta^{\frac{1}{q}}
            \intertext{Wir definieren}
            \zeta_k \definedas \frac{\abs{x_k}^p}{\norm{x}_p^p}\quad&\quad\eta_n \definedas \frac{\abs{y_n}^q}{\norm{y}_q^q}\\
            \impl \sum_{k=1}^{n} \frac{\abs{x_k}\abs{y_k}}{\norm{x}_p\norm{y}_q} &\leq \sum_{k=1}^{n} \frac{1}{p}\cdot\frac{\abs{x_k}^p}{\norm{x}_p^p} + \frac{1}{q}\cdot\frac{\abs{y_k}^q}{\norm{y}_q^q} = \frac{1}{p}+\frac{1}{q} = 1\\
            \impl \sum_{k=1}^{n} \abs{x_k}\abs{y_k} &\leq \norm{x}_p\cdot \norm{y}_q\qedhere
        \end{align*}
    \end{proof}
\end{korollar}


    \vfill

    \begin{center}
        \textbf{\LARGE Pause bis zum Sommersemester 2024}
    \end{center}

    \vfill

\end{document}
