\documentclass[11pt, twoside, a4paper]{article}

% Setup
\usepackage[margin=2.4cm, top=3.5cm]{geometry}
\usepackage[utf8]{inputenc}
\usepackage[ngerman]{babel}

% Package imports
\usepackage{amsfonts}
\usepackage{amsmath}
\usepackage{amssymb}
\usepackage{amsthm}
\usepackage{mathtools}
\usepackage{setspace}
\usepackage{float}
\usepackage{enumitem}
\usepackage{hyperref}
\usepackage[pagestyles]{titlesec}
\usepackage{fancyhdr}
\usepackage{colonequals}
\usepackage{caption}
\usepackage{tikz}

% Theorems
\newtheorem{blockelement}{Blockelement}[subsection]
\newtheoremstyle{plain}{}{}{}{}{\bfseries}{.}{ }{}
\theoremstyle{plain}
\newtheorem{bemerkung}[blockelement]{Bemerkung}
\newtheorem{definition}[blockelement]{Definition}
\newtheorem{lemma}[blockelement]{Lemma}
\newtheorem{satz}[blockelement]{Satz}
\newtheorem{notation}[blockelement]{Notation}
\newtheorem{korollar}[blockelement]{Korollar}
\newtheorem{uebung}[blockelement]{Übung}
\newtheorem{beispiel}[blockelement]{Beispiel}
\newtheorem{folgerung}[blockelement]{Folgerung}
\newtheorem{axiom}[blockelement]{Axiom}
\newtheorem{beobachtung}[blockelement]{Beobachtung}
\newtheorem{konzept}[blockelement]{Konzept}
\newtheorem{visualisierung}[blockelement]{Visualisierung}
\newtheorem{anwendung}[blockelement]{Anwendung}

% Paranthese
\newcommand{\set}[1]{\left\{#1\right\}}
\newcommand{\pair}[1]{\left(#1\right)}
\newcommand{\abs}[1]{\left|#1\right|}
\newcommand{\linterv}[1]{\left[#1\right)}
\newcommand{\rinterv}[1]{\left(#1\right]}
\newcommand{\interv}[1]{\left[#1\right]}

% Shorten commands
\newcommand{\equivalent}[0]{\Leftrightarrow{}}
\newcommand{\impl}[0]{\Rightarrow{}}
\newcommand{\definedas}[0]{\coloneqq}
\newcommand{\definedasbackwards}[0]{\eqqcolon}
\newcommand{\definedasequiv}[0]{\ratio\Leftrightarrow{}}
\newcommand{\infinity}[0]{\infty}
\newcommand{\exclude}[0]{\setminus}
\renewcommand{\emptyset}{\varnothing}
\newcommand{\annot}[2]{\overset{\text{#2}}{#1}}
\newcommand{\nn}[0]{\\[2\baselineskip]}
\newcommand{\anf}[1]{\glqq{}#1\grqq}
\newcommand{\OBDA}{o.B.d.A.}
\newcommand{\theoremescape}{\leavevmode}
\newcommand{\fromto}{\rightarrow{}}
\newcommand{\naturalnumbers}{\mathbb{N}}
\newcommand{\realnumbers}{\mathbb{R}}
\newcommand{\aligntoright}[2]{\hfill#1\hspace{#2\textwidth}~}
\newcommand{\horizontalline}[0]{\par\noindent\rule{0.05\textwidth}{0.1pt}\\}
\newcommand{\ntoinfty}[0]{n\fromto\infty}

% Envs
\newenvironment{induktionsanfang}{
    \rule{0pt}{3ex}\noindent
    \begin{minipage}[t]{0.11\textwidth}
    {I-Anfang}
    \end{minipage}
    \hfill
    \begin{minipage}[t]{0.89\textwidth}
    }
    {
    \end{minipage}
}
\newenvironment{induktionsschritt}{
    \rule{0pt}{3ex}\noindent
    \begin{minipage}[t]{0.11\textwidth}
    {I-Schritt}
    \end{minipage}
    \hfill
    \begin{minipage}[t]{0.89\textwidth}
    }
    {
    \end{minipage}
}

% Section style
\titleformat*{\section}{\LARGE\bfseries}
\titleformat*{\subsection}{\large\bfseries}

% Page styles
\newpagestyle{pagenumberonly}{
    \sethead{}{}{}
    \setfoot[][][\thepage]{\thepage}{}{}
}
\newpagestyle{headfootdefault}{
    \sethead[][][\thesubsection~\textit{\subsectiontitle}]{\thesection~\textit{\sectiontitle}}{}{}
    \setfoot[][][\thepage]{\thepage}{}{}
}
\pagestyle{headfootdefault}

\begin{document}
    \title{\vspace{3cm} Skript zur Vorlesung\\Analysis I\\bei Prof. Dr. Dirk Hundertmark}
    \author{Karlsruher Institut für Technologie}
    \date{Wintersemester 2023/24}
    \maketitle
    \begin{center}
        Dieses Skript ist inoffiziell. Es besteht kein\\ Anspruch auf Vollständigkeit oder Korrektheit.
    \end{center}
    \thispagestyle{empty}
    \newpage

    \tableofcontents
    ~\\
    Alle mit [*] markierten Kapitel sind noch nicht korrektur gelesen und bedürfen eventuell noch Änderungen.
    \newpage


    \section{Aussagenlogik}
    %%%%%%%%%%%%%%%%%%%%%%%%
% 26. Oktober 2023
%%%%%%%%%%%%%%%%%%%%%%%%

\thispagestyle{pagenumberonly}
\setcounter{subsection}{1}

\begin{definition}[Aussage]
    Eine Aussage ist eine Behauptung sprachlich oder mittels Formeln, welche entweder wahr oder falsch ist.
\end{definition}

\begin{beispiel}[Zulässige Aussagen]
    \theoremescape
    \begin{enumerate}[label=(\roman*)]
        \item Bielefeld existiert (w)
        \item $2+2=5$ (f)
        \item Es gibt unendlich viele Primzahlen (w)
    \end{enumerate}
\end{beispiel}

\begin{definition}[Aussageform]
    Eine Aussage, die von mindestens einer Variablen abhängt, nennt sich Aussageform.
    Wir schreiben zum Beispiel $H(x)$ für eine Aussage für die Variable $x$.
\end{definition}

\begin{beispiel}[Mögliche Aussageformen]
    \theoremescape
    \begin{enumerate}[label=(\roman*)]
        \item $H(x) \definedasequiv \pair{x^2-3x+2=0}$
        \item $G(x) \definedasequiv \pair{x=1\lor x=2}$
    \end{enumerate}
\end{beispiel}

\begin{konzept}[Beweisstruktur]
    \begin{equation*}
        \begin{split}
            p\\
            \text{Vorraussetzung}\\
            \text{hinreichend für }q
        \end{split}
        \begin{split}
            \qquad\impl\qquad
        \end{split}
        \begin{split}
            q\\
            \text{Behauptung}\\
            \text{notwendig für }p
        \end{split}
    \end{equation*}
    \noindent Beweis: $p\impl r_1 \impl r_2 \impl r_3 \impl \dots\impl r_n \impl q$. ($r_1,\dots, r_n$ sind bereits bekannte wahre Aussagen oder Axiome)
\end{konzept}

\begin{satz}[Regeln der Aussagenlogik]
    Seien $p,q,r$ Aussagen.
    Dann sind folgende Aussagen wahr:
    \begin{enumerate}[label=(\roman*)]
        \item $p \lor \neg p$ \aligntoright{(Tertium non datur)}{0.1}\\
        $p \impl p$\\
        $\neg (p \land \neg p)$
        \item $p\land q \equivalent q \land p$ \aligntoright{(Kommutativität)}{0.1}\\
        $p\lor q \equivalent q \lor p$
        \item $(p\land q) \land r \equivalent p \land (q\land r)$ \aligntoright{(Assoziativität)}{0.1}\\
        $(p\lor q) \lor r \equivalent p \lor (q\land r)$
        \item $\neg(p\land q) \equivalent \neg p \lor \neg q$ \aligntoright{(De Morgan)}{0.1}\\
        $\neg(p\lor q) \equivalent \neg p \land \neg q$
        \item $p\impl q \equivalent \neg p \lor q$ \aligntoright{(Definition der Implikation)}{0.1}
        \item $(p\equivalent q) \equivalent (p\impl q) \land (q\impl p)$ \aligntoright{(Definition der Äquivalenz)}{0.1}\\
        $(p\equivalent q) \equivalent (p\land q) \lor (\neg p \land \neg q)$
        \item $(p\equivalent q) \land (q\equivalent r) \impl (p \equivalent r)$ \aligntoright{(Transitivität)}{0.1}
    \end{enumerate}
\end{satz}

\newpage


    \section{Mengen}
    \thispagestyle{pagenumberonly}

\subsection{Eigenschaften von Mengen}

\begin{notation}[Konkrete Beschreibung von Mengen]
    Eine Menge ist informell formuliert eine Ansammlung von Objekten.
    Um diese Objekte konkret anzugeben, schreiben wir $M = \set{1, 2, 3}$.\\
    Eine andere Möglichkeit, eine Menge anzugeben ist über die Eigenschaften der Elemente.
    Wir schreiben: $M=\set{x|~H(x)}$. Das bedeutet, dass ein Element $x$ genau dann ein Element der Menge ist, wenn $H(x)$ gilt.
    Wir schreiben $x\in M \equivalent H(x)$.
\end{notation}

\begin{beispiel}[Mengendeklaration über Aussagenform]
    \begin{align*}
        H(x) &\definedas (x^2-3x+2 = 0)\\
        \impl \set{x|~H(x)} &= \set{1,2}
    \end{align*}
\end{beispiel}

\begin{definition}[Eigenschaften von Mengen]
    Allgemein gilt für Mengen:
    \theoremescape
    \begin{enumerate}
        \item 2 Mengen sind gleich, wenn sie die selben Elemente enthalten.
        \item Die leere Menge ($\emptyset$) ist die einzige Menge, die keine Elemente enthält.
        \item Wenn für jedes $x\in A$ auch $x\in B$ folgt, dann ist $A$ eine Teilmenge von $B$. ($A\subseteq B$)
        \item Ist $A\subseteq B$ und $A\neq B$, dann nennen wir $A$ eine echte Teilmenge von $B$. ($A\subsetneq B$)
        \item $A$ und $B$ sind disjunkt, falls aus $x\in A$ folgt, dass $x\not\in B$.
    \end{enumerate}
\end{definition}

\begin{bemerkung}
    Allgemein gilt für zwei Mengen $A$ und $B$, dass $A\subseteq B \land B\subseteq A \equivalent A = B$.\footnote{Diese Äquivalenz wird insbesondere in Beweisen häufig eingesetzt, indem durch das Zeigen, dass zwei Mengen gegenseitige Teilmenge sind, deren Gleichheit gezeigt wird.}
\end{bemerkung}

\subsection{Operationen mit Mengen}

\begin{definition}[Mengenoperationen]
    Seien $A, B$ Mengen. Dann gilt:
    \begin{align*}
        A\cap B &\definedas \set{x|~x\in A \land x\in B}\tag{Schnitt}\\
        A\cup B &\definedas \set{x|~x\in A \lor x\in B}\tag{Vereinigung}\\
        A\exclude B &\definedas \set{x|~x\in A \land x\notin B}\tag{Differenz}\\[10pt]
        \text{Wenn } A\subseteq M&\colon A^{C} = A^{C}_{M} \definedas M\exclude A\tag{Komplement}
    \end{align*}
    Allgemein gilt: $A$ und $B$ disjunkt $\equivalent A\cap B = \emptyset$
\end{definition}

\begin{visualisierung}[Darstellung von Mengen als Venn-Diagramm]
    Die Operationen auf zwei Mengen $A$ und $B$ lassen sich mittels eines Venn-Diagramms veranschaulichen:
    \begin{figure}[H]
        \centering
        \begin{tikzpicture}
            \node [draw,
            circle,
            minimum size =3cm,
            label={135:$A$}] (A) at (0,0){};

            \node [draw,
            circle,
            minimum size =3cm,
            label={45:$B$}] (B) at (1.8,0){};
            \node at (0.9,0) {$A\cap B$};
        \end{tikzpicture}
        \caption{Schnittmenge zweier Mengen als Venn-Diagramm}
    \end{figure}
\end{visualisierung}

%%%%%%%%%%%%%%%%%%%%%%%%
% 31. Oktober 2023
%%%%%%%%%%%%%%%%%%%%%%%%

\begin{lemma}[Kommutativität des Schnitts]
    \marginnote{[31. Okt]}
    $A\cap B = B \cap A$
    \begin{proof}
        \begin{align*}
            A\cap B &= \set{x|~x\in A \land x \in B}\\
            &= \set{x|~x\in B \land x \in A}\\
            &= B\cap A\qedhere
        \end{align*}
    \end{proof}
\end{lemma}

\begin{lemma}[Distributivität]
    $A\cup (B \cap C) = (A\cup B) \cap (A\cup C)$
    \begin{proof}
        \begin{align*}
            x\in A \cup \pair{B \cap C} &\equivalent x\in A \lor x\in B \cap C\\
            &\equivalent x\in A \lor \pair{x\in B \land x \in B}\\
            &\equivalent \pair{x\in A \lor x\in B}\land\pair{x\in A \lor x\in C}\\
            &\equivalent x\in A \cup B\land x \in A \cup C\\
            &\equivalent x\in \pair{A \cup B}\cap\pair{A\cup C}\qedhere
        \end{align*}
    \end{proof}
\end{lemma}

\begin{definition}[Familie von Mengen]
    Sei $J$ eine Indexmenge mit $J\neq \emptyset$. Die Mengenfamilie ist gegeben durch Mengen $A_{j}$ für jedes $j\in J$. Wir schreiben $\set{A_j}_{j\in J}$
\end{definition}

\begin{definition}[Schnitt und Vereinigung über mehrere Mengen]
    Für eine Mengenfamilie $\set{A_j}j\in J$ gilt:
    \begin{align*}
        \bigcap_{j\in J} A_j &\definedas \set{x|~\forall j\in J: x\in A_j}\\[10pt]
        \bigcup_{j\in J} A_j &\definedas \set{x|~\exists j\in J: x\in A_j}
    \end{align*}
\end{definition}

\subsection{Quantoren}

\begin{definition}[Quantoren]
    Wir definieren drei unterschiedliche Quantoren:
    \theoremescape
    \begin{enumerate}[label=(\roman*)]
        \item $\forall$: \anf{Für alle}
        \item $\exists$: \anf{Es existiert ein}
        \item $\exists!$: \anf{Es existiert genau ein}
    \end{enumerate}
    \vspace{0.2cm}
    Es seien $A,B$ Mengen und $H(x,y)$ eine Aussageform mit $x\in A$ und $y \in B$. Dann gilt:
    \begin{align*}
        \forall x \in A~\exists y \in B\colon H(x,y) \equivalent \text{Für alle $x\in A$ existiert ein $y \in B$, so dass $H(x,y)$ wahr ist}
    \end{align*}
\end{definition}

\begin{folgerung}[Negation von Quantoren]
    Es seien $A,B$ Mengen und $H(x,y)$ Aussageform mit $x\in A$ und $y \in B$. Dann gilt:
    \begin{align*}
        \neg\pair{\forall x \in A\colon H(x)} &\equivalent \exists x\in A\colon \neg H(x)\\
        \neg\pair{\exists x \in A\colon H(x)} &\equivalent \forall x\in A\colon \neg H(x)\\
        \neg \pair{\forall x \in A~\exists y\in B\colon H(x,y)}&\equivalent\exists x \in A\colon \neg\pair{\exists y\in B\colon H(x,y)}\\
        &\equivalent \exists x\in A~\forall y \in B\colon \neg H(x,y)\\
    \end{align*}
\end{folgerung}

\begin{definition}[Potenzmenge und Mengensystem]
    Sei $A$ eine Menge, so heißt $\mathcal{P}(A) \definedas \set{N|~N \subseteq A}$ Potenzmenge von $A$. Eine Teilmenge $A\subseteq \mathcal{P}(A)$ heißt Mengensystem über $A$.
\end{definition}

\begin{bemerkung}[Russels Paradoxon]
    Russel definiert: $R\definedas\set{M|~M \text{ ist Menge und $M \notin M$}}$\\
    Falls $R$ eine Menge, dann kann man fragen, ob $R\in R$ oder nicht.\\
    1. Fall: $R\notin R\impl R\in R$\qquad (Widerspruch)\\
    2. Fall: $R\in R\impl R \notin R$\qquad (Widerspruch)\\
    Lösung: $R$ ist keine Menge, sondern eine Klasse.
\end{bemerkung}

\vfill

\subsection{Kartesisches Produkt und Relationen}

\begin{definition}[Tupel]
    Seien $A$ und $B$ Mengen. Für $x\in A$ und $y\in B$ ist $(a,b) \definedas \set{a, \set{a,b}}$ das geordnete Paar oder Tupel bestehend aus $a$ und $b$.\\[10pt]
    Zwei Tupel $(a_1, b_1)$ und $(a_2, b_2)$ mit $a_1, a_2\in A, b_1, b_2\in B$ sind genau dann gleich, wenn ihr jeweils erstes und zweites Element gleich ist:
    \begin{align*}
    (a_1, b_1)
        = (a_2, b_2) \equivalent a_1 = a_2 \land b_1 = b_2
    \end{align*}
\end{definition}

\begin{definition}[Kartesisches Produkt]
    Somit ist $A \times B \definedas \set{\pair{a,b}|~a\in A \land b \in B}$ wieder eine Menge, genannt das Kartesische Produkt von $A$ und $B$.
\end{definition}

\begin{beispiel}
    \begin{align*}
        M &\definedas \set{1,2,3}\\
        N &\definedas \set{3,4}\\
        M\times N &= \set{(1,3), (1,4), (2,3), (2,4), (3,3), (3,4)}
    \end{align*}
\end{beispiel}


\begin{definition}[Relation]
    Seien $A, B$ Mengen. Eine Relation $R=\pair{A, B, G}$ besteht aus einer Menge $A$, einer Menge $B$ und einer Menge $G\subseteq A\times B$.\\
    $G$ ist der Graph von $R$, auch geschrieben als $G_R$.
    Ist $\pair{a,b}\in G$ so sagt man \anf{$a$ ist $R$-verwandt zu $b$}. Wir schreiben $aRb$ (Infix-Schreibweise).\\
    $A$ ist die Definitionsmenge von $R$ und $B$ ist die Zielmenge von $R$.\\[10pt]
    Seien $R_1 = \pair{A_1, B_1, G_1}$ und $R_2 = \pair{A_2, B_2, G_2}$ Relationen. Dann gilt:
    \begin{align*}
        R_1 = R_2 \equivalent A_1 = A_2 \land B_1 = B_2 \land G_1 = G_2
    \end{align*}
    Wir können eine Umkehrrelation $R^{-1}$ wie folgt definieren:
    \begin{align*}
        R^{-1} &\definedas \pair{B,A,G_{R^{-1}}}\\
        G_{R^{-1}} &\definedas \set{\pair{b,a} |~\pair{a,b} \in G_R}
    \end{align*}
\end{definition}

\begin{beispiel}[Kleiner-Relation]
    Es sei $A = \set{1,2,3,4}$. Wir definieren die Kleiner-Relation $R=\set{A, A, G_{<}}.$\\
    Dann gilt: $a_1 < a_2 \equivalent \pair{a_1, a_2} \in G_{<}$ und somit $G_{<} = \set{(1,2), (1,3), (1,4), (2,3), (2,4), (3,4)}$
\end{beispiel}

\vfill
\newpage

\begin{definition}[Äquivalenzrelation]
    Sei $R=\pair{A, A, G}$ eine Relation. Dann definieren wir unterschiedliche Eigenschaften, die die Relation haben kann:
    \begin{enumerate}[label=(\roman*)]
        \item $R$ ist reflexiv:\quad $\forall a\in A\colon aRa\quad\pair{\forall a \in A\colon (a,a) \in G}$
        \item $R$ ist symmetrisch:\quad $\forall a_1, a_2\in A\colon a_{1}Ra_2\equivalent a_{2}Ra_1$
        \item $R$ ist transitiv:\quad $\forall a_1, a_2, a_3 \in A\colon a_{1}Ra_{2} \land a_{2}Ra_{3} \impl a_{1}Ra_{3}$
    \end{enumerate}
    Eine Äquivalenzrelation ist eine reflexive, symmetrische und transitive Relation auf $A$.\\
    Ist $R$ eine Äquivalenzrelation und $a_{1}Ra_{2}$ so nennt man $a_{1}$ äquivalent zu $a_{2}$ bezüglich $R$.
\end{definition}

\begin{notation}[Äquivalenzklassen]
    Sei $R$ Äquivalenzrelation auf $A$. Dann gilt:
    \begin{align*}
    [a]
        _{R} \definedas \set{b\in A|~aRb}
    \end{align*} ist die Äquivalenzklasse von $a$. Wir schreiben auch $a\sim_R b$ für $aRb$ oder $a = b$ modulo $R$.
\end{notation}

\begin{beobachtung}
    Allgemein gilt für Äquivalenzklassen damit:
    \theoremescape
    \begin{enumerate}[label=(\roman*)]
        \item $\forall a\in A\colon [a]_R \neq \emptyset$
        \item $aRa \impl a\in [a]_R$
        \item $a_1, a_2\in [a]_R \impl a_1 \sim_R a, a_2 \sim_R a \annot{\impl}{sym.} a_1 \sim_R a, a \sim_R a_2 \annot{\impl}{trans.} a_1\sim_R a_2$
    \end{enumerate}
\end{beobachtung}

\begin{lemma}
    Sei $R$ Äquivalenzrelation auf $A$. Für $a_1,a_2\in A$ ist entweder $[a_1]_R = [a_2]_R$ oder $[a_1]_R \cap [a_2]_R = \emptyset$.
    \begin{proof}
        Da $[a_1]_R$, $[a_2]_R \neq \emptyset$ reicht es zu zeigen, dass: $[a_1]_R \cap [a_2]_R \neq \emptyset \impl [a_1]_R = [a_2]_R$.
        \begin{align*}
            \text{Sei }b\in [a_1]_R\cap [a_2]_R\text{ und }c\in [a_1]_R\\
            \impl b\sim_R a_1 \land c\sim_R a_1 \annot{\impl}{trans.} c\sim_R b\\
            \intertext{Da  $b\sim_R a_2$ muss nach der Transitivität gelten:}
            c\sim_R a_2 \impl [a_1]_R \subseteq [a_2]_R
        \end{align*}
        Symmetrisch lässt sich argumentieren, dass $[a_2]_R \subseteq [a_1]_R$.
    \end{proof}
\end{lemma}

\begin{korollar}
    Es sei $R$ eine Äquivalenzrelation auf $A\neq \emptyset$. Dann sind $a_1,a_2\in A$ entweder äquivalent oder sie gehören zu disjunkten Äquivalenzklassen.
\end{korollar}

\begin{definition}[Zerlegung einer Menge]
    Sei $A\neq \emptyset$ eine Menge. Dann ist eine Zerlegung $F = \set{A_j}_{j\in J}, A_j\subseteq A$ mit folgenden Eigenschaften definiert:
    \begin{enumerate}
        \item $\forall j\in J\colon A_j\neq \emptyset$
        \item Für $j_1, j_2\in J, j_1\neq j_2\colon A_{j1} \cap A_{j2} = \emptyset$
        \item $\bigcup_{j\in J}A_j = A$
    \end{enumerate}
\end{definition}

\begin{notation}[Quotient]
    Es sei $R$ Äquivalenzrelation auf $A$.
    \begin{align*}
        F\definedas\set{[a]_R|~a\in A}
    \end{align*}
    ist eine Zerlegung von $A$ (bezüglich der Äquivalenzrelation $R$). Wir schreiben $F = A/R$.
    $A/R$ ist der \anf{Quotient} von $A$ bezüglich $R$
\end{notation}

\begin{beispiel}[Restklassendefinition über Äquivalenzrelationen]
    Es sei $A = \naturalnumbers_{0} = \set{0,1,2,3,4,\dots}$ und $p\in \mathbb{N}$. $m,n\in\mathbb{N}_0$ seien genau dann äquivalent, wenn $m=n+k\cdot p$ für ein $k\in\mathbb{Z}$:
    \begin{align*}
        R_{p} = \set{\pair{m,n}\in \mathbb{N}_0 \times \mathbb{N}_0 |~\exists k\in \mathbb{Z}\text{ mit } m=n+k\cdot p}
    \end{align*}
    So definieren wir die Restklassen von $\mathbb{N}_0$ bezüglich Division mit $p$.
    \begin{align*}
        m\in [j]_{R} \equivalent m=n+k\cdot p\text{ für ein }k\in\mathbb{Z}
    \end{align*}
\end{beispiel}

%%%%%%%%%%%%%%%%%%%%%%%%
% 2. November 2023
%%%%%%%%%%%%%%%%%%%%%%%%

\subsection{Funktionen}

\begin{bemerkung}[Moralische Definition einer Funktion]
    \marginnote{[2. Nov]}
    Gegeben Mengen $A, B$, eine Relation $f$ von $A$ nach $B$. $f$ ist eine Funktion, wenn es jedem Element in $A$ genau ein Element in $B$ zuordnet.
\end{bemerkung}

\begin{notation}[Pfeilnotation]
    Wir schreiben $f: A \fromto B,~a\mapsto f\pair{a}$.
\end{notation}
\begin{folgerung}
    Zu $a\in A$ gibt es $f\pair{a}\in B \leadsto$ Tupel($a, f\pair{a}$) $\in A\times B \impl \set{\pair{a, f(a)}: a\in A} \subseteq A\times B$.
\end{folgerung}

\begin{definition}[Funktion]
    Eine Relation $R = (A, B, G_R)$ heißt Funktion (oder Abbildung), wenn
    \begin{align*}
        \forall a\in A~\exists! b\in B\colon (a,b)\in G_f.
    \end{align*}
    Wir setzen dann $f(a)\definedas b$.
\end{definition}

\begin{beispiel}[Mögliche Funktionen]
    \theoremescape
    \begin{alignat*}{2}
        f\colon &\mathbb{Z} \rightarrow \realnumbers,\quad && n\mapsto 3n^2+7\\
        g\colon &\mathbb{Z} \rightarrow \mathbb{Z},\quad && n\mapsto 3n^2+7\\
        h\colon &\linterv{0, \infinity} \rightarrow \realnumbers,\quad && x\mapsto x^2+3x+4\\
        j\colon &\realnumbers \rightarrow \realnumbers,\quad && x\mapsto x^2+3x+4\\
    \end{alignat*}
\end{beispiel}
\begin{bemerkung}
    $f$ und $g$ haben zwar die gleiche Funktionsvorschrift, sind aber dennoch unterschiedliche Funktionen, da diese wie Relationen auch über Definitionsmenge und Zielmenge definiert sind.
\end{bemerkung}

\begin{notation}[Bild und Urbild]
    Sei $f: A\rightarrow B$ eine Funktion.
    Dann gilt
    \begin{alignat*}{2}
        & M \subseteq A:~ & f\pair{M} &\definedas \set{b\in B|~\exists x\in M\colon b=f\pair{x}}\tag{Bild von $M$ unter $f$}\\
        & N \subseteq B:~ & f^{-1}\pair{N} &\definedas \set{a\in A|~f(a)\in N}\tag{Urbild von $N$ unter $f$}
    \end{alignat*}
    Außerdem ist das gesamte Bild von $f$ definiert als
    \begin{align*}
        Bild(f) \definedas f(A)
    \end{align*}
\end{notation}
\begin{definition}[Einschränkungen von Funktionen]
    Sei $f\colon~A \rightarrow B$ eine Funktion und $M \subseteq A$.
    Dann ist die Einschränkung (oder Restriktion) von $f$ auf $M$ definiert als
    \begin{align*}
        f|_{M}\colon&~M \rightarrow B,\quad x \mapsto f\pair{x}
    \end{align*}
\end{definition}
\begin{definition}[Besondere Eigenschaften von Funktionen]
    Es sei $f\colon A\fromto B$ eine Funktion.
    \theoremescape
    \begin{enumerate}[label=(\roman*)]
        \item $f$ ist injektiv, falls $\forall a_1, a_2 \in A\colon f\pair{a_1} = f\pair{a_2} \impl a_1 = a_2$.
        \item $f$ ist surjektiv, falls $Bild\pair{f} = B$.
        \item $f$ ist bijektiv, falls es injektiv und surjektiv ist. In diesem Fall existiert eine Inverse\\ $f^{-1}: B\fromto A, b\mapsto a$ mit $f(a)=b$.
    \end{enumerate}
\end{definition}

\subsection{Geordnete Mengen}

\begin{definition}[Ordnungsrelation und teilweise geordnete Menge]
    Es sei $A$ eine Menge und $R$ eine Relation auf $A$. $R$ heißt Ordnungsrelation (geschrieben \anf{$\prec$}), falls
    \begin{enumerate}[label=(\roman*)]
        \item $\forall a\in A\colon a \prec a$ (Reflexivität)
        \item $\forall a_1, a_2, a_3\in A\colon a_1\prec a_2 \land a_2 \prec a_3 \impl a_1 \prec a_3$ (Transitivität)
        \item $\forall a_1, a_2\in A\colon a_1\prec a_2 \land a_2 \prec a_1 \impl a_1 = a_2$ (Antisymmetrie)
    \end{enumerate}
    $\pair{A, \prec}$ heißt \textbf{teilweise} geordnete Menge. Nicht alle Paare $a_1$, $a_2$ müssen vergleichbar sein.
\end{definition}

\begin{notation}
    Wir schreiben $a_1\prec a_2$ als $a_{1}Ra_{2}$ oder $\pair{a_1, a_2}\in G_R$.
\end{notation}
\begin{definition}[Kette]
    $T\subseteq A$ heißt Kette (oder geordnete Menge), falls
    \begin{align*}
        a_1, a_2\in T \impl a_1\prec a_2 \lor a_2 \prec a_1
    \end{align*}
\end{definition}

\begin{beispiel}[Ordnungsrelation]
    Ordnungsrelation auf $\mathcal{P}(A)$:
    \begin{align*}
        M, N \subseteq A \quad M \prec N, \text{ falls } M\subseteq N
    \end{align*}
    \noindent Ordnungsrelation auf $\naturalnumbers$:
    \begin{align*}
    (\naturalnumbers, \prec)
        : n\prec m, \text{ falls $n$ teilt $m$}
    \end{align*}
    $2$ und $3$ sind dabei nicht vergleichbar, aber $2$ und $4$ sind vergleichbar.
\end{beispiel}

\newpage


    \section{Die Axiome der Reellen Zahlen}
    \thispagestyle{pagenumberonly}

Es gibt eine Menge $\realnumbers$, genannt Reelle Zahlen, die 3 Gruppen von Axiomen erfüllt:
\begin{enumerate}
    \item Algebraische Axiome
    \item Anordnungsaxiome
    \item Das Vollständigkeitsaxiom
\end{enumerate}

\subsection{Algebraische Axiome}
In $\realnumbers$ gibt es 2 Operationen:
\begin{enumerate}
    \item Addition \anf{+}
    \item Multiplikation \anf{$\cdot$}
\end{enumerate}

\begin{folgerung}
    $a,b\in\realnumbers\impl a+b\in \realnumbers$ und $a \cdot b\in\realnumbers$
\end{folgerung}

\begin{definition}[Eigenschaften eines Körpers]
    \theoremescape
    \begin{enumerate}[label=(I.\arabic*)]
        \item $(a+b) + c = a + (b+c)$\quad \textit{Assoziativität der Addition}
        \item $a + b = b + a$\quad \textit{Kommutativität der Addition}
        \item Es gibt genau eine Zahl genannt Null, geschrieben $0$, mit $\forall a \in\realnumbers: a+0 = a$ \quad \textit{Existenz eines neutralen Elements der Addition}
        \item $\forall a\in\realnumbers~\exists! b\in \realnumbers: a + b = 0$, geschrieben $b=-a$\quad \textit{Existenz eines inversen Elements der Addition}
        \item $\pair{a\cdot b}\cdot c = a\cdot\pair{b\cdot c}$\quad \textit{Assoziativität der Multiplikation}
        \item $a\cdot b = b \cdot a$\quad \textit{Kommutativität der Multiplikation}
        \item $\forall a \in\realnumbers, a \neq 0$ gibt es ein eindeutiges $b\neq 0$ mit $a\cdot b = 1$. Wir schreiben $b = a^{-1} = \frac{1}{a}$\quad \textit{Existenz eines inversen Elements der Multiplikation}
        \item Es gibt genau eine Zahl Eins, geschrieben 1, die von 0 verschieden ist, mit $\forall a\in\realnumbers: a\cdot 1 = a$\quad \textit{Existenz eines neutralen Elements der Multiplikation}
        \item $a\cdot\pair{b+c} = a \cdot b + a \cdot c$\quad \textit{Distributivität}
    \end{enumerate}

    \noindent Jede Menge $\mathbb{K}$, welche (I.1) bis (I.9) erfüllt, heißt \textbf{Körper}.
\end{definition}

\begin{bemerkung}
    Dass die Eindeutigkeit von 0 und 1 durch die Axiome gefordert wird, ist nicht unbedingt erforderlich.\footnote{Seien $0, 0'$ neutrale Elemente bezüglich der Addition. $\impl 0 = 0+0' = 0'+0=0'$}
\end{bemerkung}

\begin{bemerkung}
    Das inverse Elemente bezüglich Addition und Multiplikation ist eindeutig.
    \begin{proof}
        Annahme: $a+b=0$ und $a+b'=0$
        \begin{align*}
            &\impl b + 0 = b + \pair{a+b'} = b' + \pair{a+b} = b' + 0\\
            &\impl b = b'\qedhere
        \end{align*}
    \end{proof}
\end{bemerkung}

\begin{notation}
    \begin{align*}
        a-b &\definedas a+(-b)\tag{Differenz}\\[10pt]
        \frac{a}{b} &\definedas a \cdot b^{-1}\tag{Quotient}
    \end{align*}
\end{notation}

\begin{satz}[Abgeleitete Regeln]
    Es gilt\\
    (I.10)
    \begin{align}
        -\pair{-a} &= a\\[8pt]
        \pair{-a} + \pair{-b} &= -\pair{a+b}\\[8pt]
        \pair{a^{-1}}^{-1} &= a\\[8pt]
        a^{-1}\cdot b^{-1} &= \pair{a\cdot b}^{-1}\\[8pt]
        a\cdot 0 &= 0\\[8pt]
        a\cdot\pair{-b} &= -\pair{a\cdot b}\\[8pt]
        (-a)\cdot(-b) &= a\cdot b\\[8pt]
        a\cdot\pair{b-c} &= a\cdot b - a\cdot c
    \end{align}
    \noindent (I.11) Ist $a\cdot b = 0$ so ist mindestens eine der Zahlen $a$ oder $b$ gleich Null.

    \begin{proof}[Beweis zu (I.10.5)]
        Zu zeigen: $a \cdot 0 = 0$
        \begin{align*}
            a\cdot 0 + a \cdot 0 &= a \cdot\pair{0+0}\\
            &= a \cdot 0\\
            \impl \pair{a\cdot 0 + a \cdot 0} + \pair{-a \cdot 0} &= a \cdot 0 + \pair{-a\cdot 0}\\
            \impl a \cdot 0 + \pair{a\cdot 0 + \pair{-a\cdot 0}} &= 0\\
            \impl a\cdot 0 +0 &= 0\\
            \impl a\cdot 0 &= 0\qedhere
        \end{align*}
    \end{proof}
    \begin{proof}[Beweis zu (I.11)]
        Sei $a\cdot b = 0$.\\
        Ist $a\neq 0 \impl b = 1\cdot b = a^{-1}\cdot a \cdot b = a^{-1}\cdot (a\cdot b) = a^{-1} = 0 = 0$\\
        Ist $b\neq 0,$ so gilt analog, dass $a = 0$.
    \end{proof}
    \begin{uebung}
        Beweisen Sie die verbleibenden Regeln aus (I.10).
    \end{uebung}
\end{satz}

\begin{satz}[Regeln des Bruchrechnens]
    \label{satz:bruchrechnen}
    ~\\(I.12) Es gilt:
    \setcounter{equation}{0}
    \begin{align}
        \frac{a}{b} + \frac{c}{d} &= \frac{ad+cb}{bd} &b,d&\neq 0\\[10pt]
        \frac{a}{b} \cdot \frac{c}{d} &= \frac{ac}{bd} &b,d&\neq 0\\[10pt]
        \frac{\frac{a}{b}}{\frac{c}{d}} &= \frac{ad}{bc} &b,c,d&\neq 0
    \end{align}
    \begin{uebung}
        Beweisen Sie Satz~\ref{satz:bruchrechnen}.
    \end{uebung}
\end{satz}

\subsection{Die Anordnungsaxiome}

Allgemein gilt: $a,b\in\realnumbers \impl a = b \lor a \neq b$.\\
Ist $a\neq b$, besteht eine Anordnung \anf{$<$}, die verlangt, dass genau eine der Relationen $a<b$ oder $b<a$ gilt.
Das heißt $\forall a,b\in\realnumbers$ gilt genau eine der Aussagen $a<b, b<a, a= b$\\
Diese Anordnung genügt folgenden Axiomen:
\begin{axiom}[Anordnungsaxiome]
    \theoremescape
    \begin{enumerate}[label=(II.\arabic*)]
        \item $a<b \land b < c \impl a < c$ (Transitivität)
        \item $a<b, c \in\realnumbers \impl a + c < b + c$
        \item $a<b, c > 0 \impl ac < bc$
    \end{enumerate}
\end{axiom}

\begin{notation}
    \theoremescape
    \begin{enumerate}[label=-]
        \item $a < b$: a ist (echt) kleiner als b
        \item $b > a$: b ist größer als a
        \item $a\leq b$: $a=b$ oder $a < b$
        \item $a\in\realnumbers$ ist positiv, wenn $a>0$; negativ, wenn $a <0$; nicht-negativ, wenn $a\geq 0$; nicht-positiv, wenn $a\leq 0$
    \end{enumerate}
\end{notation}

\begin{beispiel}
    $a<b\equivalent b - a > 0$
    \begin{proof}
        \begin{align*}
            a &<b\\
            \impl 0 = a + \pair{-a} &< b + \pair{-a} = b - a\\[10pt]
            b-a &>0\\
            \impl a &< a + \pair{b-a} = b\qedhere
        \end{align*}
    \end{proof}
\end{beispiel}

%%%%%%%%%%%%%%%%%%%%%%%%
% 7. November 2023
%%%%%%%%%%%%%%%%%%%%%%%%

\begin{satz}[Aus den Anordnungsaxiomen abgeleitete Regeln]
    \marginnote{[7. Nov]}
    \theoremescape
    \begin{enumerate}[label=(II.\arabic*)]
        \setcounter{enumi}{3}
        \item $a<b\equivalent b-a > 0$
        \item $a<0\equivalent -a > 0$ und $a>0\equivalent -a < 0$
        \item $a<b\equivalent -b < -a$
        \item $a<b \land c < d \equivalent a+c<b+d$
        \item $ab > 0 \equivalent \pair{a>0 \land b > 0}\lor\pair{a < 0 \land b < 0}$ und $ab < 0 \equivalent \pair{a>0 \land b < 0}\lor\pair{a < 0 \land b > 0}$
        \item $a\neq 0 \impl a^2 > 0$\quad(Insbesondere $1>0$)
        \item $a<b \land c<0 \impl ac > bc$
        \item $a>0 \equivalent \frac{1}{a}>0$
        \item $a^2 < b^2 \land a > 0 \land b > 0 \impl a < b$
    \end{enumerate}
    \newpage
    \begin{proof}
        \theoremescape
        \begin{enumerate}[label=(II.\arabic*)]
            \setcounter{enumi}{3}
            \item Sei $a<b \impl 0=a+(-a) \annot{<}{(II.2)} b + (-a) = b-a$.\\
            Ist $b-a>0 \annot{\impl}{(II.2)} a<a+(b-a)=b$
            \item Setze $b\definedas 0$ in (II.4) $\impl b-a=-a>0$.\\
            2ter Teil: Ersetze $a$ durch $-a$ in (II.5). ($a>0 \impl -a < 0 \equivalent -(-a)>0 \equivalent a >0$)
            \item (II.6) folgt aus (II.5), da $a<b\equivalent b-a>0 \equivalent (-a)-(-b) > 0\equivalent -b < -a$
            \item Sei $a<b \land c < d \annot{\impl}{(II.2)} a + c < b + c \land b + c < b + d \annot{\impl}{(II.1)} a+c < b + d$
            \item $a,b>0 \annot{\impl}{(II.3)} ab > 0\cdot b = 0$ und $a,b<0 \annot{\impl}{(II.5)} -a,-b>0 \impl (-a)(-b) > 0 \impl ab > 0$.\\
            Umkehrung: Sei $ab>0 \impl a\neq 0 \land b \neq 0$. Wäre $a>0 \land b < 0 \annot{\impl}{(II.5)} -b>0$. Wie gerade gezeigt folgt $a(-b) > 0 \impl -ab > 0 \annot{\impl}{(II.5)} ab < 0$ (Widerspruch zur Annahme).\\
            Genauso zeigt man, dass die Annahme $a<0 \land b > 0$ falsch ist.\\
            (Zweite Behauptung lässt sich analog zeigen).
            \item $a\neq 0 \equivalent a > 0 \lor a < 0 \annot{\impl}{(II.8)} a^2 = a \cdot a > 0$. Ferner ist $1\neq 0 \impl 1=1\cdot 1 > 0$
            \item Sei $c<0 \impl -c > 0$ und aus $a<b$ folgt $(-c)\cdot a < (-c)\cdot b \impl -c\cdot a < -c\cdot b \impl c\cdot b < c\cdot a$
            \item $a\cdot a^{-1} = 1 > 0$ (falls $a\neq 0$) $\annot{\impl}{(II.8)} a^{-1} > 0$ sofern $a>0$ ist und aus $a^{-1}>0$ folgt $a>0$
            \item Sei $a^{2}<b^2, a>0, b>0$. Angenommen $a<b$ ist falsch, d.h. $a\geq b \impl a^{2} \geq a \cdot a \geq a\cdot b\geq b\cdot b = b^{2}\impl a^{2}\geq b^{2}$ (Widerspruch)
        \end{enumerate}
    \end{proof}
\end{satz}

\newpage

\subsection{Das Vollständigkeitsaxiom}

\begin{axiom}[Vollständigkeitsaxiom]
    Jede nicht-leere Teilmenge $M\subseteq \realnumbers$, welche nach oben beschränkt ist, besitzt eine kleinste obere Schranke, genannt das Supremum von $M$.
\end{axiom}

\begin{notation}[Supremum]
    Das Supremum einer Menge $M$ schreiben wir als $\sup M$
\end{notation}

\begin{definition}[Beschränktheit von Mengen]
    Sei $M\subseteq \realnumbers, M\neq \emptyset$.
    \begin{enumerate}[label=(\roman*)]
        \item $M$ heißt \textbf{nach oben beschränkt}, falls ein $k\in\realnumbers$ existiert mit $\forall x\in M:x\leq k$.
        Jede solche Zahl $k$ heißt obere Schranke von $M$.
        \item $M$ heißt \textbf{nach unten beschränkt}, falls ein $k\in\realnumbers$ existiert mit $\forall x\in M:x\geq k$.
        Jede solche Zahl $k$ heißt untere Schranke von $M$.
        \item $M$ heißt \textbf{beschränkt}, falls ein $k\geq 0$ existiert mit $-k\leq x \leq k\quad \forall x\in M$
    \end{enumerate}
\end{definition}

\begin{definition}[Kleinste obere und größte untere Schranke]
    Eine Zahl $k\in\realnumbers$ heißt kleinste obere (größte untere) Schranke, falls
    \begin{enumerate}
        \item es eine obere (untere) Schranke ist und
        \item es keine kleinere obere (größere untere) Schranke für $M$ gibt
    \end{enumerate}
\end{definition}

\begin{folgerung}
    \theoremescape
    Allgemein gilt
    \begin{align*}
        x \leq k \equivalent -k \leq -x
    \end{align*}
    das heißt für eine Menge $M\neq\emptyset$ gilt
    \begin{center}
        $k$ ist eine obere Schranke für $M$\\ $\equivalent -k$ ist eine untere Schranke für $-M\definedas\set{-x|~x\in M}$
    \end{center}
    und
    \begin{center}
        $k$ ist kleinste obere Schranke für M\\ $\equivalent -k$ ist die größte untere Schranke für $-M$
    \end{center}
    Das heißt das Anordnungsaxiom ist äquivalent zum \textit{Anordnungsaxiom}$^{-1}$ (Jede nicht-leere Teilmenge $M\subseteq \realnumbers$, welche nach unten beschränkt ist, besitzt eine größte untere Schranke, genannt das Infimum von $M$. Wir schreiben $\inf M$).
\end{folgerung}

\begin{beispiel}
    \theoremescape
    \begin{align*}
        M &\definedas \interv{0,1} = \set{x|~0\leq x \leq 1}\\
        \sup M &= 1 \qquad \inf M = 0\nn
        A &\definedas \pair{0,1} = \set{x|~0< x < 1}\\
        \sup A &= 1 \qquad \inf A = 0\\
    \end{align*}
\end{beispiel}

\begin{notation}
    Sei $M\subseteq \realnumbers, M\neq \emptyset$\\
    Wir schreiben $\sup M < \infty$, falls $M$ nach oben beschränkt ist, andernfalls setzen wir
    \begin{align*}
        \sup M \definedas \infty
    \end{align*}
    Falls $M$ nach unten beschränkt ist, schreiben wir $\inf M > -\infty$, andernfalls setzen wir
    \begin{align*}
        \inf M \definedas -\infty
    \end{align*}
\end{notation}

\begin{satz}[Eigenschaften des Supremums]
    \label{satz:sup}
    Sei $M\subseteq \realnumbers$
    \begin{enumerate}[label=(\roman*)]
        \item Ist $\sup M < \infty$, so folgt $\forall \varepsilon > 0~\exists x\in M$ mit $\sup\pair{M} -\varepsilon < x$
        \item Ist $\sup M = \infty$, so gilt $\forall k\geq 0~\exists x\in M$ mit $x> k$
    \end{enumerate}
    \begin{proof}
        \theoremescape
        \begin{enumerate}
            \item Wir setzen $a\definedas \sup M$. Sei $a<\infty$. Wäre (i) falsch, so folgt $\exists \varepsilon>0~\forall x\in M:a-\varepsilon > x$.\\Das heißt $a-\varepsilon$ ist eine obere Schranke für $M$. Aber $a-\varepsilon < a$; (Widerspruch)
            \item Ist $a=\infty$, so hat $M$ keine obere Schranke. Nach Def. folgt für jedes $k\in\realnumbers$ existiert ein $x\in M: x > k$\qedhere
        \end{enumerate}
    \end{proof}
\end{satz}
\begin{satz}[Eigenschaften des Infimums]
    \label{satz:inf}
    Sei $M\subseteq \realnumbers$
    \begin{enumerate}[label=(\roman*)]
        \item Ist $\inf M > -\infty$, so folgt $\forall \varepsilon > 0~\exists x\in M$ mit $x < \inf\pair{M}+ \varepsilon$
        \item Ist $\inf M = -\infty$, so gilt $\forall k\geq 0~\exists x\in M$ mit $x< -k$
    \end{enumerate}
    \begin{proof}
        Wende Satz~\ref{satz:sup} auf $-M\definedas\set{-x|~x\in M}$ an und beachte $\sup\pair{-M} = \inf\pair{M}$
    \end{proof}
\end{satz}
\begin{definition}[Maximum und Minimum]
    Es sei $M\subseteq \realnumbers, M\neq\emptyset$\\
    $m\in M$ heißt größtes Element von $M$ (Maximum), geschrieben $\max M$, falls $x\leq m~\forall x\in M$.\\
    Entsprechend: $m\in M$ heißt kleinstes Element von $M$ (Minimum), geschrieben $\min M$, falls $x\geq m~\forall x\in M$
\end{definition}

\begin{beispiel}
    Sei $M$ beschränkt, $M\neq\emptyset$
    \begin{align*}
        M&\definedas\set{x|~0\leq x< 1}\\
        \sup M &= 1\\
        \inf M &= \min M = 0\\
        M&\text{ hat kein Maximum}
    \end{align*}
\end{beispiel}

\newpage


    \section{Die natürlichen Zahlen $\naturalnumbers$ und vollständige Induktion}
    \thispagestyle{pagenumberonly}
Frage: Was sind die natürlichen Zahlen? (zum Beispiel 1, $2\definedas 1+1$, 3$\definedas 2+1$, 4$\definedas 3+1$, usw.)

\subsection{Induktive Mengen}
\begin{definition}[Induktive Menge]
    Eine Teilmenge $A\subseteq \realnumbers$ heißt induktiv, falls
    \begin{enumerate}
        \item $1\in A$ und
        \item Ist $x\in A$ so ist auch $x+1\in A$
    \end{enumerate}
\end{definition}

\begin{beispiel}
    \begin{align*}
        \rinterv{0, \infty} &= \set{x|~x\geq 0}\text{ und}\\
        \rinterv{1, \infty} &= \set{x|~x\geq 1} \text{ sind induktiv}
    \end{align*}
\end{beispiel}

\begin{beobachtung}[Schnittmengen von induktiven Mengen]
    Ist $J\neq\emptyset$ Indexmenge und $A_j$ induktive Teilmenge von $\realnumbers$ für jedes $j\in J$.
    Dann folgt daraus: $A\definedas \bigcap_{j\in J}A_j$ ist induktiv.\\
    Anders formuliert: Beliebige Schnittmengen von induktiven Mengen sind induktiv.
    \begin{proof}
        \begin{align*}
            x\in A &\equivalent \forall j\in J: x \in A_j\\
            &\impl x+1 \in A_j \quad \forall j\in J\\
            &\impl x+1 \in \pair{\bigcap_{j\in J}A_j}=A\qedhere
        \end{align*}
    \end{proof}
\end{beobachtung}

\begin{definition}[Definition von $\naturalnumbers$]
    Sei $\bar{f}\definedas\set{A\subseteq \realnumbers|~A\text{ ist induktiv}}$.
    \begin{align*}
        \naturalnumbers &\definedas \text{kleinste induktive Teilmenge von }\realnumbers\\
        &\definedas \bigcap_{A\in \bar{f}}A
    \end{align*}
    d. h. $\naturalnumbers\subseteq A$, falls $A$ induktiv ist.
\end{definition}

\begin{satz}[Induktionsprinzip]
    \label{satz:induktionsprinzip}
    Ist $M\subseteq \naturalnumbers$ induktiv $\impl M = \naturalnumbers$
    \begin{proof}
        Nach Voraussetzung ist $M\subseteq \naturalnumbers$ und aus der Definition von $\naturalnumbers$ als Schnitt aller induktiven Teilmengen von $\realnumbers$ ist auch $\naturalnumbers\subseteq M \impl \naturalnumbers = M$.
    \end{proof}
\end{satz}

%%%%%%%%%%%%%%%%%%%%%%%%
% 9. November 2023
%%%%%%%%%%%%%%%%%%%%%%%%

\subsection{Vollständige Induktion}

\begin{satz}[Induktionsbeweis]
    Für jedes $n\in\naturalnumbers$ sei eine Aussage $B_n$ gegeben derart, dass folgendes gilt:
    \begin{enumerate}
        \item $B_1$ ist wahr.
        \item Aus der Annahme, dass $B_n$ ($n\in\naturalnumbers$) wahr ist folgt, dass $B_{n+1}$ wahr ist.
    \end{enumerate}
    Dann ist $B_n$ wahr $\forall n\in\naturalnumbers$.
    \begin{proof}
        Definiere: $M\definedas\set{n\in\naturalnumbers|~B_n\text{ ist wahr}} \subseteq \naturalnumbers$.\\
        Zu zeigen: $M=\naturalnumbers$. Also reicht nach Satz~\ref{satz:induktionsprinzip} zu zeigen, dass $M$ induktiv ist.
        \begin{enumerate}
            \item $1\in\naturalnumbers$, da $B_1$ wahr ist.
            \item Ist $n\in M$ dann ist $B_n$ wahr $\impl$ $B_n+1$ ist wahr $\impl$ $n+1\in M$ $\impl$ $M$ ist induktiv $\impl$ $M=\naturalnumbers$ \qedhere
        \end{enumerate}
    \end{proof}
\end{satz}

\begin{bemerkung}[Starke Induktion]
    Die starke Induktion ist eine Variante der vollständigen Induktion. Sie ist definiert durch:
    \begin{enumerate}
        \item $B_1$ ist wahr.
        \item für alle $n\in\naturalnumbers$ gilt $B_1, B_2, B_3, \dots, B_n$ ist wahr $\impl$ $B_{n+1}$ ist wahr.
    \end{enumerate}
\end{bemerkung}

\begin{beispiel}
    Zu zeigen ist: $B_n: n < 2^{n}$
    \begin{proof}
        ~\\
        \begin{induktionsanfang}
            $B_1$ ist wahr, da $1<2^{1}=2$
        \end{induktionsanfang}
        \begin{induktionsschritt}
            Induktionsannahme $B_n$ ist wahr für ein $n=k$, d.h. $k<2^k$. Und Schluß zu zeigen: $k+1<2^{k+1} \impl 2^{k+1} = 2\cdot 2^{k} > 2k \geq k+1$\qedhere
        \end{induktionsschritt}
    \end{proof}
\end{beispiel}

\begin{uebung}
    Zeigen Sie, dass $2k\geq k+1$ für alle $k\in\naturalnumbers$ gilt.
\end{uebung}

\begin{beispiel}[Gaußsche Summenformel]
    Zu zeigen: $1+2+3+\dots+n= \frac{n\cdot(n+1)}{2}~\forall n\in\naturalnumbers$
    \begin{proof}
        $B_n: 1+2+\dots+n = \frac{n\cdot(n+1)}{2}$\\
        \begin{induktionsanfang}
            $B_1: 1=\frac{1\cdot(1+1)}{2}=\frac{2}{2}=1$
        \end{induktionsanfang}
        \begin{induktionsschritt}
            $B_n$ ist wahr für ein $k\in\naturalnumbers$, d.h. $1+2+\dots+k=\frac{k\cdot(k+1)}{2}$
            \begin{align*}
                1+2+\dots+k+(k+1) &= \frac{k\cdot(k+1)}{2}+(k+1)\\
                &= (k+1)\cdot\pair{\frac{k}{2}+1}\\
                &= \frac{(k+1)\cdot(k+2)}{2}
            \end{align*}
        \end{induktionsschritt}
        \noindent Also ist $B_n$ wahr für $n=k+1$. Nach dem Prinzip der vollständigen Induktion folgt $\forall n\in\naturalnumbers:~B_n$ ist wahr, d.h. $1+2+\dots+n=\frac{n\cdot(n+1)}{2}$\qedhere
    \end{proof}
\end{beispiel}

\begin{satz}[Eigenschaften von $\naturalnumbers$]
    \label{satz:n-eigenschaften}
    \theoremescape
    \begin{enumerate}
        \item $n\geq 1~\forall n\in\naturalnumbers$
        \item $n+m\in\naturalnumbers~\forall n,m\in\naturalnumbers$
        \item $n\cdot m\in\naturalnumbers~\forall n,m\in\naturalnumbers$
        \item Für $n\in\naturalnumbers$ gilt entweder $n=1$ oder $n-1\in\naturalnumbers$
        \item $(n-m) \in \naturalnumbers~\forall n,m\in\naturalnumbers$ mit $m<n$
    \end{enumerate}
    \begin{proof}[Beweis (1.)]
        ~\\
        \begin{induktionsanfang}
            $n\geq 1$ gilt für $n=1$
        \end{induktionsanfang}
        \begin{induktionsschritt}
            Nach Anfang: $n\geq 1$ ist wahr für ein $n=k$.\\
            Da $k+1>k\impl k+1>k\geq 1 \impl k+1\geq 1$\qedhere
        \end{induktionsschritt}
    \end{proof}
    \begin{proof}[Beweis (2.)]
        Fixiere $m\in\naturalnumbers$ für $n\in\naturalnumbers$. Behauptung: $B_n: m+n\in \naturalnumbers$\\
        \begin{induktionsanfang}
            $B_1: m+1 \in \naturalnumbers$, da $\naturalnumbers$ induktiv ist.
        \end{induktionsanfang}
        \begin{induktionsschritt}
            Nach Anfang: $B_n$ ist wahr für $n=k$, d.h. $(m+k)\in\naturalnumbers$. Somit für $k+1: m+(k+1) = (m+k)+1\in\naturalnumbers \impl$ $B_n$ ist wahr $\forall n\in\naturalnumbers$\qedhere
        \end{induktionsschritt}
    \end{proof}
    \begin{proof}[Beweis (3.)]
        Fixiere $m\in\naturalnumbers$ für $n\in\naturalnumbers$. Behauptung: $B_n: m\cdot n\in \naturalnumbers$\\
        \begin{induktionsanfang}
            $B_1: m\cdot 1=m \in \naturalnumbers$
        \end{induktionsanfang}
        \begin{induktionsschritt}
            Nach Anfang: $B_n$ ist wahr für $n=k$, d.h. $m\cdot(k+1) = mk + m \in\naturalnumbers$ gilt nach Satz 2.\qedhere
        \end{induktionsschritt}
    \end{proof}
    \begin{proof}[Beweis (4.)]
        $B_n: n= 1 \lor n-1\in\naturalnumbers$\\
        \begin{induktionsanfang}
            $B_1: 1=1$
        \end{induktionsanfang}
        \begin{induktionsschritt}
            Nehmen an, für ein $n=k$ ist $G_k$ wahr. Also ist entweder (a) $k=1$ oder (b) $(k-1)\in\naturalnumbers$.\\
            Für $n=k+1$ gilt dann im Fall (a): $(k+1)-1 = k\in\naturalnumbers$\\
            Im Fall (b): $(k+1)-1 = (k-1)+1$ und es gilt $(k-1)\in\naturalnumbers$. Daraus folgt, dass $(k-1)+1\in\naturalnumbers$, da $\naturalnumbers$ induktiv ist.\qedhere
        \end{induktionsschritt}
    \end{proof}
    \begin{proof}[Beweis (5.)]
        $B_n: n-m\in\naturalnumbers$ für jedes $m,n\in\naturalnumbers$ mit $m<n$\\
        \begin{induktionsanfang}
            $B_1$ leere Behauptung, da kein $m$ existiert, mit $m<1$ (nach (1.)).
        \end{induktionsanfang}
        \begin{induktionsschritt}
            $B_n$ wahr für ein $n=k$. Das heißt $k-1\in\naturalnumbers~\forall m\in\naturalnumbers$ mit $m<k$.\\
            Zu zeigen: $(k+1)-m\in\naturalnumbers~\forall m<k+1$. Ist $m=1\impl m-1 = 0$ und $(k+1)-m = k+1-m=k\in\naturalnumbers$.\\
            Ist $m>1 \annot{\impl}{(4.)} m-1\in\naturalnumbers$. Da $m<k+1 \impl m-1<k \impl (k+1)-m = k-(m-1)\in\naturalnumbers$ (nach Annahme).\qedhere
        \end{induktionsschritt}
    \end{proof}
\end{satz}

\begin{korollar}
    \label{korollar:4.2.7}
    Es gibt kein $n\in\naturalnumbers$ mit $0<n<1$. Ferner gilt $\forall n\in\naturalnumbers$ gibt es keine natürliche Zahl $m\in\naturalnumbers$ mit $n<m<n+1$ oder mit $n-1<m<n$.
\end{korollar}
\begin{uebung}
    Beweisen Sie dies.
\end{uebung}
\begin{notation}[Zahlenmengen]
    \begin{align*}
        \naturalnumbers_0 &\definedas \naturalnumbers\cup\set{0}\\
        \mathbb{-N} &\definedas \set{-n|~n\in\naturalnumbers}\\
        \mathbb{Z} &\definedas \naturalnumbers\cup\set{0}\cup\mathbb{-N} \tag{Menge der ganzen Zahlen}\\
        \mathbb{Q} &\definedas\set{\frac{p}{q}\middle|~p\in\mathbb{Z}, q\in\naturalnumbers} \tag{Menge der rationalen Zahlen}\\
        \realnumbers&\exclude\mathbb{Q}\tag{Irrationale Zahlen}
    \end{align*}
\end{notation}
\begin{bemerkung}
    Sei für $n\in\mathbb{Z}~B_n$ eine Aussage und $n_0\in\naturalnumbers$.\\
    Dann gilt $\pair{\forall n\geq n_0: B_n} \equivalent$ ($B_n$ ist wahr) $\land$ (Ist $B_n$ wahr für $n\geq n_0$ so ist auch $B_{n+1}$ wahr)
\end{bemerkung}

\begin{satz}
    $n,k\in\mathbb{Z} \impl a +b \in\mathbb{Z}$ und $a\cdot b \in\mathbb{Z}$
    \begin{proof}
        Folgt aus Satz~\ref{satz:n-eigenschaften} mit $-(-a) = a\quad a > 0 \impl -a < 0$
    \end{proof}
\end{satz}

\begin{satz}[Satz von Archimedes]
    $\forall x\in\realnumbers~\exists n\in\naturalnumbers: x < n$.
    (Das heißt $\naturalnumbers$ ist eine nach oben unbeschränkte Teilmenge von $\realnumbers$)
    \begin{proof}
        Angenommen die Aussage ist falsch. Das heißt $\exists x\in\realnumbers$ mit $n\leq x\quad\forall n\in\naturalnumbers$\\
        $\annot{\impl}{(Vollständ.)} a\definedas \sup\naturalnumbers$ existiert und $n<a~\forall n\in\naturalnumbers$\\
        Es gilt $a+1 > a \impl a - 1 < a$, das heißt $a-1$ ist keine obere Schranke für $\naturalnumbers$\\
        $\impl \exists n\in\naturalnumbers: a-1<n \impl a < n + 1 \in\naturalnumbers$. Widerspruch zu $a$ ist obere Schranke für $\naturalnumbers$.
    \end{proof}
\end{satz}

\begin{korollar}
    $\forall x\in\realnumbers~\exists n\in\naturalnumbers\colon -n<x$
    \begin{proof}
        Wende vorherigen Satz auf $-x$ an. $\impl \exists n\in\naturalnumbers\colon -x < n \impl x > -n$
    \end{proof}
\end{korollar}

\begin{satz}[Wohlordnungsprinzip für $\naturalnumbers$]
    Jede nichtleere Menge natürlicher Zahlen hat ein kleinstes Element.
    \begin{proof}
        Sei $M\subseteq \naturalnumbers, M\neq\emptyset$. Es gilt $\inf \naturalnumbers = 1 \impl a = \inf M \geq 1 > -\infty$. Zu zeigen: $a\in M$.\\
        Annahme: $a\not\in\naturalnumbers \impl a < m\quad\forall m \in\naturalnumbers$\\
        Satz~\ref{satz:inf} besagt, dass $\forall \varepsilon > 0~\exists m\in M: m<a+\varepsilon$.\\
        1) Wähle $\varepsilon = 1 \impl \exists m\in M: m< a + 1$.\\
        2) Wähle $\varepsilon = a -m: \impl \exists m'\in M: m'<a+\varepsilon =m$. Das heißt $m',m\in M$ mit $a<m'<m<a+1$\\
        $\impl 0 < m-m' < 1 \annot{\impl}{(5.)} m-m'\in\naturalnumbers$. Widerspruch zu Korollar~\ref{korollar:4.2.7}
    \end{proof}
\end{satz}

\newpage


    \section{Summe, Produkt, Wurzeln}
    %%%%%%%%%%%%%%%%%%%%%%%%
% 14. November 2023
%%%%%%%%%%%%%%%%%%%%%%%%

\thispagestyle{pagenumberonly}

\subsection{Summenzeichen, Produktzeichen}
\begin{definition}
    Seien $m\leq n$, $m,n\in\naturalnumbers_{0}$. Für jedes $k\in\naturalnumbers_0$, $m\leq k\leq n$, sei $a_k\in\realnumbers$.\\
    Dann setzt man:
    \begin{align*}
        \sum_{k=m}^{n}a_k &= a_m+a_{m+1}+a_{m+2}+\dots+a_{n}
        \intertext{und}
        \prod_{k=m}^{n} &= a_m\cdot a_{m+1}\cdot a_{m+2}\cdot \dots\cdot a_{n}
    \end{align*}
    Für $n\in\naturalnumbers_0$, $n<m$ setzt man $\prod_{k=m}^{n}a_k = 1$.
\end{definition}
\begin{definition}[Fakultät]
    Sei $n\in\naturalnumbers$, dann gilt:
    \begin{align*}
        n! &= 1 \cdot 2\cdot 3 \cdot \dots \cdot n
        \intertext{und wir definieren}
        0! &= 1
    \end{align*}
    Alternativ lässt sich rekursiv definieren:
    \begin{align*}
        0! &\definedas 1\\
        n! &= (n-1)! \cdot n
    \end{align*}
\end{definition}

\begin{satz} % Satz 3
    Die Anzahl aller möglichen Anordnungen einer $n$-elementigen Menge $\set{A_1, \dots, A_n}$ ist gleich $n!$.\\
    Wenn wir beispielsweise die Menge $\set{1,2,3}$ betrachten. Mögliche Anordnungen: $\set{1,2,3}$, $\set{1,3,2}$, $\set{2,1,3}$, $\set{2,3,1}$, $\set{3,1,2}$, $\set{3,2,1}$. Somit gibt es 6 Möglichkeiten, was $3!$ entspricht.
    \begin{proof}[Induktionsbeweis]
        ~\\
        \begin{induktionsanfang}
            $n=1$, es gibt eine Anordnung $\set{A_1}$ und es gilt $1! = 1$
        \end{induktionsanfang}
        \\
        \begin{induktionsschritt}
            Die Gesamtzahl aller Anordnungen von $\set{A_1, \dots, A_{n+1}}$ ist gleich
            \begin{align*}
                &(n+1)\cdot [\text{Gesamtzahl von Anordnungen von }\set{A_1, \dots, A_n}]\\
                \annot{=}{I-Ann} &(n+1) \cdot n! = (n+1)!\qedhere
            \end{align*}
        \end{induktionsschritt}
    \end{proof}
\end{satz}

\subsection{Binomischer Lehrsatz}
\begin{definition}[Binomialkoeffizient]
    Für $n,k\in\naturalnumbers_0$ setzt man:
    \begin{align*}
        \binom{n}{k} &\definedas \frac{n\cdot(n-1)\cdot\dots\cdot(n-k+1)}{k!} = \frac{n!}{k!\cdot(n-k)!} \tag{$n$ über $k$}
    \end{align*}
\end{definition}
\begin{bemerkung}[Spezielle Binomialkoeffizienten]
    $\binom{n}{0} = 1, \binom{n}{n} = 1, \binom{n}{k} = 0$ für $k>n$
\end{bemerkung}

\begin{satz}
    \label{satz:teilmengen-anzahl}
    Die Anzahl der $k$-elementigen Teilmengen einer $n$-elementigen Menge $\set{A_1, \dots, A_n}$ ist gleich $\binom{n}{k}$.
\end{satz}

\begin{hilfsatz}
    \label{hilfsatz:binom-add}
    $\forall k,n\in\naturalnumbers$ gilt $\binom{n}{k} = \binom{n-1}{k-1} + \binom{n-1}{k}$.
    \begin{proof}
        \begin{align*}
            \binom{n-1}{k} + \binom{n-1}{k-1} &= \frac{(n-1)!}{k!\cdot(n-1-k)!} + \frac{(n-1)!}{(k-1)!\cdot(n-1-k+1)!}\\
            &= \frac{(n-1)!\cdot(n-k)}{k!\cdot(n-k)!} + \frac{(n-1)!\cdot k}{k!\cdot(n-k)!}\\
            &= \frac{(n-1)!\cdot\interv{n-k+k}}{k!\cdot(n-k)!}\\
            &= \frac{n!}{k!\cdot(n-k)!} \annot{=}{Def.} \binom{n}{k}\qedhere
        \end{align*}
    \end{proof}
\end{hilfsatz}
\begin{proof}[Beweis von Satz~\ref{satz:teilmengen-anzahl} (Induktion nach $n$)]
    ~\\
    \begin{induktionsanfang}
        $n=1$, $\set{A_1}$. Wenn $k=0$, dann gibt es eine Möglichkeit und es gilt $\binom{1}{0} = 1$. Wenn $k=1$, gibt es auch eine Möglichkeit und es gilt $\binom{1}{1} = 1$.\\
    \end{induktionsanfang}
    \\
    \begin{induktionsschritt}
        $n\rightarrow n+1$\\
        Die Behauptung sei für $M_n=\set{A_1, \dots, A_n}$ schon bewiesen. Wir betrachten $M_n+1 = \set{A_1, \dots, A_{n+1}}$. Für $k=0$ und $k=n+1$ ist die Behauptung offensichtlich.\\
        Für $1\leq k \leq n$ gehört jede $k$-elementige Teilmenge von $M_{n+1}$ zu genau einer der folgenden Klassen:
        \begin{enumerate}
            \item $T_0$ besteht aus $k$-elementigen Teilmengen, die $A_{n+1}$ nicht enthalten.
            \item $T_1$ besteht aus denjenigen Teilmengen, die $A_{n+1}$ enthalten.
        \end{enumerate}
        \noindent In $T_0$ gibt es nach Induktionsannahme $\binom{n}{k}$ Elemente.\\
        In $T_1$ gibt es $\binom{n}{k-1}$ Elemente\footnotemark.\\
        Insgesamt:
        \begin{equation*}
            \binom{n}{k}+\binom{n}{k-1}\annot{=}{\ref{hilfsatz:binom-add}}\binom{n+1}{k}\qedhere
        \end{equation*}
    \end{induktionsschritt}
    \footnotetext{Wir wissen, dass $A_{n+1}$ bereits ein Element der Teilmenge ist. Damit müssen wir noch $k-1$ aus $n$ Elemente auswählen. Die Formel dafür folgt aus der Induktionsannahme}
\end{proof}

\begin{satz}[Binomischer Lehrsatz]
    \label{satz:binom-lehrsatz}
    Sei $x,y\in\realnumbers$ und $n\in\naturalnumbers$. Dann gilt:
    \begin{align*}
    (x+y)
        ^{n} &= \sum_{k=0}^{n} \binom{n}{k}\cdot x^{n-k}\cdot y^k
    \end{align*}
\end{satz}
\begin{beispiel}[Folgerung der binomischen Formel aus dem binomischen Lehrsatz]
    Es sei $n=2$. Es gilt $\binom{2}{0}=1$, $\binom{2}{1}=2$, $\binom{2}{2}=1$. Daraus folgt:
    \begin{align*}
    (x+y)
        ^{2} &= x^2+2xy+y^2
    \end{align*}
\end{beispiel}
\begin{proof}[Beweis von Satz~\ref{satz:binom-lehrsatz}]
    ~\\IA: $n=0$
    \begin{align*}
    (x+y)
        ^0&=1\\
        \sum_{k=0}^0\binom{0}{k}\cdot x^{k}\cdot y^{0-k} &= \binom{0}{k}\cdot 1\cdot 1 = 1
    \end{align*}
    Induktionsschritt: $n\rightarrow n+1$
    \begin{align*}
    (x+y)
        ^{n+1} &= (x+y)^n\cdot (x+y) = (x+y)^n \cdot x + (x+y)^n\cdot y\\
        (x+y)^n\cdot x &\annot{=}{I-An} \sum_{k=0}^{n}\binom{n}{k}\cdot x^{n-k}\cdot y^k\cdot x\\
        &=1\cdot x^{n+1} + \sum_{k=1}^{n} \binom{n}{k}\cdot x^{n+1-k}\cdot y^{k}\\
        (x+y)^n\cdot y &= \sum_{k=0}^{n}\binom{n}{k}\cdot x^{n-k}\cdot y^{k+1}\tag{$l\definedas k+1$}\\
        &= \sum_{l=1}^{n+1} \binom{n}{l-1}\cdot x^{n+1-l}\cdot y^{l}\\
        &= \sum_{k=1}^{n+1} \binom{n}{k+1}\cdot x^{n+1-k} \cdot y^{k}\\[10pt]
        \impl (x+y)^{n+1} &= x^{n+1} + \sum_{k=1}^{n} \interv{\binom{n}{k}+\binom{n}{k+1}} \cdot x^{n+1-k}\cdot y^k + y^{n+1}\\
        &\annot{=}{\ref{hilfsatz:binom-add}} \sum_{k=0}^{n+1} \binom{n+1}{k}\cdot x^{n+1-k}\cdot y^{k}\qedhere
    \end{align*}
\end{proof}

\begin{bemerkung}
    Sei $x>0$, dann gilt $(1+x)^n = 1+\underbrace{\binom{n}{1}x}_{n\cdot x} + \underbrace{\sum \dots}_{>0} > 1 + n\cdot x$
\end{bemerkung}

\subsection{Bernoullische Ungleichung}

\begin{satz}
    Es sei $n\in\naturalnumbers$ und $a\in\realnumbers$, $a > -1$. Dann gilt
    \begin{align*}
    (1+a)
        ^n \geq 1+na
    \end{align*}
    \begin{proof}
        Wir verwenden vollständige Induktion:\\
        \begin{induktionsanfang}
            $n=1 \impl 1+a = 1+a$
        \end{induktionsanfang}
        \\
        \begin{induktionsschritt}
            $n\rightarrow n+1$
            \begin{align*}
            (1+a)
                ^{n+1} = (1+a)^n\cdot (1+a) \annot{\geq}{I-Ann} (1+na)\cdot(1+a) = 1+na+a+na^2\geq 1+(n+1)\cdot a
            \end{align*}
        \end{induktionsschritt}
    \end{proof}
\end{satz}

%%%%%%%%%%%%%%%%%%%%%%%%
% 16. November 2023
%%%%%%%%%%%%%%%%%%%%%%%%

\newpage

\subsection{Wurzeln}

\begin{satz}
    Für jedes $c\in\realnumbers$, $c>0$, gibt es genau ein $x>0$, so dass $x^2 = c$ ist.
    \begin{proof}
        \textit{Eindeutigkeit}\\
        $x_1>0$, $x_2>0$: $\pair{x_1}^2 = \pair{x_2}^2 = c \impl 0 = (\pair{x_1}^2-\pair{x_2}^2) = (x_1-x_2) \cdot \underbrace{(x_1+x_2)}_{>0} \impl x_1 = x_2$\\
        \textit{Existenz}\\
        Wir definieren $M\definedas\set{z\in\realnumbers|~z\geq 0, z^2 \leq c}$. Dann gilt $0\in M \impl M\neq \emptyset$\\[10pt]
        $M$ ist beschränkt, weil $(1+c)^2=1+2c+c^2 > c$, \quad$z\in M \impl z < 1 + c$\\
        Somit $\exists \sup M$ und wir definieren $x\definedas \sup M$. Zu zeigen: $x^2 = c$\\[10pt]
        Wir nehmen an, dass $x^2<c$ und setzen $\varepsilon \definedas \min\set{1, \frac{c-x^2}{2x+1}} \impl 0 < \varepsilon \leq 1 \impl \varepsilon^2 < \varepsilon$\\
        $(x+\varepsilon)^2 = x^2 + 2\varepsilon x + \varepsilon^2 < x^2+\varepsilon\pair{2x+1}\leq x^2+c-x^2=c$\\
        $\impl x+\varepsilon\in M$ (Widerspruch) $\impl x^2 \geq c$\\[10pt]
        Wir nehmen an, dass $x^2 > c$, $\varepsilon \definedas\min\set{\frac{x^2-c}{2x}, \frac{x}{2}}$, $\varepsilon > 0$, $x-\varepsilon \geq x-\frac{x}{2}>0$\\
        $\pair{x-\varepsilon}^2 = x^2 - 2 x\varepsilon + \varepsilon^2 > x^2-2x\varepsilon \geq x^2-x^2+c\impl (x-\varepsilon)^2 > c \impl x\neq \sup M$ (Widerspruch)\\[10pt]
        $\impl x^2 = c$
    \end{proof}
\end{satz}

\begin{bemerkung}
    $x=\sqrt {c}$, $x=c^{\frac{1}{2}}$, $x$ ist die Quadratwurzel von $c$
\end{bemerkung}

\begin{satz}
    Für $n\in\naturalnumbers$ und für jedes $c\in\realnumbers$, $c\geq 0$ gibt es genau ein $x \geq 0$, $x\in\realnumbers$, so dass $x^n = c$.
    \begin{proof}
        \textit{Eindeutigkeit}\\
        $x_1>0$, $x_2>0$, $\pair{x_1}^n-\pair{x_2}^n = c$, $0=\pair{\pair{x_1}^n-\pair{x_2}^n} = \pair{x_1 - x_2}\cdot\pair{\sum_{k=0}^{n-1} \pair{x_1}^{n-k+1}\cdot \pair{x_2}^k}$\\
        $\impl x_1 = x_2$
    \end{proof}
    \noindent Die Existenz ist Aufgabe auf dem Übungsblatt.
\end{satz}

\begin{definition}[Spezielle Potenzen]
    $m,n\in\naturalnumbers$\quad $x^\frac{m}{n}\definedas \pair{x^\frac{1}{m}}^n$, $x^0 = 1$, $0^m = 0$ mit $m\neq 0$, $0^0 = 1$
\end{definition}

\subsection{Absolutbetrag}

\begin{definition}[Betrag]
    $\abs{a} \definedas \left\{ \begin{array}{lr}
                                    a  & a>0 \\
                                    0  & a=0 \\
                                    -a & a<0
    \end{array}\right.$
\end{definition}

\begin{satz}[Eigenschaften des Betrags]
    \theoremescape
    \begin{enumerate}[label=(\roman*)]
        \item $\abs{a} \geq 0$, $\abs{a} = 0 \equivalent a = 0$
        \item $\abs{\lambda\cdot a} = \abs{\lambda}\cdot\abs{a}$, $\forall \lambda, a \in \realnumbers$
        \item $\abs{a+b}\leq \abs{a}+\abs{b}$ (Dreiecksungleichung) %%% 5.14
    \end{enumerate}
    \begin{proof}[Beweis von (iii)]
        \begin{align*}
            \abs{a+b}^2 = &\pair{a+b}^2 = a^2 + 2ab + b^2\\
            \leq &\abs{a}^2 + 2\abs{a}\abs{b} + \abs{b}^2 = \pair{\abs{a}+\abs{b}}^2\\
            \impl &\abs{a+b}\leq\abs{a}+\abs{b}\qedhere
        \end{align*}
    \end{proof}
\end{satz}

\begin{definition}[Geometrische Betrachtung des Betrags]
    Man nennt $\abs{a-b}$ den Abstand zweier Punkte $a,b\in\realnumbers$ auf der Zahlengerade.
\end{definition}

\begin{satz}[Eigenschaften von Differenzen im Betrag]
    \label{satz:diff-abs}
    \theoremescape
    \begin{enumerate}[label=(\roman*)]
        \item $\abs{a-b}\geq 0$, $\abs{a-b} = 0\equivalent a = b$
        \item $\abs{a-b} = \abs{b-a}$
        \item $\abs{a-b}\leq \abs{a-c} + \abs{b-c}$ $\forall c\in\realnumbers$ %%% 5.15
    \end{enumerate}
    \begin{proof}[Beweis von (iii)]
        \begin{align*}
            \abs{a-b} &= \abs{a-c+c-b} \leq \abs{a-c}+\abs{c-b} = \abs{a-c}+\abs{b-c}\qedhere
        \end{align*}
    \end{proof}
\end{satz}

\begin{satz} %%% 5.16
    $\forall a,b\in\realnumbers$ gilt $\abs{\abs{a}-\abs{b}}\leq\abs{a-b}$
    \begin{proof}
        \begin{align*}
            \abs{a} = \abs{a-b+b} &\leq \abs{a-b} + \abs{b}\\
            \impl \abs{a}-\abs{b} &\leq \abs{a-b}\\
            \abs{b}-\abs{a} &\leq \abs{b-a} = \abs{a-b}\\[10pt]
            \impl \abs{a-b}&\geq\abs{\abs{a}-\abs{b}}\qedhere
        \end{align*}
    \end{proof}
\end{satz}

\begin{folgerung}
    \theoremescape
    \begin{enumerate}[label=(\roman*)]
        \item $\abs{a-b}\geq \abs{a}-\abs{b}$
        \item $\abs{a+b} = \abs{a-\pair{-b}} \geq \abs{a} - \abs{-b}$
    \end{enumerate}
\end{folgerung}

\begin{bemerkung}
    Durch Induktion leitet man her, dass:
    \begin{align*}
        \abs{\sum_{i=1}^{n} a_i} &\leq \sum_{i=1}^{n} \abs{a_i}\quad a_i \in\realnumbers
    \end{align*}
\end{bemerkung}

\newpage


    \section{[*] Folgen und Grenzwerte}

    \subsection{Konvergenz}
    \thispagestyle{pagenumberonly}

    \begin{definition}[Reelle Folge]
        Eine reelle Folge ist eine Abbildung $\naturalnumbers\fromto\realnumbers$. Alternative Notation: $\pair{a_n}_{n\in\naturalnumbers}$, $\pair{a_1, a_2, \dots}$
    \end{definition}

    \begin{beispiel}
        $a_n = \frac{1}{n}$, $n\geq 1$\quad $\pair{1,\frac{1}{2},\frac{1}{3}, \dots}$
    \end{beispiel}

    \begin{definition}[Konvergenzkriterium]
        Sei $\pair{a_n}_{n\in\naturalnumbers}$ eine Folge reeller Zahlen. Die Folge heißt konvergent gegen $a\in\realnumbers$, falls gilt:
        \begin{align*}
            \forall \varepsilon > 0 ~\exists n_0\in\naturalnumbers: \abs{a_n-a} &< \varepsilon\text{ für alle $n>n_0$}
            \intertext{$a$ heißt Grenzwert von $\pair{a_n}_{n\in\naturalnumbers}$ und man schreibt}
            \lim_{n\fromto \infty} a_n &= a
        \end{align*}
    \end{definition}

    \begin{beispiel}[Nachweis von Konvergenz]
        Es sei
        \begin{align*}
            \pair{a_n}_{n\in\naturalnumbers}&=1+\frac{(-1)^n}{2n}
            \intertext{wir definieren $\varepsilon$ und $n_0$}
            \varepsilon > 0,&\quad n_0 = \frac{1}{2\varepsilon}\\
            \intertext{und wenden das Konvergenzkriterium an}
            \abs{a_n-1} = \abs{1+\frac{(-1)^n}{2n}-1} &= \abs{\frac{(-1)^n}{2n}} < \frac{\abs{(-1)^n}}{\abs{\frac{2}{2\varepsilon}}} = \varepsilon\\
            \impl \lim_{n\fromto\infty} a_n &=1
        \end{align*}
    \end{beispiel}

    \begin{bemerkung}[\anf{Fast alle Elemente}]
        Wir sagen, dass fast alle Element der Folge $(a_n)_{n\in\naturalnumbers}$ eine Eigenschaft (E) haben, wenn es höchstens eine endliche Anzahl von $a_n$ existiert, die die Eigenschaft (E) nicht erfüllen.
    \end{bemerkung}

    \begin{definition}[Alternativ formuliertes Konvergenzkriterium]
        Sei $(a_n)_{n\in\naturalnumbers}$ eine Folge, dann heißt diese Folge konvergent gegen $a\in\realnumbers$, wenn $\forall \varepsilon > 0$ und für fast alle $a_n$ gilt: $\abs{a_n-a}<\varepsilon$
    \end{definition}

    \begin{definition}[Nullfolge]
        Eine Folge, die gegen $0$ konvergiert, heißt Nullfolge.
    \end{definition}

    \begin{definition}[Divergenz]
        Eine nicht-konvergente Folge heißt divergent.
    \end{definition}

    \begin{definition}[$\varepsilon$-Umgebung]
        Für $\varepsilon > 0$, $a\in\realnumbers$ versteht man unter der $\varepsilon$-Umgebung von $a$ das Intervall $\left]a-\varepsilon, a+\varepsilon\right[$
    \end{definition}

    \subsection{Geometrische Bedeutung der Konvergenz}

    \begin{visualisierung}
        ~
        \begin{figure}[H]
            \centering
            \begin{tikzpicture}
                \draw (0,0) -- (10,0);
                \draw (5,-0.25) node[below] {$a$} -- (5,0.25);
                \draw (3.8,-0.4) -- (3.9, -0.4) node[below] {$a-\varepsilon$}  -- (4,-0.4) -- (4,0.4) -- (3.8, 0.4);
                \draw (6.2,-0.4) -- (6.1, -0.4) node[below] {$a+\varepsilon$}  -- (6,-0.4) -- (6,0.4) -- (6.2, 0.4);
                \foreach \x in {1, 2.2, 3, 3.9, 4.3, 4.6, 4.8, 4.9, 4.94, 4.96, 4.98, 5.005, 5.01, 5.04, 5.1, 5.35, 5.85, 6.4}
                \draw (\x,-0.15) -- (\x,0.15);
            \end{tikzpicture}
            \caption{Geometrische Darstellung einer\\ konvergenten Folge und ihrer $\varepsilon$-Umgebung}
        \end{figure}
    \end{visualisierung}

    %%%%%%%%%%%%%%%%%%%%%%%%
    % 21. November 2023
    %%%%%%%%%%%%%%%%%%%%%%%%

    \subsection{Eigenschaften von Folgen und Konvergenzen}

    \begin{definition}[Beschränktheit von Folgen]
        Eine Folge $(a_n)_n$ heißt nach oben beschränkt, falls es ein $k\in\realnumbers$ gibt mit
        \begin{align*}
            \forall n\in \naturalnumbers\colon a_n \leq k\tag{$k$ ist obere Schranke für $(a_n)_n$}
        \end{align*}
        Beschränktheit nach unten wird analog definiert.\\
        $(a_n)_n$ heißt beschränkt, falls ein $k\geq 0$ existiert mit
        \begin{align*}
            \forall n\in\naturalnumbers\colon -k \leq a_n \leq k\quad \text{bzw.}\quad \forall n\in\naturalnumbers\colon \abs{a_n} \leq k
        \end{align*}
    \end{definition}

    \begin{satz}
        \label{satz:konv-folg-beschr}
        Jede konvergente Folge $(a_n)_{n\in\naturalnumbers}$ ist beschränkt.
        \begin{proof}
            Sei $a$ der Grenzwert von $(a_n)_n$\quad $\pair{\lim_{n\fromto\infty} a_n = a}$.\\
            Das heißt für $\varepsilon = 1 \impl \exists N\in\naturalnumbers\colon \abs{a_n-a} < 1\quad\forall n\geq N$\\
            $\abs{a_n} = \abs{a_n-a+a} \leq \underbrace{\abs{a_n-a}}_{<1} + \abs{a} < 1+\abs{a} \quad \forall n\geq N$\\
            Setze $k\definedas \max\pair{\abs{a_1}, \abs{a_2}, \dots, \abs{a_{N-1}}, 1+\abs{a}} \geq 0$\\
            $\impl \abs{a_n} \leq k \quad\forall n\in\naturalnumbers$\qedhere
        \end{proof}
    \end{satz}

    \begin{satz}[Eindeutigkeit des Grenzwerts]
        Der Grenzwert einer konvergenten Folge $(a_n)_n$ ist eindeutig.
        \begin{proof}
            Annahme: $a_n\fromto a$ und $a_n\fromto b$ mit  $a\neq b$. Dann gilt \OBDA, dass $a<b$.\\
            Wir setzen $\varepsilon = \frac{b-a}{2} > 0$.
            \begin{align*}
                \impl &\exists N_1 \in\naturalnumbers: \abs{a_n-a} < \varepsilon \quad \forall n\geq N_1\\
                \impl &\exists N_2 \in\naturalnumbers: \abs{a_n-b} < \varepsilon \quad \forall n\geq N_2\\[10pt]
                &N\definedas \max\pair{N_1, N_2}\\
                \impl &\forall n\geq N\colon a_n < a+\varepsilon = a + \frac{b-a}{2} = \frac{b+a}{2} = b-\varepsilon\\
                &\forall n\geq N: a_n > b-\varepsilon = \frac{b+a}{2}\\[10pt]
                \impl &a_n <\frac{b+a}{2} < a_n \quad \forall n\geq N\qquad\text{ (Widerspruch)}\qedhere
            \end{align*}
        \end{proof}
    \end{satz}

    \begin{satz}[Eigenschaften von konvergenten Folgen]
        \label{satz:konvergenzsaetze}
        Seien $(x_n)_n, (y_n)_n$ konvergente reelle Folgen mit $x_n\fromto a$, $y_n\fromto b$ für $n\fromto\infty$. Dann gilt:
        \begin{enumerate}[label=(\alph*)]
            \item $x_n+y_n\fromto a+b$\quad ($\lim\pair{x_n+y_n}=\lim x_n + \lim y_n$).
            \item $x_n\cdot y_n \fromto a\cdot b$\quad ($\lim\pair{x_n\cdot y_n} = \lim x_n \cdot \lim y_n$).
            \item $\lambda\cdot y_n \fromto \lambda\cdot b$\quad ($\lim\pair{\lambda\cdot y_n} = \lambda \cdot \lim y_n$). ($\lambda\in\realnumbers$)
            \item Ist $b\neq 0$, so ist $y_n \neq 0$ für fast alle $n\in\naturalnumbers$ und $\frac{x_n}{y_n}$ ist für fast alle $n$ definiert und $\lim\pair{\frac{x_n}{y_n}} = \frac{\lim x_n}{\lim y_n} = \frac{a}{b}$.
            \item $\abs{x_n} \fromto \abs{a}$\quad ($\lim_{n\fromto \infty} \abs{x_n} = \abs{\lim_{n\fromto \infty} x_n}$)
            \item Ist $x_n \leq y_n$ für fast alle $n\in\naturalnumbers$ $\impl a = \lim_{n\fromto \infty} x_n \leq b = \lim_{n\fromto \infty} y_n$.
        \end{enumerate}

        \begin{proof}
            \theoremescape
            \begin{enumerate}[label=(\alph*)]
                \item Nach Satz~\ref{satz:diff-abs} gilt
                \begin{align*}
                    \abs{x_n+y_n - (a+b)} = \abs{x_n - a + (y_n-b)} &\leq \abs{x_n - a} + \abs{y_n-b}
                    \intertext{Aufgrund der Konvergenz der Folgen gilt}
                    \forall \varepsilon > 0~\exists N_1: \abs{x_n - a} &< \frac{\varepsilon}{2} \quad \forall n\geq N_1\\
                    \forall \varepsilon > 0~\exists N_2: \abs{y_n - b} &< \frac{\varepsilon}{2} \quad \forall n\geq N_2\\
                    \intertext{Wir wählen $N=\max\pair{N_1, N_2}$}
                    \impl \abs{\pair{x_n+y_n}-(a+b)} \leq \abs{x_n - a} + \abs{y_n-b} &\leq \frac{\varepsilon}{2} + \frac{\varepsilon}{2} = \varepsilon\quad \forall n\geq N
                \end{align*}
                \item Umformen, um eine passende Ungleichung zu erreichen
                \begin{align*}
                    x_n \cdot y_n - ab &= \pair{x_n-a+a} \cdot y_n - ab\\
                    &= \pair{x_n-a}\cdot y_n + a \cdot y_n - ab\\
                    &= \pair{x_n-a}\cdot y_n + a\cdot \pair{y_n-b}\\[10pt]
                    \impl \abs{x_n\cdot y_n - ab} &= \abs{\pair{x_n-a}\cdot y_n + a\cdot \pair{y_n-b}}\\
                    &\leq \abs{x_n-a}\cdot\abs{y_n}+\abs{a}\cdot\abs{y_n-b}
                \end{align*}
                Nach Satz~\ref{satz:konv-folg-beschr} ist $y_n$ beschränkt
                \begin{align*}
                    \exists k\geq 0\colon \abs{y_n} &\leq k \quad\forall n\in\naturalnumbers
                    \intertext{Nach der Konvergenz der Folgen gilt außerdem}
                    \forall \varepsilon >0~\exists N_1, N_2\colon \abs{x_n-a} &< \frac{\varepsilon}{2(k+1)}\quad\forall n\geq N_1\\
                    \abs{y_n-b} &< \frac{\varepsilon}{2(\abs{a}+1)}\quad\forall n\geq N_2\\[10pt]
                    \impl \forall n\geq \max\pair{N_1, N_2}: \abs{x_n\cdot y_n-ab} &\leq \frac{\varepsilon}{2(k+1)}\cdot k + \abs{a}\cdot\frac{\varepsilon}{2(\abs{a}+1)} \leq \varepsilon
                \end{align*}
                \item Setze $y_n=\lambda \fromto \lambda$ und verwende (b)
                \item Wir wählen $\varepsilon\definedas\frac{\abs{b}}{2}$
                \begin{align*}
                    \exists N\in\naturalnumbers\colon \abs{y_n-b} &< \varepsilon = \frac{\abs{b}}{2}\quad\forall n\geq \naturalnumbers.\\[10pt]
                    \forall n\geq N\colon\quad\abs{y_n} &= \abs{y_n-b+b} = \abs{b+(y_n-b)}\\
                    &\geq \abs{b} - \abs{y_r-b} > \abs{b} - \frac{\abs{b}}{2} + \frac{\abs{b}}{2}\\
                    &> 0
                \end{align*}
                $\impl y_n \neq 0$ für fast alle $n$ und somit $\frac{x_n}{y_n}$ wohldefiniert für fast alle $n\in\naturalnumbers$
                \begin{align*}
                    \intertext{Wir betrachten den Spezialfall $x_n = 1$}
                    \abs{\frac{1}{y_n} - \frac{1}{b}} &= \abs{\frac{b-y_n}{y_n\cdot b}} = \frac{\abs{b-y_n}}{\abs{y_n}\cdot\abs{b}}.\\
                    \intertext{Wir wissen schon, dass $\exists N_1\colon \abs{y_n} \geq \frac{\abs{b}}{2} \quad\forall n\geq N_1$}
                    \impl \forall n\geq N_1: \abs{\frac{1}{y_n}-\frac{1}{b}} &\leq \frac{\abs{y_n-b}}{\abs{y_n}\cdot\abs{b}} \leq \frac{2\cdot\abs{y_n-b}}{\abs{b}\cdot \abs{b}}\\
                    \forall \varepsilon > 0~\exists N_2\colon \abs{y_n-b} &< \frac{\abs{b}^2}{2}\cdot\varepsilon\\
                    \impl \forall n\geq \max\pair{N_1, N_2}\colon \abs{\frac{1}{y_n}-\frac{1}{b}} &< \frac{2}{\abs{b}^2} \cdot \frac{\abs{b}^2}{2} \cdot \varepsilon = \varepsilon
                    \intertext{Jetzt wenden wir (b) an und dann gilt}
                    \lim_{n\fromto \infty} \frac{x_n}{y_n} = \lim_{n\fromto \infty} \pair{x_n\cdot\frac{1}{y_n}} &= \lim x_n \cdot \lim\frac{1}{y_n} = a\cdot \frac{1}{b} = \frac{a}{b}
                \end{align*}
                \item Wir zeigen mit der Umkehrung der Dreiecksungleichung
                \begin{align*}
                    \abs{\abs{x_n}-\abs{a}} \annot{\leq}{Dreiecks.} \abs{x_n-a}\\
                    \impl \abs{x_n} \fromto \abs{a}
                \end{align*}
                \item Angenommen $a > b$. Wir wählen $\varepsilon=\frac{a-b}{2}$.
                \begin{align*}
                    \impl \exists N_1: \abs{x_n-a} < \varepsilon \quad\forall n\geq N_1\\
                    \impl \exists N_2: \abs{y_n-b} < \varepsilon \quad\forall n\geq N_2\\
                    \intertext{Für $n\geq \max\pair{N_1,N_2}$ folgt}
                    x_n > a-\varepsilon = a + \frac{a-b}{2} = \frac{a+b}{2} = b+\frac{a-b}{2} = b + \varepsilon > y_n\\
                    \impl x_n > y_n
                \end{align*}
            \end{enumerate}
        \end{proof}
    \end{satz}

    \newpage

    \begin{satz}[Sandwich Satz]
        \label{satz:sandwich}
        Seien $(a_n)_n$, $(b_n)_n$, $(c_n)_n$ Folgen mit $\lim a_n = a$, $\lim \pair{b_n-a_n} = 0$ und $a_n \leq c_n \leq b_n$ für fast alle $n\in\naturalnumbers$.\\
        Dann folgt, dass $(c_n)_n$ und $(b_n)_n$ jeweils gegen $a$ konvergieren.
        \begin{proof}
            ~\\
            1. Schritt: $\lim b_n = a$.\\[10pt]
            Sei $\varepsilon > 0$. Da $a_n\fromto a$, $\exists N_1: \abs{a_n-a} < \frac{\varepsilon}{2}\quad\forall n\geq N_1$\\
            $0\leq b_n-a_n$ ist Nullfolge $\impl \exists N_2: \abs{b_n-a_n} < \frac{\varepsilon}{2}\quad\forall n\geq N_2$\\
            \begin{align*}
                \abs{b_n-a} &= \abs{b_n-a_n+a_n-a}\\
                &\leq \abs{b_n - a_n} + \abs{a_n-a}\\
                &\leq \frac{\varepsilon}{2} + \frac{\varepsilon}{2} = \varepsilon\quad \forall n\geq \max\pair{N_1, N_2}
            \end{align*}
            \noindent 2. Schritt: $\lim c_n = a$.
            \begin{align*}
                \lim a_n &= a = \lim b_n\\
                \impl \forall\varepsilon > 0~\exists N_1, N_2\colon &\abs{a_n-a} < \varepsilon \quad\forall n\geq N_1\\
                \text{ und } &\abs{b_n-a} < \varepsilon \quad\forall n\geq N_2\\
                a-\varepsilon &< a_n < a + \varepsilon\quad\forall n\geq N_1\\
                a-\varepsilon &< b_n < a + \varepsilon\quad\forall n\geq N_2\\
                \intertext{Auch gilt $a_n \leq c_n \leq b_n \quad\forall n\geq N_3$ und wir definieren $N\definedas\max\pair{N_1, N_2, N_3}$}
                \impl a-\varepsilon &< a_n \leq c_n \leq b_n < a + \varepsilon\quad\forall n\geq N
            \end{align*}
            Das heißt $\abs{c_n-a} < \varepsilon\quad\forall n\geq N$. Das heißt $\lim c_n = a$\qedhere
        \end{proof}
    \end{satz}

    \subsection{[*] Monotone Konvergenz}

    Wir sind bisher immer davon ausgegangen, dass wir den Grenzwert einer Folge bereits kennen. Das folgende Unterkapitel beschäftigt sich damit, die Konvergenz einer Folge nachzuweisen, wenn deren Grenzwert nicht bekannt ist.

    \begin{definition}[Monotonie]
        Eine reelle Folge $(a_n)_n$ heißt
        \begin{enumerate}[label=(\roman*)]
            \item monoton wachsend, wenn $a_n \leq a_{n+1}\quad\forall n\in\naturalnumbers$
            \item streng monoton wachsend, wenn $a_n < a_{n+1}\quad\forall n\in\naturalnumbers$
            \item monoton fallend, wenn $a_n \geq a_{n+1}\quad\forall n\in\naturalnumbers$
            \item streng monoton fallend, wenn $a_n > a_{n+1}\quad\forall n\in\naturalnumbers$
        \end{enumerate}
        \noindent Wir nennen $(a_n)_n$ (streng) monoton, falls sie (streng) monoton wachsend oder fallend ist.
    \end{definition}

    \begin{satz}[Monotone Konvergenz]
        \label{satz:monoton-konv}
        Eine monoton wachsende Folge $(a_n)_n$ konvergiert genau dann, wenn $(a_n)_n$ nach oben beschränkt ist.\\
        Und eine monoton fallende Folge $(a_n)_n$ konvergiert genau dann, wenn sie nach unten beschränkt ist.
    \end{satz}

    \newpage

    \begin{lemma}[Hilfaussage für monotone Konvergenz]
        ~\label{lemma:hilf-monoton-konv}
        Jede nach oben (bzw. unten) beschränkte Folge besitzt eine kleinste obere (bzw. größte untere) Schranke.
        \begin{proof}
            Sei $(a_n)_n$ nach oben beschränkt und $S\definedas\set{c\in\realnumbers|~a_n \leq c~\forall n} \neq \emptyset\footnote{Folgt aus der Beschränktheit}.$\\
            Dann ist $S$ nach unten beschränkt, da $a_1 \leq c\quad\forall c\in S$\\
            $\impl a\definedas \inf S$ existiert\\[10pt]
            Behauptung: $a$ ist obere Schranke für $(a_n)_n$, das heißt $a\in S$.\\[10pt]
            Angenommen $a\notin S$ ($a$ ist keine obere Schranke)\\
            $\impl \exists n_0 \in\naturalnumbers\colon a_{n_0} > a$.\\
            Wir setzen $\varepsilon\definedas a_{n_0} - a > 0$. Und $\exists c\in S\colon c < a + \varepsilon = a + a_{n_0} - a = a_{n_0}$. Widerspruch zu $c\in S$.\\[10pt]
            Das heißt $a$ ist obere Schranke für $(a_n)_n$.
        \end{proof}
    \end{lemma}

    \begin{proof}[Beweis von Satz~\ref{satz:monoton-konv}]
        Angenommen $(a_n)_n$ ist monoton wachsend und nach oben beschränkt.
        \begin{align*}
            &a\definedas\sup_{n\in\naturalnumbers} a_n\\
            \impl &a_n \leq a\quad\forall n\in\naturalnumbers
            \intertext{Sei $\varepsilon > 0 \impl a-\varepsilon$ keine obere Schranke für $(a_n)_n$ mehr.}
            \impl &\exists N\in\naturalnumbers\colon a_N \geq a-\varepsilon\\
            \intertext{Sei $n\geq N$. Dann gilt:}
            &a_N \leq a_{N+1} \leq a_{N+2} \leq \dots \leq a_{N+k} = a_n\\
            \impl &\forall n\geq N: a-\varepsilon < a_n\\
            \impl &\abs{a_n-a} < \varepsilon\qedhere
        \end{align*}
    \end{proof}

    \begin{proof}[Alternativer Beweis]
        \footnote{Dieser Beweis bezieht sich laut Vorlesung auf Lemma 14, welches allerdings nicht existiert. Gemeint ist vermutlich Satz 14 (hier Satz~\ref{satz:monoton-konv}) oder Lemma 15 (hier Lemma~\ref{lemma:hilf-monoton-konv})}
        ~\\
        Sei $(a_n)_n$ eine Folge mit $a_n = f(a), f:\naturalnumbers\fromto\realnumbers$\\
        $Bild(f) = f(\naturalnumbers)$ nach oben beschränkt\\
        $a\definedas\sup\pair{f\pair{\naturalnumbers}}\geq a_n \quad\forall n\in\naturalnumbers$
    \end{proof}

    \newpage

    %%%%%%%%%%%%%%%%%%%%%%%%
    % 23. November 2023
    %%%%%%%%%%%%%%%%%%%%%%%%

    \begin{beispiel}
        \theoremescape
        \begin{enumerate}
            \item Berechnen von $\sqrt {c}$ für $c>0$.
            \begin{align*}
                x^2 = c &\equivalent x = \frac{c}{x}\qquad x >0\\
                &\equivalent x=\frac{1}{2}\pair{x+\frac{c}{x}}
            \end{align*}
            Folge $x_n$: Wählen $x_0 > 0$. Für $n\in\naturalnumbers$ sei $x_n=\frac{1}{2}\pair{x_{n-1}+\frac{c}{x_{n-1}}}$\\
            Behauptung 1: $x_n \geq \sqrt{c}\quad\forall n\in\naturalnumbers$\\
            Bew.:
            \begin{align*}
                x_1 &= \frac{1}{2}\pair{x_0+\frac{c}{x_0}}\\
                \intertext{haben AGM\footnotemark: $\forall a,b\geq 0\colon ab\leq \pair{\frac{a+b}{2}}^2$}
                0 &\leq x^2-2xy+y^2 \equivalent xy \leq \frac{x^2+y^2}{2}
                \intertext{Setzen $x=\sqrt{a}$ und $y=\sqrt{b}$, $\sqrt{ab} \leq\frac{ab}{2}$}
                (x_1)^2 \geq \frac{1}{2}\pair{x_0-\frac{c}{x_0}}^2 \geq x_0\cdot\frac{c}{x_0} = c\\
                x_1 &\geq \sqrt{c}\\
                \intertext{Falls }
            \end{align*}
            \footnotetext{Arithmetisch-geometrisches Mittel}
            Behauptung 2: $(x_n)_n$ ist monoton fallend.\\
            \textit{Beweis}
            \begin{align*}
                x_{n+1} &= \frac{1}{2}\pair{x_n+\frac{c}{x_n}}
                \intertext{haben für $n\geq 1$: $(x_n)^2 = \pair{\frac{1}{2}\pair{x_{n+1} + \frac{c}{x_{n+1}}}}^2 \geq c$}\\
                \impl x_n &\geq \frac{c}{x_n}\\
                \impl x_{n+1} &\geq \frac{1}{2}\pair{x_n+\frac{c}{x_n}}\leq \frac{1}{2}\pair{x_n+x_n} = x_n
            \end{align*}
            Nach dem Satz der monotonen Konvergenz ist $x\definedas\lim_{n\fromto\infty} x_n$ entspricht?? $\geq\sqrt {c}$\\
            $=\lim_{n\fromto\infty} x_{n+1}$
            \begin{align*}
                x &= \lim x_n = \lim \frac{1}{2}\pair{x_{n+1} + \frac{c}{x_{n+1}}} = \frac{1}{2}\pair{x+\frac{c}{x}}\\
                \equivalent x &= \sqrt {c}
            \end{align*}



            \item $x_n = \frac{1}{n}\fromto 0$
            \begin{align*}
                \abs{\frac{1}{n}-0} &= \frac{1}{n} \tag{geg. $\varepsilon > 0$ nach $n\in\naturalnumbers$, $N>\frac{1}{\varepsilon} \equivalent \varepsilon > \frac{1}{N}$}
                \impl \forall n\geq N\colon \abs{\frac{1}{n}-0} = \frac{1}{n} \leq \frac{1}{N}< \varepsilon
            \end{align*}
            Ähnlich funktioniert $x_n = \frac{1}{\sqrt {c}}$
            \item $x_n\definedas \frac{1+2+3+\dots+n}{2} \fromto \frac{1}{2}$ für $n\fromto\infty$\\
            Bew: $1+2+3+\dots+n = \frac{n\cdot(n+1)}{2} \impl x_n = \frac{\frac{n\cdot(n+1)}{2}}{n^2} = \frac{n^2+n}{2n^2} = \frac{1+\frac{1}{n}}{2} \fromto \frac{1}{2}$
            \item Sei $0\leq q< 1$. $x_n \definedas q^n \fromto 0$\\
            Bew: Ist $q=0\impl x_n = 0^n = 0 \fromto 0$.\\
            Also sei $0<q<1 \impl \frac{1}{q}> 1$\\
            $h\definedas\frac{1}{q}-1>0\quad \frac{1}{q}= 1 + h$, $q=\frac{1}{1+h}$\\
            $\impl x_n = q^n \leq \pair{\frac{1}{1+h}}^n = \frac{1}{\pair{1+h}}^n$\\
            Nach Bernoulli: $\pair{1+h}^n\geq 1+nh$\\
            $\impl q^n = \frac{1}{()}$
            \item Sei $-1<q<1\impl x_n \definedas q^n \fromto 0$. (Übung)
            \item Sei $a>0$, $x_n\definedas a^\frac{1}{n} = \sqrt[n]{a} \fromto 1$\\
            Bew: 1. Fall $a = 1\impl x_n = 1$\\
            2. Fall $a> 1 \impl a^\frac{1}{n} > a^\frac{a}{n}=1$\\
            $h_m = a^\frac{1}{m} - 1 > 0$ und $a+h_m = a^\frac{a}{m}$\\
            $q=\pair{a+b_n}^n \geq 1+ab_1 \geq a_{b1}$\\
            $\impl h_n < \frac{a}{n}$\\
            $\abs{a^\frac{1}{n}-1} = h_n < \frac{a}{n}\impl \lim a^\frac{1}{n}= 1$\\
            3. Fall $0<a<1$\\
            \begin{align*}
                b &= \frac{1}{a} > 1\\
                \impl \lim b^\frac{1}{n} &= 1\\
                &=\lim \frac{1}{a^\frac{1}{n}} &\impl \lim a^{\frac{1}{n}} = 1
            \end{align*}
            \item $x_n = \sqrt[n]{n} = n^\frac{1}{n}\fromto 1$
            \begin{align*}
                n \geq 2 \impl \frac{1}{n^n} > 1\qquad h_n \definedas n^{\frac{1}{n}} - 1\\
                \impl n = \pair{n^\frac{1}{n}}^n = \pair{1+h_n}^n \geq 1+n\cdot h_n > n\cdot h_n \impl h_n \leq \frac{h}{n} = 1
            \end{align*}
            Aber: Binomialsatz
            \begin{align*}
                \pair{1+h_n}^n &= \underbrace{\sum_{k=0}^{n} \binom{n}{k}\pair{h_n}^k}_{\geq 0} \geq \binom{n}{0}\cdot \pair{h_n}^0 \binom{n}{1}\cdot \pair{h_n}^1 + \binom{n}{2}\cdot \pair{h_n}^2\\
                &= 1 + n\cdot h_n + \frac{n\cdot(n+1)}{2}(h_n)^2\\
                > \frac{n(n+1)}{2}\cdot h_n^2\\
                \impl h_n^2 < \frac{2n}{n(n-1)} = \frac{2}{n-1}\\
                0 < h_n < \sqrt{\frac{2}{n-1}} \fromto 0\\
                \impl \lim h_n &= 0 \equivalent \lim n^{\frac{1}{n}} = 1
            \end{align*}
        \end{enumerate}
    \end{beispiel}

    \begin{definition}
        Eine reelle Folge $(x_n)_n$ strebt gegen unendlich, falls $\forall k \geq 0~\exists N\in\naturalnumbers\colon x_n \geq k\quad \forall n> N$\\
        ($\equivalent \forall k \geq 0$ ist $x_n < k$ für endlich viele $n\in\naturalnumbers$).\\
        $(x_n)_n$ strebt gegen $-\infty$, falls $(-x_n)_n$ gegen $\infty$ strebt.
    \end{definition}

    $x_n=n$ divergiert gegen $\infty$

    \begin{satz}
        Falls $x_n\fromto \infty$ folgt $\frac{1}{x_n}\fromto 0$.\\
        Ist $(x_n)_n$ Nullfolge mit $x_n > 0$ für fast alle $n\in\naturalnumbers$, so ist $\frac{1}{x_n}\fromto \infty$\\
        Falls $x < 0$ für fast alle $n$, so folgt $\frac{1}{x_n} \fromto -\infty$
        \begin{proof}
            Sei $x_n\fromto \infty$\\
            $\impl \forall \varepsilon > 0~\exists N\in\naturalnumbers\colon x_n > \frac{1}{2}\quad\forall n \geq N$\\
            $\impl 0 < \frac{1}{x_n} < \varepsilon\quad \forall n \geq N$, das heißt $\frac{1}{x_n}\fromto 0$\\
            (Rest Übung)
        \end{proof}
    \end{satz}

    \begin{beispiel}
        $\frac{n}{2^n}\fromto 0$ (Sogar $\forall k\in\naturalnumbers\colon \frac{n^k}{2^n} \fromto 0$)

        \begin{proof}
            Zu zeigen: $x_n = \frac{2^n}{n}\fromto \infty$\\
            $2^n = (1+1)^n = \\sum_{k=0}^{n} \binom{n}{k}\geq \binom{n}{2} = \frac{n(n+1)}{2}$\\
            $\impl x_n = \frac{2^n}{n}\geq \frac{\frac{n(n-1)}{2}}{1} = \frac{1}{2}(n+1)$\\
            $\impl x_n \fromto \infty$, $\frac{1}{x_n}\fromto 0$
        \end{proof}
    \end{beispiel}

    \begin{beispiel}[Die Eulersche Zahl]
        \begin{align*}
            a_n &\definedas (1+\frac{1}{n})^n\\
            b_n &\definedas (1 + \frac{1}{n})^{n+1}\\
            \impl a_n & b_n\quad\forall n \in\naturalnumbers\\
            \intertext{Behauptung:}
            a_n &< a_{n+1}\text{ und } b_n > b_{n+1} \quad \forall n
        \end{align*}
        \begin{proof}
            Sei $n\geq 2$
            \begin{align*}
                \frac{a_n}{a_{n-1}} &= \frac{\pair{1+\frac{1}{n}}^n}{\pair{1+\frac{1}{n-1}}^{n-1}}\\
                &= \pair{1+\frac{1}{n-1}} \cdot \frac{\pair{1+\frac{1}{n}}^n}{\pair{1+\frac{1}{n-1}}^n}\\
                &= \frac{n}{n-1} \cdot \pair{\frac{\frac{n+1}{n}}{\frac{n}{n-1}}}^n = \frac{n}{n-1}\cdot \pair{\frac{(n+1)\cdot (n-1)}{n^2}}^n\\
                &= \frac{n}{n-1}\cdot \pair{\frac{n^2-1}{n^2}}^n\\
                &= \frac{n}{n-1}\cdot\pair{1-\frac{1}{n^2}}^n\\
                &\geq \frac{n}{n-1}\cdot\pair{1-\frac{n}{n^2}} = \frac{n}{n-1}\cdot\pair{1-\frac{1}{n}}\\
                &= 1 \impl a_n > a_{n-1} \quad \forall n\geq 2\\
                ???\\
                \intertext{Mit dem Satz der monotonen Konvergenz folgt:}
                \impl a_1 < a_2 < \dots < a_n < b_n < b_{n+1} < \dots < b_1\\
                \impl e\definedas \lim a_n = \lim \pair{1+\frac{1}{n}}^n\text{ existiert}\\
                b\definedas \lim b_n\text{ existiert, sogar $b=e$}
                \text{Da } 1 < \frac{b_n}{a_n} = \frac{\pair{1+\frac{1}{n}}^{n-1}}{\pair{1+\frac{1}{n}}^{n}} = 1 + \frac{1}{n} \fromto 1\\
                1 &= \lim \frac{b_n}{a_n} = \frac{\lim b_n}{\lim a_n} = \frac{b}{e} \impl b = e\\
                \pair{1+\frac{1}{n}}^n < e < \pair{1-\frac{1}{n}}^{n-1}
            \end{align*}
        \end{proof}
    \end{beispiel}

    \newpage

    \subsection{Häufungswerte und Teilfolgen}

    Wir möchten eine Folge $(a_n)_n$ \anf{massieren}.
    \begin{align*}
        a_n &= f(n), \quad f: \naturalnumbers \fromto \realnumbers
    \end{align*}

    \noindent Wir wollen die Folgeglieder umordnen oder auch beliebige weglassen. Wie machen wir das und wie lässt sich das ausdrücken?

    \begin{definition}[Umordnung]
        Sei $(a_n)_n$ eine reelle Folge. Eine Umordnung ist gegeben durch eine Bijektion
        \begin{align*}
            \sigma: \naturalnumbers\fromto \naturalnumbers\\
            b_n \definedas a_{\sigma(n)} \tag{Umordnung von $a_n$}
        \end{align*}
    \end{definition}

    \begin{definition}[Ausdünnung]
        Es sei $\kappa: \naturalnumbers\fromto\naturalnumbers$ streng monoton wachsend
        \begin{align*}
            b_n \definedas a_{\kappa(n)}\tag{Teilfolge von $a_n$}
        \end{align*}
    \end{definition}

    \begin{satz}[Konvergenz von Teilfolgen und Umordnungen]
        \label{satz:konv-teilfolgen-umordnungen}
        Für jede konvergente reelle Folge $(a_n)_n$ konvergiert jede Umordnung und jede Teilfolge gegen den selben Grenzwert.

        \begin{proof}[Beweis\footnotemark.]
            \footnotetext{Nachtrag vom 28. November 2023.}
            Sei $a_n$ konvergent gegen $a$ und $\kappa: \naturalnumbers\fromto\naturalnumbers$ monoton wachsend ($\kappa(n+1) > \kappa(n)$).\\
            $\impl \kappa(j) \geq j\quad\forall j \in\naturalnumbers$\footnotemark\\
            \footnotetext{Lässt sich per Induktion nachweisen}
            \begin{align*}
                b_j &\definedas a_{\kappa(j)}
                \intertext{$a_n\fromto a$, das heißt}
                \forall \varepsilon > 0~\exists N\in\naturalnumbers&\colon a-\varepsilon < a_n < a + \varepsilon\quad\forall n \geq N\\
                \impl \forall j\geq N&\colon a-\varepsilon < a_{\kappa(j)} < a + \varepsilon\quad\forall j \geq N\\[10pt]
                \impl \lim_{j\fromto\infty} a_{\kappa(j)} &= a\quad\text{ d.h. } \lim_{j\fromto\infty} b_j = a
            \end{align*}
            Für Umordnung: Sei $\sigma: \naturalnumbers\fromto\naturalnumbers$ Bijektion.
            \begin{align*}
                b_j &\definedas a_{\sigma(j)}\\
                \intertext{Wir haben $\forall \varepsilon > 0~\exists N\in\naturalnumbers\colon a-\varepsilon < a_n < a+\varepsilon$ und betrachten das Urbild}
                A&\definedas \sigma^{-1}\pair{\set{1,2,3, \dots, N}}\subseteq \naturalnumbers\\
                \intertext{$A$ hat endlich viele Elemente}
                L&\definedas \max\pair{\sigma^{-1}(1), \sigma^{-1}(2), \dots, \sigma^{-1}(N)}\\
                j \geq L &\impl \sigma(j) \geq N\\
                &\impl \forall j\geq L\colon a-\varepsilon < a_{\sigma(j)} < a + \varepsilon\\
                &\impl \lim_{j\fromto\infty} a_{\sigma(j)} = a\qedhere
            \end{align*}
        \end{proof}
    \end{satz}

    \begin{uebung}
        Weisen Sie nach, dass sich der Grenzwert einer Folge nicht verändert, wenn man endlich viele Elemente ändert.
    \end{uebung}

    \begin{definition}[Häufungswert]
        Sei $(a_n)_n$ reelle Folge. Eine reelle Zahl $a$ heißt Häufungswert (oder Häufungspunkt) von $a_n$, falls $\forall \varepsilon >0$ unendlich viele $a_n$ in $\pair{a-\varepsilon, a+\varepsilon}$ liegen. Das heißt:
        \begin{align*}
            a-\varepsilon < a_n < a+ \varepsilon\text{ für unendlich viele $n$}
        \end{align*}
        Das heißt $\forall L\in\naturalnumbers~\exists n> L\colon a-\varepsilon < a_n < a+\varepsilon$
    \end{definition}

    \begin{beispiel}
        \theoremescape
        \begin{enumerate}
            \item $a_n=\frac{1}{n}$ hat Häufungswert 0
            \item $a_n=(-1)^n$ hat Häufungswerte $1$, $-1$
            \item $a_n=n$ hat keinen Häufungswert
            \item $a_n=(-1)^n+\frac{1}{n}$ hat Häufungswerte 1, -1
        \end{enumerate}
    \end{beispiel}

    \begin{satz}[Häufungswertkriterium über Teilfolgen]
        \label{satz:haeufungswert-teilfolge}
        Eine reelle Zahl $a$ ist genau dann Häufungswert einer Folge $(a_n)_n$, wenn eine Teilfolge von $a_n$ existiert, die gegen $a$ konvergiert.

        \begin{proof}
            \anf{$\impl$}: $a$ sei Häufungswert von $a_n$. Das heißt
            \begin{align*}
                \forall \varepsilon > 0~\forall L\in\naturalnumbers~\exists n>L\colon a-\varepsilon < a_n < a+\varepsilon
            \end{align*}
            \begin{enumerate}[label=\arabic*)]
                \item Wir wählen $\varepsilon = 1 \impl \exists n_1\in\naturalnumbers\colon a-1 < a_{n_1} < a+1$
                \item Wir wählen $\varepsilon = \frac{1}{2}$, $L=n_1 + 1 \impl \exists n_2 > L > n_1\colon a-\frac{1}{2} < a_{n_2} < a + \frac{1}{2}$
                \item Wir wählen $\varepsilon = \frac{1}{3} \impl \exists n_3 > n_2\colon a - \frac{1}{3} < a_{n_3} < a + \frac{1}{3}$
                \item[$j$)] Wir wählen $\varepsilon = \frac{1}{j+1} \impl \exists n_{j+1} > n_j\colon a-\frac{1}{j+1} < a_{n_{j+1}} < a + \frac{1}{j+1}\qquad \impl n_j < n_{j+1}\quad\forall j \in \naturalnumbers$
            \end{enumerate}
            \noindent Wir definieren $\kappa\pair{j} \definedas n_j$ und $b_j \definedas a_{\kappa(j)}$ als eine Teilfolge von $(a_n)_n$. Es gilt
            \begin{align*}
                a-\frac{1}{j} < b_j < a + \frac{1}{j}\quad\forall j\in\naturalnumbers
            \end{align*}
            und nach Satz~\ref{satz:sandwich} konvergiert $(b_j)_j$ gegen $a$.\qedhere
        \end{proof}
        \begin{uebung}
            Beweisen Sie mittels Konvergenzkriterien und der Definition von Häufungswerten die Rückrichtung des vorherigen Satzes.
        \end{uebung}
    \end{satz}

    \newpage

    %%%%%%%%%%%%%%%%%%%%%%%%
    % 28. November 2023
    %%%%%%%%%%%%%%%%%%%%%%%%

    \subsection{Größter und kleinster Häufungswert - Limes superior und Limes inferior}

    Es sei $(a_n)_{n\in\naturalnumbers}$ eine beschränkte reelle Folge.
    \begin{align*}
        x_n \definedas \sup_{l\geq n} a_l
    \end{align*}
    ist monoton fallend, weil
    \begin{align*}
        \sup_{l\geq n} a_l = \max(a_n, \sup_{l\geq n+1} a_l) \geq \sup_{l\geq n+1} a_l = x_{n+1}
    \end{align*}
    $(x_n)_{n\in\naturalnumbers}$ ist monoton fallend und nach unten beschränkt.
    \begin{align*}
        \annot{\impl}{\ref{satz:monoton-konv}} x = \lim x_{n} = \lim_{n\fromto\infty} \sup_{l\geq n} a_l = \inf \sup_{l\geq n} a_l
    \end{align*}
    \noindent existiert.\\[10pt]
    Genauso: $y_n\definedas \inf_{l\geq n} a_l \impl y_n < y_{n+1}$ und $y_n$ ist nach oben beschränkt und damit existiert:
    \begin{align*}
        y = \lim y_n = \lim_{n\fromto\infty} \inf_{l\geq n} a_l = \sup \inf_{l\geq n} a_j
    \end{align*}

    \begin{definition}[Limes superior und inferior] % Definition 1
        Sei $(a_n)_{n\in\naturalnumbers}$ eine beschränkte reelle Folge.
        \begin{align*}
            \limsup_{n\fromto\infty} a_n \definedas \lim_{n\fromto\infty} \sup_{l\geq n} a_l = \inf \sup_{l\geq n} a_l \tag{Limes superior}\\
            \liminf_{n\fromto\infty} a_n \definedas \lim_{n\fromto\infty} \inf_{l\geq n} a_l = \sup \inf_{l\geq n} a_l \tag{Limes inferior}
        \end{align*}
        Damit gilt außerdem
        \begin{align*}
            y_n < x_n\quad \inf_{l\geq n} a_l &\leq \sup_{l\geq n} a_l\\
            \impl \liminf_{n\fromto\infty} a_n &\leq \limsup_{n\fromto\infty} a_n
            \intertext{und}
            \limsup_{n\fromto\infty} \pair{-a_n} &= -\liminf_{n\fromto\infty} a_n\\
            \liminf_{n\fromto\infty} \pair{-a_n} &= -\limsup_{n\fromto\infty} a_n
        \end{align*}
    \end{definition}

    \begin{beispiel}
        \begin{align*}
            a_n = (-1)^n \impl \liminf_{n\fromto\infty} a_n &= -1\\
            \limsup_{n\fromto\infty} a_n &= 1\\[10pt]
            \limsup_{n\fromto\infty} (-a_n) = - \liminf_{n\fromto\infty} a_n\\
            \liminf_{n\fromto\infty} (-a_n) = -\limsup_{n\fromto\infty} a_n
        \end{align*}
    \end{beispiel}
    \newpage

    \begin{lemma}[Charakterisierung von $\limsup$ und $\liminf$] % Lemma 2
        \label{lemma:limsup-charak}
        Sei $(a_n)_{n\in\naturalnumbers}$ beschränkte reelle Folge. Dann gilt:
        \begin{align*}
            a^{*} = \limsup a_n\quad\equivalent\quad \forall \varepsilon > 0~
            &
            \begin{array}{l}
                \text{ist } a_n < a^{*} + \varepsilon\text{ für fast alle $n$} \\
                \text{und } a_n > a^{*} -\varepsilon\text{ für unendlich viele $n$}
            \end{array}
            \\[10pt]
            a_{*} = \liminf a_n\quad\equivalent\quad \forall \varepsilon > 0~
            &
            \begin{array}{l}
                \text{ist } a_n > a_{*} - \varepsilon\text{ für fast alle $n$} \\
                \text{und } a_n < a_{*} + \varepsilon\text{ für unendlich viele $n$}
            \end{array}
        \end{align*}
        \begin{proof}[Beweis für ersten Teil des Lemmas, zweiter analog.]
            \anf{$\impl$}:
            \begin{align*}
                a^{*} = \limsup_{n\fromto\infty} a_n &= \inf_{n\in\naturalnumbers} \sup_{l\geq n} a_l\\
                \intertext{Angenommen $\exists \varepsilon > 0\colon a_l \geq a^{*} +\varepsilon$ für unendlich viele $l$}
                \impl \forall n\in\naturalnumbers\colon \sup_{l\geq n} a_l &\geq a^{*} + \varepsilon\\
                \impl a^{*} = \inf_{n\in\naturalnumbers} \sup_{l\geq n} a_l &\geq a^{*} + \varepsilon\quad\text{ (Widerspruch)}
                \intertext{Angenommen $\exists \varepsilon > 0\colon a_n \leq a^{*} -\varepsilon$ für fast alle $n$}
                \impl \sup_{l\geq n} a_l &\leq a^{*} - \varepsilon_0\\
                \impl a^{*} = \lim_{n\fromto\infty} \sup_{l\geq n} a_l &\leq a^{*} - \varepsilon_0\quad\text{(Widerspruch)}
            \end{align*}
            \anf{$\Leftarrow$}: Sei $a^{*}\in\realnumbers$
            \begin{align*}
                \forall \varepsilon > 0\colon a_n &< a^{*} + \varepsilon\text{ für fast alle $n$}\\
                a_n &> a^{*} - \varepsilon\text{ für unendlich viele $n$}\\
                \impl \exists k\in\naturalnumbers\colon a_l &< a^{*} + \varepsilon\quad\forall l\geq k\\[10pt]
                \impl \forall n\geq k\colon \sup_{l\geq n} a_l &\leq a^{*} + \varepsilon\\
                \sup_{l\geq n} a_l &> a^{*} - \varepsilon\\
                \impl \forall n \geq k\colon a^{*} - \varepsilon &< \sup_{l\geq n} a_l \leq a^{*} + \varepsilon\\
                \impl a^{*} - \varepsilon \leq \lim_{n\fromto\infty} \sup_{l\geq n} a_l &\leq a^{*} + \varepsilon\quad \forall \varepsilon > 0\\
                \impl \limsup_{n\fromto\infty} a_n &= a^{*}\qedhere
            \end{align*}
        \end{proof}
    \end{lemma}

    \begin{satz}[Eigenschaften von $\limsup$ und $\liminf$] % Satz 3
        \label{satz:limsup-haeufungspunkt}
        Sei $a_n$ eine beschränkte reelle Folge und $H(a_n)$ die Menge der Häufungspunkte von $a_n$. Dann gilt
        \begin{align*}
            \limsup a_n, \liminf a_n \in H(a_n)\tag{1}
        \end{align*}
        Insbesondere ist $H(a_n) \neq \emptyset$. Ferner ist

        \begin{align*}
            \forall x \in H(a_n): \liminf a_n \leq x \leq \limsup a_{n}\tag{2}
        \end{align*}

        \begin{proof}[Beweis von (1)]
            \begin{align*}
                a^{*} &\definedas\limsup a_n\\
                \annot{\impl}{\ref{lemma:limsup-charak}} \forall\varepsilon > 0\colon a_n &< a^* + \varepsilon\text{ für fast alle } n\\
                a_n &> a^*-\varepsilon\text{ für unendlich viele }n\\[10pt]
                \impl \forall\varepsilon > 0\colon a^* - \varepsilon &< a_n < a^* + \varepsilon\text{ für unendlich viele } n\\
                \impl &a^*\text{ ist Häufungswert von } (a_n)_n\qedhere
            \end{align*}
        \end{proof}
        \noindent Analog lässt sich der Beweis auch für $\liminf a_n$ führen.

        \begin{proof}[Beweis von (2)]
            Sei $a$ Häufungswert von $(a_n)_n$.
            Annahme: $a>a^*\definedas\limsup a_n$:
            \begin{align*}
                &\impl a-\varepsilon < a_n < a + \varepsilon\text{ für unendlich viele } n
                \intertext{Wähle $\varepsilon = \frac{a-a^*}{2}$}
                &\impl a - \frac{a-a^*}{2} < a_n < a + \frac{a-a^*}{2}\\
                &\impl a_n > a - \frac{a-a^*}{2} = \frac{a+a^*}{2} = a^* + \varepsilon\text{ für unendlich viele }n
                \intertext{Widerspruch zu Lemma~\ref{lemma:limsup-charak}}
                &\impl a \leq \limsup a_n\qedhere
            \end{align*}
        \end{proof}
        \noindent Mit $-a_n$ lässt sich analog zeigen, dass $a\geq \liminf a_n\quad\forall$ Häufungspunkte $a$ von $(a_n)_n$.
    \end{satz}

    \begin{notation}[Limes superior/inferior von unbeschränkten Folgen]
        Es sei $a_n$ nicht nach oben beschränkt. Dann setzten wir
        \begin{align*}
            \limsup a_n &\definedas +\infty
        \end{align*}
        Ist $a_n$ nicht nach unten beschränkt. Dann setzen wir
        \begin{align*}
            \liminf a_n &\definedas -\infty
        \end{align*}
    \end{notation}

    \begin{korollar}[Konvergenzkriterium nach limes superior und limes inferior]
        Eine reelle Folge $a_n$ konvergiert genau dann, wenn $a_n$ beschränkt ist und $\limsup a_n = \liminf a_n$.
        \begin{proof}
            \anf{$\impl$}:\\
            Jede konvergente Folge ist beschränkt.\\
            $a_n$ konvergiert gegen $a$ genau dann, wenn $H(a_n) = \set{a}$\\
            $\impl \liminf a_n = a = \limsup a_n$\\[10pt]
            \anf{$\Leftarrow$}: Sei $\liminf a_n = \limsup a_n = a$\\
            $\annot{\impl}{\ref{lemma:limsup-charak}} \forall \varepsilon > 0\colon a_n < a+\varepsilon \text{ für fast alle $n$}$\\
            und $a_n > a-\varepsilon $ für fast alle $n$\\
            $\impl \lim a_n = a$\qedhere
        \end{proof}
    \end{korollar}

    \begin{uebung}
        Zeigen Sie: Wenn $\liminf$ und $\limsup$ als reelle Zahlen existieren, dann ist die Folge beschränkt.
    \end{uebung}

    \begin{satz}[Satz von Bolzano-Weierstraß] % Satz 5
        \label{satz:bolzano-weierstrass}
        Jede beschränkte reelle Folge $a_n$ besitzt mindestens einen Häufungspunkt.

        \begin{proof}
            $a^* \definedas \limsup a_n$ ist ein Häufungspunkt von $a_n$ nach Satz~\ref{satz:limsup-haeufungspunkt}.
        \end{proof}
    \end{satz}

    \begin{korollar}
        Jede beschränkte reelle Folge $a_n$ hat eine konvergente Teilfolge.
        \begin{proof}
            Nach Satz~\ref{satz:bolzano-weierstrass} ist $H(a_n) \neq\emptyset$ und nach Satz~\ref{satz:haeufungswert-teilfolge} gibt es zu jedem Häufungspunkt eine konvergente Teilfolge.
        \end{proof}
    \end{korollar}

    \newpage

    \subsection{Das Konvergenzkriterium von Cauchy}

    \begin{definition}[Cauchy-Folge]
        Eine reelle Folge $a_n$ heißt Cauchy-Folge, falls
        \begin{align*}
            &\forall\varepsilon > 0~\exists N\in\naturalnumbers\colon \abs{a_n - a_m} < \varepsilon\quad\forall n,m \geq N\\
            (\equivalent &\forall \varepsilon > 0~\exists N\in\naturalnumbers\colon \abs{a_n - a_m} < \varepsilon\quad\forall n\geq m\geq N)
        \end{align*}
    \end{definition}

    \begin{lemma} % Lemma 2
        \label{lemma:konv-cauchy}
        Jede konvergente reelle Folge $a_n$ ist eine Cauchy-Folge.
        \begin{proof}
            Es sei $(a_n)_n\fromto a$ eine reelle Folge.
            \begin{align*}
                \forall \varepsilon > 0~\exists N\in\naturalnumbers\colon &\abs{a_n-a} < \frac{\varepsilon}{2}\quad\forall n\geq N\\
                \impl\text{ Sei }n,m\geq N \impl &\abs{a_n-a_m} = \abs{a_n-a+a-a_m}\\
                \leq &\abs{a_n-a} + \abs{a-a_m} < \frac{\varepsilon}{2} + \frac{\varepsilon}{2} = \varepsilon\qedhere
            \end{align*}
        \end{proof}
    \end{lemma}

    \begin{lemma} % Lemma 3
        \label{lemma:beschr-cauchy}
        Jede reelle Cauchy-Folge ist beschränkt.
        \begin{proof}
            Es sei $\varepsilon = 1$
            \begin{align*}
                \impl \exists N\colon \abs{a_n-a_m} &< 1\quad \forall n,m\geq N\\
                \impl \forall n,m\geq N\colon\abs{a_n} &= \abs{a_n-a_m+a_m}\\
                &\leq \abs{a_n-a_m} + \abs{a_m}\\
                &< 1 + \abs{a_m}
                \intertext{Da wir $m=N$ wählen können, gilt}
                \impl \forall n\geq N\colon \abs{a_n} &< 1 + \abs{a_N}
                \intertext{Es gibt eine Schranke ab dem $N$-ten Folgenglied und damit gilt}
                \impl \abs{a_n} &\leq \max\pair{\abs{a_1}, \abs{a_2},\dots, \abs{a_{N-1}}, 1+\abs{a_N}}\qedhere
            \end{align*}
        \end{proof}
    \end{lemma}

    \begin{lemma} % Lemma 4
        \label{lemma:cauchy-konv-teilfolge}
        Eine reelle Cauchy-Folge $a_n$ konvergiert genau dann, wenn sie eine konvergente Teilfolge hat.


        %%%%%%%%%%%%%%%%%%%%%%%%
        % 30. November 2023
        %%%%%%%%%%%%%%%%%%%%%%%%

        \begin{proof}
            ~\\
            \anf{$\impl$}: Klar, weil nach Satz~\ref{satz:konv-teilfolgen-umordnungen} jede Teilfolge einer konvergenten Folge konvergiert.\\[10pt]
            \anf{$\Leftarrow$}: Sei $b_j$ eine Teilfolge von $a_n$ mit $b_j = a_{n_j}$ und $n_1 < n_2 < \dots < n_j < n_{j+1} < \dots$\\
            Es sei $a\definedas \lim b_j$. Behauptung: $\lim a_n = a$\\[10pt]
            \noindent Wir wissen:
            \begin{align*}
                \forall\varepsilon>0~\exists N_1 \in\naturalnumbers&\colon \abs{a-a_{n_j}} < \frac{\varepsilon}{2}\quad\forall j\geq N_1\\
                \exists N_2 \in\naturalnumbers&\colon \abs{a_n-a_m} < \frac{\varepsilon}{2}\quad\forall m\geq n\geq N_2\\[10pt]
                \impl \abs{a_n-a} &= \abs{a_n-a_{n_j} + a_{n_j} - a}\\
                &\leq \abs{a_n-a_{n_j}} + \abs{a_{n_j}-a}
                \intertext{Wir wählen $j\geq\max\pair{N_1,N_2}$}
                \impl \abs{a_n-a} &\leq \abs{a_n-a_{n_j}} + \abs{a_{n_j}-a}\\
                &<\frac{\varepsilon}{2} + \frac{\varepsilon}{2} = \varepsilon \quad\forall n\geq N\\
                \impl \lim_{n\fromto\infty} a_n &= a\qedhere
            \end{align*}
        \end{proof}
    \end{lemma}

    \begin{satz} % Satz 5
        \label{satz:jede-konv-cauchy}
        Jede reelle Folge konvergiert genau dann, wenn sie eine Cauchy-Folge ist.

        \begin{proof}
            \anf{$\impl$}: Folgt direkt aus Lemma~\ref{lemma:konv-cauchy}\\[10pt]
            \anf{$\Leftarrow$}: Sei $(a_n)_{n\in\naturalnumbers}$ eine Cauchy-Folge $\annot{\impl}{Lemma~\ref{lemma:beschr-cauchy}} (a_n)_n$ ist beschränkt $\annot{\impl}{Satz~\ref{satz:bolzano-weierstrass}} (a_n)_n$ hat eine konvergente Teilfolge $\annot{\impl}{Lemma~\ref{lemma:cauchy-konv-teilfolge}} (a_n)_n$ ist konvergent
        \end{proof}
    \end{satz}

    \newpage


    \section{[*] Dichtheit von $\mathbb{Q}$ in $\mathbb{R}$}

    Wir kennen bereits folgende Mengen:

    \begin{align*}
        \mathbb{Q}&\definedas\set{\frac{m}{n}\middle|~m\in\mathbb{Z} \land n\in\naturalnumbers}\tag{Rationale Zahlen}\\
        \mathbb{Q}&\subseteq \realnumbers\quad \mathbb{Q}\neq\mathbb{R}\text{, da } \sqrt{2}\notin \mathbb{Q}\\
        \mathbb{R}&\exclude\mathbb{Q} \tag{Irrationale Zahlen}
    \end{align*}

    \begin{bemerkung}[Größenvergleich der irrationalen und rationalen Zahlen]
        $\realnumbers\exclude\mathbb{Q}$ ist sehr viel größer als $\mathbb{Q}$. (Wird später noch behandelt)
    \end{bemerkung}

    \begin{definition}[Dichte Teilmenge] % Definition 1
        Sei $A\subseteq B (\subseteq \realnumbers)$. $A$ heißt dicht in $B$, falls
        \begin{align*}
            \forall b\in B~\exists (a_n)_{n\in\naturalnumbers}, (a_n \in A~\forall n\in\naturalnumbers)\text{ mit }\lim_{n\fromto\infty} a_n = b
        \end{align*}
        Das heißt für jedes $b\in B$ existiert eine Folge in $A$, die gegen $b$ konvergiert.
    \end{definition}

    \begin{notation}
        Statt $\forall n\in\naturalnumbers\colon a_n \in A$ schreiben wir auch $(a_n)_n\subseteq A$.
    \end{notation}
    \horizontalline
    \noindent Wir wollen nun zeigen, dass $\mathbb{Q}$ dicht in $\realnumbers$ ist. Ziel:
    \begin{align*}
        \forall x\in\realnumbers~\exists (a_n)_{n\in\naturalnumbers}, a_n\in\mathbb{Q}\text{ mit }\lim_{n\fromto\infty} a_n =x
    \end{align*}

    \begin{lemma}[Zwischenwerte von reellen Zahlen] % Lemma 2
        \label{lemma:zwisch-reelle-zahlen}
        \theoremescape
        \begin{enumerate}[label=(\roman*)]
            \item $\forall x,y\in\realnumbers$ mit $y-x>1~\exists m\in\mathbb{Z}$ mit $x < m < y$
            \item $\forall x,y\in\realnumbers$ mit $y>x~\exists q\in\mathbb{Q}\colon x<q<y$
        \end{enumerate}
        \begin{proof}
            \theoremescape
            \begin{enumerate}[label=(\roman*)]
                \item Sei $y>x+1$. Wir definieren
                \begin{align*}
                    A&\definedas\set{p\in\mathbb{Z}|~p>x} \subseteq \mathbb{Z}
                    \intertext{$A$ ist nach unten beschränkt und nach Satz~\ref{satz:von-archimedes} gilt $A\neq \emptyset$. Nach Satz~\ref{satz:wohlordnungsprinzip} erweitert auf $\mathbb{Z}$ folgt}
                    \impl m&\definedas\min(A)\text{ existiert,}\quad m\in A\\
                    \impl m-1&\notin A,\quad m-1\in \mathbb{Z}\\
                    \impl x &< m,\quad m-1 \leq x\\
                    \impl m&\leq x+1< x+(y-x) = y\\
                    \impl x&<m<y
                \end{align*}
                \item Sei $y>x\equivalent y-x>0$
                \begin{align*}
                    \annot{\impl}{\ref{satz:von-archimedes}} \exists n\in\naturalnumbers\colon n\cdot\pair{y-x}&>1\\
                    \annot{\impl}{(i)} \exists m\in \mathbb{Z}\colon nx &< m < ny\\
                    \equivalent x &< \frac{m}{n}< y\\
                    \text{Wähle }q&=\frac{m}{n}\qedhere
                \end{align*}
            \end{enumerate}
        \end{proof}
    \end{lemma}

    \begin{folgerung}[Dichtheit von $\mathbb{Q}$ in $\realnumbers$]
        Anwendung: $\forall x\in\realnumbers~\exists (a_n)_n\in\mathbb{Q}$ mit $q_n\fromto x$
        \begin{proof}
            Für $n\in\naturalnumbers$ wähle $y=x+\frac{1}{n}$
            \begin{align*}
                \annot{\impl}{Lemma~\ref{lemma:zwisch-reelle-zahlen}} \exists q_n\in\mathbb{Q}\colon x &< q_n < y = x + \frac{1}{n}\\
                x &< q_n < x + \frac{1}{n}\quad (\fromto x)
                \intertext{Nach dem Sandwich-Satz~(\ref{satz:sandwich}) gilt}
                \impl q_n &\fromto x
            \end{align*}
        \end{proof}
    \end{folgerung}

    \begin{lemma} % Lemma 3
        $\forall x\in\realnumbers~\forall n\in\naturalnumbers~\exists m_n\in\mathbb{Z}\colon$
        \begin{align*}
            \frac{m_n-1}{n} \leq x < \frac{m_n}{n}
        \end{align*}
        \begin{uebung}
            Beweisen Sie das vorherige Lemma.\\
            \textit{Hinweis}: Betrachte $A_n\definedas\set{p\in\mathbb{Z}|~p>nx}\subseteq\mathbb{Z}$. Behauptung: $m_n\definedas \min A_n$ ermöglicht den Beweis.
        \end{uebung}
    \end{lemma}

    \begin{bemerkung}[Siehe Walter: Analysis 1, Kapitel 3.8 und 4.8]
        Sei $a > 0$
        \begin{align*}
            \impl\forall n\in\naturalnumbers\colon a^{\frac{1}{n}} = \sqrt[n]{a}\text{ existiert}
            \intertext{Da $a^n \definedas \prod_{j=1}^n a\quad a^{-n} \definedas \frac{1}{a^n}\quad a^0 \definedas 1$ für $n\in\naturalnumbers$ gilt}
            a^m\text{ definiert }\forall m\in\mathbb{Z}
        \end{align*}
        \underline{Definiere}: $m\in\mathbb{Z}$, $n\in\naturalnumbers$, $a^\frac{m}{n}\definedas \pair{a^\frac{1}{m}}^m$\\
        Check Wohldefiniertheit. Das heißt ist $q=\frac{m_1}{n_1} = \frac{m_2}{n_2}$ muss gelten
        \begin{align*}
            \pair{a^\frac{1}{n}}^{m_2} = a^\frac{m_1}{n_1} = a^\frac{m_2}{n_2} = \pair{a^\frac{1}{n_2}}^{m_2}
        \end{align*}
        Für $x\in\realnumbers$ wähle $a_n\in\mathbb{Q}$, $q_n\fromto x$ und definiere
        \begin{align*}
            a^x \definedas\lim_{n\fromto\infty} a^{q_n}
        \end{align*}
        Check: $a^x\cdot a^y = a^{x+y}$, $a^{x}b^x = (ab)^x$, $a^{-x} = \frac{1}{a^x}$ usw.
    \end{bemerkung}

    \newpage


    \section{[*] Reihen (und Konvergenz von Reihen)}

    Bedeutung von endlichen Summen ist klar.\\
    Frage: Gegeben eine reelle Folge $a_n$. Was ist $\pair{a_1 + a_2 + a_3 + \dots = \sum_{n=1}^{\infty} a_n}$?

    \subsection{[*] Konvergenz-Kriterien für Reihen}

    \begin{definition}[Reihen als Partialsummen] % Definition 1
        Das Symbol
        \begin{align*}
            \sum_{n=1}^{\infty} a_n\tag{Sei $a_n$ eine reelle Folge}
        \end{align*}
        wird folgendermaßen verwendet:

        \begin{enumerate}[label=\alph*)]
            \item Es steht für die Folge der Partialsummen:
            \begin{align*}
                s_n&\definedas \sum_{j=1}^{n} a_j\quad\forall n\in\naturalnumbers
            \end{align*}
            \item Die Reihe $\sum_{n=1}^{\infty} a_n$ konvergiert, falls der Grenzwert der Partialsummen $\lim_{n\fromto\infty} s_n$ existiert.\\
            Wir setzen
            \begin{align*}
                \sum_{n=1}^{\infty} a_n &\definedas \lim_{n\fromto\infty} s_n
            \end{align*}
            \item Konvergiert $(s_n)_n$ nicht, so heißt $\sum_{n=1}^{\infty} a_n$ divergent. Falls $s_n$ bestimmt divergiert so setzen wir
            \begin{align*}
                \sum_{n=1}^{\infty} a_n &\definedas \infty \tag{Wenn $\lim_{n\fromto\infty} s_n = \infty$}\\
                \sum_{n=1}^{\infty} a_n &\definedas -\infty \tag{Wenn $\lim_{n\fromto\infty} s_n = -\infty$}
            \end{align*}
        \end{enumerate}
    \end{definition}

    \begin{satz}[Monotone Konvergenz für Reihen] % Satz 2
        \label{satz:mont-konv-reihen}
        Sei $a_n$ eine reelle Folge mit $\forall n\colon a_n\geq 0$. Dann konvergiert die Reihe $\sum_{j=1}^{\infty} a_j$ genau dann, wenn die Folge der Partialsummen $s_n$ nach oben beschränkt ist.

        \begin{proof}
            Betrachte
            \begin{align*}
                s_{n+1} &= \sum_{j=1}^{n+1} a_j = \sum_{j=1}^n a_j + a_{n+1} \geq \sum_{j=1}^{n} a_j = s_n
            \end{align*}
            und wende Satz~\ref{satz:monoton-konv} an.
        \end{proof}
    \end{satz}

    \begin{korollar} % Korollar 3
        Für eine Reihe $\sum_{n=1}^{\infty} a_n$ mit $a_n\geq 0$ gilt entweder
        \begin{align*}
            \sum_{n=1}^{\infty} a_n < \infty\quad\text{oder}\quad\sum_{n=1}^{\infty} a_n = \infty
        \end{align*}

        \begin{proof}
            Folgt direkt aus Satz~\ref{satz:mont-konv-reihen}.
        \end{proof}
    \end{korollar}

    \begin{bemerkung}
        Oft hat man Reihen der Form
        \begin{align*}
            \sum_{n=0}^{\infty} a_n &= a_0+a_1+a_2+\dots\\
            s_n&\definedas \sum_{j=0}^{n} a_j \tag{$n\in\naturalnumbers_0$}
            \intertext{oder}
            s_n &\definedas a_0 + \sum_{j=1}^{\infty} a_j\tag{$n\in\naturalnumbers$}
            \intertext{Sofern ein Limes existiert, gilt dann}
            \sum_{n=0}^{\infty} a_n &\definedas \lim s_n\\[10pt]
            \intertext{Allgemein für $v\in\mathbb{Z}\quad a_v, a_{v+1}, a_{v+2}, \dots$}
            \sum_{n=v}^{\infty} a_n &= a_v + a_{v+1} + \dots\\
            s_n &\definedas \sum_{j=v}^{n} a_j\text{ def. Folge $(s_n)_{n\geq v}$}
        \end{align*}
    \end{bemerkung}

    \begin{beispiel}[Geometrische Folge und Reihe]
        \footnote{Wir setzen $0^0 = 1$}
        \begin{align*}
            q\neq 1 \impl \sum_{j=0}^{n} q^j = \frac{1-q^{n+1}}{1-q}\quad\forall n\in\naturalnumbers_0\\
            \text{Ist } \abs{q} < 1\colon \sum_{n=0}^{\infty} q^n\text{ konvergiert und } \sum_{n=0}^{\infty} q^n = \frac{1}{1-q}\tag{geometrische Reihe}
        \end{align*}
        Zum Beispiel $q=\frac{1}{2}$
        \begin{align*}
            \sum_{j=0}^{n} q^j &= \frac{1-q^{n+1}}{1-q}\\
            \equivalent \pair{1-q}\cdot \sum_{j=0}^{n} q^j &= 1-q^{n+1}\\
            \frac{1}{2}\sum_{j=0}^{n} \pair{\frac{1}{2}}^j &= 1 - \pair{\frac{1}{2}}^{n+1}\\
            \impl 1 - \pair{\frac{1}{2}}^{n+1} &= \sum_{j=0}^{n} \frac{1}{2}\cdot\pair{\frac{1}{2}}^j\\
            &= \sum_{j=0}^{n} \pair{\frac{1}{2}}^{j+1} = \sum_{j=1}^{n+1} \pair{\frac{1}{2}}^j
        \end{align*}
        Das heißt es sollte gelten
        \begin{align*}
            \sum_{j=1}^{n} \pair{\frac{1}{2}}^j &= 1-\pair{\frac{1}{2}}^n\quad \forall n\in\naturalnumbers
        \end{align*}

        \noindent Dass dieser Zusammenhang gelten muss, lässt sich einfach veranschaulichen. Die linke Seite der Gleichung kann als Summe über Teilflächen des Einheitsquadrats\footnote{Original: \anf{Kuchen}} visualisiert werden.\\
        Erst wird eine Hälfte, dann ein Viertel, dann ein Achtel (usw.) des Quadrats hinzugefügt. Der zurückbleibende Flächeninhalt ist immer genauso groß wie das zuletzt hinzugefügte Stück. Dieser Term wird durch den rechten Teil der Gleichung beschrieben.

        \begin{proof}[Beweis der Reihenformel]
            \begin{align*}
                s_n &\definedas \sum_{j=0}^{n} q^n\\
                q\cdot s_n &= q\cdot \sum_{j=0}^{n} q^j = \sum_{j=0}^{n} q^{j+1}\\
                &= \sum_{j=1}^{n+1} q^j\tag{Indexshift}\\
                \impl (1-q)\cdot s_n = s_n - q\cdot s_n &= \sum_{j=0}^{n}  q^j - \sum_{j=1}^{n+1} q^j\tag{Reißverschlusssumme}\\
                = q^0 - q^{n+1} &= 1 - q^{n+1}\\
                \impl s_n &= \frac{1-q^{n+1}}{1-q}\qedhere
            \end{align*}
        \end{proof}

        %%%%%%%%%%%%%%%%%%%%%%%%
        % 05. Dezember 2023
        %%%%%%%%%%%%%%%%%%%%%%%%

        \begin{proof}[Beweis der Konvergenz für $\abs{q} < 1$]
            \begin{align*}
                s_n\definedas \sum_{j=0}^{n} q^j &= \frac{1-q^{n+1}}{1-q}\\
                \lim_{n\fromto\infty} q^n &= 0 = \lim_{n\fromto\infty} q^{n+1}\tag{Weil $\abs{q} < 1$}\\
                \annot{\impl}{\ref{satz:konvergenzsaetze}} \lim s_n &= \frac{1-\lim_{n\fromto\infty} q^{n+1}}{1-q} = \frac{1-0}{1-q} = \frac{1}{1-q}\qedhere
            \end{align*}
        \end{proof}
        \begin{bemerkung}
            Ist $\abs{q}\geq 1$, dann ist $1+\underbrace{q}_{\geq 1}+\underbrace{q^2}_{\geq 1}+\underbrace{q^3}_{\geq 1}+\dots + \underbrace{q^n}_{\geq 1} \geq 1+n\fromto\infty$.
        \end{bemerkung}
    \end{beispiel}

    % TODO: Wahrscheinlichkeitstheoretische Anschauung
    \newpage

    \begin{beispiel}[Harmonische Reihe]
        \begin{align*}
            s_n &\definedas \sum_{j=1}^{n} \frac{1}{j}
        \end{align*}
        $(s_n)_n$ ist monoton wachsend, aber nicht nach oben beschränkt. Das heißt $\lim_{n\fromto\infty} s_n = \infty \impl \sum_{n=1}^{\infty} \frac{1}{n} = \infty$.

        \begin{proof}
            \begin{align*}
                s_{2n} - s_n &= \sum_{j=n+1}^{2n} \frac{1}{j} \geq \sum_{j=n+1}^{n} \frac{1}{2n} = n \cdot\frac{1}{2n} = \frac{1}{2}\\
                \impl s_2 - s_1 &\geq \frac{1}{2}\\
                \impl s_2 &\geq s_1 + \frac{1}{2} = 1 + \frac{1}{2} > \frac{1}{2}\\
                s_4 - s_2 &\geq \frac{1}{2}\\
                \impl s_4 &\geq s_2 + \frac{1}{2} > \frac{1}{2} + \frac{1}{2} = 1\\
                s_8 - s_4 &\geq \frac{1}{2}\\
                \impl s_8 &\geq s_4 + \frac{1}{2} \geq \frac{3}{2}\\
                \annot{\impl}{Induktion} s_{\pair{2^j}} &> \frac{j}{2}\quad\forall j\in\naturalnumbers
            \end{align*}
            \noindent Also ist $s_{\pair{2^j}}$ nicht nach oben beschränkt $\impl$ $(s_n)_n$ nicht nach oben beschränkt.
            \begin{align*}
                \annot{\impl}{\ref{satz:monoton-konv}} \sum_{n=1}^{\infty} \frac{1}{n} = \lim_{n\fromto\infty} s_n &= +\infty\qedhere
            \end{align*}
        \end{proof}
    \end{beispiel}

    \begin{satz} % Satz 6
        Seien $\sum_{n=1}^{\infty} a_n$, $\sum_{n=1}^{\infty} b_n$ konvergente Reihen. Dann ist
        \begin{align*}
            \forall\lambda, \mu \in\realnumbers\colon \sum_{n=1}^{\infty} \pair{\lambda\cdot a_n + \mu\cdot b_n}
        \end{align*}
        konvergent und es gilt
        \begin{align*}
            \sum_{n=1}^{\infty} \pair{\lambda\cdot a_n + \mu\cdot b_n} = \lambda \cdot \sum_{n=1}^{\infty} a_n + \mu\cdot \sum_{n=1}^{\infty} b_n
        \end{align*}
        \begin{proof}
            \begin{align*}
                s_n &\definedas \sum_{j=1}^{n} a_n \quad t_n \definedas \sum_{j=1}^{n} b_n\\
                d_n &\definedas \sum_{j=1}^{n} \pair{\lambda\cdot a_n + \mu\cdot b_n} =   \lambda \cdot \sum_{n=1}^{n} a_n + \mu\cdot \sum_{n=1}^{n} b_n\\
                &= \lambda\cdot s_n + \mu\cdot t_n \fromto \lambda\cdot s + \mu\cdot t\tag{Mit $s$ und $t$ als Limes}
            \end{align*}
        \end{proof}
    \end{satz}

    \newpage

    \begin{satz}[Majoranten-Kriterium] % Satz 7
        \label{satz:majoranten-kriterium}
        Gegeben zwei Folgen $0\leq a_n \leq b_n~\forall n\in\naturalnumbers$. Konvergiert
        \begin{align*}
            \sum_{n=1}^{\infty} b_n\quad\text{so konvergiert auch}\quad\sum_{n=1}^{\infty} a_n
        \end{align*}
        und es gilt
        \begin{align*}
            0\leq \sum_{n=1}^{\infty} a_n \leq \sum_{n=1}^{\infty} b_n
        \end{align*}
        \begin{proof}
            \begin{align*}
                &s_n = \sum_{j=1}^{n} a_j\quad t_n = \sum_{j=1}^{n} b_j\\
                \impl &s_{n+1} = s_n + a_{n+1} \geq s_n\\
                &t_{n+1} = t_n + b_{n+1} \geq t_n\\
                \impl &(s_n)_n,~(t_n)_n\text{ sind monoton wachsend}\\
                \intertext{Mit der Konvergenz von $(t_n)_n$ und $(t_n)_n$ monoton wachsend folgt mit Satz~\ref{satz:mont-konv-reihen}}
                \impl &(t_n)_n\text{ ist beschränkt}\\
                \impl &\exists M\geq 0\colon t_n \leq M\quad\forall n\in\naturalnumbers\\
                0\leq a_n \leq b_n \impl &0\leq s_n \leq t_n \leq M \quad\forall n\in\naturalnumbers\\
                \impl &(s_n)_n\text{ ist nach oben beschränkt und monoton wachsend}\\
                \annot{\impl}{\ref{satz:mont-konv-reihen}} &\lim_{n\fromto\infty} s_n\text{ existiert }\\
                \quad s &\definedas \lim_{n\fromto\infty} s_n \leq \lim_{n\fromto\infty} t_n = \sum_{n=1}^{\infty} b_n\qedhere\\
                \intertext{Außerdem gilt:}
                t_n - s_n &= \sum_{j=1}^{n} b_j - \sum_{j=1}^{n} a_j = \sum_{j=1}^{n} n \pair{b_j-a_j} \geq 0
            \end{align*}
        \end{proof}
    \end{satz}

    \begin{satz}[Minoranten-Kriterium] % Satz 8
        Sei $0\leq b_n \leq a_n~\forall n\in\naturalnumbers$ und
        \begin{align*}
            \sum_{n=1}^{\infty} b_n &= \infty\\
            \impl \sum_{n=1}^{\infty} a_n&\text{ divergiert auch bestimmt gegen }\infty
        \end{align*}
        \begin{proof}
            \begin{align*}
                t_n &= \sum_{j=1}^{n} b_n \quad s_n = \sum_{j=1}^{n} a_n
                \intertext{Analog zum Beweis des Majoranten-Kriteriums gilt:}
                (s_n)_n,~(t_n)_n&\text{ sind monoton wachsend und }t_n \leq s_n \quad\forall n\in\naturalnumbers
                \intertext{Dann lässt sich folgern}
                \sum_{n=1}^{\infty} b_n = \infty &\equivalent (t_n)_n \text{ wächst über alle Grenzen}\\
                &\impl (s_n)_n\text{ wächst über alle Grenzen}\\
                &\impl \lim_{n\fromto\infty} s_n = \infty = \sum_{n=1}^{\infty} a_n\qedhere
            \end{align*}
        \end{proof}
    \end{satz}

    \begin{beispiel}[Anwendung des Minoranten-Kriteriums]
        Es sei
        \begin{align*}
            a_n\geq \frac{c}{n}\quad\forall n\in\naturalnumbers\tag{$c>0$}
            \intertext{Nach Minorantenkriterium und der Divergenz der harmonischen Reihe gilt}
            \impl \sum_{n=1}^{\infty} a_n = \infty
        \end{align*}
    \end{beispiel}
    \begin{beispiel}[Anwendung des Majoranten-Kriteriums]
        Es sei wieder $c>0$. Dann folgt aus
        \begin{align*}
            0\leq a_n \leq c\cdot q^n\land 0\leq q < 1
            \intertext{nach Majoranten-Kriterium und dem Konvergenzkriterium der geometrischen Reihe, dass}
            \impl \sum_{n=1}^{\infty} a_n\text{ konvergiert}\tag{$b_n=c\cdot q^n$}
        \end{align*}
    \end{beispiel}

    \begin{bemerkung}[Abgeschwächtes Majoranten-Kriterium]
        Die Konvergenz/Divergenz von Reihen (und Folgen) ändert sich nicht, wenn man endlich viele Summanden (Folgeglieder) abändert.\\
        Für das Majoranten-Kriterium reicht also, dass $0\leq a_n \leq b_n$ für fast alle $n\in\naturalnumbers$, damit
        \begin{align*}
            \sum_{n=1}^{\infty} b_n\text{ konvergiert }\impl \sum_{n=1}^{\infty} a_n\text{ konvergiert}
        \end{align*}
        Das gleiche gilt analog für das Minoranten-Kriterium
    \end{bemerkung}

    \newpage

    \begin{satz}[Cauchyscher Verdichtungssatz] % Satz 9
        \label{satz:cauchy-verdichtung}
        Sei $(a_n)_n$ eine monoton fallende Nullfolge. Dann gilt
        \begin{align*}
            &\sum_{n=0}^{\infty} a_n\text{ konvergiert}\\
            \equivalent &\sum_{n=0}^{\infty} 2^n\cdot a_{\pair{2^n}}\text{ konvergiert}\tag{Verdichtete Reihe}
        \end{align*}
        \begin{proof}
            \anf{$\Leftarrow$} 1. Schritt. Zu zeigen: $a_n\geq 0\quad\forall n\in\naturalnumbers$.
            \begin{align*}
                a_n&\geq a_{n+1} \geq a_{n+2} \geq \dots \geq a_{n+l} \fromto 0\text{ für } l\fromto\infty\\
                \impl a_n &\geq 0\quad\forall n\in\naturalnumbers
            \end{align*}
            2. Schritt:
            \begin{align*}
                s_n &\definedas \sum_{j=0}^{n} a_j\\
                t_n &\definedas \sum_{\nu=0}^{n} 2^{\nu} \cdot a_{\pair{2^\nu}}
                \intertext{Jedes $\overline{n}\in\naturalnumbers$ können wir eindeutig schreiben als $\overline{n}=2^\nu + l$ mit $\nu\in\naturalnumbers_0,~0\leq l < 2^\nu$~\footnotemark.\endgraf \noindent Sei $1\leq n< 2^k$ für ein $k\in\naturalnumbers_0$}
                s_n &= \sum_{j=0}^{n} a_j = a_0 + \sum_{j=1}^{n} a_j\\
                &\leq a_0 + \sum_{j=1}^{2^k-1} a_j = a_0 + \sum_{\nu=0}^{k-1} \pair{\sum_{l=0}^{2^\nu - 1} \underbrace{a_{\pair{2^\nu + l}}}_{\leq a_{\pair{2^\nu}}}}\\
                &\leq a_0 + \sum_{\nu=0}^{k-1} \pair{\sum_{l=0}^{2^\nu-1} a_{\pair{2^\nu}}}\\
                &= a_0 + \sum_{\nu = 0}^{k-1} 2^{\nu}\cdot a_{\pair{2^\nu}}\\
                &= a_0 + t_{k-1}\\[10pt]
                \impl &\forall n < 2^k\text{ gilt }s_n\leq a_0 + t_{k-1}
            \end{align*}
            \footnotetext{Es lässt sich zeigen, dass $l$ und $\nu$ in diesem Fall eindeutig sind.}
            Angenommen
            \begin{align*}
                \sum_{\nu = 0}^{\infty} 2^{\nu} \cdot a_{2^\nu}\text{ konvergent} &\equivalent \lim_{n\fromto\infty} t_n = t\text{ existiert}\\
                &\impl s_n \leq a_0 + \lim_{k\fromto\infty} t_{k-1} = a_0 + t\quad\forall n\in\naturalnumbers
                \intertext{Somit ist $a_0 + l$ eine obere Schranke von $(s_n)_n$. Da $s_n\leq s_{n+1} \annot{\impl}{\ref{satz:monoton-konv}} \lim_{n\fromto\infty} s_n$ existiert}
                \impl \sum_{n=0}^{\infty} a_n&\text{ konvergent}
            \end{align*}
            \anf{$\impl$} Sei $n\geq 2^k$
            \begin{align*}
                s_n &= \sum_{j=0}^{n} a_j \geq \sum_{j=0}^{2^k} a_j\\
                &= \sum_{\nu=0}^{k} \pair{\sum_{l=0}^{2^\nu-1} \underbrace{a_{\pair{2^\nu + l}}}_{\geq a_{\pair{2^\nu+1}}}}\\
                &\geq \sum_{\nu=0}^{k} \pair{\sum_{l=0}^{2^\nu-1} a_{\pair{2^\nu+1}}} = \sum_{\nu=0}^{k} 2^{\nu}\cdot a_{\pair{2^\nu}} + 1\\
                &= \sum_{\nu = 1}^{k+1} 2^{\nu-1} a_{\pair{2^\nu}} = \frac{1}{2}\sum_{\nu=1}^{k+1} 2^{\nu} a_{\pair{2^\nu}}\\
                &= \frac{1}{2}\pair{\sum_{\nu=0}^{k+1} 2^\nu a_{\pair{2^\nu}} - a_0}\\
                &= t_{k+1} - a_0\\
                \impl t_k \leq t_{k+1} &\leq s_n + a_0\quad\forall 2^k \geq n\\
                \impl t_k &\leq \lim_{n\fromto\infty} s_n + a_0 = s + a_0 < \infty\\
                \text{sofern} \sum_{n=0}^{\infty} a_0\text{ konvergiert}
            \end{align*}
            Da $t_k < t_{k+1}$ konvergiert, konvergiert auch $\sum_{\nu=0}^{\infty} 2^\nu \cdot a_{2^\nu}$
        \end{proof}
    \end{satz}

    \begin{beispiel}[Cauchyscher Verdichtungssatz als Konvergenzkriterium]
        Es sei
        \begin{align*}
            a_n &= \frac{1}{n^\alpha}\\
            2^{n} \cdot a_{\pair{2^\nu}} &= \frac{2^n}{(2^n)^\alpha} = \frac{2^n}{2^{n\cdot\alpha}}\\
            &= 2^{n-n\cdot\alpha} = 2^{(1-\alpha)\cdot n} = \pair{2^{1-\alpha}}^n = q^n
            \intertext{Damit $q^n$ und damit auch $(a_n)_n$ konvergiert muss wie bereits gezeigt gelten}
            q &\definedas 2^{1-\alpha} < 1 \equivalent 1-\alpha < 0 \equivalent \alpha > 1
            \intertext{Es sei}
            a_n &= \frac{1}{n}\\
            2^n \cdot a_{\pair{2^\nu}} &= 2^n \cdot \frac{1}{2^n} = 1\impl (a_n)_n\text{ konvergiert}
        \end{align*}
    \end{beispiel}

    %%%%%%%%%%%%%%%%%%%%%%%%
    % 7. Dezember 2023
    %%%%%%%%%%%%%%%%%%%%%%%%

    \begin{definition}[Alternierende Reihe] % Definition 10
        Sei $(a_n)_n$ eine reelle Folge nicht-negativer reeller Zahlen. Dann heißt
        \begin{align*}
            \sum_{n=1}^{\infty} (-1)^{n+1} \cdot a_n = a_1 - a_2 + a_3 - a_4 \dots
        \end{align*}
        alternierende Reihe. Alternativ $a_n\geq 0$, $n\in\naturalnumbers_0$.
        \begin{align*}
            \sum_{n=0}^{\infty} (-1)^n\cdot a_n = a_0 - a_1 + a_2 - a_3 \dots
        \end{align*}
    \end{definition}

    \begin{satz}[Leibniz-Kriterium] % Satz 11
        Sei $(a_n)_n$ eine monoton fallende Nullfolge. Dann konvergiert
        \begin{align*}
            \sum_{n=1}^{\infty} (-1)^{n+1} \cdot a_n
        \end{align*}

        \begin{proof}
            Idee: Wir unterscheiden zwischen geraden und ungeraden $n$.
            \begin{align*}
                s_n &\definedas \sum_{j=1}^{n} (-1)^{j+1} \cdot a_j\\
                s_{2(n+1)} &= \sum_{j=1}^{2n+2} (-1)^{j+1} \cdot a_j\\
                &= \sum_{j=1}^{2n} (-1)^{j+1} \cdot a_j + (-1)^{(2n+1)+1} \cdot a_{2n+1} + (-1)^{(2n+2)+1} \cdot a_{2n+2}\\
                &= s_{2n} + \underbrace{a_{2n+1} - a_{2n+2}}_{\geq 0}\\
                &\geq s_{2n}
                \intertext{Also ist $(s_{2n})_n$ monoton wachsend}
                s_{2(n+1)+1} &= s_{2n+3} = \sum_{j=1}^{2n+3} (-1)^{j+1} \cdot a_j\\
                &= s_{2n+1} + (-1)^{(2n+2)+1} \cdot a_{2n+2} + (-1)^{(2n+3)+1} \cdot a_{2n+3}\\
                &= s_{2n+1} \underbrace{- a_{2n+2} + a_{2n+3}}_{\leq 0}\\
                &\leq s_{2n+1}
            \end{align*}
            Also ist $(s_{2n+1})_n$ ist monoton fallend und
            \begin{align*}
                s_{2n+1} - s_{2n} &= \sum_{j=1}^{2n+1} (-1)^{j+1} \cdot a_j - \sum_{j=1}^{2n} (-1)^{j+1}\cdot a_j\\
                &= (-1)^{(2n+1)+1} \cdot a_{2n+1} = a_{2n+1} \geq 0
                \intertext{$\impl \abs{s_{2n+1}-s_{2n}} = a_{2n+1}$ und $s_{2n+1} \geq s_{2n}$}
                0 \leq a_1 - a_2 &= s_2 \leq s_{2n} \leq s_{2n+1} \leq s_1 = a_1\\
                \annot{\impl}{\ref{satz:monoton-konv}} s_g &\definedas \lim_{n\fromto\infty} s_{2n}\text{ existiert}\\
                s_u &\definedas \lim_{n\fromto\infty} s_{2n-1}\text{ existiert}\\
                \text{und } s_g &= s_u\\
                \impl (s_n)_n&\text{ konvergiert gegen $s=s_g=s_u$}\qedhere
            \end{align*}
        \end{proof}
        \begin{uebung}
            Weisen Sie nach, dass eine Folge konvergiert, wenn die Teilfolgen der geraden und der ungeraden Folgeglieder konvergieren und die Differenz eine Nullfolge ist.
        \end{uebung}
    \end{satz}

    \begin{beispiel}
        \begin{align*}
            &\sum_{n=1}^{\infty} \pair{-1}^{n+1} \cdot \frac{1}{\sqrt{n}}\text{ konvergiert}\\
            \text{aber } &\sum_{n=1}^{\infty} \frac{1}{\sqrt {n}} \text{ divergiert}\\
            &\sum_{n=1}^{\infty} (-1)^{n} \cdot \frac{1}{n}\text{ konvergiert}\\
            &\sum_{n=2}^{\infty} (-1)^n \cdot \frac{1}{\log\pair{n}}\text{ konvergiert}
        \end{align*}
    \end{beispiel}

    \begin{satz}[Cauchy-Kriterium] % Satz 12
        \label{satz:cauchy-kriterium}
        Sei $(a_n)_n$ eine Folge reeller Zahlen. Dann konvergiert
        \begin{align*}
            \sum_{n=1}^{\infty} a_n
        \end{align*}
        genau dann, wenn
        \begin{align*}
            \forall \varepsilon > 0~\exists N_{\varepsilon}\in\naturalnumbers\colon \abs{\sum_{j=n+1}^{m} a_j} <\varepsilon\quad\forall m>n\geq N_{\varepsilon}
        \end{align*}
        \begin{proof}
            \begin{align*}
                s_n &\definedas \sum_{j=1}^{n} a_j\\
                \intertext{$(s_n)_n$ konvergiert nach Satz~\ref{satz:jede-konv-cauchy} genau dann, wenn es eine Cauchy-Folge ist. Das heißt}
                \forall\varepsilon > 0~\exists N_{\varepsilon}\in\naturalnumbers\colon &\abs{s_{m} - s_{n}} < \varepsilon\quad\forall m> n \geq N_{\varepsilon}\\
                \abs{s_m - s_n} &= \sum_{j=1}^{m} a_j = \sum_{j=1}^{n} a_j\\
                &= \sum_{j=n+1}^{m} a_j\qedhere
            \end{align*}
        \end{proof}
    \end{satz}

    \begin{korollar} % Korollar 12
        \label{korollar:folge-von-reihe-nullfolge}
        Ist die reelle Reihe $\sum_{n=1}^{\infty} a_n$ konvergent, so ist $(a_n)_n$ eine Nullfolge.
        \begin{proof}
            Nach Satz~\ref{satz:cauchy-kriterium} $\impl$
            \begin{align*}
                \forall \varepsilon > 0~\exists N_{\varepsilon}\in\naturalnumbers\colon &\abs{\sum_{j=n+1}^{n+p} a_j} < \varepsilon\quad\forall n\geq N_{\varepsilon}, p\in\naturalnumbers\\
                \intertext{Wähle $p=1$}
                \sum_{j=n+1}^{n+1} a_{j} &= a_{j+1}\\
                \impl \abs{a_{n+1}} &< \varepsilon\quad \forall n\geq N_{\varepsilon}\\
                \impl \lim_{\ntoinfty} a_{n+1} &= 0\\
                \impl \lim_{\ntoinfty} a_{n} &= 0\qedhere
            \end{align*}
        \end{proof}
    \end{korollar}

    \subsection{[*] Absolut konvergente Reihen und Umordnungen}

    \begin{definition}[Absolute Konvergenz] % Definition 1
        Eine Reihe$\sum_{n=1}^{\infty} a_n$ heißt absolut konvergent, falls
        \begin{align*}
            \sum_{n=1}^{\infty} \abs{a_n}
        \end{align*}
        konvergiert. Das heißt falls
        \begin{align*}
            \sum_{n=1}^{\infty} \abs{a_n} < \infty
        \end{align*}
    \end{definition}

    \begin{satz}[Absolute Konvergenz als Konvergenzkriterium] % Satz 2
        \label{satz:absolut-konvergenz-konvergenkriterium}
        Ist eine Folge $\sum_{n}^{\infty} a_n$ absolut konvergent, so ist sie auch konvergent und
        \begin{align*}
            \abs{\sum_{n=1}^{\infty} a_n} \leq \sum_{n=1}^{\infty} \abs{a_n}
        \end{align*}

        \begin{proof}
            Wir haben für $m>n$
            \begin{align*}
                \abs{\sum_{j=n+1}^{m} a_j} \leq \sum_{j=n+1}^{m} \abs{a_j}
            \end{align*}
            Wir nehmen an, dass $\sum_{n=1}^{\infty} \abs{a_n}$ konvergiert. Das heißt Cauchy ist erfüllt.
            \begin{align*}
                \impl \forall \varepsilon > 0~\exists N_{\varepsilon}\colon \sum_{j=n+1}^{m} \abs{a_j} &< \varepsilon \quad\forall m> n\geq N_{\varepsilon}\\
                \impl \abs{\sum_{j=n+1}^{m} a_j} &< \varepsilon\quad\forall m>n\geq N_{\varepsilon}\\
                \impl \sum_{j=1}^{\infty} a_j&\text{ konvergiert}\qedhere
            \end{align*}
        \end{proof}
    \end{satz}

    \begin{beispiel}[Konvergenz ohne absolute Konvergenz]
        \begin{align*}
            \sum_{n=1}^{\infty} (-1)^{n+1}\cdot \frac{1}{n}
        \end{align*}
        ist konvergent, aber nicht absolut konvergent.
    \end{beispiel}

    \begin{definition}[Majorante] % Def 3
        Die Reihe
        \begin{align*}
            \sum_{n=1}^{\infty} c_n\quad c_n\geq 0~\forall n\in\naturalnumbers
        \end{align*}
        ist eine Majorante der Reihe $\sum_{n}^{\infty} a_n$ falls
        \begin{align*}
            \abs{a_n} &\leq c_n\text{ für fast alle }n\in\naturalnumbers
            \intertext{Das heißt}
            \exists n_0\in\naturalnumbers\colon \abs{a_n} &< c_n\quad\forall n\geq n_0
        \end{align*}
    \end{definition}

    \begin{satz}[Majoranten-Konvergenz-Kriterium für Reihen] % Satz 4
        \label{satz:majorante-reihen}
        Hat die Reihe $\sum_{n=1}^{\infty} a_n$ eine konvergente Majorante $\sum_{n=1}^{\infty} c_n$, so ist diese Reihe absolut konvergent und somit auch konvergent.
        \begin{proof}
            Folgt aus Satz~\ref{satz:absolut-konvergenz-konvergenkriterium} und Satz~\ref{satz:majoranten-kriterium}.
        \end{proof}
    \end{satz}

    \begin{satz}[Quotientenkriterium] % Satz 5
        \label{satz:quotientenkriterium}
        Sei
        \begin{align*}
            s_n &\definedas \sum_{n=0}^{\infty} a_n
        \end{align*}
        eine Reihe mit $a_n\neq 0$. Ferner gebe es ein $0\leq q < 1$ so dass
        \begin{align*}
            \frac{\abs{a_{n+1}}}{\abs{a_n}} \leq q\text{ für fast alle } n\in\naturalnumbers\\
            \impl \sum_{n=0}^{\infty} a_n\text{ absolut konvergent}
        \end{align*}

        \begin{proof}
            Wir haben $n_0\in\naturalnumbers$
            \begin{align*}
                \frac{\abs{a_{n+1}}}{\abs{a_n}} &\leq q\quad\forall n\geq n_0\\[10pt]
                \abs{a_{n+1}} \leq q\cdot\abs{a_n} &\leq q^2 \cdot \abs{a_{n-1}} \leq q^3 \cdot \abs{a_{n-2}}\\[10pt]
                \leq \dots &\leq q^{p+1} \cdot\abs{a_{n_0}} = q^{n-n_0+1} \cdot\abs{a_{n_0}}\tag{$p\in\naturalnumbers$, $n=n_0+p$}\\[10pt]
                &= q^{n+1} \cdot q^{-n_0} \cdot\abs{a_{n_0}}\\[10pt]
                \impl \abs{a_n} &\leq \underbrace{q^n \cdot K}_{\definedasbackwards c_n}\tag{$K \definedas q^{-n_0} \cdot \abs{a_{n_0}}$}\\
                \impl \abs{a_{n_0}} &\leq c_n\quad\forall n\geq n_0\\
                \intertext{und}
                \sum_{n=0}^{\infty} c_n &= \sum_{n=0}^{\infty} K\cdot q^{n} = K\cdot \sum_{n=0}^{\infty} q^{n} < \infty
                \intertext{Da $0< q < 1$}
                \impl \sum_{n=c}^{\infty} &a_n\text{ hat die konvergente Majorante } K \cdot \sum_{n=c}^{\infty} q^{n}
            \end{align*}
            Damit lässt sich aus Satz~\ref{satz:majorante-reihen} folgern, dass die Reihe konvergiert.
        \end{proof}
    \end{satz}

    \begin{bemerkung}[Quotientenkriterium über $\limsup$]
        \theoremescape
        \begin{enumerate}[label=(\roman*)]
            \item Ist $\limsup_{n\fromto\infty} \frac{\abs{a_{n+1}}}{\abs{a_n}} < 1 \equivalent$ Quotientenkriterium
            \item Ist $\limsup_{n\fromto\infty} \frac{\abs{a_{n+1}}}{\abs{a_n}} > 1\impl \sum_{n=0}^{\infty} a_0$ divergent
            \item Ist $\limsup_{n\fromto\infty} \frac{\abs{a_{n+1}}}{\abs{a_n}} = 1\impl \text{Keine Aussage über absolute Konvergenz von } \sum_{n=0}^{\infty} a_n$ möglich
        \end{enumerate}
        \begin{proof}[Beweis (ii).]
            \begin{align*}
                \text{Ist } \overline{q}\definedas\liminf_{n\fromto\infty} \frac{\abs{a_{n+1}}}{\abs{a_n}} &> 1\\[8pt]
                \impl \forall \varepsilon > 0~\exists n_0\colon \frac{\abs{a_{n+1}}}{\abs{a_n}} \geq \overline{q} - \varepsilon = \frac{\overline{q}+1}{2} &\geq q > 1\tag{Wähle $\varepsilon=\frac{\overline{q}-1}{2} > 0$}\\[8pt]
                \impl \frac{\abs{a_n+1}}{\abs{a_n}} &\geq q > 1\quad\forall n\geq n_0
                \intertext{Wir wenden ein ähnliches Prinzip wie im vorherigen Beweis an}
                \impl \abs{a_{n+1}} &\geq q^{n+1} \cdot q^{-n_0}\cdot\abs{a_{n_0}}\\[8pt]
                \impl \abs{a_n} &\geq q^n\cdot K\tag{$K=q^{-n_0}\cdot\abs{a_{n_0}}$}\\[8pt]
                \impl \sum a_j\text{ divergiert nach}&\text{ Korollar~\ref{korollar:folge-von-reihe-nullfolge}}\qedhere
            \end{align*}
        \end{proof}
    \end{bemerkung}

    \begin{beispiel}[Divergenz bei nicht-eindeutigem Quotientenkriterium]
        $a_n \definedas \frac{1}{n}$
        \begin{align*}
            \frac{\abs{a_{n+1}}}{\abs{a_n}} = \frac{a_{n+1}}{a_n} &= \frac{n}{n+1} = 1 + \frac{1}{n} \fromto 1
            \intertext{Und}
            \sum_{n=1}^{\infty} \frac{1}{n}&\text{ divergiert (Harmonische Reihe)}
        \end{align*}
    \end{beispiel}

    \begin{beispiel}[Konvergenz bei nicht-eindeutigem Quotientenkriterium]
        $a_n \definedas \frac{1}{n^2}$
        \begin{align*}
            \frac{a_{n+1}}{a_n} = \frac{n^2}{(n+1)^2} &= \pair{\frac{n}{n+1}}^2 = \pair{1-\frac{1}{n+1}}^2 \fromto 1
            \intertext{Aber}
            \sum_{n=1}^{\infty} &a_n
            \intertext{konvergiert absolut:}
            \pair{1-\frac{1}{n}}^2 &= 1 - \frac{2}{n+1}+ \pair{\frac{1}{n+1}}^2\\
            &= 1 - \frac{2}{n+1}\cdot\pair{1-\frac{1}{2(n+1)}}\\
            &\leq 1- \frac{s-\delta}{n+1}\tag{Für $\delta>0$ und fast alle $n$}
        \end{align*}
    \end{beispiel}

    \newpage

    \begin{beispiel}[Eulersche Zahl über Reihendarstellung]
        Die Reihe
        \begin{align*}
            \sum_{n=0}^{\infty} \frac{1}{n!}
        \end{align*}
        ist absolut konvergent.
        \begin{align*}
            a_n &= \frac{1}{n!}\\
            \frac{a_{n+1}}{a_n} &= \frac{n!}{(n+1)!} = \frac{1}{n+1}\fromto 0
        \end{align*}
        Behauptung:
        \begin{align*}
            \sum_{n=0}^{\infty} \frac{1}{n!} &= e\tag{Eulersche Zahl}
        \end{align*}
        \begin{proof}
            Wir wenden die bereits gezeigt Formel für $e$ an:
            \begin{align*}
                e &= \lim_{\ntoinfty} \pair{1+\frac{1}{n}}^n\\[10pt]
                \pair{1+\frac{1}{n}}^n &= \sum_{k=0}^{n}\binom{n}{k} \pair{\frac{1}{n}}^k\\
                &= \sum_{k=0}^{n} \frac{n\cdot(n-1)\cdot(n-k+1)}{k!\cdot n^k}\\
                &= \sum_{k=0}^{n} \frac{1}{k!} \cdot \prod_{j=0}^{k-1} \frac{n-j}{n}\\
                &\leq \sum_{k=0}^{n} \frac{1}{k!} \leq \sum_{k=0}^{\infty} \frac{1}{k!}\\[10pt]
                \impl e &\leq \sum_{n=0}^{\infty} \frac{1}{n!}
            \end{align*}
            \begin{center}
            [Der zweite Teil des Beweises wird in der nächsten Vorlesung behandelt.]
            \end{center}
        \end{proof}
    \end{beispiel}

    % \renewcommand{\thepage}{(Noch in Bearbeitung)}
\end{document}
