\thispagestyle{pagenumberonly}

Bedeutung von endlichen Summen ist klar.\\
Frage: Gegeben eine reelle Folge $a_n$. Was ist $\pair{a_1 + a_2 + a_3 + \dots = \sum_{n=1}^{\infty} a_n}$?

\subsection{Konvergenz-Kriterien für Reihen}

\begin{definition}[Reihen als Partialsummen] % Definition 1
    Das Symbol
    \begin{align*}
        \sum_{n=1}^{\infty} a_n\tag{Sei $a_n$ eine reelle Folge}
    \end{align*}
    wird folgendermaßen verwendet:

    \begin{enumerate}[label=\alph*)]
        \item Es steht für die Folge der Partialsummen:
        \begin{align*}
            s_n&\definedas \sum_{j=1}^{n} a_j\quad\forall n\in\N
        \end{align*}
        \item Die Reihe $\sum_{n=1}^{\infty} a_n$ konvergiert, falls der Grenzwert der Partialsummen $\lim_{n\fromto\infty} s_n$ existiert.\\
        Wir setzen
        \begin{align*}
            \sum_{n=1}^{\infty} a_n &\definedas \lim_{n\fromto\infty} s_n
        \end{align*}
        \item Konvergiert $(s_n)_n$ nicht, so heißt $\sum_{n=1}^{\infty} a_n$ divergent. Falls $s_n$ bestimmt divergiert so setzen wir
        \begin{align*}
            \sum_{n=1}^{\infty} a_n &\definedas \infty \tag{Wenn $\lim_{n\fromto\infty} s_n = \infty$}\\
            \sum_{n=1}^{\infty} a_n &\definedas -\infty \tag{Wenn $\lim_{n\fromto\infty} s_n = -\infty$}
        \end{align*}
    \end{enumerate}
\end{definition}

\begin{satz}[Monotone Konvergenz für Reihen] % Satz 2
    \label{satz:mont-konv-reihen}
    Sei $a_n$ eine reelle Folge mit $\forall n\colon a_n\geq 0$. Dann konvergiert die Reihe $\sum_{j=1}^{\infty} a_j$ genau dann, wenn die Folge der Partialsummen $(s_n)_n$ nach oben beschränkt ist.

    \begin{proof}
        Betrachte
        \begin{align*}
            s_{n+1} &= \sum_{j=1}^{n+1} a_j = \sum_{j=1}^n a_j + a_{n+1} \geq \sum_{j=1}^{n} a_j = s_n
        \end{align*}
        und wende Satz~\ref{satz:monoton-konv} an.
    \end{proof}
\end{satz}

\begin{korollar} % Korollar 3
    Für eine Reihe $\sum_{n=1}^{\infty} a_n$ mit $a_n\geq 0$ gilt entweder
    \begin{align*}
        \sum_{n=1}^{\infty} a_n < \infty\quad\text{oder}\quad\sum_{n=1}^{\infty} a_n = \infty
    \end{align*}

    \begin{proof}
        Folgt direkt aus Satz~\ref{satz:mont-konv-reihen}.
    \end{proof}
\end{korollar}

\begin{bemerkung}
    Oft hat man Reihen der Form
    \begin{align*}
        \sum_{n=0}^{\infty} a_n &= a_0+a_1+a_2+\dots\\
        s_n&\definedas \sum_{j=0}^{n} a_j \tag{$n\in\N_0$}
        \intertext{oder}
        s_n &\definedas a_0 + \sum_{j=1}^{\infty} a_j\tag{$n\in\N$}
        \intertext{Sofern ein Limes existiert, gilt dann}
        \sum_{n=0}^{\infty} a_n &\definedas \lim s_n\\[10pt]
        \intertext{Allgemein für $v\in\Z$ und $a_v, a_{v+1}, a_{v+2}, \ldots\in\R$}
        \sum_{n=v}^{\infty} a_n &= a_v + a_{v+1} + \dots\\
        s_n &\definedas \sum_{j=v}^{n} a_j\text{ def. Folge $(s_n)_{n\geq v}$}
    \end{align*}
\end{bemerkung}

\begin{beispiel}[Geometrische Summe und Reihe]
    \footnote{Wir setzen $0^0 = 1$}
    \begin{align*}
        q\neq 1 \impl \sum_{j=0}^{n} q^j = \frac{1-q^{n+1}}{1-q}\quad\forall n\in\N_0\\
        \text{Ist } \abs{q} < 1\colon \sum_{n=0}^{\infty} q^n\text{ konvergiert und } \sum_{n=0}^{\infty} q^n = \frac{1}{1-q}\tag{geometrische Reihe}
    \end{align*}
    Zum Beispiel $q=\frac{1}{2}$
    \begin{align*}
        \sum_{j=0}^{n} q^j &= \frac{1-q^{n+1}}{1-q}\\
        \equivalent \pair{1-q}\cdot \sum_{j=0}^{n} q^j &= 1-q^{n+1}\\
        \frac{1}{2}\cdot\sum_{j=0}^{n} \pair{\frac{1}{2}}^j &= 1 - \pair{\frac{1}{2}}^{n+1}\\
        \impl 1 - \pair{\frac{1}{2}}^{n+1} &= \sum_{j=0}^{n} \frac{1}{2}\cdot\pair{\frac{1}{2}}^j\\
        &= \sum_{j=0}^{n} \pair{\frac{1}{2}}^{j+1} = \sum_{j=1}^{n+1} \pair{\frac{1}{2}}^j
    \end{align*}
    Das heißt es sollte gelten
    \begin{align*}
        \sum_{j=1}^{n} \pair{\frac{1}{2}}^j &= 1-\pair{\frac{1}{2}}^n\quad \forall n\in\N
    \end{align*}

    \noindent Dass dieser Zusammenhang gelten muss, lässt sich einfach veranschaulichen. Die linke Seite der Gleichung kann als Summe über Teilflächen des Einheitsquadrats\footnote{Original: \anf{Kuchen}} visualisiert werden.\\
    Erst wird eine Hälfte, dann ein Viertel, dann ein Achtel (usw.) des Quadrats hinzugefügt. Der zurückbleibende Flächeninhalt ist immer genauso groß wie das zuletzt hinzugefügte Stück. Dieser Term wird durch den rechten Teil der Gleichung beschrieben.

    \begin{proof}[Beweis der Reihenformel]
        \begin{align*}
            s_n &\definedas \sum_{j=0}^{n} q^n\\
            q\cdot s_n &= q\cdot \sum_{j=0}^{n} q^j = \sum_{j=0}^{n} q^{j+1}\\
            &= \sum_{j=1}^{n+1} q^j\tag{Indexshift}\\
            \impl (1-q)\cdot s_n = s_n - q\cdot s_n &= \sum_{j=0}^{n}  q^j - \sum_{j=1}^{n+1} q^j\tag{Reißverschlusssumme}\\
            = q^0 - q^{n+1} &= 1 - q^{n+1}\\
            \impl s_n &= \frac{1-q^{n+1}}{1-q}\qedhere
        \end{align*}
    \end{proof}

    %%%%%%%%%%%%%%%%%%%%%%%%
    % 05. Dezember 2023
    %%%%%%%%%%%%%%%%%%%%%%%%

    \begin{proof}[Beweis der Konvergenz für $\abs{q} < 1$]
        \marginnote{[5. Dez]}
        \begin{align*}
            s_n\definedas \sum_{j=0}^{n} q^j &= \frac{1-q^{n+1}}{1-q}\\
            \lim_{n\fromto\infty} q^n &= 0 = \lim_{n\fromto\infty} q^{n+1}\tag{Weil $\abs{q} < 1$}\\
            \annot{\impl}{\ref{satz:konvergenzsaetze}} \lim s_n &= \frac{1-\lim_{n\fromto\infty} q^{n+1}}{1-q} = \frac{1-0}{1-q} = \frac{1}{1-q}\qedhere
        \end{align*}
    \end{proof}
    \begin{bemerkung}
        Ist $\abs{q}\geq 1$, dann ist $1+\underbrace{q}_{\geq 1}+\underbrace{q^2}_{\geq 1}+\underbrace{q^3}_{\geq 1}+\dots + \underbrace{q^n}_{\geq 1} \geq 1+n\fromto\infty$.
    \end{bemerkung}
\end{beispiel}

\begin{beispiel}[Harmonische Reihe]
    \begin{align*}
        s_n &\definedas \sum_{j=1}^{n} \frac{1}{j}
    \end{align*}
    $(s_n)_n$ ist monoton wachsend, aber nicht nach oben beschränkt. Das heißt
    \begin{align*}
        \lim_{n\fromto\infty} s_n = \infty \impl \sum_{n=1}^{\infty} \frac{1}{n} = \infty
    \end{align*}

    \newpage

    \begin{proof}
        \begin{align*}
            s_{2n} - s_n &= \sum_{j=n+1}^{2n} \frac{1}{j} \geq \sum_{j=n+1}^{n} \frac{1}{2n} = n \cdot\frac{1}{2n} = \frac{1}{2}\\
            \impl s_2 - s_1 &\geq \frac{1}{2}\\
            \impl s_2 &\geq s_1 + \frac{1}{2} = 1 + \frac{1}{2} > \frac{1}{2}\\
            s_4 - s_2 &\geq \frac{1}{2}\\
            \impl s_4 &\geq s_2 + \frac{1}{2} > \frac{1}{2} + \frac{1}{2} = 1\\
            s_8 - s_4 &\geq \frac{1}{2}\\
            \impl s_8 &\geq s_4 + \frac{1}{2} \geq \frac{3}{2}\\
            \annot{\impl}{Induktion} s_{\pair{2^j}} &> \frac{j}{2}\quad\forall j\in\N
        \end{align*}
        \noindent Also ist $s_{\pair{2^j}}$ nicht nach oben beschränkt $\impl$ $(s_n)_n$ nicht nach oben beschränkt.
        \begin{align*}
            \annot{\impl}{\ref{satz:monoton-konv}} \sum_{n=1}^{\infty} \frac{1}{n} = \lim_{n\fromto\infty} s_n &= +\infty\qedhere
        \end{align*}
    \end{proof}
\end{beispiel}

\begin{satz} % Satz 6
    Seien $\sum_{n=1}^{\infty} a_n$, $\sum_{n=1}^{\infty} b_n$ konvergente Reihen. Dann ist
    \begin{align*}
        \forall\lambda, \mu \in\R\colon \sum_{n=1}^{\infty} \pair{\lambda\cdot a_n + \mu\cdot b_n}
    \end{align*}
    konvergent und es gilt
    \begin{align*}
        \sum_{n=1}^{\infty} \pair{\lambda\cdot a_n + \mu\cdot b_n} = \lambda \cdot \sum_{n=1}^{\infty} a_n + \mu\cdot \sum_{n=1}^{\infty} b_n
    \end{align*}
    \begin{proof}
        \begin{align*}
            s_n &\definedas \sum_{j=1}^{n} a_n \quad t_n \definedas \sum_{j=1}^{n} b_n\\
            d_n &\definedas \sum_{j=1}^{n} \pair{\lambda\cdot a_n + \mu\cdot b_n} =   \lambda \cdot \sum_{n=1}^{n} a_n + \mu\cdot \sum_{n=1}^{n} b_n\\
            &= \lambda\cdot s_n + \mu\cdot t_n \fromto \lambda\cdot s + \mu\cdot t\tag{Mit $s$ und $t$ als Limes}
        \end{align*}
    \end{proof}
\end{satz}

\begin{satz}[Majoranten-Kriterium] % Satz 7
    \label{satz:majoranten-kriterium}
    Gegeben zwei Folgen $0\leq a_n \leq b_n~\forall n\in\N$. Konvergiert
    \begin{align*}
        \sum_{n=1}^{\infty} b_n\quad\text{so konvergiert auch}\quad\sum_{n=1}^{\infty} a_n
    \end{align*}
    und es gilt
    \begin{align*}
        0\leq \sum_{n=1}^{\infty} a_n \leq \sum_{n=1}^{\infty} b_n
    \end{align*}
    \begin{proof}
        \begin{align*}
            &s_n = \sum_{j=1}^{n} a_j\quad t_n = \sum_{j=1}^{n} b_j\\
            \impl &s_{n+1} = s_n + a_{n+1} \geq s_n\\
            &t_{n+1} = t_n + b_{n+1} \geq t_n\\
            \impl &(s_n)_n,~(t_n)_n\text{ sind monoton wachsend}\\
            \intertext{Mit der Konvergenz von $(t_n)_n$ und $(t_n)_n$ monoton wachsend folgt mit Satz~\ref{satz:mont-konv-reihen}}
            \impl &(t_n)_n\text{ ist beschränkt}\\
            \impl &\exists M\geq 0\colon t_n \leq M\quad\forall n\in\N\\
            0\leq a_n \leq b_n \impl &0\leq s_n \leq t_n \leq M \quad\forall n\in\N\\
            \impl &(s_n)_n\text{ ist nach oben beschränkt und monoton wachsend}\\
            \annot{\impl}{\ref{satz:mont-konv-reihen}} &\lim_{n\fromto\infty} s_n\text{ existiert }\\
            \quad s &\definedas \lim_{n\fromto\infty} s_n \leq \lim_{n\fromto\infty} t_n = \sum_{n=1}^{\infty} b_n\qedhere\\
            \intertext{Außerdem gilt:}
            t_n - s_n &= \sum_{j=1}^{n} b_j - \sum_{j=1}^{n} a_j = \sum_{j=1}^{n} \underbrace{b_j-a_j}_{\geq 0} \geq 0
        \end{align*}
    \end{proof}
\end{satz}

\begin{satz}[Minoranten-Kriterium] % Satz 8
    Sei $0\leq b_n \leq a_n~\forall n\in\N$ und
    \begin{align*}
        \sum_{n=1}^{\infty} b_n &= \infty\\
        \impl \sum_{n=1}^{\infty} a_n&\text{ divergiert auch bestimmt gegen }\infty
    \end{align*}
    \begin{proof}
        \begin{align*}
            t_n &= \sum_{j=1}^{n} b_n \quad s_n = \sum_{j=1}^{n} a_n
            \intertext{Analog zum Beweis des Majoranten-Kriteriums gilt:}
            (s_n)_n,~(t_n)_n&\text{ sind monoton wachsend und }t_n \leq s_n \quad\forall n\in\N
            \intertext{Dann lässt sich folgern}
            \sum_{n=1}^{\infty} b_n = \infty &\equivalent (t_n)_n \text{ wächst über alle Grenzen}\\
            &\impl (s_n)_n\text{ wächst über alle Grenzen}\\
            &\impl \lim_{n\fromto\infty} s_n = \infty = \sum_{n=1}^{\infty} a_n\qedhere
        \end{align*}
    \end{proof}
\end{satz}

\begin{beispiel}[Anwendung des Minoranten-Kriteriums]
    Es sei
    \begin{align*}
        a_n\geq \frac{c}{n}\quad\forall n\in\N\tag{$c>0$}
        \intertext{Nach Minorantenkriterium und der Divergenz der harmonischen Reihe gilt}
        \impl \sum_{n=1}^{\infty} a_n = \infty
    \end{align*}
\end{beispiel}
\begin{beispiel}[Anwendung des Majoranten-Kriteriums]
    Es sei wieder $c>0$. Dann folgt aus
    \begin{align*}
        0\leq a_n \leq c\cdot q^n\tag{$0\leq q < 1$}
        \intertext{nach Majoranten-Kriterium und dem Konvergenzkriterium der geometrischen Reihe, dass}
        \impl \sum_{n=1}^{\infty} a_n\text{ konvergiert}\tag{$b_n=c\cdot q^n$}
    \end{align*}
\end{beispiel}

\begin{bemerkung}[Abgeschwächtes Majoranten-Kriterium]
    Die Konvergenz/Divergenz von Reihen (und Folgen) ändert sich nicht, wenn man endlich viele Summanden (Folgeglieder) abändert.
    Für das Majoranten-Kriterium reicht also, dass $0\leq a_n \leq b_n$ für fast alle $n\in\N$, damit
    \begin{align*}
        \sum_{n=1}^{\infty} b_n\text{ konvergiert }\impl \sum_{n=1}^{\infty} a_n\text{ konvergiert}
    \end{align*}
    Das gleiche gilt analog für das Minoranten-Kriterium
\end{bemerkung}

\vfill

\begin{satz}[Cauchyscher Verdichtungssatz] % Satz 9
    \label{satz:cauchy-verdichtung}
    Sei $(a_n)_n$ eine monoton fallende Nullfolge. Dann gilt
    \begin{align*}
        &\sum_{n=0}^{\infty} a_n\text{ konvergiert}\\
        \equivalent &\sum_{n=0}^{\infty} 2^n\cdot a_{\pair{2^n}}\text{ konvergiert}\tag{Verdichtete Reihe}
    \end{align*}
    \begin{proof}
        \anf{$\Leftarrow$} 1. Schritt. Zu zeigen: $a_n\geq 0\quad\forall n\in\N$.
        \begin{align*}
            a_n\geq a_{n+1} &\geq a_{n+2} \geq \dots \geq a_{n+l} \fromto 0\text{ für } l\fromto\infty\\
            \impl a_n &\geq 0\quad\forall n\in\N
            \intertext{2. Schritt}
            s_n \definedas \sum_{j=0}^{n} a_j \quad&\quad t_n \definedas \sum_{\nu=0}^{n} 2^{\nu} \cdot a_{\pair{2^\nu}}
            \intertext{Jedes $\overline{n}\in\N$ können wir eindeutig schreiben als $\overline{n}=2^\nu + l$ mit $\nu\in\N_0,~0\leq l < 2^\nu$. Sei $1\leq n< 2^k$ für ein $k\in\N_0$}
        \end{align*}
        \newpage
        \noindent Damit gilt
        \begin{align*}
            s_n &= \sum_{j=0}^{n} a_j = a_0 + \sum_{j=1}^{n} a_j\\
            &\leq a_0 + \sum_{j=1}^{2^k-1} a_j = a_0 + \sum_{\nu=0}^{k-1} \pair{\sum_{l=0}^{2^\nu - 1} \underbrace{a_{\pair{2^\nu + l}}}_{\leq a_{\pair{2^\nu}}}}\\
            &\leq a_0 + \sum_{\nu=0}^{k-1} \pair{\sum_{l=0}^{2^\nu-1} a_{\pair{2^\nu}}}\\
            &= a_0 + \sum_{\nu = 0}^{k-1} 2^{\nu}\cdot a_{\pair{2^\nu}} = a_0 + t_{k-1}\\[10pt]
            \impl \forall n < 2^k\text{ gilt }s_n&\leq a_0 + t_{k-1}
            \intertext{Angenommen}
            \sum_{\nu = 0}^{\infty} 2^{\nu} \cdot a_{2^\nu}\text{ konvergent} &\equivalent \lim_{n\fromto\infty} t_n = t\text{ existiert}\\
            \impl s_n \leq a_0 + \lim_{k\fromto\infty} t_{k-1} &= a_0 + t\quad\forall n\in\N
            \intertext{Somit ist $a_0 + t$ eine obere Schranke von $(s_n)_n$. Da $s_n\leq s_{n+1} \annot{\impl}{\ref{satz:monoton-konv}} \biglim{n\fromto\infty} s_n$ existiert}
            \impl \sum_{n=0}^{\infty} a_n&\text{ konvergent}
            \intertext{\anf{$\impl$} Sei $n\geq 2^k$}
            s_n &= \sum_{j=0}^{n} a_j \geq \sum_{j=0}^{2^k} a_j\\
            &= \sum_{\nu=0}^{k} \pair{\sum_{l=0}^{2^\nu-1} \underbrace{a_{\pair{2^\nu + l}}}_{\geq a_{\pair{2^{\nu+1}}}}}\\
            &\geq \sum_{\nu=0}^{k} \pair{\sum_{l=0}^{2^\nu-1} a_{\pair{2^{\nu+1}}}} = \sum_{\nu=0}^{k} 2^{\nu}\cdot a_{\pair{2^{\nu + 1}}}\\
            &= \sum_{\nu = 1}^{k+1} 2^{\nu-1} a_{\pair{2^\nu}} = \frac{1}{2}\sum_{\nu=1}^{k+1} 2^{\nu} a_{\pair{2^\nu}}\\
            &= \frac{1}{2}\pair{\sum_{\nu=0}^{k+1} 2^\nu a_{\pair{2^\nu}} - a_0}\\
            &= t_{k+1} - a_0\\
            \impl t_k \leq t_{k+1} &\leq s_n + a_0\quad\forall 2^k \geq n\\
            \impl t_k \leq \lim_{n\fromto\infty} s_n + a_0 &= s + a_0 < \infty\text{ sofern } \sum_{n=0}^{\infty} a_n\text{ konvergiert}
        \end{align*}
        Da $t_k \leq t_{k+1}$ konvergiert, konvergiert auch $\sum_{\nu=0}^{\infty} 2^\nu \cdot a_{2^\nu}$
    \end{proof}
\end{satz}

\begin{beispiel}[Cauchyscher Verdichtungssatz als Konvergenzkriterium]
    Es sei
    \begin{align*}
        a_n &= \frac{1}{n^\alpha}\\
        2^{n} \cdot a_{\pair{2^n}} &= \frac{2^n}{(2^n)^\alpha} = \frac{2^n}{2^{n\cdot\alpha}}\\
        &= 2^{n-n\cdot\alpha} = 2^{(1-\alpha)\cdot n} = \pair{2^{1-\alpha}}^n = q^n
        \intertext{Damit $q^n$ und damit auch $(a_n)_n$ konvergiert muss wie bereits gezeigt gelten}
        q &\definedas 2^{1-\alpha} < 1 \equivalent 1-\alpha < 0 \equivalent \alpha > 1
    \end{align*}
\end{beispiel}

\begin{beispiel}[Cauchyscher Verdichtungssatz bei Divergenz]
    Es sei $a_n = \frac{1}{n}$
    \begin{align*}
        2^n \cdot a_{\pair{2^n}} &= 2^n \cdot \frac{1}{2^n} = 1
        \intertext{Weil die verdichtete Folge keine Nullfolge ist, gilt}
        \impl \sum_{n=0}^{\infty} a_n &\text{ divergiert}
    \end{align*}
\end{beispiel}

%%%%%%%%%%%%%%%%%%%%%%%%
% 7. Dezember 2023
%%%%%%%%%%%%%%%%%%%%%%%%

\begin{definition}[Alternierende Reihe] % Definition 10
    \marginnote{[7. Dez]}
    Sei $(a_n)_n$ eine reelle Folge nicht-negativer reeller Zahlen. Dann heißt
    \begin{align*}
        \sum_{n=1}^{\infty} (-1)^{n+1} \cdot a_n = a_1 - a_2 + a_3 - a_4 \dots
    \end{align*}
    alternierende Reihe. Alternativ $a_n\geq 0$, $n\in\N_0$.
    \begin{align*}
        \sum_{n=0}^{\infty} (-1)^n\cdot a_n = a_0 - a_1 + a_2 - a_3 \dots
    \end{align*}
\end{definition}

\begin{satz}[Leibniz-Kriterium] % Satz 11
    Sei $(a_n)_n$ eine monoton fallende Nullfolge. Dann konvergiert
    \begin{align*}
        \sum_{n=1}^{\infty} (-1)^{n+1} \cdot a_n
    \end{align*}

    \begin{proof}
        Idee: Wir unterscheiden zwischen geraden und ungeraden $n$.
        \begin{align*}
            s_n &\definedas \sum_{j=1}^{n} (-1)^{j+1} \cdot a_j\\
            s_{2(n+1)} &= \sum_{j=1}^{2n+2} (-1)^{j+1} \cdot a_j\\
            &= \pair{\sum_{j=1}^{2n} (-1)^{j+1} \cdot a_j} + (-1)^{(2n+1)+1} \cdot a_{2n+1} + (-1)^{(2n+2)+1} \cdot a_{2n+2}\\
            &= s_{2n} + \underbrace{a_{2n+1} - a_{2n+2}}_{\geq 0}\\
            &\geq s_{2n}
            \intertext{Also ist $(s_{2n})_n$ monoton wachsend}
            s_{2(n+1)+1} &= s_{2n+3} = \sum_{j=1}^{2n+3} (-1)^{j+1} \cdot a_j\\
            &= s_{2n+1} + (-1)^{(2n+2)+1} \cdot a_{2n+2} + (-1)^{(2n+3)+1} \cdot a_{2n+3}\\
            &= s_{2n+1} \underbrace{- a_{2n+2} + a_{2n+3}}_{\leq 0}\\
            &\leq s_{2n+1}
        \end{align*}
        Also ist $(s_{2n+1})_n$ ist monoton fallend und
        \begin{align*}
            s_{2n+1} - s_{2n} &= \sum_{j=1}^{2n+1} (-1)^{j+1} \cdot a_j - \sum_{j=1}^{2n} (-1)^{j+1}\cdot a_j\\
            &= (-1)^{(2n+1)+1} \cdot a_{2n+1} = a_{2n+1} \geq 0
            \intertext{$\impl \abs{s_{2n+1}-s_{2n}} = a_{2n+1}$ und $s_{2n+1} \geq s_{2n}$}
            0 \leq a_1 - a_2 &= s_2 \leq s_{2n} \leq s_{2n+1} \leq s_1 = a_1\\
            \annot{\impl}{\ref{satz:monoton-konv}} s_g &\definedas \lim_{n\fromto\infty} s_{2n}\text{ existiert}\\
            s_u &\definedas \lim_{n\fromto\infty} s_{2n-1}\text{ existiert}\\
            \text{und } s_g &= s_u\\
            \impl (s_n)_n&\text{ konvergiert gegen $s=s_g=s_u$}\qedhere
        \end{align*}
    \end{proof}
\end{satz}

\begin{uebung}
    Weisen Sie nach, dass eine Folge konvergiert, wenn die Teilfolgen der geraden und der ungeraden Folgeglieder konvergieren und die Differenz eine Nullfolge ist.
\end{uebung}

\begin{beispiel}
    \begin{align*}
        &\sum_{n=1}^{\infty} \pair{-1}^{n+1} \cdot \frac{1}{\sqrt{n}}\text{ konvergiert}\\
        \text{aber } &\sum_{n=1}^{\infty} \frac{1}{\sqrt {n}} \text{ divergiert}\\
        &\sum_{n=1}^{\infty} (-1)^{n} \cdot \frac{1}{n}\text{ konvergiert}\\
        &\sum_{n=2}^{\infty} (-1)^n \cdot \frac{1}{\log\pair{n}}\text{ konvergiert}
    \end{align*}
\end{beispiel}

\begin{satz}[Cauchy-Kriterium] % Satz 12
    \label{satz:cauchy-kriterium}
    Sei $(a_n)_n$ eine Folge reeller Zahlen. Dann konvergiert
    \begin{align*}
        \sum_{n=1}^{\infty} a_n
    \end{align*}
    genau dann, wenn
    \begin{align*}
        \forall \varepsilon > 0~\exists N_{\varepsilon}\in\N\colon \abs{\sum_{j=n+1}^{m} a_j} <\varepsilon\quad\forall m>n\geq N_{\varepsilon}
    \end{align*}
    \begin{proof}
        \begin{align*}
            s_n &\definedas \sum_{j=1}^{n} a_j\\
            \intertext{$(s_n)_n$ konvergiert nach Satz~\ref{satz:jede-konv-cauchy} genau dann, wenn es eine Cauchy-Folge ist. Das heißt}
            \forall\varepsilon > 0~\exists N_{\varepsilon}\in\N\colon &\abs{s_{m} - s_{n}} < \varepsilon\quad\forall m> n \geq N_{\varepsilon}\\
            \abs{s_m - s_n} &= \sum_{j=1}^{m} a_j - \sum_{j=1}^{n} a_j\\
            &= \sum_{j=n+1}^{m} a_j\qedhere
        \end{align*}
    \end{proof}
\end{satz}

\begin{korollar} % Korollar 12
    \label{korollar:folge-von-reihe-nullfolge}
    Ist die reelle Reihe $\sum_{n=1}^{\infty} a_n$ konvergent, so ist $(a_n)_n$ eine Nullfolge.
    \begin{proof}
        Nach Satz~\ref{satz:cauchy-kriterium} gilt
        \begin{align*}
            \forall \varepsilon > 0~\exists N_{\varepsilon}\in\N\colon &\abs{\sum_{j=n+1}^{n+p} a_j} < \varepsilon\quad\forall n\geq N_{\varepsilon}, p\in\N\\
            \intertext{Wähle $p=1$}
            \sum_{j=n+1}^{n+1} a_{j} &= a_{j+1}\\
            \impl \abs{a_{n+1}} &< \varepsilon\quad \forall n\geq N_{\varepsilon}\\
            \impl \lim_{\ntoinf} a_{n+1} &= 0\\
            \impl \lim_{\ntoinf} a_{n} &= 0\qedhere
        \end{align*}
    \end{proof}
\end{korollar}

\newpage

\subsection{Absolut konvergente Reihen und Umordnungen}

\begin{definition}[Absolute Konvergenz] % Definition 1
    Eine Reihe $\displaystyle\sum_{n=1}^{\infty} a_n$ heißt absolut konvergent, falls
    \begin{align*}
        \sum_{n=1}^{\infty} \abs{a_n}
    \end{align*}
    konvergiert. Das heißt falls
    \begin{align*}
        \sum_{n=1}^{\infty} \abs{a_n} < \infty
    \end{align*}
\end{definition}

\begin{satz}[Absolute Konvergenz als Konvergenzkriterium] % Satz 2
    \label{satz:absolut-konvergenz-konvergenkriterium}
    Ist eine Folge $\sum_{n}^{\infty} a_n$ absolut konvergent, so ist sie auch konvergent und
    \begin{align*}
        \abs{\sum_{n=1}^{\infty} a_n} \leq \sum_{n=1}^{\infty} \abs{a_n}
    \end{align*}

    \begin{proof}
        Wir haben für $m>n$
        \begin{align*}
            \abs{\sum_{j=n+1}^{m} a_j} \leq \sum_{j=n+1}^{m} \abs{a_j}
        \end{align*}
        Wir nehmen an, dass $\displaystyle\sum_{n=1}^{\infty} \abs{a_n}$ konvergiert. Das heißt das Cauchy-Kriterium ist erfüllt.
        \begin{align*}
            \impl \forall \varepsilon > 0~\exists N_{\varepsilon}\colon \sum_{j=n+1}^{m} \abs{a_j} &< \varepsilon \quad\forall m> n\geq N_{\varepsilon}\\
            \impl \abs{\sum_{j=n+1}^{m} a_j} &< \varepsilon\quad\forall m>n\geq N_{\varepsilon}\\
            \impl \sum_{j=1}^{\infty} a_j&\text{ konvergiert}\qedhere
        \end{align*}
    \end{proof}
\end{satz}

\begin{beispiel}[Konvergenz ohne absolute Konvergenz]
    Die Reihe
    \begin{align*}
        \sum_{n=1}^{\infty} (-1)^{n+1}\cdot \frac{1}{n}
    \end{align*}
    ist konvergent, aber nicht absolut konvergent.
\end{beispiel}

\begin{definition}[Majorante] % Def 3
    Die Reihe
    \begin{align*}
        \sum_{n=1}^{\infty} c_n\quad c_n\geq 0~\forall n\in\N
    \end{align*}
    ist eine Majorante der Reihe $\sum_{n=1}^{\infty} a_n$ falls
    \begin{align*}
        \abs{a_n} &\leq c_n\text{ für fast alle }n\in\N
        \intertext{Das heißt}
        \exists n_0\in\N\colon \abs{a_n} &< c_n\quad\forall n\geq n_0
    \end{align*}
\end{definition}

\begin{satz}[Majoranten-Konvergenz-Kriterium für Reihen] % Satz 4
    \label{satz:majorante-reihen}
    Hat die Reihe $\sum_{n=1}^{\infty} a_n$ eine konvergente Majorante $\sum_{n=1}^{\infty} c_n$, so ist diese Reihe absolut konvergent und somit auch konvergent.
    \begin{proof}
        Folgt aus Satz~\ref{satz:absolut-konvergenz-konvergenkriterium} und Satz~\ref{satz:majoranten-kriterium}.
    \end{proof}
\end{satz}

\begin{satz}[Quotientenkriterium] % Satz 5
    \label{satz:quotientenkriterium}
    Sei
    \begin{align*}
        s_n &\definedas \sum_{n=0}^{\infty} a_n
    \end{align*}
    eine Reihe mit $a_n\neq 0$. Ferner gebe es ein $0\leq q < 1$ so dass
    \begin{align*}
        \frac{\abs{a_{n+1}}}{\abs{a_n}} \leq q\text{ für fast alle } n\in\N\\
        \impl \sum_{n=0}^{\infty} a_n\text{ absolut konvergent}
    \end{align*}

    \begin{proof}
        Wir haben $n_0\in\N$
        \begin{align*}
            \frac{\abs{a_{n+1}}}{\abs{a_n}} &\leq q\quad\forall n\geq n_0\\[10pt]
            \abs{a_{n+1}} &\leq q\cdot\abs{a_n} \leq q^2 \cdot \abs{a_{n-1}} \leq q^3 \cdot \abs{a_{n-2}}\\[10pt]
            &\leq \dots \leq q^{p+1} \cdot\abs{a_{n_0}} = q^{n-n_0+1} \cdot\abs{a_{n_0}}\tag{$p\in\N$, $n=n_0+p$}\\[10pt]
            &= q^{n+1} \cdot q^{-n_0} \cdot\abs{a_{n_0}}\\[10pt]
            \impl \abs{a_n} &\leq \underbrace{q^n \cdot K}_{\definedasbackwards c_n}\tag{$K \definedas q^{-n_0} \cdot \abs{a_{n_0}}$}\\
            \impl \abs{a_{n}} &\leq c_n\quad\forall n\geq n_0\\
            \intertext{und}
            \sum_{n=0}^{\infty} c_n &= \sum_{n=0}^{\infty} K\cdot q^{n} = K\cdot \sum_{n=0}^{\infty} q^{n} < \infty
            \intertext{Da $0< q < 1$}
            \impl \sum_{n=c}^{\infty} &a_n\text{ hat die konvergente Majorante } K \cdot \sum_{n=c}^{\infty} q^{n}
        \end{align*}
        Damit lässt sich aus Satz~\ref{satz:majorante-reihen} folgern, dass die Reihe konvergiert.
    \end{proof}
\end{satz}

\begin{bemerkung}[Quotientenkriterium über $\limsup$ und $\liminf$]
    \theoremescape
    \begin{enumerate}[label=(\roman*)]
        \item Ist $\displaystyle \limsup_{n\fromto\infty} \frac{\abs{a_{n+1}}}{\abs{a_n}} < 1 \equivalent$ Quotientenkriterium
        \item Ist $\displaystyle \liminf_{n\fromto\infty} \frac{\abs{a_{n+1}}}{\abs{a_n}} > 1\impl \sum_{n=0}^{\infty} a_n$ divergent
        \item Ist $\displaystyle \limsup_{n\fromto\infty} \frac{\abs{a_{n+1}}}{\abs{a_n}} = 1\impl \text{Keine Aussage über absolute Konvergenz von } \sum_{n=0}^{\infty} a_n$ möglich
    \end{enumerate}
    \begin{proof}[Beweis (ii).]
        \begin{align*}
            \text{Ist } \overline{q}\definedas\liminf_{n\fromto\infty} \frac{\abs{a_{n+1}}}{\abs{a_n}} &> 1\\[8pt]
            \impl \forall \varepsilon > 0~\exists n_0\colon \frac{\abs{a_{n+1}}}{\abs{a_n}} \geq \overline{q} - \varepsilon = \frac{\overline{q}+1}{2} &\definedasbackwards q > 1\tag{Wähle $\varepsilon=\frac{\overline{q}-1}{2} > 0$}\\[8pt]
            \impl \frac{\abs{a_{n+1}}}{\abs{a_n}} &\geq q > 1\quad\forall n\geq n_0
            \intertext{Wir wenden ein ähnliches Prinzip wie im vorherigen Beweis an}
            \impl \abs{a_{n+1}} &\geq q^{n+1} \cdot q^{-n_0}\cdot\abs{a_{n_0}}\\[8pt]
            \impl \abs{a_n} &\geq q^n\cdot K\tag{$K=q^{-n_0}\cdot\abs{a_{n_0}}$}\\[8pt]
            \impl \sum_{n=0}^{\infty} a_j\text{ divergiert nach}&\text{ Korollar~\ref{korollar:folge-von-reihe-nullfolge}}&&\qedhere
        \end{align*}
    \end{proof}
\end{bemerkung}

\begin{beispiel}[Divergenz bei nicht-eindeutigem Quotientenkriterium]
    Es sei $a_n \definedas \frac{1}{n}$
    \begin{align*}
        \frac{\abs{a_{n+1}}}{\abs{a_n}} = \frac{a_{n+1}}{a_n} &= \frac{n}{n+1} = 1 + \frac{1}{n} \fromto 1
        \intertext{Und}
        \sum_{n=1}^{\infty} \frac{1}{n}&\text{ divergiert (Harmonische Reihe)}
    \end{align*}
\end{beispiel}

\begin{beispiel}[Konvergenz bei nicht-eindeutigem Quotientenkriterium]
    Es sei $a_n \definedas \frac{1}{n^2}$
    \begin{align*}
        \frac{a_{n+1}}{a_n} = \frac{n^2}{(n+1)^2} &= \pair{\frac{n}{n+1}}^2 = \pair{1-\frac{1}{n+1}}^2 \fromto 1
        \intertext{Aber}
        \sum_{n=1}^{\infty} &a_n
        \intertext{konvergiert absolut:}
        \pair{1-\frac{1}{n}}^2 &= 1 - \frac{2}{n+1}+ \pair{\frac{1}{n+1}}^2\\
        &= 1 - \frac{2}{n+1}\cdot\pair{1-\frac{1}{2\cdot(n+1)}}\\
        &\leq 1- \frac{2-\delta}{n+1}\tag{Für $\delta>0$ und fast alle $n$}
    \end{align*}
\end{beispiel}

\newpage

\begin{beispiel}[Eulersche Zahl über Reihendarstellung]
    Die Reihe
    \begin{align*}
        \sum_{n=0}^{\infty} \frac{1}{n!}
    \end{align*}
    ist absolut konvergent.
    \begin{align*}
        a_n &= \frac{1}{n!}\\
        \frac{a_{n+1}}{a_n} = \frac{n!}{(n+1)!} &= \frac{1}{n+1}\fromto 0
    \end{align*}
    Behauptung:
    \begin{align*}
        e'\definedas\sum_{n=0}^{\infty} \frac{1}{n!} &= e\tag{Eulersche Zahl}
    \end{align*}
    \begin{proof}
        Wir wenden die bereits gezeigte Formel $e=\biglim{\ntoinf} \pair{1+\frac{1}{n}}^n$ an:
        \begin{align*}
            \pair{1+\frac{1}{n}}^n &= \sum_{k=0}^{n}\binom{n}{k} \pair{\frac{1}{n}}^k\\
            &= \sum_{k=0}^{n} \frac{n\cdot(n-1)\cdot \ldots \cdot (n-k+1)}{k!\cdot n^k}\\
            &= \sum_{k=0}^{n} \frac{1}{k!} \cdot \underbrace{\prod_{j=0}^{k-1} \frac{n-j}{n}}_{\leq 1} \leq \sum_{k=0}^{n} \frac{1}{k!}\\
            \impl e &\leq \sum_{n=0}^{\infty} \frac{1}{n!}
            \intertext{Noch zu zeigen: $e'\leq e$}
            \pair{1+\frac{1}{n}}^n &= \sum_{k=0}^{n} \frac{1}{k!} \cdot \prod_{j=0}^{k-1} \pair{1-\frac{j}{n}}\\
            &\geq \sum_{k=0}^{m} \frac{1}{k!}\cdot \prod_{j=0}^{k-1} \pair{1-\frac{j}{n}}\tag{$\forall n>m$}\\
            \intertext{Wir halten $m\in\N$ fest}
            \impl e &= \lim_{\ntoinf} \pair{1+\frac{1}{n}}^n \geq \lim_{\ntoinf} \sum_{k=0}^{m} \frac{1}{k!}\cdot \prod_{j=0}^{k-1} \pair{1-\frac{j}{n}}\\
            &= \sum_{k=0}^{m} \frac{1}{k!} \cdot \prod_{j=0}^{k-1} \underbrace{\lim_{\ntoinf} \pair{1-\frac{j}{n}}}_{=1}\\
            &= \sum_{n=0}^{m} \frac{1}{k!}\tag{$\forall m\in\N$}\\
            \impl e\geq \lim_{\ntoinf} \sum_{k=0}^{m} \frac{1}{k!} &= \sum_{k=0}^{\infty} \frac{1}{k!} = e\\
            \impl e &\leq e' \land e'\leq e\\
            \impl e &= e'\qedhere
        \end{align*}
    \end{proof}

    %%%%%%%%%%%%%%%%%%%%%%%%
    % 12. Dezember 2023
    %%%%%%%%%%%%%%%%%%%%%%%%

    \begin{bemerkung}[Ännäherung von $e$ über Reihen]
        \marginnote{[12. Dez]}
        \begin{align*}
            s_n &\definedas \sum_{k=0}^{n} \frac{1}{k!}\\
            r_{n,p} &\definedas \sum_{k=n+1}^{n+p} \frac{1}{k!}\\
            r_{n,p} &= \sum_{k=n+1}^{n+p} \frac{1}{k!} = \frac{1}{(n+1)!} + \frac{1}{(n+2)!} + \dots + \frac{1}{(n+p)!}\\
            &= \frac{1}{(n+1)!}\cdot\interv{1+\frac{1}{n+2}+\frac{1}{(n+2)\cdot(n+3)} + \dots + \frac{1}{(n+2)\cdot(n+3)\cdot\ldots\cdot(n+p)}}\\
            &> \frac{1}{(n+1)!}
            \intertext{Wir betrachten den zweiten Faktor}
            &1+\frac{1}{n+2}+\frac{1}{(n+2)\cdot(n+3)} + \dots + \frac{1}{(n+2)\cdot(n+3)\cdot\ldots\cdot(n+p)}\\
            &< 1 + \frac{1}{2} + \frac{1}{2\cdot 3}+\dots + \frac{1}{2\cdot 3 \cdot\ldots\cdot p}\\
            &= \sum_{k=0}^{p} \frac{1}{k!} - 1 < \sum_{k=0}^{\infty} \frac{1}{k!} - 1 = e-1
            \intertext{Wir kombinieren die Abschätzung über beide Faktoren und erhalten}
            \impl &\frac{1}{(n+1)!} < r_{n,p} < \frac{e-1}{(n+1)!}\\
            \equivalent &\frac{1}{(n+1)!} < s_{n+p} - s_{n} < \frac{e-1}{(n+1)!}\\
            &s_{n+p} - s_{n} \fromto e - s_n\text{ für }p\fromto\infty\\
            \impl &\frac{1}{(n+1)!} \leq e - s_n \leq \frac{e-1}{(n+1)!}
            \intertext{Wir erhalten also ein Verfahren, um einen Näherungswert für $e$ zu bestimmen}
            &2,5 \leq e \leq 3\quad(n=1)\\
            &2,66 \leq e \leq 2,8\quad(n=2)\\
            \vdots\\
            &e = 2,71828182\ldots
        \end{align*}
    \end{bemerkung}
\end{beispiel}


\begin{bemerkung}[Ausblick: Definition von Exponentialfunktionen über Reihen]
    \begin{align*}
        a_n &\definedas\sum_{n=0}^{\infty} \frac{x^n}{n!}\\
        \frac{\abs{a_{n+1}}}{\abs{a_n}} &= \abs{\frac{x^{n+1}\cdot n!}{(n+1)!\cdot x^n}} = \frac{\abs{x}}{n+1}\fromto 0
    \end{align*}
    Diese Konvergenz werden wir später in Kapitel~\ref{subsec:exp} nutzen, um die Exponentialfunktion über
    \begin{align*}
        \exp(x) &= \sum_{k=0}^{\infty} \frac{x^n}{n!} = e^x
    \end{align*}
    zu definieren.
\end{bemerkung}

\begin{satz}[Nach Lambert 1707] % Satz 7
    Die eulersche Zahl $e$ ist irrational.
    \begin{proof}
        Wir haben
        \begin{align*}
            \frac{1}{(n+1)!} &\leq e-s_n \leq \frac{e-1}{(n+1)!} < \frac{2}{(n+1)!}
            \intertext{Angenommen $e=\frac{p}{q}$\quad$p,q\in\N$. Wir nehmen $p=q\cdot m$}
            \impl \frac{1}{(q+1)!} &\leq \frac{p}{q} - s_q < \frac{2}{(q+1)!}\\
            0 &< \frac{1}{q+1} \leq \frac{p}{q}\cdot q! - q!\cdot s_q < \frac{2}{q+1}<1
            \intertext{Es lässt sich zeigen, dass der dritte Term eine ganze Zahl ist:}
            \frac{p}{q}\cdot q! &= p\cdot(q-1)!\in\N\\
            q!\cdot s_q &= q! \cdot \sum_{k=0}^{q} \frac{1}{k!}\in\N
        \end{align*}
        Damit ergibt sich ein Widerspruch, weil keine ganze Zahl zwischen $0$ und $1$ liegt.
    \end{proof}
\end{satz}

\begin{satz}[Wurzelkriterium] % Satz 8
    \label{satz:wurzelkriterium}
    Sei $(a_n)_n$ eine reelle Folge.
    \begin{enumerate}[label=(\roman*)]
        \item Ist $\displaystyle \limsup_{\ntoinf} \sqrt[n]{\abs{a_n}} < 1$, dann konvergiert $\sum_{n=0}^{\infty} a_n$ absolut.
        \item Ist $\displaystyle \limsup_{\ntoinf} \sqrt[n]{\abs{a_n}} > 1$, dann ist $\sum_{n=0}^{\infty} a_n$ divergent.
        \item Ist $\displaystyle \limsup_{\ntoinf} \sqrt[n]{\abs{a_n}} = 1$, so ist keine Aussage möglich.
    \end{enumerate}
    \begin{proof}[Beweis (iii)]
        \begin{align*}
            a_n &\definedas \frac{1}{n^p}\\
            \sqrt[n]{\abs{a_n}} &= \pair{\frac{1}{\sqrt[n]{n}}}^p \fromto \pair{\frac{1}{1}}^p = 1\tag{$n\fromto\infty$}\\
            \sum_{n=1}^{\infty} \frac{1}{n^p}&\text{ divergiert für $p=1$ und konvergiert für $p>1$}\\
            \impl &\text{ keine Aussage möglich}\qedhere
        \end{align*}
    \end{proof}
    \begin{proof}[Beweis (i)]
        Sei
        \begin{align*}
            \hat{q}&\definedas \limsup \sqrt[n]{\abs{a_n}} < 1
            \intertext{Wähle $\varepsilon\definedas \frac{1-\hat{q}}{2} > 0$}
            \annot{\impl}{\ref{lemma:limsup-charak}} \sqrt[n]{\abs{a_n}} &< \hat{q} + \varepsilon = \frac{1+\hat{q}}{2}\definedasbackwards q < 1\text{ für fast alle $n$}\\
            \impl \exists n_{0}\in\N\colon \sqrt[n]{\abs{a_n}} &\leq q\quad\forall n\geq n_0\\
            \equivalent \abs{a_n} &\leq q^n\quad\forall n\geq n_0\\
            \intertext{Das heißt $\sum_{n=0}^{\infty} q^n$ ist eine konvergente Majorante für $\sum_{n=0}^{\infty} a_n$}
            \impl \sum_{n=0}^{\infty} a_n&\text{ ist absolut konvergent}\qedhere
        \end{align*}
    \end{proof}
    \begin{proof}[Beweis (ii)]
        Sei
        \begin{align*}
            \hat{q}&\definedas \limsup_{\ntoinf} \sqrt[n]{\abs{a_n}} > 1\\
            \varepsilon &\definedas \frac{\hat{q}-1}{2}>0\\
            q&\definedas \hat{q}-\varepsilon = \frac{1+\hat{q}}{2} > 1\\
            \impl \sqrt[n]{\abs{a_n}} &> \hat{q}-\varepsilon = q > 1\tag{für unendlich viele $n$}\\
            \impl \abs{a_n} &> q^n\tag{für unendlich viele $n$}\\
            \impl (a_n)_n&\text{ keine Nullfolge}
        \end{align*}
        Somit konvergiert $\sum_{n=0}^{\infty} a_n$ nicht.
    \end{proof}
\end{satz}

\begin{definition}[Umordnung von Reihen] % Definition 9
    Seien
    \begin{align*}
        \sum_{n=0}^{\infty} a_n\quad \sum_{n=0}^{\infty} b_n
    \end{align*}
    Reihen mit Gliedern $a_n, b_n\in\R$, $n\in\N_0$. Wir nennen $\sum_{n=0}^{\infty} b_n$ eine Umordnung von $\sum_{n=0}^{\infty} a_n$, falls eine Bijektion
    \begin{align*}
        \sigma: \N_0 \fromto \N_0
    \end{align*}
    existiert mit $b_n = a_{\sigma(n)}~\forall n\in\N_0$.\\
    Ähnlich für Reihen
    \begin{align*}
        \sum_{n=1}^{\infty} a_n\quad \sum_{n=1}^{\infty} b_n
    \end{align*}
    $\sum_{n=1}^{\infty} b_n$ Umordnung von $\sum_{n=1}^{\infty} a_n$, falls Bijektion $\sigma: \N \fromto\N$ existiert mit $b_n = a_{\sigma(n)}$.
\end{definition}

\begin{definition}[Unbedingte Konvergenz] % Definition 10
    Eine Reihe $\sum_{n=0}^{\infty} a_n$ heißt unbedingt konvergent, falls jede Umordnung $\sum_{n=0}^{\infty} b_n$ von dieser Reihe ebenfalls konvergiert und die selbe Summe hat.\\
    Andernfalls heißt $\sum_{n=0}^{\infty} a_n$ bedingt konvergent.
\end{definition}

\begin{satz}[Dirichlet 1837] % Satz 11
    Eine Reihe $\sum_{n=0}^{\infty} a_n$, $a_n\in\R$ ist absolut konvergent genau dann, wenn sie unbedingt konvergiert.
    \begin{proof}
        \anf{$\impl$} Sei $\sum_{n=0}^{\infty} a_n$ absolut konvergent
        \begin{align*}
            \impl \sum_{n=0}^{\infty} \abs{a_n}\text{ konvergent} &\equivalent \sum_{n=0}^{\infty} \abs{a_n} < \infty
            \intertext{Wir wenden das Cauchy-Kriterium an}
            \forall \varepsilon>0~\exists N\in\N\colon \sum_{j=n+1}^{n+p} \abs{a_j} &< \varepsilon\quad\forall n\geq N, p\in\N\\
            \impl \sum_{j=n+1}^{\infty}\abs{a_j} = \lim_{p\fromto\infty} \sum_{j=n+1}^{n+p} \abs{a_j} &\leq \varepsilon\quad\forall n\geq N\tag{3}
            \intertext{Wir definieren}
            s_n \definedas \sum_{j=0}^{n} a_j&\quad \sum_{n=0}^{\infty} b_n\text{ Umordnung von } \sum_{n=0}^{\infty} a_n\\
            b_n &= a_{\sigma(n)}\tag{$\sigma: \N_0 \fromto\N_0$ Bijektion}\\
            t_n &\definedas \sum_{j=0}^{n} b_j
            \intertext{Wir wissen $s_n\fromto s$. Zu zeigen: $t_n\fromto s$}
            \set{1,2,\dots, N} &\subseteq \set{\sigma(1), \sigma(2), \dots, \sigma(M)}\tag{Nehmen $M\in\N$}\\
            \intertext{Ist dann $n\geq M$, dann ist}
            \set{a_1, a_2, a_3, \dots, a_N}\subseteq \set{b_1, b_2, \dots b_M} &= \set{a_{\sigma(1)}, a_{\sigma(2)}, \dots, a_{\sigma(M)}}
            \intertext{Also treten alle Glieder $a_1, \dots, a_n$ in der Summe $s_n$ in $t_n = t_1 + t_2 + \dots + t_n$ auf. Diese Terme heben sich in $s_n-t_n$ gegenseitig auf, sofern $n\geq M$ ist}
            \impl \abs{s_n-t_n} &\leq \sum_{\substack{j\geq N+1\\ j=\sigma(k)}}^{} \abs{a_j}\tag{für ein $k\in\set{1,\dots, M}$}\\
            &\leq \sum_{j=N+1}^{\infty} \abs{a_j} \leq \varepsilon\\
            \impl s_n - t_n &\fromto 0\text{ für }\ntoinf
        \end{align*}
        Da $(s_n)_n$ gegen $s$ konvergiert, konvergiert auch $(t_n)_n$ gegen $s$. Somit konvergiert $\sum_{n=0}^{\infty} b_n$ und hat die selbe Summe wie $\sum_{n=0}^{\infty} a_n$.\\[10pt]
        \anf{$\Leftarrow$} Angenommen $\sum_{n=0}^{\infty} a_n$ ist unbedingt konvergent, aber nicht absolut konvergent.
        \begin{align*}
            p_n \definedas (a_n)_{+} &\definedas \max\pair{0, a_n}\\
            q_n \definedas (a_n)_{-} &\definedas \max\pair{0,-a_n} = -\min\pair{0,a_n}\\
            \impl \abs{a_n} &= p_n + q_n\quad\forall n\in\N_0\\
            \intertext{Wir haben $\sum_{n=0}^{\infty} a_n$ konvergiert, aber $\sum_{n=0}^{\infty} \abs{a_n} = \infty$. Behauptung:}
            \sum_{n=0}^{\infty} p_n = \infty &\text{ und } \sum_{n=0}^{\infty} q_n = \infty
            \intertext{Angenommen $0\leq \sum_{n=0}^{\infty} p_n < \infty$.}
            \impl \sum_{n=0}^{\infty} \overbrace{\pair{p_n - a_n}}^{=q_n}&\text{ konvergiert}\\
            \impl \sum_{n=0}^{\infty} q_n &< \infty
            \intertext{Da $\abs{a_n} = p_n + q_n$}
            \impl \sum_{n=0}^{\infty} \abs{a_n} = \sum_{n=0}^{\infty} \pair{p_n+q_n} &= \sum_{n=0}^{\infty} p_n + \sum_{n=0}^{\infty} q_n < \infty
        \end{align*}
        Widerspruch zu $\sum_{n=0}^{\infty} a_n$ ist nicht absolut konvergent.\qedhere\newpage
        \noindent \textit{Konstruktiver Beweis für Rückrichtung}. Setze $\sigma_0=0$ und bestimme induktiv $(\sigma_n)_n,~\sigma_n < \sigma_{n+1}$ mit
        \begin{align*}
            p_0 + p_1 + \dots + p_{\sigma_n} > n + q_0 + q_1 + \dots + q_n~\forall n\in\N
        \end{align*}
        $\sigma_1\definedas$ kleinste natürliche Zahl mit $p_0 + p_1 + p_{\sigma_1} > 1 + q_0 + q_1$.\\
        $\sigma_2\definedas$ kleinste natürliche Zahl $\geq \sigma_1 + 1\colon p_0 + \dots + p_{\sigma_2} > 2 + q_0 + q_1 + q_2$.\\
        Machen induktiv weiter. Gegeben $\sigma_n$ wähle $\sigma_{n+1} =$ kleinste natürliche Zahl $\geq \sigma_n + 1$ mit
        \begin{align*}
            p_0 + p_1 + \dots + p_{\sigma_{n+1}} > n + q_0 + q_1 + \dots + q_{n+1}
        \end{align*}
        Umordnung $p_0 - q_0 + p_1 + \dots + p_{\sigma_1} - q_1 + p_{\sigma_{1+1}} + \dots + p_{\sigma_2} - q_2 + p_{\sigma_{2+1}} + \dots + p_{\sigma_3} - q_3 + \dots$.
        \begin{align*}
            &= p_0 - q_0 + \sum_{\vartheta = 1}^{\infty}\pair{ \pair{\sum_{l=\sigma_{\vartheta - 1} + 1}^{\sigma_{\vartheta}} p_l} - q_{\vartheta}}\\
            t_{k} &= p_0 - q_0 + \sum_{\vartheta = 1}^{k}\pair{ \pair{\sum_{l=\sigma_{\vartheta - 1} + 1}^{\sigma_{\vartheta}} p_l} - q_{\vartheta}}\tag{Partialsummen der Umordnung}\\
            &= p_0 + p_1 + \dots + p_{\sigma_k} - \pair{q_0 + q_1 + \dots + q_k} > k \text{ unbeschränkt}\\
            \impl &\pair{t_n}_n \text{ divergiert nach } +\infty \text { im Widerspruch zur Annahme.}\\
        \end{align*}
    \end{proof}
\end{satz}

%%%%%%%%%%%%%%%%%%%%%%%%
% 14. Dezember 2023
%%%%%%%%%%%%%%%%%%%%%%%%

\begin{satz}[Nach Riemann 1854]
    \marginnote{[14. Dez]}
    Ist $\sum a_n$ konvergent, aber nicht absolut konvergent. Dann gibt es zu jedem $c\in\R$ eine Umordnung $\sum_{}^{} b_n$ von $\sum_{}^{} a_n$ so, dass $\sum b_n$ konvergiert und den Wert $c$ hat ($\sum b_n = c$).

    \begin{proof}
    (Später)
    \end{proof}
\end{satz}

\newpage