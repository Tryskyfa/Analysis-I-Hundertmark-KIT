\imaginarysubsection{Potenzreihen}
\thispagestyle{pagenumberonly}

Wir untersuchen Reihen der Form
\begin{align*}
    \sum_{n=0}^{\infty} a_n\cdot z^n \text{ oder } \sum_{n=0}^{\infty} a_n \cdot \pair{z-z_0}^n
\end{align*}
für $z_0\in\C$ fest, $z\in\C$ oder $\R$, $a_n\in\C$. Partialsummen:
\begin{align*}
    s_n(z) &\definedas \sum_{j=0}^{n} a_j\cdot z^j
\end{align*}
Frage: Konvergenz?
\begin{beispiel}
    \begin{align*}
        \exp(z) \definedas &\sum_{n=0}^{\infty} \frac{z^n}{n!} \text{ konvergiert unabhängig von } z \\
        &\sum_{n=0}^{\infty} n! \cdot z^n \text{ konvergiert nur für } z=0
    \end{align*}
\end{beispiel}

\begin{definition}[Konvergenzradius]
    Wir setzen
    \begin{align*}
        R &\definedas \sup \set{\abs{z}: z\in\C \text{ und } \sum_{n=0}^{\infty} a_n \cdot z^n \text{ konvergent } }
    \end{align*}
    $R$ ist der Konvergenzradius.
\end{definition}

\begin{satz}
    \label{satz:potenzreihe-kreisscheibe}
    Die Potenzreihe $\sum_{n=0}^{\infty} a_n z^n$ konvergiert absolut für jedes $z$ in der Kreisscheibe
    \begin{align*}
        B_R (0) &= \set{z\in\C: \abs{z} < R}
    \end{align*}
    Für jedes $\abs{z} > R$ divergiert $ \sum_{n=0}^{\infty} a_n \cdot z^n$.
    \begin{proof}
        Sei $z_1\neq 0$ und $ \sum_{n=0}^{\infty} a_n \cdot z^n$ konvergent. Damit ist $\pair{a_n\cdot z^n}$ eine Nullfolge und damit beschränkt.
        \begin{align*}
            \impl K&\definedas \sup_{n\in\N} \set{a_n z^n} < \infty
            \intertext{Sei $0 < R < \abs{z_1}$, $0 < \theta \definedas \frac{R}{\abs{z_1}} < 1$}
            z \in \overline{B_r (0)} &= \set{z\in\C: \abs{z} \leq R}\\[10pt]
            \abs{a_n z^n} = \abs{a_n}\cdot\abs{z^n} = \abs{a_n} \cdot \abs{z}^n &= \abs{a_n} \cdot \abs{z_1}^n\cdot \pair{\frac{\abs{z}}{\abs{z_1}}}^n \leq K\cdot \theta^n\quad \forall n \geq 0\\
            \impl \sum_{n=0}^{\infty} K\cdot\theta^n \text{ konvergente} &\text{ Majorante für } \sum_{n=0}^{\infty} a_n z^n \text{ sofern } 0 \leq z \leq R < \abs{z_1}\\
            \impl \sum_{n=0}^{\infty} a_n z^n &\text{ konvergiert absolut }
        \end{align*}
        Damit konvergiert $\sum_{n=0}^{\infty} a_n z^n$ absolut für alle $\abs{z} \leq R$. Angenommen $\sum_{n=0}^{\infty} a_n z^n$ konvergiert für ein $\abs{z} > R$, dann ergibt sich ein Widerspruch zur Definition von $R$.
    \end{proof}
\end{satz}

\begin{bemerkung}
    Konvergieren $ \sum_{n=0}^{\infty} a_n z^n$ und $\sum_{n=0}^{\infty} b_n z^n$ für $\abs{z} < R$. Dann gilt
    \begin{align*}
        \sum_{n=0}^{\infty} \pair{\lambda a_n + \mu b_n} z^n &= \lambda \sum_{n=0}^{\infty} a_n z^n + \mu \sum_{n=0}^{\infty} b_n z^n
    \end{align*}
\end{bemerkung}

\begin{bemerkung}
    Konvergieren $ \sum_{n=0}^{\infty} a_n z^n$ und $ \sum_{n=0}^{\infty} b_n z^n$ auf $B_R (0)$. Dann gilt
    \begin{align*}
        \pair{\sum_{n=0}^{\infty} a_n z^n} \cdot \pair{\sum_{n=0}^{\infty} b_n z^n} = \sum_{n=0}^{\infty} \pair{\sum_{\nu=0}^{\infty} a_{\nu} b_{n-\nu}} z^n \tag{Cauchy-Produkt}
    \end{align*}
\end{bemerkung}

\begin{bemerkung}
    Wir können auch Potenzreihen der folgenden Form betrachten
    \begin{align*}
        \sum_{n=0}^{\infty} a_n z^n\quad \sum_{n=0}^{\infty} a_n \cdot \pair{z-z_0}^n \text{ mit } a_n\in\R^d \text{ oder } \C^d
    \end{align*}
\end{bemerkung}

\begin{satz} % Satz 2
    Die Potenzreihe $ \sum_{n=0}^{\infty} a_n z^n$ konvergiert absolut $\forall z\in B_R (0)$ und divergiert $\forall\abs{z} > R$.

    \begin{proof}
        Abschreiben des Beweises von Satz~\ref{satz:potenzreihe-kreisscheibe}.
    \end{proof}
\end{satz}

\begin{lemma} % Lemma 3
    \label{lemma:potenzreihen-abschaetzung}
    Konvergiert $\sum_{n=0}^{\infty} a_n z^n$ für ein $z=z_1 \neq 0$ und ist $0 < r < \abs{z_1}$. So ist $\sum_{n=0}^{\infty} a_n z^n$ auf $B_r (0) = \set{z\in\C: \abs{z} \leq r}$ beschränkt. Das heißt
    \begin{align*}
        \exists M_r \geq 0\colon\abs{\sum_{n=0}^{\infty} a_n z^n} \leq M_r\quad\forall \abs{z} \leq r
    \end{align*}
    \begin{proof}
        Wir setzen
        \begin{align*}
            \theta &\definedas \frac{r}{\abs{z_1}} < 1
            \intertext{Da $ \sum_{n=0}^{\infty}  a_n z_1^n$ konvergiert ist $a_n z_1^n$ eine Nullfolge, also beschränkt}
            \impl K &= \sup_{n\in\N_0} \abs{a_n z_1^n} = \sup_{n\in\N} \abs{a_n} \abs{z_1}^n < \infty\\
            \intertext{Ist $\abs{z} < r$}
            \abs{a_n z^n} &= \abs{a_n} \abs{z}^n = \abs{a_n} \abs{z_1}^n \pair{\frac{\abs{z}}{\abs{z_1}}}^n\\
            \impl \abs{\sum_{n=0}^{\infty} a_n z^n} &\leq \sum_{n=0}^{\infty} \abs{a_n z}^n \leq \sum_{n=0}^{\infty} K \theta^n = \frac{K}{1-\theta} < \infty\quad \forall \abs{z} \leq r\qedhere
        \end{align*}
    \end{proof}
\end{lemma}

\begin{lemma} % Lemma 4
    \label{lemma:temp-4}
    Konvergiert $\sum_{n=0}^{\infty} a_n z^n$ für ein $z=z_1 \neq 0$ und ist $0 < r < \abs{z_1}$. Dann existiert für alle $0< r < \abs{z_1}$ und $k\in\N_0$ ein $M_{k,r}$ mit
    \begin{align*}
        \abs{\sum_{n=k+1}^{\infty} a_n z^n} &\leq M_{k,r}\cdot\abs{z}^{k+1}\quad\forall \abs{z} \leq r
    \end{align*}

    \begin{proof}
        $\sum_{n=0}^{\infty} a_n z^n$ konvergiert
        \begin{align*}
            \impl &\sum_{n=k+1}^{\infty} a_n z^n \text{ konvergiert auch }\\
            \impl &\sum_{n=k+1}^{\infty} a_n z^{n-(k+1)} = z^{-(k+1)} \sum_{n=k+1}^{\infty} a_n z^n \text{ konvergiert}
        \end{align*}
        Nach Lemma~\ref{lemma:potenzreihen-abschaetzung} existiert für $0<r<z_1$ ein $M_{k,r} \geq 0$ sodass
        \begin{align*}
            \abs{\sum_{n=k+1}^{\infty} a_n z^{n-(k+1)}} &\leq M_{r,k}\quad\forall \abs{z} \leq r < \abs{z_1}\\[10pt]
            \abs{\sum_{n=k+1}^{\infty} a_n z^{n-(k+1)}} &= \abs{z^{-(k+1)} \sum_{n=k+1}^{\infty} a_n z^n}\\[10pt]
            &= \frac{1}{\abs{z}^{k+1}} \abs{\sum_{n=k+1}^{\infty} a_n z^n}\\[10pt]
            \impl \abs{\sum_{n=k+1}^{\infty} a_n z^n} &\leq M_{k,r} \abs{z}^{k+1}\qedhere
        \end{align*}
    \end{proof}
\end{lemma}

\begin{anwendung}[Fehlerabschätzung]
    \begin{align*}
        \sum_{n=0}^{\infty} a_n z^n &= \underbrace{\sum_{n=0}^{k} a_n z^n}_{s_k (z)} + \underbrace{\sum_{n=k+1}^{\infty} a_n z^n}_{Fehler}\\
        \abs{\text{Fehler}} &\leq M_{k,r}\cdot \abs{z}^{k+1} \leq M_{k,r}\cdot \theta^{k+1} \tag{$\theta = \frac{r}{\abs{z_1}}$}
    \end{align*}
\end{anwendung}

%%%%%%%%%%%%%%%%%%%%%%%%
% 18. Januar 2024
%%%%%%%%%%%%%%%%%%%%%%%%

\begin{lemma}[Bessere Version von Lemma~\ref{lemma:temp-4}]
    \marginnote{[18. Jan]}
    \label{lemma:potenzreihen-abschaetzung-besser}
    Sei $\sum_{n=0}^{\infty} a_n z^n$ konvergent für ein $z=z_1 \neq 0$ und ist $0 < r < \abs{z_1}$. Dann existiert für alle $0< r < \abs{z_1}$ und $k\in\N_0$ ein $M_{k,z_1}\geq 0$ mit
    \begin{align*}
        \abs{\sum_{n=k+1}^{\infty} a_n z^n} &\leq M_{r,z_1}\cdot \abs{\frac{z}{z_1}}^{k+1}\quad\forall \abs{z} \leq r
    \end{align*}

    \begin{proof}
        \begin{align*}
            \abs{\frac{z}{z_1}} &\leq \frac{\abs{z}}{\abs{z_1}} \leq \frac{r}{\abs{z_1}} \definedasbackwards \theta < 1
            \intertext{Die Reihe $\sum_{n=0}^{\infty} a_n z_1^n$ konvergiert, damit ist $a_n z_1^n$ eine Nullfolge}
            K &\definedas \sup_{n\geq 0} \abs{a_n z_1^n} < \infty\\
            \impl \abs{a_n z^n} &= \abs{a_n z_1^n\cdot\pair{\frac{z}{z_1}}^n} = \abs{a_n z_1^n}\cdot \abs{\frac{z}{z_1}}^n\\
            \abs{\frac{z}{z_1}} &\leq \frac{r}{\abs{z_1}} = \theta < 1\\
            \impl \abs{\sum_{n=k+1}^{\infty} a_n z^n} &\leq \sum_{n=k+1}^{\infty} \abs{a_n z^n} \leq \sum_{n=k+1}^{\infty} K \cdot \abs{\frac{z}{z_1}}^n\\
            &= K\cdot \abs{\frac{z}{z_1}}^{k+1} \cdot \sum_{n=0}^{\infty} \abs{\frac{z}{z_1}}^n\\
            &= \frac{K}{1-\theta}\cdot \abs{\frac{z}{z_1}}^{k+1} = \underbrace{ \frac{K}{1-\abs{\frac{r}{z_1}}} }_{M_{r,z_1}}\cdot \pair{\frac{\abs{z}}{\abs{z_1}}}^{k+1}\qedhere
        \end{align*}
    \end{proof}
\end{lemma}

\begin{satz} % Satz 5
    \label{satz:12-5}
    Sei $\sum_{n=0}^{\infty} a_n z^n$ eine Potenzreihe, die für ein $z=c_1\neq 0$ konvergiert. Sei außerdem $(z_j)_j$ eine Folge mit $0< \abs{z_j} < \abs{c_1}$, $z_j \fromto 0$ für $j\fromto\infty$. Dann gilt folgende Implikation
    \begin{align*}
        \sum_{n=0}^{\infty} a_n\cdot (z_j)^n &= 0\quad\forall j\in\N\\
        \impl a_n &= 0\quad\forall n\in\N_0
    \end{align*}
    \begin{proof}
        Angenommen nicht alle $a_n=0$
        \begin{align*}
            \impl B &\definedas \set{n\in\N_0: a_n \neq 0} \neq \emptyset\\
            \intertext{$B$ hat ein kleinstes Element, nennen wir $n_0$}
            \impl a_0 &= a_1 = \dots = a_{n_0-1} = 0\quad a_{n_0}\neq 0\\
            f(z) &= \sum_{n=0}^{\infty} a_n z^n = \sum_{n=n_0}^{\infty} a_n z^n = a_{n_0} z^{n_0} + \sum_{n=n_0 + 1}^{\infty} a_n z^n\tag{1}
            \intertext{Nach Voraussetzung gilt}
            f(z_j) &= 0
            \intertext{Damit gilt nach (1) und Lemma~\ref{lemma:potenzreihen-abschaetzung-besser}}
            \impl \abs{a_{n_0} (z_j)^{n_0}} &= \abs{- \sum_{n=n_0 + 1}^{\infty} a_n z_j^n} \leq M_{r, c_1}\cdot \pair{\frac{\abs{z_j}}{\abs{c_1}}}^{n_0 +1}\\
            \impl \abs{a_{n_0}} &\leq M_{r,z}\cdot \abs{c_1}^{-(n_0+1)}\cdot\abs{z_j}\fromto 0\\
            \impl a_{n_0} &= 0
        \end{align*}
        Damit ergibt sich ein Widerspruch zu $a_{n_0} \neq 0$.\qedhere.
    \end{proof}
\end{satz}

\begin{satz} % Satz 6
    Seien $\sum_{n=0}^{\infty} a_n z^{n}$, $ \sum_{n=0}^{\infty} b_n z^n$ Potenzreihen welche für ein $z=w\neq 0$ konvergieren. Sei außerdem $(z_j)_j$ eine Folge mit $z_j \neq 0$, $z_j \fromto 0$ für $j\fromto\infty$. Dann gilt folgende Implikation
    \begin{align*}
        \sum_{n=0}^{\infty} a_n z_j^n &= \sum_{n=0}^{\infty} b_n z_j^n\text{ für fast alle } z_j\\
        \impl a_n &= b_n \quad\forall n\in\N_0
    \end{align*}
    \begin{proof}
        \begin{align*}
            c_n &\definedas a_n - b_n
            \intertext{O.B.d.A sind alle $\abs{z_j} < \abs{w}$}
            \impl h(z) &\definedas \sum_{n=0}^{\infty} c_n z^n \text{ konvergiert für } z=w\neq 0\\
            \text{ und } h(z_j) &= 0\quad \forall j\\
            \annot{\impl}{Satz~\ref{satz:12-5}} c_n &= 0\quad \forall n\in\N_0\\
            \equivalent a_n &= b_n \quad\forall n\in\N_0\qedhere
        \end{align*}
    \end{proof}
\end{satz}

\begin{satz}[Konvergenzradius über $\limsup$] % Satz 7
    Für den Konvergenzradius $R$ einer Potenzreihe $ \sum_{n=0}^{\infty} a_n z^n$ gilt
    \begin{align*}
        R &= \frac{1}{\displaystyle\limsup_{n\fromto\infty}\sqrt[n]{\abs{a_n}}}
    \end{align*}

    \begin{proof}
        Schritt 1: Zu zeigen: Für $\abs{z} < R$ konvergiert die Potenzreihe.
        \begin{align*}
            M &= \limsup_{n\fromto\infty}\sqrt[n]{\abs{a_n}}\\
            \impl \forall \varepsilon > 0~\exists N \colon \sqrt[n]{\abs{a_n}} &\leq M+\varepsilon\quad\forall n\geq N\\
            \impl \forall \varepsilon > 0 \text{ ist } \sqrt[n]{\abs{a_n}} &> M- \varepsilon \text{ für }\infty\text{-viele } n
        \end{align*}
        Schritt 2: Zu zeigen: Für $\abs{z} > R$ konvergiert die Potenzreihe nicht. (Übung)
    \end{proof}
\end{satz}

\begin{korollar}
    Die Potenzreihen
    \begin{align*}
        \sum_{n=0}^{\infty} a_n z^n \text{ und } \sum_{n=1}^{\infty} n a_n z^{n-1}
    \end{align*}
    haben den gleichen Konvergenzradius.
    \begin{proof}
        Folgt mit vorherigem Satz und $\sqrt[n]{n}\fromto 1$ für $n\fromto\infty$.
    \end{proof}
\end{korollar}

\newpage