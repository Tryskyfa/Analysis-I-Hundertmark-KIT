%%%%%%%%%%%%%%%%%%%%%%%%
% 15. Februar 2024
%%%%%%%%%%%%%%%%%%%%%%%%

\thispagestyle{pagenumberonly}

\subsection{Konvexe und konkave Funktionen}

\begin{skizze}[Konvexe Funktion]
    \marginnote{[15. Feb]}
    Wähle ein $\lambda\in\pair{0,1}$ und formuliere die Interpolation $r=\lambda x + \pair{1-\lambda}\cdot y$.
    \begin{figure}[H]
        \centering
        \begin{tikzpicture}
            \draw[->] (-1, 0) -- (3, 0);
            \draw[->] (0, -1) -- (0, 4);
            \draw (-0.95*0.5, 0.1) -- (-0.95*0.5, -0.1) node[below] {$x$};
            \draw (5.2*0.5, 0.1) -- (5.2*0.5, -0.1) node[below] {$y$};

            \fill (-0.95*.5,0.7875*.5) circle[radius=1.5pt] node[left] {$f(x)$};
            \fill (5.2*.5,5.4*.5) circle[radius=1.5pt] node[right] {$f(y)$};
            \draw[domain=-2:6, smooth, variable=\x] plot ({0.5*\x}, {(0.2*(\x-0.25)^2+0.5)*0.5}) node[anchor=east] {$f$};
            \draw[domain=-0.95:5.2, smooth, variable=\x, dashed] plot ({0.5*\x}, {(0.75*\x+1.5)*0.5});
        \end{tikzpicture}
        \caption{Konvexe Funktion mit eingezeichneter Sekante}
    \end{figure}
    \noindent Wir erhalten die Sekantengleichung $\lambda f(x) +\pair{1-\lambda}\cdot f(y)$. Die Funktion ist konvex, wenn sie unter der Sekante verläuft. Das heißt
    \begin{align*}
        \lambda f(x) +\pair{1-\lambda} f(y) \geq f\of{x+\pair{1-\lambda}\cdot y}\quad\forall\lambda\in\pair{0,1} \text{ und } x,y\in D
    \end{align*}
\end{skizze}

\begin{definition}[Konvexität/Konkavität]
    Sei $f: D\fromto \R$ wobei $D\subseteq \R$ ein Intervall. Die Funktion $f$ heißt konvex (konkav), falls für alle $x,y\in D$ und alle $\lambda \in\pair{0,1}$ gilt
    \begin{align*}
        f\of{\lambda x + \pair{1-\lambda}\cdot y} \underset{(\geq)}{\leq} \lambda f(x) + \pair{1-\lambda}\cdot f(y)
    \end{align*}
\end{definition}

\begin{beispiel}
    Wir betrachten $f: \R\fromto\R$ mit $x\mapsto x^2$. Sei $x,y\in\R$ und \OBDA sei $y\geq x$, $\lambda\in\pair{0,1}$. Dann ist zu zeigen
    \begin{align*}
        f\of{\lambda x + \pair{1-\lambda}\cdot y} &\leq \lambda f\of{x} + \pair{1-\lambda}\cdot  f\of{y}\\
        \equivalent \pair{\lambda x+\pair{1-\lambda}\cdot y}^2 &\leq \lambda x^{2} + \pair{1-\lambda}\cdot y^2\\
        \equivalent \lambda^2 x^2 + 2\lambda\pair{1-\lambda}\cdot xy + \pair{1-\lambda}^{2}\cdot y^2 & \leq \lambda x^2 + \pair{1-\lambda}\cdot y^2\\
        \equivalent \lambda\cdot\pair{\lambda -1}\cdot x^2 + \pair{1-\lambda}\cdot\pair{1-\lambda-1}\cdot y^2 + \lambda\pair{1-\lambda}\cdot 2xy &\leq 0\\
        \equivalent \lambda\pair{\lambda -1}\cdot\interv{x^2+y^2-2xy} &\leq 0\\
        \equivalent x^2+y^2-2xy &\geq 0\\
        \equivalent \pair{x-y}^2 &\geq 0
    \end{align*}
\end{beispiel}

\hfill

\begin{satz}[Konvexität und zweite Ableitung] % Satz 1
    Sei $D\subseteq\R$ ein offenes Intervall und $f: D\fromto\R$ eine auf $D$ zweimal differenzierbare Abbildung. Dann gilt
    \begin{align*}
        f \text{ ist konvex }\quad \equivalent\quad \forall x\in D\colon f''(x)\geq 0
    \end{align*}
    \newpage
    \begin{proof}
        \anf{$\Leftarrow$}: Da $f'' > 0$ auf $D$, ist $f'$ auf $D$ monoton wachsend. Sei $x,y\in D$, \OBDA sei $y>x$ und $\lambda\in\pair{0,1}$, $r\definedas\lambda x + \pair{1-\lambda}\cdot y$.
        Es gilt nach Konstruktion, dass $x < r < y$. Wir untersuchen die Intervalle $D_1\definedas\pair{x,r}$ und $D_{2}\definedas\pair{r,y}$. Nach Satz~\ref{satz:mittelwertsatz} gilt $\exists \xi_1\in D_1,\xi_2\in D_2, \xi_1 < \xi_2$ sodass
        \begin{align*}
            \frac{f(r)-f(x)}{r-x} = f'\of{\xi_1} &\leq f'\of{\xi_2} = \frac{f(y)-f(r)}{y-r}
            \intertext{Beachte $r-x = \lambda x + \pair{1-\lambda}\cdot y - x = \pair{1-\lambda}\pair{y-x}$ und $y-r = \lambda\pair{y-x}$}
            \impl \frac{f(r)-f(x)}{\pair{1-\lambda}\pair{y-x}} &\leq \frac{f(y)-f(r)}{\lambda\pair{y-x}}\\
            \impl f(r) &\leq \pair{1-\lambda}\cdot f(y) + \lambda f(x)\\
            \impl f(\lambda x + \pair{1-\lambda}\cdot y) &\leq \pair{1-\lambda}\cdot f(y) + \lambda f(x)\\
            \impl f &\text{ ist konvex}
            \intertext{\anf{$\impl$}: Wir führen einen Beweis mit Kontraposition. Angenommen es existiert ein $x_0\in D$ so, dass $f''(x_0) < 0$. Wir definieren die Hilfsfunktion}
            \varphi\of{x} &= f(x) + c\cdot\pair{x-x_0}\\
            \impl \varphi'\of{x} &= f'\of{x} + c\\
            \impl \varphi'\of{x_0} &= f'\of{x_0} + c
            \intertext{Wähle $c=-f'\of{x_0}$}
            \impl \varphi'\of{x_0} &= 0
            \intertext{Es gilt $\varphi''\of{x_0} = f''\of{x_0} < 0$}
            \impl \varphi \text{ hat in } &x_0\in D \text{ ein isoliertes Maximum}\\
            \impl \exists t > 0\colon \varphi\of{x_0-t} &< \varphi\of{x_0} \land \varphi\of{x_0+t} < \varphi\of{x_0}\\
            \impl f(x_0) = \varphi\of{x_0} &> \frac{1}{2}\pair{\varphi\of{x_0+t} + \varphi\of{x_0-t}}\\
            \impl f(x_0) &> \frac{1}{2}\pair{f(x_0+t)+ct + f(x_0-t)-ct}\\
            \impl f(x_0) &> \frac{1}{2}\pair{f(x_0+t)+f(x_0-t)}
            \intertext{Wir wählen $y=x_0+t$, $x=x_0-t$, $\lambda = \frac{1}{2}$}
            \impl f(\lambda x + \pair{1-\lambda}\cdot y) &> \frac{1}{2}f(x) + \pair{1-\lambda}f(y)\\
            \impl f &\text{ ist nicht konvex}\qedhere
        \end{align*}
    \end{proof}
\end{satz}

\begin{beispiel}
    Wir betrachten $f: \pair{0,\infty}\fromto\R$, $x\mapsto x^r$ für $r\in\R$
    \begin{align*}
        f'\of{x} &= r\cdot x^{r-1}\\
        f''(x) &= r\cdot\pair{r-1}\cdot x^{r-2}
    \end{align*}
    Für $r\geq 1$ und $r\leq 0$ ist $f$ auf $\pair{0,\infty}$ konvex, für $r\in\pair{0,1}$ konkav.
\end{beispiel}

\begin{beispiel}
    Die Funktion $f: \R\fromto\R$, $x\mapsto e^{\lambda x}$
    \begin{align*}
        f''(x) &= \lambda^2\cdot e^{\lambda x} > 0
    \end{align*}
    ist konvex auf $\R$.
\end{beispiel}

\begin{beispiel}
    Die Funktion $f: \pair{0, \infty}\fromto\R$, $x\mapsto\ln\of{x}$
    \begin{align*}
        f''(x) &= \frac{-1}{x^2} < 0
    \end{align*}
    ist konkav auf $\pair{0, \infty}$.
\end{beispiel}

\begin{beispiel}
    Die Funktion $f: \interv{-\frac{\pi}{2}, \frac{\pi}{2}}$, $x\mapsto \arctan\of{x}$
    \begin{align*}
        f'(x) &= \frac{1}{1+x^2} = \pair{1+x^2}^{-1}\\
        \impl f''(x) &= -\pair{1+x^2}^{-2}\cdot 2x
    \end{align*}
    ist konvex im Intervall $\interv{-\frac{\pi}{2}, 0}$ und konkav in $\interv{0, \frac{\pi}{2}}$.
\end{beispiel}

\begin{lemma} % Lemma 2
    \label{lemma:konvex-verkettung}
    Seien $D_1, D_2\sbset\R$ Intervalle und $f: D_1\fromto D_2$ und $g: D_2\fromto \R$. Dann gilt
    \begin{enumerate}[label=(\roman*)]
        \item Falls $f$ konvex, $g$ konvex und monoton wachsend $\impl g\circ f: D_1\fromto\R$ ist konvex
        \item Falls $f$ konkav, $g$ konvex und monoton fallend $\impl g \circ f: D_1\fromto \R$ ist konvex
    \end{enumerate}

    \begin{proof}[Beweis (i)]
        Es seien $x,y\in D_1$ mit $x < y$ und $\lambda\in\pair{0,1}$ dann ist
        \begin{align*}
            \pair{g\circ f}\of{\lambda x + \pair{1-\lambda}\cdot y} &= g\of{f\of{\lambda x + \pair{1-\lambda}\cdot y}}
            \intertext{Wegen der Konvexität von $f$ und der Monotonie von $g$ gilt}
            \impl \pair{g\circ f}\of{\lambda x + \pair{1-\lambda}\cdot y} &\leq g\of{\lambda f\of{x} + \pair{1-\lambda}\cdot f\of{y}}\\
            \impl \pair{g\circ f}\of{\lambda x + \pair{1-\lambda}\cdot y} &\leq \lambda g\of{f\of{x}} + \pair{1-\lambda}\cdot g\of{f\of{y}}\\
            \impl \pair{g\circ f}\of{\lambda x + \pair{1-\lambda}\cdot y} &\leq \lambda\pair{g\circ f}\of{x} + \pair{1-\lambda}\cdot\pair{g\circ f}\of{y}\qedhere
        \end{align*}
    \end{proof}
    \noindent Der Beweis von (ii) funktioniert analog.
\end{lemma}

\begin{lemma} % Lemma 3
    \label{lemma:konvex-umkehrabbildung}
    Es seien $D, B\sbset\R$ Intervalle und $f: D\fromto B$ bijektiv mit $f^{-1}: B\fromto D$ Umkehrabbildung. Dann gilt
    \begin{enumerate}[label=(\roman*)]
        \item $f$ ist monoton wachsend und konvex $\equivalent f^{-1}$ ist monoton wachsend und konkav
        \item $f$ ist monoton fallend und konvex $\equivalent f^{-1}$ ist monoton fallend und konvex
        \item $f$ ist monoton fallend und konkav $\equivalent f^{-1}$ ist monoton fallend und konkav
    \end{enumerate}
\end{lemma}

\begin{uebung}
    Beweisen Sie Lemma~\ref{lemma:konvex-umkehrabbildung}.
\end{uebung}

\newpage

\subsection{Ungleichungen von Jensen und Hölder}
\begin{satz}[Ungleichung von Jensen]
    \label{satz:ungleichung-jensen}
    Sei $f: D\fromto\R$ konvex (konkav), sowie $x_1,\dots, x_n\in D$ und $\lambda_1, \dots, \lambda_n\in\pair{0,1}$ mit $ \sum_{}^{} \lambda_i = 1$. Dann ist
    \begin{align*}
        f\of{\sum_{i=1}^{n} \lambda_i x_i}\underset{(\geq)}{\leq}~ \sum_{i=1}^{n} \lambda_i f\of{x_i}
    \end{align*}

    \begin{proof}
        Wir nutzen vollständige Induktion.\\
        \begin{induktionsanfang}
            $n=2$ gilt wegen Definition der Konvexität.
        \end{induktionsanfang}
        \begin{induktionsvoraussetzung}
            Es gelte die Behauptung für ein festes, aber beliebiges $n\in\N$ mit $n\geq 2$.
        \end{induktionsvoraussetzung}
        \begin{induktionsschritt}
            $n\fromto n+1$
            \begin{align*}
                f\of{\sum_{i=1}^{n+1} \lambda_i x_i} &= f\of{\underbrace{\pair{1-\lambda_{n+1}}}_{1-\theta}\cdot \underbrace{\sum_{i=1}^{n} \frac{\lambda_i}{1-\lambda_{n+1}}\cdot x_i}_{x} + \underbrace{\lambda_{n+1}}_{\theta}\cdot \underbrace{x_{n+1}}_{y}}
                \intertext{Nach der Konvexität von $f$ gilt damit}
                &\leq \pair{1-\lambda_{n+1}}\cdot f\of{\sum_{i=1}^{n} \frac{\lambda_i}{1-\lambda_{n+1}}\cdot x_i} + \lambda_{n+1} \cdot f\of{x_{n+1}}\\
                \annot[{&}]{\leq}{IV} \pair{1-\lambda_{n+1}}\cdot \pair{\sum_{i=1}^{n} \frac{\lambda_i}{1-\lambda_{n+1}}\cdot f\of{x_i}} + \lambda_{n+1}\cdot f(x_{n+1})\\
                &= \sum_{i=1}^{n+1} \lambda_i f\of{x_i}
            \end{align*}
            Nach Induktion folgt die Behauptung.\qedhere
        \end{induktionsschritt}
    \end{proof}
\end{satz}


\begin{korollar}[Ungleichung vom arithmetisch-geometrischen Mittel] % Korollar 1
    Es seien $x_1, \dots, x_n\in \R$. Dann gilt
    \begin{align*}
        \sqrt[n]{\prod_{i=1}^{n} x_i} &\leq \frac{1}{n}\cdot \sum_{i=1}^{n} x_i
    \end{align*}
    \begin{proof}
        $f: \pair{0,\infty}$, $x\mapsto \ln\of{x}$ ist konkav.
        \begin{align*}
            \annot{\impl}{\ref{satz:ungleichung-jensen}} \ln\of{ \sum_{i=1}^{n} \lambda_i x_i} &\geq \sum_{i=1}^{n} \lambda_i \ln\of{x_i}\\
            \impl \sum_{i=1}^{n} \lambda_i x_i &\geq \exp\of{\sum_{i=1}^{n} \lambda_i\cdot \ln\of{x_i}}\\
            &= \prod_{i=1}^{n}  e^{\ln\of{x_i^{\lambda_i}}}\\
            &= \prod_{i=1}^{n}  x_i^{\lambda_i}
            \intertext{Wähle $\lambda_i = \frac{1}{n}$}
            \impl \frac{1}{n}\cdot\sum_{i=1}^{n} x_i &\geq \sqrt[n]{\prod_{i=1}^{n} x_i}\qedhere
        \end{align*}
    \end{proof}
\end{korollar}

\begin{korollar}[Ungleichung von Hölder\footnote{Otto Hölder (1859 - 1937), deutscher Mathematiker, der durch seine Arbeit zur Hölder-Ungleichung, Hölder-Stetigkeit und Kompositionsreihen bekannt ist und der Schwiegervater der Enkelin Marius Sophus Lies.}] % Korollar 2
    \label{korollar:hoelder}
    Seien $p,q\in\pair{1,\infty}$ mit $\frac{1}{p} + \frac{1}{q} = 1$, $x = \pair{x_1, \dots, x_n}\in\C^n$, $y = \pair{y_1, \dots, y_n}\in\C^n$. Wir definieren die $p$-Norm
    \begin{align*}
        \norm{x}_p &\definedas \pair{\sum_{i=1}^{n} \abs{x_i}^p}^{\frac{1}{p}}
        \intertext{Dann gilt}
        \sum_{i=1}^{n} \abs{x_i y_i} &\leq \norm{x}_p \cdot \norm{y}_q
    \end{align*}
    \begin{proof}
        Da $\frac{1}{p} + \frac{1}{q} = 1$ und $\ln$ konkav gilt für beliebige $\zeta, \eta\in\R_+$
        \begin{align*}
            \ln\of{\frac{1}{p} \cdot \zeta + \frac{1}{q} \cdot \eta} &\geq \frac{1}{p}\ln\of{\zeta} + \frac{1}{q}\ln\of{\eta}\\
            \impl \frac{\zeta}{p} + \frac{\eta}{q} &\geq \zeta^{\frac{1}{p}} \eta^{\frac{1}{q}}
            \intertext{Wir definieren}
            \zeta_k \definedas \frac{\abs{x_k}^p}{\norm{x}_p^p}\quad&\quad\eta_n \definedas \frac{\abs{y_n}^q}{\norm{y}_q^q}\\
            \impl \sum_{k=1}^{n} \frac{\abs{x_k}\abs{y_k}}{\norm{x}_p\norm{y}_q} &\leq \sum_{k=1}^{n} \frac{1}{p}\cdot\frac{\abs{x_k}^p}{\norm{x}_p^p} + \frac{1}{q}\cdot\frac{\abs{y_k}^q}{\norm{y}_q^q} = \frac{1}{p}+\frac{1}{q} = 1\\
            \impl \sum_{k=1}^{n} \abs{x_k}\abs{y_k} &\leq \norm{x}_p\cdot \norm{y}_q\qedhere
        \end{align*}
    \end{proof}
\end{korollar}
