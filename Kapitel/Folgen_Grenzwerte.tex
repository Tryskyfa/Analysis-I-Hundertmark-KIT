\subsection{Konvergenz}
\thispagestyle{pagenumberonly}

\begin{definition}[Reelle Folge]
    Eine reelle Folge ist eine Abbildung $\N\fromto\R$. Alternative Notation: $\pair{a_n}_{n\in\N}$, $\pair{a_1, a_2, \dots}$
\end{definition}

\begin{beispiel}
    $a_n = \frac{1}{n}$, $n\geq 1$\quad $\pair{1,\frac{1}{2},\frac{1}{3}, \dots}$
\end{beispiel}

\begin{definition}[Konvergenzkriterium]
    Sei $\pair{a_n}_{n\in\N}$ eine Folge reeller Zahlen. Die Folge heißt konvergent gegen $a\in\R$, falls gilt:
    \begin{align*}
        \forall \varepsilon > 0 ~\exists n_0\in\N\colon \abs{a_n-a} &< \varepsilon\text{ für alle $n>n_0$}
        \intertext{$a$ heißt Grenzwert von $\pair{a_n}_{n\in\N}$ und man schreibt}
        \lim_{n\fromto \infty} a_n &= a
    \end{align*}
\end{definition}

\begin{beispiel}[Nachweis von Konvergenz]
    Es sei
    \begin{align*}
        \pair{a_n}_{n\in\N}&=1+\frac{(-1)^n}{2n}
        \intertext{wir definieren $\varepsilon$ und $n_0$}
        \varepsilon > 0,&\quad n_0 = \frac{1}{2\varepsilon}\\
        \intertext{und wenden das Konvergenzkriterium an}
        \abs{a_n-1} = \abs{1+\frac{(-1)^n}{2n}-1} &= \abs{\frac{(-1)^n}{2n}} < \frac{\abs{(-1)^n}}{\abs{\frac{2}{2\varepsilon}}} = \varepsilon\\
        \impl \lim_{n\fromto\infty} a_n &=1
    \end{align*}
\end{beispiel}

\begin{bemerkung}[\anf{Fast alle Elemente}]
    Wir sagen, dass fast alle Element der Folge $(a_n)_{n\in\N}$ eine Eigenschaft (E) haben, wenn es höchstens eine endliche Anzahl von $a_n$ existiert, die die Eigenschaft (E) nicht erfüllen.
\end{bemerkung}

\begin{definition}[Alternativ formuliertes Konvergenzkriterium]
    Sei $(a_n)_{n\in\N}$ eine Folge, dann heißt diese Folge konvergent gegen $a\in\R$, wenn $\forall \varepsilon > 0$ und für fast alle $a_n$ gilt: $\abs{a_n-a}<\varepsilon$
\end{definition}

\begin{definition}[Nullfolge]
    Eine Folge, die gegen $0$ konvergiert, heißt Nullfolge.
\end{definition}

\begin{definition}[Divergenz]
    Eine nicht-konvergente Folge heißt divergent.
\end{definition}

\begin{definition}[$\varepsilon$-Umgebung]
    Für $\varepsilon > 0$, $a\in\R$ versteht man unter der $\varepsilon$-Umgebung von $a$ das Intervall $\left]a-\varepsilon, a+\varepsilon\right[$
\end{definition}

\subsection{Geometrische Bedeutung der Konvergenz}

\begin{visualisierung}
    ~
    \begin{figure}[H]
        \centering
        \begin{tikzpicture}
            \draw (0,0) -- (10,0);
            \draw (5,-0.25) node[below] {$a$} -- (5,0.25);
            \draw (3.8,-0.4) -- (3.9, -0.4) node[below] {$a-\varepsilon$}  -- (4,-0.4) -- (4,0.4) -- (3.8, 0.4);
            \draw (6.2,-0.4) -- (6.1, -0.4) node[below] {$a+\varepsilon$}  -- (6,-0.4) -- (6,0.4) -- (6.2, 0.4);
            \foreach \x in {1, 2.2, 3, 3.9, 4.3, 4.6, 4.8, 4.9, 4.94, 4.96, 4.98, 5.005, 5.01, 5.04, 5.1, 5.35, 5.85, 6.4}
            \draw (\x,-0.15) -- (\x,0.15);
        \end{tikzpicture}
        \caption{Geometrische Darstellung einer\\ konvergenten Folge und ihrer $\varepsilon$-Umgebung}
    \end{figure}
\end{visualisierung}

%%%%%%%%%%%%%%%%%%%%%%%%
% 21. November 2023
%%%%%%%%%%%%%%%%%%%%%%%%

\subsection{Eigenschaften von Folgen und Konvergenzen}

\begin{definition}[Beschränktheit von Folgen]
    \marginnote{[21. Nov]}
    Eine Folge $(a_n)_n$ heißt nach oben beschränkt, falls es ein $k\in\R$ gibt mit
    \begin{align*}
        \forall n\in \N\colon a_n \leq k\tag{$k$ ist obere Schranke für $(a_n)_n$}
    \end{align*}
    Beschränktheit nach unten wird analog definiert.\\
    $(a_n)_n$ heißt beschränkt, falls ein $k\geq 0$ existiert mit
    \begin{align*}
        \forall n\in\N\colon -k \leq a_n \leq k\quad \text{bzw.}\quad \forall n\in\N\colon \abs{a_n} \leq k
    \end{align*}
\end{definition}

\begin{satz}
    \label{satz:konv-folg-beschr}
    Jede konvergente Folge $(a_n)_{n\in\N}$ ist beschränkt.
    \begin{proof}
        Sei $a$ der Grenzwert von $(a_n)_n$~$\pair{\lim_{n\fromto\infty} a_n = a}$. Das heißt für $\varepsilon = 1$
        \begin{align*}
            \impl \exists N\in\N\colon \abs{a_n-a} &< 1\quad &\forall n\geq N\\[8pt]
            \abs{a_n} = \abs{a_n-a+a} \leq \abs{a_n-a} + \abs{a} &< 1+\abs{a} \quad &\forall n\geq N
            \intertext{Setze $k\definedas \max\pair{\abs{a_1}, \abs{a_2}, \dots, \abs{a_{N-1}}, 1+\abs{a}} \geq 0$}
            \impl \abs{a_n} &\leq k \quad&\forall n\in\N &\qedhere
        \end{align*}
    \end{proof}
\end{satz}

\begin{satz}[Eindeutigkeit des Limes]
    Der Limes einer konvergenten Folge $(a_n)_n$ ist eindeutig.
    \begin{proof}
        Annahme: $(a_n)_n\fromto a$ und $(a_n)_n\fromto b$ mit  $a\neq b$. Dann gilt \OBDA, dass $a<b$. Setzen $\varepsilon = \frac{b-a}{2} > 0$.
        \begin{align*}
            \impl &\exists N_1 \in\N\colon \abs{a_n-a} < \varepsilon \quad \forall n\geq N_1\\
            \impl &\exists N_2 \in\N\colon \abs{a_n-b} < \varepsilon \quad \forall n\geq N_2\\[10pt]
            &N\definedas \max\pair{N_1, N_2}\\
            \impl &\forall n\geq N\colon a_n < a+\varepsilon = a + \frac{b-a}{2} = \frac{b+a}{2} = b-\varepsilon\\
            &\forall n\geq N\colon a_n > b-\varepsilon = \frac{b+a}{2}\\
            \impl &a_n <\frac{b+a}{2} < a_n \quad \forall n\geq N\qquad\text{ (Widerspruch)}\qedhere
        \end{align*}
    \end{proof}
\end{satz}

\begin{satz}[Eigenschaften von konvergenten Folgen]
    \label{satz:konvergenzsaetze}
    Seien $(x_n)_n, (y_n)_n$ konvergente reelle Folgen mit $x_n\fromto a$, $y_n\fromto b$ für $n\fromto\infty$. Dann gilt:
    \begin{enumerate}[label=(\alph*)]
        \item $x_n+y_n\fromto a+b$\quad ($\lim\pair{x_n+y_n}=\lim x_n + \lim y_n$).
        \item $x_n\cdot y_n \fromto a\cdot b$\quad ($\lim\pair{x_n\cdot y_n} = \lim x_n \cdot \lim y_n$).
        \item $\lambda\cdot y_n \fromto \lambda\cdot b$\quad ($\lim\pair{\lambda\cdot y_n} = \lambda \cdot \lim y_n$). ($\lambda\in\R$)
        \item Ist $b\neq 0$, so ist $y_n \neq 0$ für fast alle $n\in\N$ und $\frac{x_n}{y_n}$ ist für fast alle $n$ definiert und $\lim\pair{\frac{x_n}{y_n}} = \frac{\lim x_n}{\lim y_n} = \frac{a}{b}$.
        \item $\abs{x_n} \fromto \abs{a}$\quad ($\lim \abs{x_n} = \abs{\lim x_n}$)
        \item Ist $x_n \leq y_n$ für fast alle $n\in\N$ $\impl a = \lim x_n \leq b = \lim y_n$.
    \end{enumerate}

    \begin{proof}
        \theoremescape
        \begin{enumerate}[label=(\alph*)]
            \item Nach Satz~\ref{satz:diff-abs} gilt
            \begin{align*}
                \abs{x_n+y_n - (a+b)} = \abs{x_n - a + (y_n-b)} &\leq \abs{x_n - a} + \abs{y_n-b}
                \intertext{Aufgrund der Konvergenz der Folgen gilt}
                \forall \varepsilon > 0~\exists N_1\colon \abs{x_n - a} &< \frac{\varepsilon}{2} \quad \forall n\geq N_1\\
                \forall \varepsilon > 0~\exists N_2\colon \abs{y_n - b} &< \frac{\varepsilon}{2} \quad \forall n\geq N_2\\
                \intertext{Wir wählen $N=\max\pair{N_1, N_2}$}
                \impl \abs{\pair{x_n+y_n}-(a+b)} \leq \abs{x_n - a} + \abs{y_n-b} &\leq \frac{\varepsilon}{2} + \frac{\varepsilon}{2} = \varepsilon\quad \forall n\geq N
            \end{align*}
            \item Umformen, um eine passende Ungleichung zu erreichen
            \begin{align*}
                x_n \cdot y_n - ab &= \pair{x_n-a+a} \cdot y_n - ab\\
                &= \pair{x_n-a}\cdot y_n + a \cdot y_n - ab\\
                &= \pair{x_n-a}\cdot y_n + a\cdot \pair{y_n-b}\\[10pt]
                \impl \abs{x_n\cdot y_n - ab} &= \abs{\pair{x_n-a}\cdot y_n + a\cdot \pair{y_n-b}}\\
                &\leq \abs{x_n-a}\cdot\abs{y_n}+\abs{a}\cdot\abs{y_n-b}
                \intertext{Nach Satz~\ref{satz:konv-folg-beschr} ist $y_n$ beschränkt}
                \exists k\geq 0\colon \abs{y_n} &\leq k \quad\forall n\in\N
                \intertext{Nach der Konvergenz der Folgen gilt außerdem}
                \forall \varepsilon >0~\exists N_1, N_2\colon \abs{x_n-a} &< \frac{\varepsilon}{2(k+1)}\quad\forall n\geq N_1\\
                \abs{y_n-b} &< \frac{\varepsilon}{2(\abs{a}+1)}\quad\forall n\geq N_2\\[10pt]
                \impl \forall n\geq \max\pair{N_1, N_2}\colon \abs{x_n\cdot y_n-ab} &\leq \frac{\varepsilon}{2(k+1)}\cdot k + \abs{a}\cdot\frac{\varepsilon}{2(\abs{a}+1)} \leq \varepsilon
            \end{align*}
            \item Setze $y_n=\lambda \fromto \lambda$ und verwende (b)
            \item Wir wählen $\varepsilon\definedas\frac{\abs{b}}{2}$
            \begin{align*}
                \exists N\in\N\colon \abs{y_n-b} &< \varepsilon = \frac{\abs{b}}{2}\quad\forall n\geq \N.\\[10pt]
                \forall n\geq N\colon\quad\abs{y_n} &= \abs{y_n-b+b} = \abs{b+(y_n-b)}\\
                &\geq \abs{b} - \abs{y_n-b} > \abs{b} - \frac{\abs{b}}{2} + \frac{\abs{b}}{2}\\
                &> 0
            \end{align*}
            $\impl y_n \neq 0$ für fast alle $n$ und somit $\frac{x_n}{y_n}$ wohldefiniert für fast alle $n\in\N$
            \begin{align*}
                \intertext{Wir betrachten den Spezialfall $x_n = 1$}
                \abs{\frac{1}{y_n} - \frac{1}{b}} &= \abs{\frac{b-y_n}{y_n\cdot b}} = \frac{\abs{b-y_n}}{\abs{y_n}\cdot\abs{b}}.\\
                \intertext{Wir wissen schon, dass $\exists N_1\colon \abs{y_n} \geq \frac{\abs{b}}{2} \quad\forall n\geq N_1$}
                \impl \forall n\geq N_1\colon \abs{\frac{1}{y_n}-\frac{1}{b}} &\leq \frac{\abs{y_n-b}}{\abs{y_n}\cdot\abs{b}} \leq \frac{2\cdot\abs{y_n-b}}{\abs{b}\cdot \abs{b}}\\
                \forall \varepsilon > 0~\exists N_2\colon \abs{y_n-b} &< \frac{\abs{b}^2}{2}\cdot\varepsilon\\
                \impl \forall n\geq \max\pair{N_1, N_2}\colon \abs{\frac{1}{y_n}-\frac{1}{b}} &< \frac{2}{\abs{b}^2} \cdot \frac{\abs{b}^2}{2} \cdot \varepsilon = \varepsilon
                \intertext{Jetzt wenden wir (b) an und dann gilt}
                \lim_{n\fromto \infty} \frac{x_n}{y_n} = \lim_{n\fromto \infty} \pair{x_n\cdot\frac{1}{y_n}} &= \lim x_n \cdot \lim\frac{1}{y_n} = a\cdot \frac{1}{b} = \frac{a}{b}
            \end{align*}
            \item Wir zeigen mit der Umkehrung der Dreiecksungleichung
            \begin{align*}
                \abs{\abs{x_n}-\abs{a}} \annot[{&}]{\leq}{Dreiecks.} \abs{x_n-a}\\
                \impl \abs{x_n} &\fromto \abs{a}
            \end{align*}
            \item Angenommen $a > b$. Wir wählen $\varepsilon=\frac{a-b}{2}$.
            \begin{align*}
                \impl \exists N_1\colon \abs{x_n-a} &< \varepsilon \quad\forall n\geq N_1\\
                \impl \exists N_2\colon \abs{y_n-b} &< \varepsilon \quad\forall n\geq N_2\\
                \intertext{Für $n\geq \max\pair{N_1,N_2}$ folgt}
                x_n &> a-\varepsilon = a + \frac{a-b}{2}\\
                &= \frac{a+b}{2} = b+\frac{a-b}{2} = b + \varepsilon\\
                &> y_n\\[10pt]
                \impl x_n &> y_n \tag{Widerspruch, Annahme falsch}\\
                \impl a &\leq b
            \end{align*}
        \end{enumerate}
    \end{proof}
\end{satz}

\newpage

\begin{satz}[Sandwich Satz]
    \label{satz:sandwich}
    Seien $(a_n)_n$, $(b_n)_n$, $(c_n)_n$ Folgen mit $\lim a_n = a$, $\lim \pair{b_n-a_n} = 0$ und $a_n \leq c_n \leq b_n$ für fast alle $n\in\N$.\\
    Dann folgt, dass $(c_n)_n$ und $(b_n)_n$ jeweils gegen $a$ konvergieren.
    \begin{proof}
        ~\\
        1. Schritt: $\lim b_n = a$.\\[10pt]
        Sei $\varepsilon > 0$. Da $a_n\fromto a$, $\exists N_1\colon \abs{a_n-a} < \frac{\varepsilon}{2}\quad\forall n\geq N_1$\\
        $0\leq b_n-a_n$ ist Nullfolge $\impl \exists N_2\colon \abs{b_n-a_n} < \frac{\varepsilon}{2}\quad\forall n\geq N_2$\\
        \begin{align*}
            \abs{b_n-a} &= \abs{b_n-a_n+a_n-a}\\
            &\leq \abs{b_n - a_n} + \abs{a_n-a}\\
            &\leq \frac{\varepsilon}{2} + \frac{\varepsilon}{2} = \varepsilon\quad \forall n\geq \max\pair{N_1, N_2}
        \end{align*}
        \noindent 2. Schritt: $\lim c_n = a$.
        \begin{align*}
            \lim a_n &= a = \lim b_n\\
            \impl \forall\varepsilon > 0~\exists N_1, N_2\colon \abs{a_n-a} &< \varepsilon \quad\forall n\geq N_1\\
            \text{ und } \abs{b_n-a} &< \varepsilon \quad\forall n\geq N_2\\[10pt]
            a-\varepsilon < a_n &< a + \varepsilon\quad\forall n\geq N_1\\
            a-\varepsilon < b_n &< a + \varepsilon\quad\forall n\geq N_2\\
            \intertext{Auch gilt $a_n \leq c_n \leq b_n~\forall n\geq N_3$ und wir definieren $N\definedas\max\pair{N_1, N_2, N_3}$}
            \impl a-\varepsilon < a_n \leq c_n &\leq b_n < a + \varepsilon\quad\forall n\geq N
        \end{align*}
        Das heißt $\abs{c_n-a} < \varepsilon~\forall n\geq N$. Das heißt $\lim c_n = a$\qedhere
    \end{proof}
\end{satz}

\subsection{Monotone Konvergenz}

Wir sind bisher immer davon ausgegangen, dass wir den Grenzwert einer Folge bereits kennen. Das folgende Unterkapitel beschäftigt sich damit, die Konvergenz einer Folge nachzuweisen, wenn deren Grenzwert nicht bekannt ist.

\begin{definition}[Monotonie]
    Eine reelle Folge $(a_n)_n$ heißt
    \begin{enumerate}[label=(\roman*)]
        \item monoton wachsend, wenn $a_n \leq a_{n+1}\quad\forall n\in\N$
        \item streng monoton wachsend, wenn $a_n < a_{n+1}\quad\forall n\in\N$
        \item monoton fallend, wenn $a_n \geq a_{n+1}\quad\forall n\in\N$
        \item streng monoton fallend, wenn $a_n > a_{n+1}\quad\forall n\in\N$
    \end{enumerate}
    \noindent Wir nennen $(a_n)_n$ (streng) monoton, falls sie (streng) monoton wachsend oder fallend ist.
\end{definition}

\begin{satz}[Monotone Konvergenz]
    \label{satz:monoton-konv}
    Eine monoton wachsende Folge $(a_n)_n$ konvergiert genau dann, wenn $(a_n)_n$ nach oben beschränkt ist.\\
    Und eine monoton fallende Folge $(a_n)_n$ konvergiert genau dann, wenn sie nach unten beschränkt ist.
\end{satz}

\newpage

\begin{lemma}[Hilfaussage für monotone Konvergenz]
    ~\label{lemma:hilf-monoton-konv}
    Jede nach oben (unten) beschränkte Folge besitzt eine kleinste obere (größte untere) Schranke.
    \begin{proof}
        Sei $(a_n)_n$ nach oben beschränkt und $S\definedas\set{c\in\R|~a_n \leq c~\forall n} \neq \emptyset\footnote{Folgt aus der Beschränktheit}.$ Dann ist $S$ nach unten beschränkt, da $\forall c\in S\colon a_1 \leq c$.
        \begin{align*}
            &\impl a\definedas \inf S \text{ existiert}
            \intertext{Behauptung: $a$ ist obere Schranke für $(a_n)_n$, das heißt $a\in S$. Annahme: $a\notin S$ ($a$ ist keine obere Schranke)}
            &\impl \exists n_0 \in\N\colon a_{n_0} > a
            \intertext{Wir setzen $\varepsilon\definedas a_{n_0} - a > 0$}
            &\impl \exists c\in S\colon c < a + \varepsilon = a + a_{n_0} - a = a_{n_0}
        \end{align*}
        Widerspruch zu $c\in S$. Das heißt $a$ ist obere Schranke für $(a_n)_n$.
    \end{proof}
\end{lemma}

\begin{proof}[Beweis von Satz~\ref{satz:monoton-konv}]
    Angenommen $(a_n)_n$ ist monoton wachsend und nach oben beschränkt.
    \begin{align*}
        &a\definedas\sup_{n\in\N} a_n\\
        \impl &a_n \leq a\quad\forall n\in\N
        \intertext{Sei $\varepsilon > 0 \impl a-\varepsilon$ keine obere Schranke für $(a_n)_n$ mehr.}
        \impl &\exists N\in\N\colon a_N \geq a-\varepsilon\\
        \intertext{Sei $n\geq N$. Dann gilt:}
        &a_N \leq a_{N+1} \leq a_{N+2} \leq \dots \leq a_{N+k} = a_n\\
        \impl &\forall n\geq N\colon a-\varepsilon < a_n\\
        \impl &\abs{a_n-a} < \varepsilon\qedhere
    \end{align*}
\end{proof}

\begin{proof}[Alternativer Beweis]
    \footnote{Dieser Beweis bezieht sich laut Vorlesung auf Lemma 14, welches allerdings nicht existiert. Gemeint ist vermutlich Lemma 15 (hier Lemma~\ref{lemma:hilf-monoton-konv})}
    ~\\
    Sei $(a_n)_n$ eine Folge mit $a_n = f(a),~f:\N\fromto\R$
    \begin{align*}
        \impl \bild(f) &= f(\N) \text{ nach oben beschränkt }\\
        \impl a&\definedas\sup\pair{f\pair{\N}}\geq a_n \quad\forall n\in\N\qedhere
    \end{align*}
\end{proof}

\newpage

%%%%%%%%%%%%%%%%%%%%%%%%
% 23. November 2023
%%%%%%%%%%%%%%%%%%%%%%%%

\begin{beispiel}[Berechnen von $\sqrt {c}$ für $c>0$]
    \marginnote{[23. Nov]}
    \begin{align*}
        x^2 = c &\equivalent x = \frac{c}{x}\tag{$x >0$}\\
        &\equivalent x=\frac{1}{2}\pair{x+\frac{c}{x}}
    \end{align*}
    Folge $x_n$. Wählen $x_0 > 0$. Für $n\in\N$ sei $x_n=\frac{1}{2}\pair{x_{n-1}+\frac{c}{x_{n-1}}}$\\
    Behauptung 1: $x_n \geq \sqrt{c}\quad\forall n\in\N$
    \begin{proof}
        \begin{align*}
            x_1 &= \frac{1}{2}\pair{x_0+\frac{c}{x_0}}
        \end{align*}
        \begin{mdframed}
            Einschub: Arithmetisch-geometrisches Mittel (AGM)
            \begin{align*}
                0 \leq (x-y)^2 &= x^2-2xy+y^2 \equivalent xy \leq \frac{x^2+y^2}{2}
                \intertext{Wir setzen $x=\sqrt{a}$ und $y=\sqrt{b}$}
                \impl \forall a,b \geq 0\colon \sqrt {ab} &\leq \frac{a+b}{2}
            \end{align*}
        \end{mdframed}

        \begin{align*}
        (x_1)
            ^2 &= \pair{\frac{1}{2}\pair{x_0+\frac{c}{x_0}}}^2 \annot{\geq}{AGM} x_0\cdot\frac{c}{x_0} = c\\
            \impl x_1 &\geq \sqrt{c}\\
            \intertext{Zu zeigen: Falls $x_n \geq 0\impl x_{n+1}\geq \sqrt{c}$}
            \pair{x_{n+1}}^2 &= \pair{\frac{1}{2}\pair{x_n + \frac{c}{x_n}}}^2 \annot{\geq}{AGM} x_n \cdot \frac{c}{x_n} = c\qedhere
        \end{align*}
    \end{proof}
    \vspace{0.5cm}

    \noindent Behauptung 2: $(x_n)_n$ ist monoton fallend.
    \begin{proof}
        \begin{align*}
            x_{n+1} &= \frac{1}{2}\pair{x_n+\frac{c}{x_n}}\\
            (x_n)^2 &= \pair{\frac{1}{2}\pair{x_{n+1} + \frac{c}{x_{n+1}}}}^2 \geq c\tag{$n\geq 1$}\\
            \impl x_n &\geq \frac{c}{x_n}\\
            \impl x_{n+1} &= \frac{1}{2}\pair{x_n+\frac{c}{x_n}}\leq \frac{1}{2}\pair{x_n+x_n} = x_n\qedhere
        \end{align*}
    \end{proof}

    \noindent Nach dem Satz der monotonen Konvergenz (\ref{satz:monoton-konv}) existiert ein Grenzwert $x$ mit
    \begin{align*}
        x&\definedas \lim_{n\fromto\infty} x_n = \lim_{n\fromto\infty} x_{n-1}\\
        x &= \lim_{n\fromto\infty} x_n = \lim_{n\fromto\infty} \frac{1}{2}\pair{x_{n-1}+\frac{c}{x_{n-1}}} = \frac{1}{2}\pair{x+\frac{c}{x}}\\
        \equivalent x &= \sqrt{c}
    \end{align*}
\end{beispiel}

\begin{beispiel}[Harmonische Folge]
    Es sei $x_n = \frac{1}{n}\fromto 0$
    \begin{align*}
        \abs{\frac{1}{n}-0} &= \frac{1}{n}
        \intertext{geg. $\varepsilon > 0$ wähle $N\in\N$ mit $N>\frac{1}{\varepsilon} \equivalent \varepsilon > \frac{1}{N}$}
        \impl \forall n\geq N\colon \abs{\frac{1}{n}-0} &= \frac{1}{n} \leq \frac{1}{N}< \varepsilon
    \end{align*}
    Ähnlich funktioniert $x_n = \frac{1}{\sqrt {n}}$
\end{beispiel}

\begin{beispiel}
    \begin{align*}
        x_n&\definedas \frac{1+2+3+\dots+n}{n^2} \fromto \frac{1}{2}\text{ für }n\fromto\infty
        \intertext{Mit Gaußscher Summenformel ($1+2+3+\dots+n = \frac{n\cdot(n+1)}{2}$)}
        \impl x_n &= \frac{\frac{n\cdot(n+1)}{2}}{n^2} = \frac{n^2+n}{2n^2} = \frac{1+\frac{1}{n}}{2} \fromto \frac{1}{2}
    \end{align*}
\end{beispiel}

\begin{beispiel}[Geometrische Folge (1)]
    \label{beispiel:geometrische-folge}
    Sei $0\leq q< 1$. Dann gilt $x_n \definedas q^n \fromto 0$
    \begin{proof}
        Ist $q=0\impl x_n = 0^n = 0 \fromto 0$. Also sei $0<q<1$
        \begin{align*}
            \impl \frac{1}{q} &> 1\\
            h &\definedas\frac{1}{q}-1>0\\
            \impl \frac{1}{q} &= 1 + h\\
            \impl q &= \frac{1}{1+h}\\
            \impl x_n &= q^n = \pair{\frac{1}{1+h}}^n = \frac{1}{(1+h)^n}
            \intertext{Nach Bernoulli (\ref{satz:bernoulli-ungleichung}) gilt $(1+h)^n \geq 1+nh>nh$}
            \impl q^n &= \frac{1}{(1+h)^n}< \frac{1}{nh}\fromto 0 \text{ für } \ntoinf\\
            \impl \abs{q^n-0} &= q^n < \frac{1}{nh} \fromto 0 \text{ für } \ntoinf
            \intertext{Für $\varepsilon>0$ wähle $N\geq \frac{1}{\varepsilon\cdot h}\equivalent \frac{1}{N\cdot h}\leq \varepsilon$}
            \impl \forall n\geq N&\colon \abs{q^n-0} = q^n \leq \frac{1}{nh} \leq \frac{1}{Nh} < \varepsilon\qedhere
        \end{align*}
    \end{proof}
\end{beispiel}

\begin{uebung}[Geometrische Folge (2)]
    Es sei $-1<q<1$. Weisen Sie basierend auf Beispiel~\ref{beispiel:geometrische-folge} nach, dass dann $x_n \definedas q^n \fromto 0$ gilt.
\end{uebung}

\newpage

\begin{beispiel}
    Sei $a>0$, $x_n\definedas a^\frac{1}{n} = \sqrt[n]{a} \fromto 1$ für $\ntoinf$.\\
    \begin{proof}
        Wir unterscheiden in drei Fälle.
        \theoremescape
        \begin{enumerate}[label=\arabic*.]
            \item $a=1$
            \begin{align*}
                \impl x_n &= 1
                \intertext{\item $a> 1$}
                \impl a^\frac{1}{n} &> 1^\frac{1}{n}=1\\
                h_n &\definedas a^\frac{1}{n} - 1 > 0 \text{ und } 1+h_n = a^\frac{1}{n}\\
                a&=\pair{1+h_n}^n \geq 1+n\cdot h_n > n\cdot h_n\\
                \impl h_n &< \frac{a}{n}\\
                \abs{a^\frac{1}{n}-1} &= a^\frac{1}{n}-1 = h_n < \frac{a}{n}\\
                \impl \lim_{\ntoinf} a^\frac{1}{n}&= 1
                \intertext{\item $0<a<1$}
                b &\definedas \frac{1}{a} > 1\\
                \impl \lim_{\ntoinf} b^\frac{1}{n} &= 1\\
                \lim_{\ntoinf} b^\frac{1}{n} &=\lim_{\ntoinf} \frac{1}{a^\frac{1}{n}}\\
                \impl \lim_{\ntoinf} a^{\frac{1}{n}} &= 1\qedhere
            \end{align*}
        \end{enumerate}
    \end{proof}
\end{beispiel}

\begin{beispiel}
    Es sei $x_n = \sqrt[n]{n} = n^\frac{1}{n}\fromto 1$
    \begin{align*}
        n^{\frac{1}{n}} &> 1 \text{ für } n\geq 2\\
        h_n &\definedas n^{\frac{1}{n}} - 1\\
        \impl n = \pair{n^\frac{1}{n}}^n = \pair{1+h_n}^n \annot[{&}]{\geq}{\ref{satz:bernoulli-ungleichung}} 1+n\cdot h_n > n\cdot h_n\\
        \impl h_n &\leq \frac{n}{n} = 1
        \intertext{Wir wenden den Binomischen Lehrsatz (\ref{satz:binom-lehrsatz}) an}
        n = \pair{1+h_n}^n &= \sum_{k=0}^{n} \underbrace{\binom{n}{k}\pair{h_n}^k}_{\geq 0}\\
        &\geq \binom{n}{0}\cdot \pair{h_n}^0 \binom{n}{1}\cdot \pair{h_n}^1 + \binom{n}{2}\cdot \pair{h_n}^2\tag{$n\geq 2$}\\
        &= 1 + n\cdot h_n + \frac{n\cdot(n-1)}{2}\cdot(h_n)^2\\
        &> \frac{n\cdot (n-1)}{2}\cdot \pair{h_n}^2\\
        \impl h_n^2 &< \frac{2n}{n\cdot (n-1)} = \frac{2}{n-1}\\
        0 &< h_n < \sqrt{\frac{2}{n-1}} \fromto 0 \text{ für } \ntoinf\\
        \impl \lim h_n &= 0\\
        \equivalent \lim n^{\frac{1}{n}} &= 1
    \end{align*}
\end{beispiel}

\begin{definition}[Divergenz gegen unendlich]
    Eine reelle Folge $(x_n)_n$ strebt gegen unendlich, falls
    \begin{align*}
        &\forall k \geq 0~\exists N\in\N\colon x_n \geq k\quad \forall n> N\\[8pt]
        (\equivalent &\forall k \geq 0 \text{ ist } x_n < k \text{ für endlich viele } n\in\N)
    \end{align*}
    Die Folge $(x_n)_n$ strebt gegen $-\infty$, falls $(-x_n)_n$ gegen $\infty$ strebt.
\end{definition}

\begin{notation}
    Wenn die Folge $(x_n)_n$ gegen unendlich strebt, schreiben wir $x_n\fromto\infty$ oder $\lim_{n\fromto\infty} x_n = +\infty$.
\end{notation}

\begin{beispiel}[Monoton wachsende, divergente Folge]
    $x_n=n$ divergiert gegen $\infty$
\end{beispiel}


\begin{satz}[Eigenschaften von Kehrwerten von Folgen]
    \theoremescape
    \begin{enumerate}[label=(\alph*)]
        \item Falls $x_n\fromto \infty$ für $\ntoinf$, so folgt $\frac{1}{x_n}\fromto 0$ für $\ntoinf$.
        \item Ist $(x_n)_n$ eine Nullfolge mit $x_n > 0$ für fast alle $n\in\N$, so ist $\frac{1}{x_n}\fromto \infty$ für $\ntoinf$.\\
        Falls $x_n < 0$ für fast alle $n$, so folgt $\frac{1}{x_n} \fromto -\infty$ für $\ntoinf$.
    \end{enumerate}

    \begin{proof}[Beweis (a)]
        Sei $x_n\fromto \infty$
        \begin{align*}
            \impl \forall \varepsilon > 0~\exists N\in\N\colon x_n &> \frac{1}{\varepsilon}\quad\forall n \geq N\\
            \impl 0 < \frac{1}{x_n} &< \varepsilon\quad \forall n \geq N
        \end{align*}
        das heißt $\frac{1}{x_n}\fromto 0$.
    \end{proof}
\end{satz}

\begin{uebung}
    Weisen Sie den Teil (b) des vorherigen Satzes nach.
\end{uebung}

\begin{beispiel}
    $\frac{n}{2^n}\fromto 0$ für $\ntoinf$ (Sogar $\forall k\in\N\colon \frac{n^k}{2^n} \fromto 0$ für $\ntoinf$)

    \begin{proof}
        Zu zeigen: $x_n = \frac{2^n}{n}\fromto \infty$
        \begin{align*}
            2^n = (1+1)^n &= \sum_{k=0}^{n} \binom{n}{k}\geq \binom{n}{2} = \frac{n\cdot (n-1)}{2}\\
            &\impl x_n = \frac{2^n}{n}\geq \frac{\frac{n\cdot (n-1)}{2}}{n} = \frac{1}{2}\cdot (n-1)\\
            &\impl x_n \fromto \infty\\
            &\impl \frac{1}{x_n}\fromto 0\qedhere
        \end{align*}
    \end{proof}
\end{beispiel}

\begin{beispiel}[Die Eulersche Zahl $e$]
    \begin{align*}
        a_n &\definedas \pair{1+\frac{1}{n}}^n\quad b_n \definedas \pair{1 + \frac{1}{n}}^{n+1}\\
        \impl a_n &< b_n\quad\forall n \in\N
    \end{align*}
    Behauptung: $\forall n\in\N\colon a_n < a_{n+1}$ und $b_n > b_{n+1}$.
    \begin{proof}
        Sei $n\geq 2$

        \begin{align*}
            \frac{a_n}{a_{n-1}} &= \frac{\pair{1+\frac{1}{n}}^n}{\pair{1+\frac{1}{n-1}}^{n-1}} = \pair{1+\frac{1}{n-1}} \cdot \frac{\pair{1+\frac{1}{n}}^n}{\pair{1+\frac{1}{n-1}}^n}\\
            &= \frac{n}{n-1} \cdot \pair{\frac{\frac{n+1}{n}}{\frac{n}{n-1}}}^n = \frac{n}{n-1}\cdot \pair{\frac{(n+1)\cdot (n-1)}{n^2}}^n\\
            &= \frac{n}{n-1}\cdot \pair{\frac{n^2-1}{n^2}}^n = \frac{n}{n-1}\cdot\pair{1-\frac{1}{n^2}}^n\\
            \annot[{&}]{\geq}{\ref{satz:bernoulli-ungleichung}} \frac{n}{n-1}\cdot\pair{1-\frac{n}{n^2}} = \frac{n}{n-1}\cdot\pair{1-\frac{1}{n}} = 1\\
            \impl a_n &> a_{n-1}\quad\forall n\geq 2\\[10pt]
            \frac{b_{n-1}}{b_n} &= \frac{\pair{1+\frac{1}{n-1}}^n}{\pair{1+\frac{1}{n}}^{n+1}} = \pair{1+\frac{1}{n}}^{-1}\cdot\pair{\frac{1+\frac{1}{n-1}}{1+\frac{1}{n}}}^n\\
            &= \frac{n}{n+1}\cdot\pair{\frac{\frac{n}{n-1}}{\frac{n+1}{n}}}^n = \frac{n}{n+1}\cdot\pair{\frac{n^2}{(n+1)\cdot(n-1)}}^n\\
            &= \frac{n}{n+1}\cdot\pair{\frac{n^2-1+1}{n^2-1}}^n = \frac{n}{n+1}\cdot\pair{1+\frac{1}{n^2-1}}^n\\
            &> \frac{n}{n+1}\cdot\pair{1+\frac{1}{n^2}}^n \annot{\geq}{\ref{satz:bernoulli-ungleichung}} \frac{n}{n+1}\cdot\pair{1+\frac{n}{n^2}} = 1\\
            \impl b_n & < b_{n-1}\quad\forall n\in\N\\[10pt]
            \impl a_1 < a_2 < \dots < a_n &< b_n < \dots < b_2 < b_1
            \intertext{Mit dem Satz der monotonen Konvergenz (\ref{satz:monoton-konv}) folgt daraus}
            e &\definedas \lim_{n\fromto\infty} a_n = \lim_{n\fromto\infty} \pair{1+\frac{1}{n}}^n \text{ existiert } \\
            b &\definedas \lim_{n\fromto\infty} b_n \text{ existiert und sogar } b = e
            \intertext{Da $1<\frac{b_n}{a_n} = \frac{\pair{1+\frac{1}{n}}^{n-1}}{\pair{1+\frac{1}{n}}^n} = 1 + \frac{1}{n}\fromto 1$}
            \impl 1 = \lim_{n\fromto\infty} \frac{b_n}{a_n} &= \frac{\lim_{n\fromto\infty} b_n}{\lim_{n\fromto\infty} a_n} = \frac{b}{e}\\[10pt]
            \impl b &= e\qedhere\\[10pt]
            \pair{1+\frac{1}{n}}^n &< e < \pair{1+\frac{1}{n}}^{n+1}
        \end{align*}
    \end{proof}
\end{beispiel}

\newpage

\subsection{Häufungswerte und Teilfolgen}

Wir möchten eine Folge $(a_n)_n$ \anf{massieren}.
\begin{align*}
    a_n &= f(n), \quad f: \N \fromto \R
\end{align*}

\noindent Wir wollen die Folgeglieder umordnen oder auch beliebige weglassen. Wie machen wir das und wie lässt sich das ausdrücken?

\begin{definition}[Umordnung]
    Sei $(a_n)_n$ eine reelle Folge. Eine Umordnung ist gegeben durch eine Bijektion
    \begin{align*}
        \sigma: \N\fromto \N\\
        b_n \definedas a_{\sigma(n)} \tag{Umordnung von $(a_n)_n$}
    \end{align*}
\end{definition}

\begin{definition}[Ausdünnung]
    Es sei $\kappa: \N\fromto\N$ streng monoton wachsend
    \begin{align*}
        b_n \definedas a_{\kappa(n)}\tag{Teilfolge von $(a_n)_n$}
    \end{align*}
\end{definition}

\begin{satz}[Konvergenz von Teilfolgen und Umordnungen]
    \label{satz:konv-teilfolgen-umordnungen}
    Für jede konvergente reelle Folge $(a_n)_n$ konvergiert jede Umordnung und jede Teilfolge gegen den selben Grenzwert.

    \begin{proof}[Beweis\footnotemark.]
        \footnotetext{Nachtrag vom 28. November 2023.}
        Sei $(a_n)_n$ konvergent gegen $a$ und $\kappa: \N\fromto\N$ monoton wachsend ($\kappa(n+1) > \kappa(n)$).\\
        $\impl \kappa(j) \geq j\quad\forall j \in\N$\footnotemark
        \footnotetext{Lässt sich per Induktion nachweisen}
        \begin{align*}
            b_j &\definedas a_{\kappa(j)}
            \intertext{$a_n\fromto a$, das heißt}
            \forall \varepsilon > 0~\exists N\in\N&\colon a-\varepsilon < a_n < a + \varepsilon\quad\forall n \geq N\\
            \impl \forall \varepsilon > 0~\exists N\in\N&\colon a-\varepsilon < a_{\kappa(j)} < a + \varepsilon\quad\forall j \geq N\\[10pt]
            \impl \lim_{j\fromto\infty} a_{\kappa(j)} &= a\quad\text{ d.h. } \lim_{j\fromto\infty} b_j = a
        \end{align*}
        Für Umordnung: Sei $\sigma: \N\fromto\N$ Bijektion.
        \begin{align*}
            b_j &\definedas a_{\sigma(j)}\\
            \intertext{Wir haben $\forall \varepsilon > 0~\exists N\in\N\colon a-\varepsilon < a_n < a+\varepsilon$ und betrachten das Urbild}
            A&\definedas \sigma^{-1}\pair{\set{1,2,3, \dots, N}}\subseteq \N\\
            \intertext{$A$ hat endlich viele Elemente}
            L&\definedas \max\pair{\sigma^{-1}(1), \sigma^{-1}(2), \dots, \sigma^{-1}(N)}\\
            j \geq L &\impl \sigma(j) \geq N\\
            &\impl \forall j\geq L\colon a-\varepsilon < a_{\sigma(j)} < a + \varepsilon\\
            &\impl \lim_{j\fromto\infty} a_{\sigma(j)} = a\qedhere
        \end{align*}
    \end{proof}
\end{satz}

\begin{uebung}
    Weisen Sie nach, dass sich der Grenzwert einer Folge nicht verändert, wenn man endlich viele Elemente ändert.
\end{uebung}

\begin{definition}[Häufungswert]
    Sei $(a_n)_n$ reelle Folge. Eine reelle Zahl $a$ heißt Häufungswert (oder Häufungspunkt) von $a_n$, falls $\forall \varepsilon >0$ unendlich viele $a_n$ in $\pair{a-\varepsilon, a+\varepsilon}$ liegen. Das heißt:
    \begin{align*}
        a-\varepsilon < a_n < a+ \varepsilon\text{ für unendlich viele $n$}
    \end{align*}
    Das heißt $\forall L\in\N~\exists n> L\colon a-\varepsilon < a_n < a+\varepsilon$.
\end{definition}

\begin{beispiel}
    \theoremescape
    \begin{enumerate}
        \item $a_n=\frac{1}{n}$ hat Häufungswert 0.
        \item $a_n=(-1)^n$ hat Häufungswerte $1$, $-1$.
        \item $a_n=n$ hat keinen Häufungswert.
        \item $a_n=(-1)^n+\frac{1}{n}$ hat Häufungswerte 1, -1.
    \end{enumerate}
\end{beispiel}

\begin{satz}[Häufungswertkriterium über Teilfolgen]
    \label{satz:haeufungswert-teilfolge}
    Eine reelle Zahl $a$ ist genau dann Häufungswert einer Folge $(a_n)_n$, wenn eine Teilfolge von $a_n$ existiert, die gegen $a$ konvergiert.

    \begin{proof}
        \anf{$\impl$}: $a$ sei Häufungswert von $(a_n)_n$. Das heißt
        \begin{align*}
            \forall \varepsilon > 0~\forall L\in\N~\exists n>L\colon a-\varepsilon < a_n < a+\varepsilon
        \end{align*}
        \begin{enumerate}[label=\arabic*)]
            \item Wir wählen $\varepsilon = 1 \impl \exists n_1\in\N\colon a-1 < a_{n_1} < a+1$
            \item Wir wählen $\varepsilon = \frac{1}{2}$, $L=n_1 + 1 \impl \exists n_2 > L > n_1\colon a-\frac{1}{2} < a_{n_2} < a + \frac{1}{2}$
            \item Wir wählen $\varepsilon = \frac{1}{3} \impl \exists n_3 > n_2\colon a - \frac{1}{3} < a_{n_3} < a + \frac{1}{3}$
            \item[$j$)] Wir wählen $\varepsilon = \frac{1}{j+1} \impl \exists n_{j+1} > n_j\colon a-\frac{1}{j+1} < a_{n_{j+1}} < a + \frac{1}{j+1}\qquad \impl n_j < n_{j+1}\quad\forall j \in \N$
        \end{enumerate}
        \noindent Wir definieren $\kappa\pair{j} \definedas n_j$ und $b_j \definedas a_{\kappa(j)}$ als eine Teilfolge von $(a_n)_n$. Es gilt
        \begin{align*}
            a-\frac{1}{j} < b_j < a + \frac{1}{j}\quad\forall j\in\N
        \end{align*}
        und nach Satz~\ref{satz:sandwich} konvergiert $(b_j)_j$ gegen $a$.\qedhere
    \end{proof}
    \begin{uebung}
        Beweisen Sie mittels Konvergenzkriterien und der Definition von Häufungswerten die Rückrichtung des vorherigen Satzes.
    \end{uebung}
\end{satz}

\newpage

%%%%%%%%%%%%%%%%%%%%%%%%
% 28. November 2023
%%%%%%%%%%%%%%%%%%%%%%%%

\subsection{Größter und kleinster Häufungswert - Limes superior und Limes inferior}
\marginnote{[28. Nov]}

Es sei $(a_n)_{n}$ eine beschränkte reelle Folge.
\begin{align*}
    x_n \definedas \sup_{l\geq n} a_l
\end{align*}
ist monoton fallend, weil
\begin{align*}
    \sup_{l\geq n} a_l = \max(a_n, \sup_{l\geq n+1} a_l) \geq \sup_{l\geq n+1} a_l = x_{n+1}
\end{align*}
$(x_n)_{n}$ ist monoton fallend und nach unten beschränkt.
\begin{align*}
    \annot{\impl}{\ref{satz:monoton-konv}} x = \lim_{\ntoinf} x_{n} = \lim_{n\fromto\infty} \sup_{l\geq n} a_l = \inf \sup_{l\geq n} a_l
\end{align*}
\noindent existiert.\\[10pt]
Genauso: $y_n\definedas \inf_{l\geq n} a_l \impl y_n < y_{n+1}$ und $y_n$ ist nach oben beschränkt und damit existiert:
\begin{align*}
    y = \lim_{\ntoinf} y_n = \lim_{n\fromto\infty} \inf_{l\geq n} a_l = \sup \inf_{l\geq n} a_j
\end{align*}

\begin{definition}[Limes superior und inferior] % Definition 1
    Sei $(a_n)_{n}$ eine beschränkte reelle Folge.
    \begin{align*}
        \limsup_{n\fromto\infty} a_n \definedas \lim_{n\fromto\infty} \sup_{l\geq n} a_l = \inf \sup_{l\geq n} a_l \tag{Limes superior}\\
        \liminf_{n\fromto\infty} a_n \definedas \lim_{n\fromto\infty} \inf_{l\geq n} a_l = \sup \inf_{l\geq n} a_l \tag{Limes inferior}
    \end{align*}
    Damit gilt außerdem
    \begin{align*}
        y_n < x_n\quad \inf_{l\geq n} a_l &\leq \sup_{l\geq n} a_l\\
        \impl \liminf_{n\fromto\infty} a_n &\leq \limsup_{n\fromto\infty} a_n
        \intertext{und}
        \limsup_{n\fromto\infty} \pair{-a_n} &= -\liminf_{n\fromto\infty} a_n\\
        \liminf_{n\fromto\infty} \pair{-a_n} &= -\limsup_{n\fromto\infty} a_n
    \end{align*}
\end{definition}

\begin{beispiel}
    \begin{align*}
        a_n = (-1)^n \impl \liminf_{n\fromto\infty} a_n &= -1\\
        \limsup_{n\fromto\infty} a_n &= 1\\[10pt]
        \limsup_{n\fromto\infty} (-a_n) = - \liminf_{n\fromto\infty} a_n\\
        \liminf_{n\fromto\infty} (-a_n) = -\limsup_{n\fromto\infty} a_n
    \end{align*}
\end{beispiel}
\newpage

\begin{lemma}[Charakterisierung von $\limsup$ und $\liminf$] % Lemma 2
    \label{lemma:limsup-charak}
    Sei $(a_n)_{n}$ beschränkte reelle Folge. Dann gilt:
    \begin{align*}
        a^{*} = \limsup a_n\quad\equivalent\quad \forall \varepsilon > 0~
        &
        \begin{array}{l}
            \text{ist } a_n < a^{*} + \varepsilon\text{ für fast alle $n$} \\
            \text{und } a_n > a^{*} -\varepsilon\text{ für unendlich viele $n$}
        \end{array}
        \\[10pt]
        a_{*} = \liminf a_n\quad\equivalent\quad \forall \varepsilon > 0~
        &
        \begin{array}{l}
            \text{ist } a_n > a_{*} - \varepsilon\text{ für fast alle $n$} \\
            \text{und } a_n < a_{*} + \varepsilon\text{ für unendlich viele $n$}
        \end{array}
    \end{align*}
    \begin{proof}[Beweis für ersten Teil des Lemmas, zweiter analog.]
        \anf{$\impl$}:
        \begin{align*}
            a^{*} = \limsup_{n\fromto\infty} a_n &= \inf_{n\in\N} \sup_{l\geq n} a_l\\
            \intertext{Angenommen $\exists \varepsilon > 0\colon a_l \geq a^{*} +\varepsilon$ für unendlich viele $l$}
            \impl \forall n\in\N\colon \sup_{l\geq n} a_l &\geq a^{*} + \varepsilon\\
            \impl a^{*} = \inf_{n\in\N} \sup_{l\geq n} a_l &\geq a^{*} + \varepsilon\quad\text{ (Widerspruch)}
            \intertext{Angenommen $\exists \varepsilon_0 > 0\colon a_n \leq a^{*} -\varepsilon_0$ für fast alle $n$}
            \impl \sup_{l\geq n} a_l &\leq a^{*} - \varepsilon_0\\
            \impl a^{*} = \lim_{n\fromto\infty} \sup_{l\geq n} a_l &\leq a^{*} - \varepsilon_0\quad\text{(Widerspruch)}
        \end{align*}
        \anf{$\Leftarrow$}: Sei $a^{*}\in\R$
        \begin{align*}
            \forall \varepsilon > 0\colon a_n &< a^{*} + \varepsilon\text{ für fast alle $n$}\\
            a_n &> a^{*} - \varepsilon\text{ für unendlich viele $n$}\\
            \impl \exists k\in\N\colon a_l &< a^{*} + \varepsilon\quad\forall l\geq k\\[10pt]
            \impl \forall n\geq k\colon \sup_{l\geq n} a_l &\leq a^{*} + \varepsilon\\
            \sup_{l\geq n} a_l &> a^{*} - \varepsilon\\
            \impl \forall n \geq k\colon a^{*} - \varepsilon &< \sup_{l\geq n} a_l \leq a^{*} + \varepsilon\\
            \impl a^{*} - \varepsilon \leq \lim_{n\fromto\infty} \sup_{l\geq n} a_l &\leq a^{*} + \varepsilon\quad \forall \varepsilon > 0\\
            \impl \limsup_{n\fromto\infty} a_n &= a^{*}\qedhere
        \end{align*}
    \end{proof}
\end{lemma}

\begin{satz}[Eigenschaften von $\limsup$ und $\liminf$] % Satz 3
    \label{satz:limsup-haeufungspunkt}
    Sei $(a_n)_n$ eine beschränkte reelle Folge und $H(a_n)$ die Menge der Häufungspunkte von $a_n$. Dann gilt
    \begin{align*}
        \limsup a_n, \liminf a_n \in H(a_n)\tag{1}
    \end{align*}
    Insbesondere ist $H(a_n) \neq \emptyset$. Ferner ist
    \begin{align*}
        \forall x \in H(a_n)\colon \liminf a_n \leq x \leq \limsup a_{n}\tag{2}
    \end{align*}

    \begin{proof}[Beweis von (1)]
        \begin{align*}
            a^{*} &\definedas\limsup a_n\\
            \annot{\impl}{\ref{lemma:limsup-charak}} \forall\varepsilon > 0\colon a_n &< a^* + \varepsilon\text{ für fast alle } n\\
            a_n &> a^*-\varepsilon\text{ für unendlich viele }n\\[10pt]
            \impl \forall\varepsilon > 0\colon a^* - \varepsilon &< a_n < a^* + \varepsilon\text{ für unendlich viele } n\\
            \impl &a^*\text{ ist Häufungswert von } (a_n)_n\qedhere
        \end{align*}
    \end{proof}
    \noindent Analog lässt sich der Beweis auch für $\liminf a_n$ führen.

    \begin{proof}[Beweis von (2)]
        Sei $a$ Häufungswert von $(a_n)_n$.
        Annahme: $a>a^*\definedas\limsup a_n$:
        \begin{align*}
            &\impl a-\varepsilon < a_n < a + \varepsilon\text{ für unendlich viele } n
            \intertext{Wähle $\varepsilon = \frac{a-a^*}{2}$}
            &\impl a - \frac{a-a^*}{2} < a_n < a + \frac{a-a^*}{2}\\
            &\impl a_n > a - \frac{a-a^*}{2} = \frac{a+a^*}{2} = a^* + \varepsilon\text{ für unendlich viele }n
            \intertext{Widerspruch zu Lemma~\ref{lemma:limsup-charak}}
            &\impl a \leq \limsup a_n\qedhere
        \end{align*}
    \end{proof}
    \noindent Mit $-a_n$ lässt sich analog zeigen, dass $a\geq \liminf a_n\quad\forall$ Häufungspunkte $a$ von $(a_n)_n$.
\end{satz}

\begin{notation}[Limes superior/inferior von unbeschränkten Folgen]
    Es sei $(a_n)_n$ nicht nach oben beschränkt. Dann setzten wir
    \begin{align*}
        \limsup a_n &\definedas +\infty
    \end{align*}
    Ist $(a_n)_n$ nicht nach unten beschränkt. Dann setzen wir
    \begin{align*}
        \liminf a_n &\definedas -\infty
    \end{align*}
\end{notation}

\begin{korollar}[Konvergenzkriterium nach limes superior und limes inferior]
    Eine reelle Folge $(a_n)_n$ konvergiert genau dann, wenn $(a_n)_n$ beschränkt ist und $\limsup a_n = \liminf a_n$.
    \begin{proof}
        \anf{$\impl$}:\\
        Jede konvergente Folge ist beschränkt. $(a_n)_n$ konvergiert gegen $a$ genau dann, wenn $H(a_n) = \set{a}$\\
        $\impl \liminf a_n = a = \limsup a_n$\\[10pt]
        \anf{$\Leftarrow$}: Sei $\liminf a_n = \limsup a_n = a$
        \begin{align*}
            \annot{\impl}{\ref{lemma:limsup-charak}} \forall \varepsilon > 0\colon a_n &< a+\varepsilon \text{ für fast alle $n$}\\
            \text{ und } a_n &> a-\varepsilon \text{ für fast alle } n\\
            \impl \lim a_n &= a\qedhere
        \end{align*}
    \end{proof}
\end{korollar}

\begin{uebung}
    Zeigen Sie: Wenn $\liminf$ und $\limsup$ als reelle Zahlen existieren, dann ist die Folge beschränkt.
\end{uebung}

\begin{satz}[Satz von Bolzano-Weierstraß] % Satz 5
    \label{satz:bolzano-weierstrass}
    Jede beschränkte reelle Folge $(a_n)_n$ besitzt mindestens einen Häufungspunkt.

    \begin{proof}
        $a^* \definedas \limsup a_n$ ist ein Häufungspunkt von $(a_n)_n$ nach Satz~\ref{satz:limsup-haeufungspunkt}.
    \end{proof}
\end{satz}

\begin{korollar}
    \label{korollar:beschr-konv-teilfolge}
    Jede beschränkte reelle Folge $(a_n)_n$ hat eine konvergente Teilfolge.
    \begin{proof}
        Nach Satz~\ref{satz:bolzano-weierstrass} ist $H(a_n) \neq\emptyset$ und nach Satz~\ref{satz:haeufungswert-teilfolge} gibt es zu jedem Häufungspunkt eine konvergente Teilfolge.
    \end{proof}
\end{korollar}

\vfill

\subsection{Das Konvergenzkriterium von Cauchy}

\begin{definition}[Cauchy-Folge]
    Eine reelle Folge $(a_n)_n$ heißt Cauchy-Folge, falls
    \begin{align*}
        &\forall\varepsilon > 0~\exists N\in\N\colon \abs{a_n - a_m} < \varepsilon\quad\forall n,m \geq N\\
        (\equivalent &\forall \varepsilon > 0~\exists N\in\N\colon \abs{a_n - a_m} < \varepsilon\quad\forall n\geq m\geq N)
    \end{align*}
\end{definition}

\begin{lemma} % Lemma 2
    \label{lemma:konv-cauchy}
    Jede konvergente reelle Folge $(a_n)_n$ ist eine Cauchy-Folge.
    \begin{proof}
        Es sei $(a_n)_n\fromto a$ eine reelle Folge.
        \begin{align*}
            \forall \varepsilon > 0~\exists N\in\N\colon &\abs{a_n-a} < \frac{\varepsilon}{2}\quad\forall n\geq N\\
            \impl\text{ Sei }n,m\geq N \impl &\abs{a_n-a_m} = \abs{a_n-a+a-a_m}\\
            \leq &\abs{a_n-a} + \abs{a-a_m} < \frac{\varepsilon}{2} + \frac{\varepsilon}{2} = \varepsilon\qedhere
        \end{align*}
    \end{proof}
\end{lemma}

\begin{lemma} % Lemma 3
    \label{lemma:beschr-cauchy}
    Jede reelle Cauchy-Folge ist beschränkt.
    \begin{proof}
        Es sei $\varepsilon = 1$
        \begin{align*}
            \impl \exists N\colon \abs{a_n-a_m} &< 1\quad \forall n,m\geq N\\
            \impl \forall n,m\geq N\colon\abs{a_n} &= \abs{a_n-a_m+a_m}\\
            &\leq \abs{a_n-a_m} + \abs{a_m}\\
            &< 1 + \abs{a_m}
            \intertext{Da wir $m=N$ wählen können, gilt}
            \impl \forall n\geq N\colon \abs{a_n} &< 1 + \abs{a_N}
            \intertext{Es gibt eine Schranke ab dem $N$-ten Folgenglied und damit gilt}
            \impl \abs{a_n} &\leq \max\pair{\abs{a_1}, \abs{a_2},\dots, \abs{a_{N-1}}, 1+\abs{a_N}}\qedhere
        \end{align*}
    \end{proof}
\end{lemma}

\vfill

\newpage

\begin{lemma} % Lemma 4
    \label{lemma:cauchy-konv-teilfolge}
    Eine reelle Cauchy-Folge $(a_n)_n$ konvergiert genau dann, wenn sie eine konvergente Teilfolge hat.

    %%%%%%%%%%%%%%%%%%%%%%%%
    % 30. November 2023
    %%%%%%%%%%%%%%%%%%%%%%%%

    \begin{proof}
        \marginnote{[30. Nov]}
        ~\\
        \anf{$\impl$}: Klar, weil nach Satz~\ref{satz:konv-teilfolgen-umordnungen} jede Teilfolge einer konvergenten Folge konvergiert.\\[10pt]
        \anf{$\Leftarrow$}: Sei $(b_j)_j$ eine Teilfolge von $(a_n)_n$ mit $b_j = a_{n_j}$ und $n_1 < n_2 < \dots < n_j < n_{j+1} < \dots$\\
        Es sei $a\definedas \lim_{j\fromto\infty} b_j$. Behauptung: $\lim_{n\fromto\infty} a_n = a$\\[10pt]
        \noindent Wir wissen:
        \begin{align*}
            \forall\varepsilon>0~\exists N_1 \in\N&\colon \abs{a-a_{n_j}} < \frac{\varepsilon}{2}\quad\forall j\geq N_1\\
            \exists N_2 \in\N&\colon \abs{a_n-a_m} < \frac{\varepsilon}{2}\quad\forall m\geq n\geq N_2\\[10pt]
            \impl \abs{a_n-a} &= \abs{a_n-a_{n_j} + a_{n_j} - a}\\
            &\leq \abs{a_n-a_{n_j}} + \abs{a_{n_j}-a}
            \intertext{Wir wählen $j\geq\max\pair{N_1,N_2}$}
            \impl \abs{a_n-a} &\leq \abs{a_n-a_{n_j}} + \abs{a_{n_j}-a}\\
            &<\frac{\varepsilon}{2} + \frac{\varepsilon}{2} = \varepsilon \quad\forall n\geq N\\
            \impl \lim_{n\fromto\infty} a_n &= a\qedhere
        \end{align*}
    \end{proof}
\end{lemma}

\begin{satz} % Satz 5
    \label{satz:jede-konv-cauchy}
    Jede reelle Folge konvergiert genau dann, wenn sie eine Cauchy-Folge ist.

    \begin{proof}
        \anf{$\impl$}: Folgt direkt aus Lemma~\ref{lemma:konv-cauchy}\\[10pt]
        \anf{$\Leftarrow$}: Sei $(a_n)_{n\in\N}$ eine Cauchy-Folge $\annot{\impl}{Lemma~\ref{lemma:beschr-cauchy}} (a_n)_n$ ist beschränkt $\annot{\impl}{Satz~\ref{korollar:beschr-konv-teilfolge}} (a_n)_n$ hat eine konvergente Teilfolge $\annot{\impl}{Lemma~\ref{lemma:cauchy-konv-teilfolge}} (a_n)_n$ ist konvergent
    \end{proof}
\end{satz}

\newpage