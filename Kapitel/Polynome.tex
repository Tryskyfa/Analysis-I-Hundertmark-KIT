\subsection{Reelle Polynome}
\thispagestyle{pagenumberonly}

\begin{definition}[Reelles Polynom]
    Es sei $a_0,a_1, \dots, a_n\in\R$,~$a_n \neq 0,~\in\R$. Dann ist
    \begin{align*}
        P(x) &\definedas a_0 + a_1 \cdot x + a_2 \cdot x^2 + \dots + a_n \cdot x^n
    \end{align*}
    ein reelles Polynom vom Grad $n$ mit $\grad(P) = n$.
\end{definition}

\begin{definition}[Nullpolynom]
    $P(x) = 0$ ist das Nullpolynom. Ein Polynom $P$ ist nicht-trivial, wenn es nicht das Nullpolynom ist.
\end{definition}

\begin{bemerkung}[Analytische Polynome]
    Ähnlich zu reellen Polynomen lassen sich auch analytische Polynome definieren. Es sei $a_0, a_1, \dots, a_n \in\C,~a_0\neq 0,~z\in\C$. Dann ist
    \begin{align*}
        P(z) &\definedas a_0 + a_1\cdot z + a_2\cdot z^2 + \dots + a_n\cdot z^n
    \end{align*}
    eine analytisches Polynom mit $\grad(P) = n$.
\end{bemerkung}

\begin{definition}[Grad von speziellen Polynomen]
    Wir definieren den Grad von konstanten Polynomen der Form $P(x) = a_0$ als 0 und den Grad des Nullpolynoms als $-1$.
\end{definition}

\begin{satz}[Eigenschaften von Polynomen]
    \theoremescape
    \label{satz:eigenschaften-polynome}
    \begin{enumerate}[label=(\roman*)]
        \item Sei $P$ ein Polynom mit $\grad(P) = n$ und $\lambda\in\R,~\lambda\neq 0$. Dann ist $Grad(\lambda P) = n$.
        \item Seien $P, Q$ nicht-triviale Polynome mit $\grad(P) = n$ und $\grad(Q)=m$. Dann gilt $PQ$ ist ein Polynom mit $\grad(PQ)=n+m$.
        \begin{proof}
            \begin{align*}
                P(x) &= \sum_{j=0}^{n} a_j x^j\\
                Q(x) &= \sum_{i=0}^{n} b_i x^i\\
                P(x) \cdot Q(x) &= \pair{\sum_{j=0}^{n} a_j x^j}\cdot \pair{\sum_{i=0}^{n} b_i x^i}\\
                &= \sum_{j=0}^{n} \sum_{l=0}^{m} a_{j} b_l\cdot x^{j+l}
                \intertext{Wir setzen $k= j+l\in\set{0,1,\dots,n+m}$}
                &= \sum_{k=0}^{n+m} \pair{\sum_{\substack{j=0\\ 0\leq k-j\leq m}}^{n} a_j b_{k-j}} x^k\\
                &= a_n b_m\cdot x^{n+m} + \textit{Terme niedrigerer Ordnung}\qedhere
            \end{align*}
        \end{proof}
        \item Entwicklung in einem anderen Punkt. Wir schreiben $x = \eta + \zeta$
        \begin{align*}
            P(x) &= \sum_{j=0}^{n} a_j x^j = \sum_{j=0}^{n} a_j \cdot\pair{\eta + \zeta}^j\\
            \annot[{&}]{=}{\ref{satz:binom-lehrsatz}} \sum_{j=0}^{n} a_j \cdot\pair{\sum_{l=0}^{j} \binom{j}{l}\cdot\eta^{l}\cdot\zeta^{j-l}}
            \intertext{Es gilt $\binom{j}{l}=0$, falls $l>j$. Also können wir auch schreiben}
            &= \sum_{j=0}^{n} a_j \cdot\pair{\sum_{l\geq 0} \binom{j}{l}\cdot\eta^{l}\cdot\zeta^{j-l}}\\
            &= \sum_{l\geq 0} \pair{\sum_{j=0}^{n} a_j\cdot \binom{j}{l}\cdot\zeta^{j-l}}\cdot\eta^l\\
            &= \sum_{l\geq 0} \underbrace{\pair{\sum_{j=0}^{n} \binom{j}{l}\cdot a_j\cdot\zeta^{j-l}}}_{\definedasbackwards b_l}\cdot\pair{x-\zeta}^l\\
            &= \sum_{l=0}^{n} b_l \cdot\pair{x-\zeta}^l
            \intertext{Wir betrachten $l=0$}
            b_0 &= \sum_{j=0}^{n} \binom{j}{0}\cdot a_j\cdot\zeta^{j} = \sum_{j=0}^{n} a_j \zeta^j = P(\zeta)\\
            \impl P(x) &= P(\zeta) + \sum_{l=1}^{n} b_l \cdot\pair{x-\zeta}^l\\
            &= P(\zeta) + (x-\zeta) \cdot \underbrace{\sum_{l=0}^{n-1} b_{l+1} \cdot \pair{x-\zeta}^l}_{\definedasbackwards Q(x-\zeta)}\\
            &= P(\zeta) + (x-\zeta)\cdot Q(x-\zeta)
        \end{align*}
        Dabei gilt außerdem $\grad(Q) = n-1$. Diesen Zusammenhang werden wir später in Satz~\ref{satz:nullstellen-polynome} verwenden, um eine Aussage über die Anzahl an Nullstellen eines Polynoms zu treffen.
    \end{enumerate}
\end{satz}

%%%%%%%%%%%%%%%%%%%%%%%%
% 9. Januar 2024
%%%%%%%%%%%%%%%%%%%%%%%%

\begin{satz}[Nullstellensatz für Polynome]
    \marginnote{[9. Jan]}
    \label{satz:nullstellen-polynome}
    Jedes nicht-triviale Polynom von Grad $n$ hat höchstens $n$ Nullstellen.

    \begin{proof}
        Induktion in $n$.~\\
        \begin{induktionsanfang}
            Für $n=0$ (konstantes Polynom) stimmt die Behauptung.
        \end{induktionsanfang}
        \begin{induktionsschritt}
            Angenommen die Behauptung stimmt für Polynom von Grad $n$. Sei $P$ Polynom von Grad $n+1$.\\
            1. Fall: $P$ hat keine Nullstelle $\impl$ Die Behauptung stimmt für $P$.\\
            2. Fall: $P(\zeta) = 0$ für ein $\zeta\in\R \annot{\impl}{\ref{satz:eigenschaften-polynome}} P(x) = (x-\zeta)\cdot Q\pair{x-\zeta}$. $Q$ ist Polynom von Grad $n$ mit höchstens $n$ Nullstellen. $\impl$ Anzahl der Nullstellen von $P$ ist $\leq n+1$.\qedhere
        \end{induktionsschritt}
    \end{proof}
\end{satz}

\begin{korollar}
    Sind $P,Q$ Polynome von Grad $\leq n$ und stimmen $P,Q$ an $n+1$ verschiedenen Stellen überein, so ist $P=Q$.

    \begin{proof}
        $h\definedas P-Q$ ist Polynom von Grad $\leq n$. Nach Satz~\ref{satz:nullstellen-polynome} hat $h$ damit höchstens $n$ Nullstellen, sofern $h$ nicht-trivial ist. Aber nach Voraussetzung existieren paarweise verschiedene $x_1,\dots, x_{n+1}$ mit
        \begin{alignat*}{2}
            P(x_j) &= Q(x_j)\quad&\forall j=1,\dots,n+1\\
            \impl h(x_j) &= 0\quad&\forall j=1,\dots,n+1
        \end{alignat*}
        Damit ist $h$ nach Satz~\ref{satz:nullstellen-polynome} trivial, das heißt $(\forall x\colon H(x) = 0) \impl (\forall x\colon P(x)=Q(x))$.
    \end{proof}
\end{korollar}

\begin{bemerkung}
    Der vorherige Beweis lässt sich auch auf analytische Polynome in den komplexen Zahlen übertragen.
\end{bemerkung}

\begin{korollar}[Koeffizientenvergleich]
    Zwei Polynome $P,Q$ sind genau dann gleich, wenn sie dieselben Koeffizienten haben.
\end{korollar}

\newpage