%%%%%%%%%%%%%%%%%%%%%%%%
% 26. Oktober 2023
%%%%%%%%%%%%%%%%%%%%%%%%

\imaginarysubsection{Aussagenlogik}
\thispagestyle{pagenumberonly}

\begin{definition}[Aussage]
    \marginnote{[26. Okt]}
    Eine Aussage ist eine Behauptung sprachlich oder mittels Formeln, welche entweder wahr oder falsch ist.
\end{definition}

\begin{beispiel}[Zulässige Aussagen]
    \theoremescape
    \begin{enumerate}[label=(\roman*)]
        \item Bielefeld existiert (w)
        \item $2+2=5$ (f)
        \item Es gibt unendlich viele Primzahlen (w)
    \end{enumerate}
\end{beispiel}

\begin{definition}[Aussageform]
    Eine Aussage, die von mindestens einer Variablen abhängt, nennt sich Aussageform.
    Wir schreiben zum Beispiel $H(x)$ für eine Aussage für die Variable $x$.
\end{definition}

\begin{beispiel}[Mögliche Aussageformen]
    \theoremescape
    \begin{enumerate}[label=(\roman*)]
        \item $H(x) \definedasequiv \pair{x^2-3x+2=0}$
        \item $G(x) \definedasequiv \pair{x=1\lor x=2}$
    \end{enumerate}
\end{beispiel}

\begin{konzept}[Beweisstruktur]
    \begin{equation*}
        \begin{split}
            p\\
            \text{Vorraussetzung}\\
            \text{hinreichend für }q
        \end{split}
        \begin{split}
            \qquad\impl\qquad
        \end{split}
        \begin{split}
            q\\
            \text{Behauptung}\\
            \text{notwendig für }p
        \end{split}
    \end{equation*}
    \noindent Beweis: $p\impl r_1 \impl r_2 \impl r_3 \impl \dots\impl r_n \impl q$. ($r_1,\dots, r_n$ sind bereits bekannte wahre Aussagen oder Axiome)
\end{konzept}

\begin{satz}[Regeln der Aussagenlogik]
    Seien $p,q,r$ Aussagen.
    Dann sind folgende Aussagen wahr:
    \begin{enumerate}[label=(\roman*)]
        \item $p \lor \neg p$ \aligntoright{(Tertium non datur)}{0.1}\\
        $p \impl p$\\
        $\neg (p \land \neg p)$
        \item $p\land q \equivalent q \land p$ \aligntoright{(Kommutativität)}{0.1}\\
        $p\lor q \equivalent q \lor p$
        \item $(p\land q) \land r \equivalent p \land (q\land r)$ \aligntoright{(Assoziativität)}{0.1}\\
        $(p\lor q) \lor r \equivalent p \lor (q\land r)$
        \item $\neg(p\land q) \equivalent \neg p \lor \neg q$ \aligntoright{(De Morgan)}{0.1}\\
        $\neg(p\lor q) \equivalent \neg p \land \neg q$
        \item $p\impl q \equivalent \neg p \lor q$ \aligntoright{(Definition der Implikation)}{0.1}
        \item $(p\equivalent q) \equivalent (p\impl q) \land (q\impl p)$ \aligntoright{(Definition der Äquivalenz)}{0.1}\\
        $(p\equivalent q) \equivalent (p\land q) \lor (\neg p \land \neg q)$
        \item $(p\equivalent q) \land (q\equivalent r) \impl (p \equivalent r)$ \aligntoright{(Transitivität)}{0.1}
    \end{enumerate}
\end{satz}

\newpage