\thispagestyle{pagenumberonly}

Es gibt eine Menge $\realnumbers$, genannt reelle Zahlen, die 3 Gruppen von Axiomen erfüllt:
\begin{enumerate}
    \item Algebraische Axiome
    \item Anordnungsaxiome
    \item Das Vollständigkeitsaxiom
\end{enumerate}

\subsection{Algebraische Axiome}
In $\realnumbers$ gibt es 2 Operationen:
\begin{enumerate}
    \item Addition \anf{+}
    \item Multiplikation \anf{$\cdot$}
\end{enumerate}

\begin{folgerung}
    $a,b\in\realnumbers\impl a+b\in \realnumbers$ und $a \cdot b\in\realnumbers$
\end{folgerung}

\begin{definition}[Eigenschaften eines Körpers]
    \theoremescape
    \begin{enumerate}[label=(I.\arabic*)]
        \item $(a+b) + c = a + (b+c)$\quad \textit{Assoziativität der Addition}
        \item $a + b = b + a$\quad \textit{Kommutativität der Addition}
        \item Es gibt genau eine Zahl genannt Null, geschrieben $0$, mit $\forall a \in\realnumbers\colon a+0 = a$ \quad \textit{Existenz eines neutralen Elements der Addition}
        \item $\forall a\in\realnumbers~\exists! b\in \realnumbers\colon a + b = 0$, geschrieben $b=-a$\quad \textit{Existenz eines inversen Elements der Addition}
        \item $\pair{a\cdot b}\cdot c = a\cdot\pair{b\cdot c}$\quad \textit{Assoziativität der Multiplikation}
        \item $a\cdot b = b \cdot a$\quad \textit{Kommutativität der Multiplikation}
        \item $\forall a \in\realnumbers, a \neq 0$ gibt es ein eindeutiges $b\neq 0$ mit $a\cdot b = 1$. Wir schreiben $b = a^{-1} = \frac{1}{a}$\quad \textit{Existenz eines inversen Elements der Multiplikation}
        \item Es gibt genau eine Zahl Eins, geschrieben 1, die von 0 verschieden ist, mit $\forall a\in\realnumbers\colon a\cdot 1 = a$\quad \textit{Existenz eines neutralen Elements der Multiplikation}
        \item $a\cdot\pair{b+c} = a \cdot b + a \cdot c$\quad \textit{Distributivität}
    \end{enumerate}

    \noindent Jede Menge $\mathbb{K}$, welche (I.1) bis (I.9) erfüllt, heißt \textbf{Körper}.
\end{definition}

\begin{bemerkung}
    Dass die Eindeutigkeit von 0 und 1 durch die Axiome gefordert wird, ist nicht unbedingt erforderlich.\footnote{Seien $0, 0'$ neutrale Elemente bezüglich der Addition. $\impl 0 = 0+0' = 0'+0=0'$}
\end{bemerkung}

\begin{bemerkung}
    Das inverse Elemente bezüglich Addition und Multiplikation ist eindeutig.
    \begin{proof}
        Annahme: $a+b=0$ und $a+b'=0$
        \begin{align*}
            &\impl b + 0 = b + \pair{a+b'} = b' + \pair{a+b} = b' + 0\\
            &\impl b = b'\qedhere
        \end{align*}
    \end{proof}
\end{bemerkung}

\begin{notation}
    \begin{align*}
        a-b &\definedas a+(-b)\tag{Differenz}\\[10pt]
        \frac{a}{b} &\definedas a \cdot b^{-1}\tag{Quotient}
    \end{align*}
\end{notation}

\begin{satz}[Abgeleitete Regeln]
    Es gilt\\
    (I.10)
    \begin{align}
        -\pair{-a} &= a\\[8pt]
        \pair{-a} + \pair{-b} &= -\pair{a+b}\\[8pt]
        \pair{a^{-1}}^{-1} &= a\\[8pt]
        a^{-1}\cdot b^{-1} &= \pair{a\cdot b}^{-1}\\[8pt]
        a\cdot 0 &= 0\\[8pt]
        a\cdot\pair{-b} &= -\pair{a\cdot b}\\[8pt]
        (-a)\cdot(-b) &= a\cdot b\\[8pt]
        a\cdot\pair{b-c} &= a\cdot b - a\cdot c
    \end{align}
    \noindent (I.11) Ist $a\cdot b = 0$ so ist mindestens eine der Zahlen $a$ oder $b$ gleich Null.

    \begin{proof}[Beweis zu (I.10.5)]
        Zu zeigen: $a \cdot 0 = 0$
        \begin{align*}
            a\cdot 0 + a \cdot 0 &= a \cdot\pair{0+0}\\
            &= a \cdot 0\\
            \impl \pair{a\cdot 0 + a \cdot 0} + \pair{-a \cdot 0} &= a \cdot 0 + \pair{-a\cdot 0}\\
            \impl a \cdot 0 + \pair{a\cdot 0 + \pair{-a\cdot 0}} &= 0\\
            \impl a\cdot 0 +0 &= 0\\
            \impl a\cdot 0 &= 0\qedhere
        \end{align*}
    \end{proof}
    \begin{proof}[Beweis zu (I.11)]
        Sei $a\cdot b = 0$.\\
        Ist $a\neq 0 \impl b = 1\cdot b = a^{-1}\cdot a \cdot b = a^{-1}\cdot (a\cdot b) = a^{-1} = 0 = 0$\\
        Ist $b\neq 0,$ so gilt analog, dass $a = 0$.
    \end{proof}
    \begin{uebung}
        Beweisen Sie die verbleibenden Regeln aus (I.10).
    \end{uebung}
\end{satz}

\begin{satz}[Regeln des Bruchrechnens]
    \label{satz:bruchrechnen}
    ~\\(I.12) Es gilt:
    \setcounter{equation}{0}
    \begin{alignat}{4}
        \frac{a}{b} + \frac{c}{d} &= \frac{ad+cb}{bd} &\text{ für } b,d&\neq 0\\[10pt]
        \frac{a}{b} \cdot \frac{c}{d} &= \frac{ac}{bd} &\text{ für } b,d&\neq 0\\[10pt]
        \frac{\frac{a}{b}}{\frac{c}{d}} &= \frac{ad}{bc} &\text{ für } b,c,d&\neq 0
    \end{alignat}
    \begin{uebung}
        Beweisen Sie Satz~\ref{satz:bruchrechnen}.
    \end{uebung}
\end{satz}

\subsection{Die Anordnungsaxiome}

Allgemein gilt: $a,b\in\realnumbers \impl a = b \lor a \neq b$.\\
Ist $a\neq b$, besteht eine Anordnung \anf{$<$}, die verlangt, dass genau eine der Relationen $a<b$ oder $b<a$ gilt.
Das heißt $\forall a,b\in\realnumbers$ gilt genau eine der Aussagen $a<b$, $b<a$, $a=b$.\\
Diese Anordnung genügt folgenden Axiomen:
\begin{axiom}[Anordnungsaxiome]
    \theoremescape
    \begin{enumerate}[label=(II.\arabic*)]
        \item $a<b \land b < c \impl a < c$\quad \textit{Transitivität}
        \item $a<b,~c \in\realnumbers \impl a + c < b + c$
        \item $a<b,~c > 0 \impl ac < bc$
    \end{enumerate}
\end{axiom}

\begin{notation}
    \theoremescape
    \begin{enumerate}[label=-]
        \item $a < b$: a ist (echt) kleiner als b
        \item $b > a$: b ist größer als a
        \item $a\leq b$: $a=b$ oder $a < b$
        \item $a\in\realnumbers$ ist positiv, wenn $a>0$; negativ, wenn $a <0$; nicht-negativ, wenn $a\geq 0$; nicht-positiv, wenn $a\leq 0$
    \end{enumerate}
\end{notation}

\begin{beispiel}
    $a<b\equivalent b - a > 0$
    \begin{proof}
        \begin{align*}
            a &<b\\
            \impl 0 = a + \pair{-a} &< b + \pair{-a} = b - a\\[10pt]
            b-a &>0\\
            \impl a &< a + \pair{b-a} = b\qedhere
        \end{align*}
    \end{proof}
\end{beispiel}

%%%%%%%%%%%%%%%%%%%%%%%%
% 7. November 2023
%%%%%%%%%%%%%%%%%%%%%%%%

\begin{satz}[Aus den Anordnungsaxiomen abgeleitete Regeln]
    \marginnote{[7. Nov]}
    \theoremescape
    \begin{enumerate}[label=(II.\arabic*)]
        \setcounter{enumi}{3}
        \item $a<b\equivalent b-a > 0$
        \item $a<0\equivalent -a > 0$ und $a>0\equivalent -a < 0$
        \item $a<b\equivalent -b < -a$
        \item $a<b \land c < d \equivalent a+c<b+d$
        \item $ab > 0 \equivalent \pair{a>0 \land b > 0}\lor\pair{a < 0 \land b < 0}$ und $ab < 0 \equivalent \pair{a>0 \land b < 0}\lor\pair{a < 0 \land b > 0}$
        \item $a\neq 0 \impl a^2 > 0$\quad(Insbesondere $1>0$)
        \item $a<b \land c<0 \impl ac > bc$
        \item $a>0 \equivalent \frac{1}{a}>0$
        \item $a^2 < b^2 \land a > 0 \land b > 0 \impl a < b$
    \end{enumerate}
    \newpage
    \begin{proof}
        \theoremescape
        \begin{enumerate}[label=(II.\arabic*)]
            \setcounter{enumi}{3}
            \item Sei $a<b \impl 0=a+(-a) \annot{<}{(II.2)} b + (-a) = b-a$.\\
            Ist $b-a>0 \annot{\impl}{(II.2)} a<a+(b-a)=b$
            \item Setze $b\definedas 0$ in (II.4) $\impl b-a=-a>0$.\\
            2ter Teil: Ersetze $a$ durch $-a$ in (II.5). ($a>0 \impl -a < 0 \equivalent -(-a)>0 \equivalent a >0$)
            \item (II.6) folgt aus (II.5), da $a<b\equivalent b-a>0 \equivalent (-a)-(-b) > 0\equivalent -b < -a$
            \item Sei $a<b \land c < d \annot{\impl}{(II.2)} a + c < b + c \land b + c < b + d \annot{\impl}{(II.1)} a+c < b + d$
            \item $a,b>0 \annot{\impl}{(II.3)} ab > 0\cdot b = 0$ und $a,b<0 \annot{\impl}{(II.5)} -a,-b>0 \impl (-a)(-b) > 0 \impl ab > 0$.\\
            Umkehrung: Sei $ab>0 \impl a\neq 0 \land b \neq 0$. Wäre $a>0 \land b < 0 \annot{\impl}{(II.5)} -b>0$. Wie gerade gezeigt folgt $a(-b) > 0 \impl -ab > 0 \annot{\impl}{(II.5)} ab < 0$ (Widerspruch zur Annahme).\\
            Genauso zeigt man, dass die Annahme $a<0 \land b > 0$ falsch ist.\\
            (Zweite Behauptung lässt sich analog zeigen).
            \item $a\neq 0 \equivalent a > 0 \lor a < 0 \annot{\impl}{(II.8)} a^2 = a \cdot a > 0$. Ferner ist $1\neq 0 \impl 1=1\cdot 1 > 0$
            \item Sei $c<0 \impl -c > 0$ und aus $a<b$ folgt $(-c)\cdot a < (-c)\cdot b \impl -c\cdot a < -c\cdot b \impl c\cdot b < c\cdot a$
            \item $a\cdot a^{-1} = 1 > 0$ (falls $a\neq 0$) $\annot{\impl}{(II.8)} a^{-1} > 0$ sofern $a>0$ ist und aus $a^{-1}>0$ folgt $a>0$
            \item Sei $a^{2}<b^2, a>0, b>0$. Angenommen $a<b$ ist falsch, d.h. $a\geq b \impl a^{2} \geq a \cdot a \geq a\cdot b\geq b\cdot b = b^{2}\impl a^{2}\geq b^{2}$ (Widerspruch)
        \end{enumerate}
    \end{proof}
\end{satz}

\newpage

\subsection{Das Vollständigkeitsaxiom}

\begin{axiom}[Vollständigkeitsaxiom]
    \label{axiom:vollstaendigkeitsaxiom}
    Jede nicht-leere Teilmenge $M\subseteq \realnumbers$, welche nach oben beschränkt ist, besitzt eine kleinste obere Schranke, genannt das Supremum von $M$.
\end{axiom}

\begin{notation}[Supremum]
    Das Supremum einer Menge $M$ schreiben wir als $\sup M$.
\end{notation}

\begin{definition}[Beschränktheit von Mengen]
    Sei $M\subseteq \realnumbers, M\neq \emptyset$.
    \begin{enumerate}[label=(\roman*)]
        \item $M$ heißt \textbf{nach oben beschränkt}, falls ein $k\in\realnumbers$ existiert mit $\forall x\in M\colon x\leq k$.
        Jede solche Zahl $k$ heißt obere Schranke von $M$.
        \item $M$ heißt \textbf{nach unten beschränkt}, falls ein $k\in\realnumbers$ existiert mit $\forall x\in M\colon x\geq k$.
        Jede solche Zahl $k$ heißt untere Schranke von $M$.
        \item $M$ heißt \textbf{beschränkt}, falls ein $k\geq 0$ existiert mit $-k\leq x \leq k\quad \forall x\in M$
    \end{enumerate}
\end{definition}

\begin{definition}[Kleinste obere und größte untere Schranke]
    Eine Zahl $k\in\realnumbers$ heißt kleinste obere (größte untere) Schranke, falls
    \begin{enumerate}
        \item es eine obere (untere) Schranke ist und
        \item es keine kleinere obere (größere untere) Schranke für $M$ gibt
    \end{enumerate}
\end{definition}

\begin{folgerung}
    \theoremescape
    Allgemein gilt
    \begin{align*}
        x \leq k \equivalent -k \leq -x
    \end{align*}
    das heißt für eine Menge $M\neq\emptyset$ gilt
    \begin{center}
        $k$ ist eine obere Schranke für $M$\\ $\equivalent -k$ ist eine untere Schranke für $-M\definedas\set{-x|~x\in M}$
    \end{center}
    und
    \begin{center}
        $k$ ist kleinste obere Schranke für M\\ $\equivalent -k$ ist die größte untere Schranke für $-M$
    \end{center}
    Das heißt das Anordnungsaxiom ist äquivalent zum \textit{Anordnungsaxiom}$^{-1}$ (Jede nicht-leere Teilmenge $M\subseteq \realnumbers$, welche nach unten beschränkt ist, besitzt eine größte untere Schranke, genannt das Infimum von $M$. Wir schreiben $\inf M$).
\end{folgerung}

\begin{beispiel}
    \theoremescape
    \begin{align*}
        M &\definedas \interv{0,1} = \set{x|~0\leq x \leq 1}\\
        \sup M &= 1 \qquad \inf M = 0\nn
        A &\definedas \pair{0,1} = \set{x|~0< x < 1}\\
        \sup A &= 1 \qquad \inf A = 0\\
    \end{align*}
\end{beispiel}

\begin{notation}
    Sei $M\subseteq \realnumbers, M\neq \emptyset$\\
    Wir schreiben $\sup M < \infty$, falls $M$ nach oben beschränkt ist, andernfalls setzen wir
    \begin{align*}
        \sup M \definedas \infty
    \end{align*}
    Falls $M$ nach unten beschränkt ist, schreiben wir $\inf M > -\infty$, andernfalls setzen wir
    \begin{align*}
        \inf M \definedas -\infty
    \end{align*}
\end{notation}

\begin{satz}[Eigenschaften des Supremums]
    \label{satz:sup}
    Sei $M\subseteq \realnumbers$
    \begin{enumerate}[label=(\roman*)]
        \item Ist $\sup M < \infty$, so folgt $\forall \varepsilon > 0~\exists x\in M$ mit $\sup\pair{M} -\varepsilon < x$
        \item Ist $\sup M = \infty$, so gilt $\forall k\geq 0~\exists x\in M$ mit $x> k$
    \end{enumerate}
    \begin{proof}
        \theoremescape
        \begin{enumerate}[label=(\roman*)]
            \item Wir setzen $a\definedas \sup M$. Sei $a<\infty$. Wäre (i) falsch, so folgt $\exists \varepsilon>0~\forall x\in M\colon a-\varepsilon > x$.\\Das heißt $a-\varepsilon$ ist eine obere Schranke für $M$. Aber $a-\varepsilon < a$ (Widerspruch)
            \item Ist $a=\infty$, so hat $M$ keine obere Schranke. Nach Def. folgt für jedes $k\in\realnumbers$ existiert ein $x\in M\colon x > k$\qedhere
        \end{enumerate}
    \end{proof}
\end{satz}
\begin{satz}[Eigenschaften des Infimums]
    \label{satz:inf}
    Sei $M\subseteq \realnumbers$
    \begin{enumerate}[label=(\roman*)]
        \item Ist $\inf M > -\infty$, so folgt $\forall \varepsilon > 0~\exists x\in M$ mit $x < \inf\pair{M}+ \varepsilon$
        \item Ist $\inf M = -\infty$, so gilt $\forall k\geq 0~\exists x\in M$ mit $x< -k$
    \end{enumerate}
    \begin{proof}
        Wende Satz~\ref{satz:sup} auf $-M\definedas\set{-x|~x\in M}$ an und beachte $\sup\pair{-M} = \inf\pair{M}$
    \end{proof}
\end{satz}
\begin{definition}[Maximum und Minimum]
    Es sei $M\subseteq \realnumbers, M\neq\emptyset$. $m\in M$ heißt größtes Element von $M$ (Maximum), geschrieben $\max M$, falls
    \begin{align*}
        x\leq m\quad\forall x\in M
    \end{align*}
    Entsprechend: $m\in M$ heißt kleinstes Element von $M$ (Minimum), geschrieben $\min M$, falls
    \begin{align*}
        x\geq m\quad\forall x\in M
    \end{align*}
\end{definition}

\begin{beispiel}
    Sei $M$ beschränkt, $M\neq\emptyset$
    \begin{align*}
        M&\definedas\set{x|~0\leq x< 1}\\
        \sup M &= 1\\
        \inf M &= \min M = 0\\
        M&\text{ hat kein Maximum}
    \end{align*}
\end{beispiel}

\newpage